\documentclass[11pt]{report}

% Encodage & langue
\usepackage[utf8]{inputenc}
\usepackage[T1]{fontenc}
\usepackage[french]{babel}

% Packages graphiques
\usepackage{tikz}
\usepackage{tikz-cd}
\usetikzlibrary{arrows.meta, positioning, calc}

% Maths
\usepackage{amssymb, amsthm}
\usepackage{amsmath}

% Bibliographie
\usepackage[backend=biber, style=numeric, citestyle=numeric, maxnames=3]{biblatex}
\addbibresource{bibliographie/bibliographie.bib}

% Marges
\usepackage[a4paper,margin=2.5cm]{geometry}

% Liens
\usepackage[colorlinks=true, linkcolor=blue, citecolor=blue, urlcolor=blue]{hyperref}

% Figures
\usepackage{graphicx}
\usepackage{float}
\usepackage{caption}
\usepackage{subcaption}

% En-têtes et pieds de page
\usepackage{fancyhdr}
\pagestyle{fancy}
\fancyhf{}
\fancyhead[L]{\leftmark}
\fancyhead[R]{\thepage}
\fancyfoot[C]{Rapport de stage - \textit{\authorname}}

% Table des matières
\usepackage{tocloft}
\setcounter{tocdepth}{3}

% Espacement
\usepackage{setspace}
\onehalfspacing

% Listes
\usepackage{enumitem}

% Code source (si nécessaire)
\usepackage{listings}
\usepackage{xcolor}

% Configuration listings
\lstset{
    basicstyle=\ttfamily\small,
    keywordstyle=\color{blue},
    commentstyle=\color{green!60!black},
    stringstyle=\color{red},
    showstringspaces=false,
    breaklines=true,
    frame=single,
    numbers=left,
    numberstyle=\tiny\color{gray}
}

% Théorèmes (si nécessaire pour un rapport technique)
\newtheorem{theoreme}{Théorème}[section]
\newtheorem{proposition}[theoreme]{Proposition}
\newtheorem{lemme}[theoreme]{Lemme}
\newtheorem{corollaire}[theoreme]{Corollaire}
\theoremstyle{definition}
\newtheorem{definition}[theoreme]{Définition}
\newtheorem{exemple}[theoreme]{Exemple}
\theoremstyle{remark}
\newtheorem{remarque}[theoreme]{Remarque}

% Commandes personnalisées pour les dérivées (si nécessaire)
\newcommand{\pt}{\partial_t}
\newcommand{\px}{\partial_x}
\newcommand{\pxx}{\partial_x^{(2)}}

% Dérivées temporelles
\newcommand{\dt}[1]{\partial_t #1}
\newcommand{\dtt}[1]{\partial_{tt} #1}

% Dérivées spatiales
\newcommand{\dx}[1]{\partial_x #1}
\newcommand{\dxx}[1]{\partial_{xx} #1}
\newcommand{\dxxx}[1]{\partial_{xxx} #1}

% Dérivées totales
\newcommand{\Dt}[1]{\frac{d #1}{dt}}
\newcommand{\Dtt}[1]{\frac{d^2 #1}{dt^2}}
\newcommand{\Dx}[1]{\frac{d #1}{dx}}
\newcommand{\Dxx}[1]{\frac{d^2 #1}{dx^2}}

% Informations du rapport
\newcommand{\authorname}{Alexandre \textsc{Edeline}}
\newcommand{\studentschool}{ENSTA Paris - Institut Polytechnique de Paris}
\newcommand{\companyname}{CMAP - École Polytechnique}
\newcommand{\companylocation}{Palaiseau, France}
\newcommand{\supervisor}{Marc \textsc{Massot} et Christian \textsc{Tenaud}}
\newcommand{\academicsupervisor}{Patrick \textsc{Ciarlet}}
\newcommand{\internshipperiod}{du 14/04/2025 au 15/09/2025}
\newcommand{\reporttitle}{Compression de maillage et problèmes d'évolution}
\newcommand{\reportsubtitle}{Intégration temporelle et multirésolution adaptative pour les EDP en temps.}

\begin{document}

% Page de titre
\begin{titlepage}
    \centering
    
    % Logo de l'école (à adapter)
    % \includegraphics[width=0.3\textwidth]{logo_ecole.png}
    
    \vspace{2cm}
    
    {\LARGE \textbf{RAPPORT DE STAGE}}
    
    \vspace{1cm}
    
    {\Large \reporttitle}
    
    \ifx\reportsubtitle\empty
    \else
        \vspace{0.5cm}
        {\large \reportsubtitle}
    \fi
    
    \vspace{2cm}
    
    \begin{tabular}{ll}
        \textbf{Étudiant :} & \authorname \\
        \textbf{École :} & \studentschool \\
        \textbf{Période :} & \internshipperiod \\
    \end{tabular}
    
    \vspace{2cm}
    
    \begin{tabular}{ll}
        \textbf{Laboratoire :} & \companyname \\
        \textbf{Maîtres de stages :} & \supervisor \\
        \textbf{Tuteur académique :} & \academicsupervisor \\
    \end{tabular}
    
    \vfill
    
    % Logo de l'entreprise (à adapter)
    % \includegraphics[width=0.2\textwidth]{logo_entreprise.png}
    
    \vspace{1cm}
    
    {\large \today}
    
\end{titlepage}

% Page blanche
\newpage
\thispagestyle{empty}
\mbox{}

% Remerciements
\newpage
\section*{Remerciements}
\addcontentsline{toc}{section}{Remerciements}

Je tiens à remercier...
\newpage
\section*{Abstracts}\addcontentsline{toc}{section}{Abstracts}
\subsection*{Résumé en Français}
    \textbf{Mots-clés :} Schémas Numériques, Simulation des EDP d'Évolution, Multirésolution Adaptative, Méthodes ImEx, 
    Advection-Diffusion-Réaction, Analyse d'erreur numérique, Analyse de stabilité.\par
    \noindent\rule{\textwidth}{0.4pt}
    % \vspace{0.3cm}
    % Ce papier documente mon projet de fin d'études qui a pris place au laboratoire du Centre de Mathématiques Appliquées de l'École Polytechnique (CMAP).
    % Cette expérience en recherche académique a été une opportunité exceptionnelle car elle m'a permis de mieux comprendre les rouages de la recherche,
    % de mettre en application et en relation les concepts et savoirs-faire acquis au cours de mes études, d'améliorer la communication et le partage de mon travail,
    % d'échanger avec des chercheurs d'horizons divers
    % et découvrir des thèmes et des problématiques scientifiques qu'y m'étaient inconnues, en somme de parfaire mon parcours académique et assurer une heureuse transition avec le monde professionnel.
    % Mon travail de recherche porte sur des méthodes modernes pour la simulation des équations d'advection-diffusion-réaction (ADR), des EDP assez capricieuses,
    % aux applications multiples, régissants entre autres les phénomènes de combustions. Ce rapport contient une introduction aux défis que portent ces équations et 
    % une introduction aux stratégies imaginées pour relever ses défis, le tout accompagnés de quelques rappels mathématiques bienvenus.
    % Il inclut bien sûr une présentation de mes contributions: deux études, une portant sur \textbf{la multi-résolution adaptative},
    % j'y présente une étude théorique de l'erreur qu'elle apporte sur un cas particulier;
    % et l'autre sur \textbf{les méthodes Runge et Kutta Implicites-Explicites (ImEx)},
    % je présente une analyse sur les équations d'ADR de ces méthodes et les compare a un autre méthode plus standard. 
    Les systèmes couplant mécanique des fluides et chimie complexe se modélisent par les équations d'advection-diffusion-réaction (ADR), 
    une classe d'équations aux dérivées partielles dont la résolution numérique requiert à la fois des stratégies d'adaptation de maillage 
    (par exemple la multirésolution adaptative, MRA) et des méthodes d'intégration temporelle spécifiques (splitting, schémas ImEx).
    Ce travail étudie les interactions entre ces deux composantes (adaptation spatiale et intégration temporelle).
    Il apporte (i) une comparaison empirique de l'effet de la MRA sur les schémas de splitting et les schémas ImEx, 
    ainsi qu'une une analyse (ii) théorique et (iii) numérique des couplage émergeant entre une méthode de Runge-Kutta classique et la MRA sur un problème de diffusion traité par méthode des lignes.
\subsection*{English abstract}
    \textbf{Keywords  :} Numerical Schemes, Evolution PDE Simulation, Adaptive Multiresolution, ImEx Methods,
    Advection-Diffusion-Reaction, Numerical Error Analysis, Stability Analysis.\par
    \noindent\rule{\textwidth}{0.4pt}
    Systems coupling fluid mechanics and complex chemistry are modeled by advection-diffusion-reaction (ADR) equations, 
    a class of partial differential equations whose numerical resolution requires both spatial mesh adaptation strategies 
    (such as adaptive multiresolution, MRA) and dedicated time integration methods (splitting, ImEx schemes).
    This work investigates the interactions between these two components (spatial adaptation and temporal integration).
    It provides (i) an empirical comparison of the effect of MRA on splitting and ImEx schemes, 
    and (ii) a theoretical and (iii) numerical analysis of the coupling that emerges between a classical Runge–Kutta method and MRA on a diffusion problem solved with the method of lines.
\newpage
\tableofcontents

\newpage

% Liste des figures (si nécessaire)
\listoffigures
\addcontentsline{toc}{section}{Liste des figures}

\newpage

% Liste des tableaux (si nécessaire)
\listoftables
\addcontentsline{toc}{section}{Liste des tableaux}

\newpage

% Corps du rapport
\section{Introduction}

\subsection{Contexte du stage}
Présentation de l'entreprise/laboratoire d'accueil, du contexte général du stage.

\subsection{Problématique et objectifs}
Description de la problématique abordée et des objectifs fixés pour le stage.

\subsection{Organisation du rapport}
Brève description de la structure du rapport.

\newpage
\chapter{Présentation du laboratoire}
\subsection{Historique et activités}
Le Centre de Mathématiques Appliquées de l'École Polytechnique\footnote{\href{https://cmap.ip-paris.fr}{https://cmap.ip-paris.fr}} (CMAP) a été créé en 1974 lors du déménagement de l'École Polytechnique vers Palaiseau. 
Cette création répond au besoin émergent de mathématiques appliquées face au développement des méthodes de conception et de simulation par calcul numérique dans de nombreuses applications industrielles de l'époque(nucléaire, aéronautique, recherche pétrolière, spatial, automobile).
Le laboratoire fut fondé grâce à l'impulsion de trois professeurs : Laurent \textsc{Schwartz}, Jacques-Louis \textsc{Lions} et Jacques \textsc{Neveu}. Jean-Claude \textsc{Nédélec} en fut le premier directeur, et la première équipe de chercheurs associés comprenait P.A. \textsc{Raviart}, P. \textsc{Ciarlet}, R. \textsc{Glowinski}, R. \textsc{Temam}, J.M. \textsc{Thomas} et J.L. \textsc{Lions}. 
Les premières recherches se concentraient principalement sur l'analyse numérique des équations aux dérivées partielles.
Le CMAP s'est diversifié au fil des décennies, intégrant notamment les probabilités dès 1976, puis le traitement d'images dans les années 1990 et les mathématiques financières à partir de 1997. 
Le laboratoire a formé plus de 230 docteurs depuis sa création et a donné naissance à plusieurs startups spécialisées dans les applications industrielles des mathématiques appliquées.

\subsection{La recherche au CMAP}
Le CMAP comprend trois pôles  de recherche: le pôle analyse, le pôle probabilités et le pôle décision et données. Chaque pôle acceuil en son sein plusieurs équipes :
\begin{enumerate}
    \item \textbf{Analyse}
        \begin{itemize}
            \item[$\diamond$] EDP pour la physique.
            \item[$\diamond$] Mécanique, Matériaux, Optimisation de Formes.
            \item[$\diamond$] HPC@Maths (calcul haute performance).
            \item[$\diamond$] PLATON (quantification des incertitudes en calcul scientifique), avec l'INRIA.
        \end{itemize}
    \item \textbf{Probabilités}
        \begin{itemize}
            \item[$\diamond$] Mathématiques financières.
            \item[$\diamond$] Population, système particules en interaction.
            \item[$\diamond$] ASCII (interactions stochastiques coopératives), avec l'INRIA.
            \item[$\diamond$] MERGE (évolution, reproduction, croissance et émergence), avec l'INRIA.
        \end{itemize}
    \item \textbf{Décision et données}
        \begin{itemize}
            \item[$\diamond$] Statistiques, apprentissage, simulation, image.
            \item[$\diamond$] RandOpt (optimisation aléatoire).
            \item[$\diamond$] Tropical (algèbre $(\max , +)$), avec l’INRIA.
        \end{itemize}
\end{enumerate}
J'ai intégré l'équipe \textbf{HPC@Maths} \textbf{pole analyse}.
De nombreuses équipe sont partagées entre le CMAP et l'INRIA ce qui démontre l'aspect appliqué du laboratoire.

\subsection{L'équipe HPC@Math et l'envrionnement de travail}
\paragraph{L'équipe HPC@Math}
    L'équipe HPC@Math\footnote{\href{https://initiative-hpc-maths.gitlab.labos.polytechnique.fr/site/index.html}{https://initiative-hpc-maths.gitlab.labos.polytechnique.fr/site/index.html}} travaille à l'interface des mathématiques de la physique (mécanique des fluides, thrermodynamique) et de l'informatique pour développer 
    des méthodes numériques complètes (schéma, nalayse d'erreur, implémentation) pour la simulation des EDP. 
    L'éuipe se centre sur les problèmes multi-échelles; les EDPs cibles qui typiquemebnt étudiées sont les équations d'advection-réaction-diffusion qui représente 
    de manière générale le couplage entre la mécanique des fluides, la thermodynamique et la chimie (typiquement un problème de combustion).
    Tout cela se fait dans le contexte HPC (high performance computing). Le HPC désigne l'usage optimimal des ressources informatiques disponibles
    cela peut être développer une simulation efficace sur une petite machine comme des schéma hautement parallélisable 
    dans des paradigmes de calculs hybrides ou dans des contextes hexascale
    \footnote{Plateformes de calculs ayant une capacité de calcul théorique de $10^{16}$ opérations par seconde (hexaflops).}. 
    Ainsi l'application des méthodes développées
    est au coeur des réflexions de l'équipe. 
\paragraph{Envrionnement de travail}


\newpage
\chapter{Description du travail objectifs et état de l'art}
    Ce préambule mathématique présente divers concepts innervant dans les travaux du stage (chapitre \ref{par:contrib}). Le lecteur habitué peut ignorer ce chapitre et 
le consulter ponctuellement au besoin. Les sujets suivants y sont introduits:
\begin{enumerate}
    \item
    \item
    \item
    \item
    \item
\end{enumerate}
    \section{Présentation du sujet et problématique générale}
    Ce travail participe à l'élaboration de méthodes numériques pour l'approximation des équations aux dérivées partielles d'évolution.
En particulier les équations d'advection-diffusion-réaction (présentation en \ref{par:adv-diff-reaction}). Elles décrivent par exemple les systèmes physiques couplant
mécanique des fluides, thermodynamique et réactions chimiques\footnote{Typiquement des problèmes de combustion.}.
Ces équations sont difficiles à simuler du fait de leur caractère multi-échelle
\footnote{Une réaction chimique a des temps et distances typiques généralement plusieurs ordres de grandeur plus faibles que les temps et distances typiques de la mécanique des fluides.}.
Pour gérer les différentes échelles spatiales, des méthodes de compression de maillage sont souvent mises en oeuvre. 
La méthode de compression utilisée et étudiée ici est la multirésolution adaptative \cite{harten1994}.
Les différentes échelles temporelles\footnote{En termes techniques, les différents termes des équations étudiées ont des raideurs très différentes.}
sont usuellement gérées par force brute ou par séparation d'opérateurs. 
Pour pallier le problème de la large gamme d'échelles temporelles rencontrées, une approche hybride est ici étudiée: les méthodes implicites-explicites (ImEx) \cite{ASCHER1997151}.
Ce travail vise donc principalement à comprendre comment la multirésolution adaptative interagit avec les différentes méthodes d'intégration temporelle.
Il s'intéresse aux questions suivantes:
\begin{itemize}
\item[$\diamond$] {Comment les effets de la compression de maillage par multirésolution adaptative (MRA) sur les solutions numériques
dépendent du problème étudié et de la méthode numérique sur lesquelles elle se greffent ?}
\item[$\diamond$] {Comment évoluent les propriétés des méthodes ImEx selon les caractéristiques des opérateurs des équations de diffusion-réaction}
                \footnote{Même si l'objectif est bien les équations d'advection-diffusion-réaction, l'étude s'est concentrée par simplicité sur l'interaction entre phénomènes de diffusion et de réactions.}
                { ?}
\end{itemize}
    \section{Quelques notions techniques}
    \subsection{Intégrations des EDOs}


Introduisons d'abord la notion de raideur d'un sysètme dynamqiue. 
\begin{definition}[Problème raide]
    Un système dynamqiue, est dit raide si les méthodes explicites ne sont pas adaptées à sa résolution.
    En termes plus mathématiques le système 
    \begin{align}
    \frac{\text d u}{\text{d}t} = A u \quad u(t) \in \mathbb{R}^d \: \forall t\geq 0.
    \end{align}
    est dit raide si l'opérateur $A$ possède de \text{grandes} valeurs propres négatives
    \footnote{Ici \textit{grand} est à comprendre au sens de \textit{grande aplitude devant d'autres valeurs propres}.}.
\end{definition}

\begin{exemple}[Équation de Dhalquist]
    Pour saisir de manière plus intuitive le concept de raideur, prenons le cas symple de l'équation de Dhalquist définissant le système suivant
    \footnote{C'est le cas le plus simple d'une valeur propre négative}:
    \begin{align}
        \frac{\text d u }{\text d t} = - \lambda u\quad \lambda > 0.
    \end{align}
    La solution est classiquement : $u(t) = u(0)e^{-\lambda t}$. Ainsi passé quelque $\lambda$ la dynamqiue du système est au point mort. 
    Grossièrement la dynamqiue digne d'intérêt du système se concentre entre $t=0$ et $t=\frac{\lambda}{10}$. Au delà, $u(t>\frac{\lambda}{10}) = o(u(t=0))$, la dynamqiue est terminée.
    Ainsi le lecteur comprend aisément que si l'on souhaite simuler le comportement d'un tel système, il faut prendre des pas de temps petits devant $\vert \lambda \vert^{-1}$.
    Si $\lambda$ est de grande amplitude cela peut devenir très contraignant... Si l'on souhaite utiliser des méthodes explicites, c'est encore pire car la raideur du sysème 
    n'est plus un simple contrainte de précision mais de stabilité. En effet si l'on cherche à approximer le sysètme par un schéma d'Euler explicite, alors : 
    $U^{n+1} = U^n (1 - \lambda \Delta t)$ alors la contrainte de stabilité est $\Delta t \, \lambda < 1$ ce qui est contraignant si $\lambda$ est grand. 
    Si $\lambda = 10^5$ alors il faut avoir $\Delta t /approx 10^{-5}$ donc pour simuler le système entre $t=0$ et $t=1$ il faut cent-milles points !
    On comprend mieux la définition précédente \textit{Un système dynamqiue, est dit raide si les méthodes explicites ne sont pas adaptées à sa résolution.}
\end{exemple}


\subsection{Intégration des EDPs}
\begin{definition}[Méthode des lignes]
    Une méthode des lignes est une famille de méthodes numériques pour approximer les EDP d'évolutions
    Elle consiste à discrétiser les opérateurs spatiaux de l'équation afin d'obtenir une équation semi-discrétisée en espace,
    puis à utiliser une technique d'intégration en temps, pour obtenir la discrétisation complète de l'équation.
\end{definition}
\begin{tikzpicture}[node distance=4.3cm,box/.style={rectangle, draw, thick, minimum width=2.5cm, minimum height=1cm, align=center},arrow/.style={-{Stealth[length=3mm]}, thick},label/.style={font=\small, align=center}]
\node[box, fill=blue!20] (edp) {EDP};
\node[box, fill=green!20, right=of edp] (edo) {Équation\\semi-discrétisée\\(EDO)};
\node[box, fill=orange!20, right=of edo] (schema) {Schéma\\final};
\draw[arrow] (edp) -- node[above, label] {Discrétisation des\\opérateurs spatiaux} (edo);
\draw[arrow] (edo) -- node[above, label] {Méthode d'intégration\\temporelle} (schema);
\node[above=.1cm of edo, font=\bfseries] {Méthode des lignes};
\end{tikzpicture}


\begin{definition}[Volumes finis]
end{definition}
\subsection{}
    \section{Objectifs}
    \section{Méthode de travail et outils}

\newpage
\chapter{Contribution}
    Cet partie vise à présenter mes travaux. Rappelons qu'il se sont concentrés sur la résolutions des EDPs d'advection-diffusion-réaction grâce. 
aux méthodes modernes de compression de maillage (multirésolution adaptative) et de découplage des opérateurs (ImEx et splitting).\par
Dans un premier temps, je présente l'analyse d'une méthode ImEx appliquée à l'équation de Nagumo (une équation de diffusion-réaction).
J'étudie la stabilité des méthodes de manière générale puis dans le cas spécifique de cette équation. Par la suite, j'ai mené une comparaison expérimentale
avec une méthode concurrente: la séparation d'opérateur.\par
Dans un second temps je présente une étude de l'impact de la multirésolution adaptative sur la convergence d'une méthode des lignes sur un problème de diffusion.
Cette étude comprend un pan théorique (obtention de l'équation équivalente du schéma avec multirésolution adaptative) 
et un pan expérimental (étude de convergence grâce au logiciel Samurai).
    % Travail 1 
        % Réalisaiton
        % conclusion
    % Travail 2
        % Réalisaiton
        % conclusion
    % Travail 3 
        % Réalisaiton
        % conclusion

\section{Conclusion}
    %Conclusion scientifique
    %Conclusion profesionelle et personelle

\newpage

% Bibliographie
\printbibliography

\newpage

% Annexes
\appendix

\section{Annexe A : Titre de l'annexe}
Contenu de la première annexe.

\section{Annexe B : Titre de l'annexe}
Contenu de la deuxième annexe.

% Vous pouvez ajouter d'autres annexes selon vos besoins :
% - Code source
% - Données supplémentaires
% - Schémas détaillés
% - Résultats complémentaires

\end{document}