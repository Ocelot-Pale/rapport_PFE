\documentclass[11pt]{article}

% Encodage & langue
\usepackage[utf8]{inputenc}
\usepackage[T1]{fontenc}
\usepackage[french]{babel}

% Packages graphiques
\usepackage{tikz}
\usepackage{tikz-cd}
\usetikzlibrary{arrows.meta, positioning, calc}

% Maths
\usepackage{amssymb, amsthm}
\usepackage{amsmath}

% Bibliographie
\usepackage[backend=biber, style=numeric, citestyle=numeric, maxnames=3]{biblatex}
\addbibresource{bibliographie/bibliographie.bib}

% Marges
\usepackage[a4paper,margin=2.5cm]{geometry}

% Liens
\usepackage[colorlinks=true, linkcolor=blue, citecolor=blue, urlcolor=blue]{hyperref}

% Figures
\usepackage{graphicx}
\usepackage{float}
\usepackage{caption}
\usepackage{subcaption}

% En-têtes et pieds de page
\usepackage{fancyhdr}
\pagestyle{fancy}
\fancyhf{}
\fancyhead[L]{\leftmark}
\fancyhead[R]{\thepage}
\fancyfoot[C]{Rapport de stage - \textit{\authorname}}

% Table des matières
\usepackage{tocloft}
\setcounter{tocdepth}{3}

% Espacement
\usepackage{setspace}
\onehalfspacing

% Listes
\usepackage{enumitem}

% Code source (si nécessaire)
\usepackage{listings}
\usepackage{xcolor}

% Configuration listings
\lstset{
    basicstyle=\ttfamily\small,
    keywordstyle=\color{blue},
    commentstyle=\color{green!60!black},
    stringstyle=\color{red},
    showstringspaces=false,
    breaklines=true,
    frame=single,
    numbers=left,
    numberstyle=\tiny\color{gray}
}

% Théorèmes (si nécessaire pour un rapport technique)
\newtheorem{theoreme}{Théorème}[section]
\newtheorem{proposition}[theoreme]{Proposition}
\newtheorem{lemme}[theoreme]{Lemme}
\newtheorem{corollaire}[theoreme]{Corollaire}
\theoremstyle{definition}
\newtheorem{definition}[theoreme]{Définition}
\newtheorem{exemple}[theoreme]{Exemple}
\theoremstyle{remark}
\newtheorem{remarque}[theoreme]{Remarque}

% Commandes personnalisées pour les dérivées (si nécessaire)
\newcommand{\pt}{\partial_t}
\newcommand{\px}{\partial_x}
\newcommand{\pxx}{\partial_x^{(2)}}

% Dérivées temporelles
\newcommand{\dt}[1]{\partial_t #1}
\newcommand{\dtt}[1]{\partial_{tt} #1}

% Dérivées spatiales
\newcommand{\dx}[1]{\partial_x #1}
\newcommand{\dxx}[1]{\partial_{xx} #1}
\newcommand{\dxxx}[1]{\partial_{xxx} #1}

% Dérivées totales
\newcommand{\Dt}[1]{\frac{d #1}{dt}}
\newcommand{\Dtt}[1]{\frac{d^2 #1}{dt^2}}
\newcommand{\Dx}[1]{\frac{d #1}{dx}}
\newcommand{\Dxx}[1]{\frac{d^2 #1}{dx^2}}

% Informations du rapport
\newcommand{\authorname}{Alexandre \textsc{Edeline}}
\newcommand{\studentschool}{ENSTA Paris - IP Paris}
\newcommand{\companyname}{CMAP - École Polytechnique}
\newcommand{\companylocation}{Palaiseau, France}
\newcommand{\supervisor}{Marc \textsc{Massot} et Christian \textsc{Tenaud}}
\newcommand{\academicsupervisor}{Patrick \textsc{Ciarlet}}
\newcommand{\internshipperiod}{du 14/04/2025 au 15/09/2025}
\newcommand{\reporttitle}{Titre du rapport de stage}
\newcommand{\reportsubtitle}{Sous-titre}

\begin{document}

% Page de titre
\begin{titlepage}
    \centering
    
    % Logo de l'école (à adapter)
    % \includegraphics[width=0.3\textwidth]{logo_ecole.png}
    
    \vspace{2cm}
    
    {\LARGE \textbf{RAPPORT DE STAGE}}
    
    \vspace{1cm}
    
    {\Large \reporttitle}
    
    \ifx\reportsubtitle\empty
    \else
        \vspace{0.5cm}
        {\large \reportsubtitle}
    \fi
    
    \vspace{2cm}
    
    \begin{tabular}{ll}
        \textbf{Étudiant :} & \authorname \\
        \textbf{École :} & \studentschool \\
        \textbf{Période :} & \internshipperiod \\
    \end{tabular}
    
    \vspace{2cm}
    
    \begin{tabular}{ll}
        \textbf{Laboratoire :} & \companyname \\
        \textbf{Lieu :} & \companylocation \\
        \textbf{Maîtres de stages :} & \supervisor \\
        \textbf{Tuteur académique :} & \academicsupervisor \\
    \end{tabular}
    
    \vfill
    
    % Logo de l'entreprise (à adapter)
    % \includegraphics[width=0.2\textwidth]{logo_entreprise.png}
    
    \vspace{1cm}
    
    {\large \today}
    
\end{titlepage}

% Page blanche
\newpage
\thispagestyle{empty}
\mbox{}

% Remerciements
\newpage
\section*{Remerciements}
\addcontentsline{toc}{section}{Remerciements}

Je tiens à remercier...
\newpage
\section*{Résumé}
\addcontentsline{toc}{section}{Résumé}
\begin{abstract}
    Résumé du rapport de stage en français (150-300 mots). 
    Présenter brièvement le contexte, les objectifs, la méthodologie, les principaux résultats et conclusions.\par
    English abstract of the internship report (150-300 words).
    Briefly present the context, objectives, methodology, main results and conclusions.\\
\end{abstract}
\textbf{Mots-clés :} mot-clé 1, mot-clé 2, mot-clé 3, mot-clé 4, mot-clé 5\\
\textbf{Keywords:} keyword 1, keyword 2, keyword 3, keyword 4, keyword 5

\section*{Abstract}
\addcontentsline{toc}{section}{Abstract}
    
    
\newpage
\tableofcontents

\newpage

% Liste des figures (si nécessaire)
\listoffigures
\addcontentsline{toc}{section}{Liste des figures}

\newpage

% Liste des tableaux (si nécessaire)
\listoftables
\addcontentsline{toc}{section}{Liste des tableaux}

\newpage

% Corps du rapport
\section{Introduction}

\subsection{Contexte du stage}
Présentation de l'entreprise/laboratoire d'accueil, du contexte général du stage.

\subsection{Problématique et objectifs}
Description de la problématique abordée et des objectifs fixés pour le stage.

\subsection{Organisation du rapport}
Brève description de la structure du rapport.
\newpage
\section{Présentation du laboratoire}
\subsection{Historique et activités}
Le Centre de Mathématiques Appliquées de l'École Polytechnique\footnote{\href{https://cmap.ip-paris.fr}{https://cmap.ip-paris.fr}} (CMAP) a été créé en 1974 lors du déménagement de l'École Polytechnique vers Palaiseau. 
Cette création répond au besoin émergent de mathématiques appliquées face au développement des méthodes de conception et de simulation par calcul numérique dans de nombreuses applications industrielles de l'époque(nucléaire, aéronautique, recherche pétrolière, spatial, automobile).
Le laboratoire fut fondé grâce à l'impulsion de trois professeurs : Laurent \textsc{Schwartz}, Jacques-Louis \textsc{Lions} et Jacques \textsc{Neveu}. Jean-Claude \textsc{Nédélec} en fut le premier directeur, et la première équipe de chercheurs associés comprenait P.A. \textsc{Raviart}, P. \textsc{Ciarlet}, R. \textsc{Glowinski}, R. \textsc{Temam}, J.M. \textsc{Thomas} et J.L. \textsc{Lions}. 
Les premières recherches se concentraient principalement sur l'analyse numérique des équations aux dérivées partielles.
Le CMAP s'est diversifié au fil des décennies, intégrant notamment les probabilités dès 1976, puis le traitement d'images dans les années 1990 et les mathématiques financières à partir de 1997. 
Le laboratoire a formé plus de 230 docteurs depuis sa création et a donné naissance à plusieurs startups spécialisées dans les applications industrielles des mathématiques appliquées.

\subsection{La recherche au CMAP}
Le CMAP comprend trois pôles  de recherche: le pôle analyse, le pôle probabilités et le pôle décision et données. Chaque pôle acceuil en son sein plusieurs équipes :
\begin{enumerate}
    \item \textbf{Analyse}
        \begin{itemize}
            \item[$\diamond$] EDP pour la physique.
            \item[$\diamond$] Mécanique, Matériaux, Optimisation de Formes.
            \item[$\diamond$] HPC@Maths (calcul haute performance).
            \item[$\diamond$] PLATON (quantification des incertitudes en calcul scientifique), avec l'INRIA.
        \end{itemize}
    \item \textbf{Probabilités}
        \begin{itemize}
            \item[$\diamond$] Mathématiques financières.
            \item[$\diamond$] Population, système particules en interaction.
            \item[$\diamond$] ASCII (interactions stochastiques coopératives), avec l'INRIA.
            \item[$\diamond$] MERGE (évolution, reproduction, croissance et émergence), avec l'INRIA.
        \end{itemize}
    \item \textbf{Décision et données}
        \begin{itemize}
            \item[$\diamond$] Statistiques, apprentissage, simulation, image.
            \item[$\diamond$] RandOpt (optimisation aléatoire).
            \item[$\diamond$] Tropical (algèbre $(\max , +)$), avec l’INRIA.
        \end{itemize}
\end{enumerate}
J'ai intégré l'équipe \textbf{HPC@Maths} \textbf{pole analyse}.
De nombreuses équipe sont partagées entre le CMAP et l'INRIA ce qui démontre l'aspect appliqué du laboratoire.

\subsection{L'équipe HPC@Math et l'envrionnement de travail}
\paragraph{L'équipe HPC@Math}
    L'équipe HPC@Math
\paragraph{Envrionnement de travail}

\newpage

\section{État de l'art}

\subsection{Contexte scientifique/technique}
Présentation du domaine d'étude, des connaissances existantes.

\subsection{Travaux antérieurs}
Revue des travaux précédents dans le domaine.

\subsection{Positionnement du stage}
Comment le travail de stage s'inscrit dans ce contexte.

\section{Méthodologie}

\subsection{Approche adoptée}
Description de l'approche méthodologique choisie.

\subsection{Outils et techniques utilisés}
Présentation des outils, logiciels, techniques employés.

\subsection{Planification du travail}
Organisation temporelle du stage, étapes principales.

\section{Réalisations et résultats}

\subsection{Première réalisation}
Description détaillée de la première contribution.

\subsection{Deuxième réalisation}
Description détaillée de la deuxième contribution.

\subsection{Résultats obtenus}
Présentation et analyse des résultats.

\subsection{Validation et tests}
Méthodes de validation, tests effectués, critères d'évaluation.

\section{Discussion}

\subsection{Analyse des résultats}
Interprétation critique des résultats obtenus.

\subsection{Limites et perspectives}
Discussion des limites du travail et des perspectives d'amélioration.

\subsection{Apports du stage}
Ce que le stage a apporté au projet/entreprise.

\section{Bilan personnel}

\subsection{Compétences acquises}
Compétences techniques et transversales développées.

\subsection{Difficultés rencontrées}
Principales difficultés et comment elles ont été surmontées.

\subsection{Intégration professionnelle}
Expérience de travail en équipe, adaptation au milieu professionnel.

\section{Conclusion}

Synthèse des contributions, bilan du stage, perspectives personnelles et professionnelles.

\newpage

% Bibliographie
\printbibliography

\newpage

% Annexes
\appendix

\section{Annexe A : Titre de l'annexe}
Contenu de la première annexe.

\section{Annexe B : Titre de l'annexe}
Contenu de la deuxième annexe.

% Vous pouvez ajouter d'autres annexes selon vos besoins :
% - Code source
% - Données supplémentaires
% - Schémas détaillés
% - Résultats complémentaires

\end{document}