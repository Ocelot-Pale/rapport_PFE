\documentclass[11pt]{report}
\usepackage{booktabs}
\usepackage{makecell} % pour les retours à la ligne dans une cellule

% Encodage & langue
\usepackage[T1]{fontenc}
\usepackage[french]{babel}
\usepackage{fourier}  % Utopia + symboles mathématiques
\usepackage[utf8]{inputenc}
% Packages graphiques
\usepackage{tikz}
\usepackage{url}
\usepackage{tikz-cd}
\usetikzlibrary{arrows.meta, positioning, calc}
\newcommand{\figref}[1]{fig.~\ref{#1}}
% Maths
\usepackage{amssymb, amsthm}
\usepackage{amsmath}
% Bibliographie
\usepackage[backend=biber, style=numeric, citestyle=numeric, maxnames=3]{biblatex}
\addbibresource{bibliographie/bibliographie.bib}

% Marges
\usepackage[a4paper,margin=2.5cm]{geometry}

% Liens
\usepackage[colorlinks=true, linkcolor=blue, citecolor=blue, urlcolor=blue]{hyperref}

% Figures
\usepackage{graphicx}
\usepackage{float}
\usepackage{caption}
\usepackage{subcaption}
\usepackage{listings}
\usepackage{xcolor}
\usepackage[section]{placeins} % insère un \FloatBarrier à chaque \section

\lstset{
    language=Python,
    basicstyle=\ttfamily\small,
    numbers=left,
    numberstyle=\tiny\color{gray},
    keywordstyle=\color{blue},
    commentstyle=\color{green!50!black},
    stringstyle=\color{red!70!black},
    frame=single,
    breaklines=true,
}
\usepackage{soul}
\newcommand{\hlblue}[1]{\sethlcolor{navy!20}\hl{#1}}
\newcommand{\hlperu}[1]{\sethlcolor{peru!20}\hl{#1}}

% En-têtes et pieds de page
\usepackage{fancyhdr}
\pagestyle{fancy}
\fancyhf{}
\fancyhead[L]{\leftmark}
\fancyhead[R]{\thepage}
\fancyfoot[C]{Rapport de stage - \textit{\authorname}}

% Table des matières
\usepackage{tocloft}
\setcounter{tocdepth}{2}
\setcounter{secnumdepth}{4}  % Numéroter jusqu'aux paragraphs
\renewcommand{\thesubsubsection}{\Alph{subsubsection}}
\renewcommand{\theparagraph}{\thesubsubsection.\roman{paragraph}}

% Espacement
\usepackage{setspace}
\onehalfspacing

% Listes
\usepackage{enumitem}
\usepackage{contour}
\contourlength{0.01em} % règle l’épaisseur du bord

% Code source (si nécessaire)
\usepackage{listings}
\usepackage{xcolor}
\definecolor{navy}{rgb}{0.0,0.0,0.5}                 % #000080
\definecolor{crimson}{rgb}{0.8627,0.0784,0.2353}     % #DC143C
\definecolor{peru}{rgb}{0.8039,0.5216,0.2471}        % #CD853F
\definecolor{antiquewhite}{rgb}{0.9804,0.9216,0.8431}% #FAEBD7
% Configuration listings
\lstset{
    basicstyle=\ttfamily\small,
    keywordstyle=\color{blue},
    commentstyle=\color{green!60!black},
    stringstyle=\color{red},
    showstringspaces=false,
    breaklines=true,
    frame=single,
    numbers=left,
    numberstyle=\tiny\color{gray}
}

% Théorèmes (si nécessaire pour un rapport technique)
\newtheorem{theoreme}{Théorème}[section]
\newtheorem{proposition}[theoreme]{Proposition}
\newtheorem{lemme}[theoreme]{Lemme}
\newtheorem{corollaire}[theoreme]{Corollaire}
\theoremstyle{definition}
\newtheorem{definition}[theoreme]{Définition}
\newtheorem{exemple}[theoreme]{Exemple}
\theoremstyle{remark}
\newtheorem{remarque}[theoreme]{Remarque}

% Commandes personnalisées pour les dérivées (si nécessaire)
\newcommand{\dlbar}{\overline{\Delta l}}  % barre plus longue
\newcommand{\doublehat}[1]{\hat{\hat{#1}}}
\newcommand{\pt}{\partial_t}
\newcommand{\px}{\partial_x}
\newcommand{\pxx}{\partial_x^{(2)}}

% Dérivées temporelles
\newcommand{\dt}[1]{\partial_t #1}
\newcommand{\dtt}[1]{\partial_{tt} #1}

% Dérivées spatiales
\newcommand{\dx}[1]{\partial_x #1}
\newcommand{\dxx}[1]{\partial_{xx} #1}
\newcommand{\dxxx}[1]{\partial_{xxx} #1}

% Dérivées totales
\newcommand{\Dt}[1]{\frac{d #1}{dt}}
\newcommand{\Dtt}[1]{\frac{d^2 #1}{dt^2}}
\newcommand{\Dx}[1]{\frac{d #1}{dx}}
\newcommand{\Dxx}[1]{\frac{d^2 #1}{dx^2}}

% Informations du rapport
\newcommand{\authorname}{Alexandre \textsc{Edeline}}
\newcommand{\studentschool}{ENSTA Paris - Institut Polytechnique de Paris}
\newcommand{\companyname}{CMAP - École Polytechnique}
\newcommand{\companylocation}{Palaiseau, France}
\newcommand{\supervisor}{Marc \textsc{Massot} et Christian \textsc{Tenaud}}
\newcommand{\academicsupervisor}{Patrick \textsc{Ciarlet}}
\newcommand{\internshipperiod}{du 14/04/2025 au 17/10/2025}
\newcommand{\reporttitle}{Étude des interactions entre adaptation spatiale par multirésolution et schémas numériques pour les équations d'advection-diffusion-réaction}
\newcommand{\reportsubtitle}{Développement de méthodes de haute résolution avec contrôle d’erreur pour la simulation des EDP évolutive multi-échelles en temps et en espace}

\begin{document}

% Page de titre
\begin{titlepage}
    \noindent
    \includegraphics[height=0.3\textwidth]{media/0_cover/cmap.jpg}%
    \hfill
    \includegraphics[height=0.3\textwidth]{media/0_cover/Logo_ENSTA_Paris.jpg}
    \centering
    

    
    \vspace{2cm}
    
    {\LARGE \textbf{RAPPORT DE STAGE}}
    
    \vspace{1cm}
    
    {\Large \reporttitle}
    
    % \ifx\reportsubtitle\empty
    % \else
    %     \vspace{0.5cm}
    %     {\large \reportsubtitle}
    % \fi
    
    \vspace{2cm}
    
    \begin{tabular}{ll}
        \textbf{Étudiant :} & \authorname \\
        \textbf{École :} & \studentschool \\
        \textbf{Période :} & \internshipperiod \\
    \end{tabular}
    
    \vspace{2cm}
    
    \begin{tabular}{ll}
        \textbf{Laboratoire :} & \companyname \\
        \textbf{Maîtres de stages :} & \supervisor \\
        \textbf{Tuteur académique :} & \academicsupervisor \\
    \end{tabular}
    
    \vfill
    
    % Logo de l'entreprise (à adapter)
    % \includegraphics[width=0.2\textwidth]{logo_entreprise.png}
    
    \vspace{1cm}
    
    {\large \today}
    
\end{titlepage}

% Page blanche
\newpage
\thispagestyle{empty}
\mbox{}

% Remerciements
\newpage
\section*{Remerciements}
\addcontentsline{toc}{section}{Remerciements}
    Je remercie chaleureusement Marc Massot pour sa bienveillance, sa disponibilité et plus généralement pour m'avoir accompagné au cours de ce stage. 
    Je tiens à faire particulièrement l'éloge de ses capacités de superviseur: guider la compréhension, encourager l'initiative,
    orienter la recherche et les expériences, aiguiller la rédaction ; un authentique mentor.\par


    J'exprime également toute ma gratitude à Christian Tenaud avec qui les échanges ont beaucoup oeuvré à la rigueur et la clarification de mes travaux.\par


    Merci à toute l'équipe du HPC@Math qui m'a beaucoup soutenu et dont le travail a beaucoup facilité mes recherches.
    Je suis sincèrement reconnaissant à Pierre, Laurent et Loïc pour la \textit{"hotline samurai"} et plus généralement pour leur aide efficace, patiente et bienveillante qui m'a permis de surpasser des problèmes informatiques
    parfois idiots, parfois insidieux et souvent obscurs...
    Je souligne au passage l'effort de l'équipe Samurai pour intégrer les fonctionnalités demandées en un temps record !\par


    Je remercie Richard et Ward pour leur aimable relecture et leurs conseils.
    Je remercie Antoine pour m'avoir aidé sur un \textit{bug} qui me semblait insurmontable.
    Je salue tous les doctorant du CMAP, les remerciant pour les débats passionnants (parfois absurdes) et je remercie tous les chercheurs, toujours disponible pour expliquer et vulgariser leurs recherches.\par 


    Je remercie Théophane, pour son soutiens et nos échanges.
    Je remercie enfin mes parents dont le soutient m'a toujours permis de me concentrer exclusivement et sans inquiétude sur mes études.
\newpage
\section*{Abstracts}\addcontentsline{toc}{section}{Abstracts}
\subsection*{Résumé en Français}
    \textbf{Mots-clés :} Schémas Numériques, Simulation des EDP d'Évolution, Multirésolution Adaptative, Méthodes ImEx, 
    Advection-Diffusion-Réaction, Analyse d'erreur numérique, Analyse de stabilité.\par
    \noindent\rule{\textwidth}{0.4pt}
    % \vspace{0.3cm}
    % Ce papier documente mon projet de fin d'études qui a pris place au laboratoire du Centre de Mathématiques Appliquées de l'École Polytechnique (CMAP).
    % Cette expérience en recherche académique a été une opportunité exceptionnelle car elle m'a permis de mieux comprendre les rouages de la recherche,
    % de mettre en application et en relation les concepts et savoirs-faire acquis au cours de mes études, d'améliorer la communication et le partage de mon travail,
    % d'échanger avec des chercheurs d'horizons divers
    % et découvrir des thèmes et des problématiques scientifiques qu'y m'étaient inconnues, en somme de parfaire mon parcours académique et assurer une heureuse transition avec le monde professionnel.
    % Mon travail de recherche porte sur des méthodes modernes pour la simulation des équations d'advection-diffusion-réaction (ADR), des EDP assez capricieuses,
    % aux applications multiples, régissants entre autres les phénomènes de combustions. Ce rapport contient une introduction aux défis que portent ces équations et 
    % une introduction aux stratégies imaginées pour relever ses défis, le tout accompagnés de quelques rappels mathématiques bienvenus.
    % Il inclut bien sûr une présentation de mes contributions: deux études, une portant sur \textbf{la multi-résolution adaptative},
    % j'y présente une étude théorique de l'erreur qu'elle apporte sur un cas particulier;
    % et l'autre sur \textbf{les méthodes Runge et Kutta Implicites-Explicites (ImEx)},
    % je présente une analyse sur les équations d'ADR de ces méthodes et les compare a un autre méthode plus standard. 
    Les systèmes couplant mécanique des fluides et chimie complexe se modélisent par les équations d'advection-diffusion-réaction (ADR), 
    une classe d'équations aux dérivées partielles dont la résolution numérique requiert à la fois des stratégies d'adaptation de maillage 
    (par exemple la multirésolution adaptative, MRA) et des méthodes d'intégration temporelle spécifiques (splitting, schémas ImEx).
    Ce travail étudie les interactions entre ces deux composantes (adaptation spatiale et intégration temporelle).
    Il apporte (i) une comparaison empirique de l'effet de la MRA sur les schémas de splitting et les schémas ImEx, 
    ainsi qu'une une analyse (ii) théorique et (iii) numérique des interactions entre schéma numérique et MRA sur des problèmes de diffusion et diffusion réaction.
\subsection*{English abstract}
    \textbf{Keywords  :} Numerical Schemes, Evolution PDE Simulation, Adaptive Multiresolution, ImEx Methods,
    Advection-Diffusion-Reaction, Numerical Error Analysis, Stability Analysis.\par
    \noindent\rule{\textwidth}{0.4pt}
    Systems coupling fluid mechanics and complex chemistry are modeled by advection-diffusion-reaction (ADR) equations, 
    a class of partial differential equations whose numerical resolution requires both spatial mesh adaptation strategies 
    (such as adaptive multiresolution, MRA) and dedicated time integration methods (splitting, ImEx schemes).
    This work investigates the interactions between these two components (spatial adaptation and temporal integration).
    It provides (i) an empirical comparison of the effect of MRA on splitting and ImEx schemes, 
    and (ii) a theoretical and (iii) numerical analysis of the interaction that emerges between the numerical scheme and MRA adaptation on a diffusion problem and a diffusion-reaction problem.
\newpage
\newpage
\tableofcontents

\newpage

% Liste des figures (si nécessaire)
\listoffigures
\addcontentsline{toc}{section}{Liste des figures}


\newpage

\newpage
\chapter{Introduction}
\section{Présentation du laboratoire}
\subsection{Le laboratoire.}
Le Centre de Mathématiques Appliquées de l'École Polytechnique\footnote{\href{https://cmap.ip-paris.fr}{https://cmap.ip-paris.fr}} (CMAP) a été créé en 1974.
Cette création répond au besoin émergent de mathématiques appliquées face au développement des méthodes de conception et de simulation par calcul numérique dans de nombreuses applications industrielles de l'époque (nucléaire, aéronautique, recherche pétrolière, spatial, automobile).
Le laboratoire fut fondé grâce à l'impulsion de trois professeurs : Laurent \textsc{Schwartz}, Jacques-Louis \textsc{Lions} et Jacques \textsc{Neveu}. Jean-Claude \textsc{Nédélec} en fut le premier directeur, et la première équipe de chercheurs associés comprenait P.A. \textsc{Raviart}, P. \textsc{Ciarlet}, R. \textsc{Glowinski}, R. \textsc{Temam}, J.M. \textsc{Thomas} et J.L. \textsc{Lions}. 
Les premières recherches se concentraient principalement sur l'analyse numérique des équations aux dérivées partielles.
Le CMAP s'est diversifié au fil des décennies, intégrant notamment les probabilités dès 1976, puis le traitement d'images dans les années 1990 et les mathématiques financières à partir de 1997. 

Le laboratoire a formé plus de 230 docteurs depuis sa création et a donné naissance à plusieurs \textit{startups} spécialisées dans les applications industrielles des mathématiques appliquées.
Le CMAP comprend trois pôles  de recherche: le pôle analyse, le pôle probabilités et le pôle décision et données.
J'ai intégré l'équipe \textbf{HPC@Maths} du \textbf{pole analyse}.
\subsection{L'équipe HPC@Math}
    L'équipe HPC@Math\footnote{\href{https://initiative-hpc-maths.gitlab.labos.polytechnique.fr/site/index.html}{https://initiative-hpc-maths.gitlab.labos.polytechnique.fr/site/index.html}} 
    travaille à l'interface des mathématiques, de l'informatique et de la physique (mécanique des fluides, thermodynamique) pour développer 
    des méthodes numériques efficaces pour la simulation des EDP. 
    L'équipe s'intéresse entre-autre aux problèmes multi-échelles, au travers des équations d'advection-réaction-diffusion.
    Les travaux de l'équipe s'inscrive toujours dans contexte HPC (high performance computing). 
    Ce terme désigne l'usage optimal des ressources informatiques disponibles et de leur architectures. 
    Chaque méthode est développée en se demandant si elle pourra être facilement parallélisé, si elle pourra être déployer sur GPU ou sur un cluster de calcul de manière efficace. 
\section{Présentation du sujet}
% Ce travail participe à l'élaboration de méthodes numériques pour l'approximation des équations aux dérivées partielles d'évolution.
% En particulier les équations d'advection-diffusion-réaction (présentation en \ref{par:adv-diff-reaction}). Elles décrivent par exemple les systèmes physiques couplant
% mécanique des fluides, thermodynamique et réactions chimiques\footnote{Typiquement des problèmes de combustion.}.
% Ces équations sont difficiles à simuler du fait de leur caractère multi-échelle
% \footnote{Une réaction chimique a des temps et distances typiques généralement plusieurs ordres de grandeur plus faibles que les temps et distances typiques de la mécanique des fluides.}.
% Pour gérer les différentes échelles spatiales, des méthodes de compression de maillage sont souvent mises en oeuvre. 
% La méthode de compression utilisée et étudiée ici est la multirésolution adaptative \cite{harten1994}.
% Les différentes échelles temporelles\footnote{En termes techniques, les différents termes des équations étudiées ont des raideurs très différentes.}
% sont usuellement gérées par force brute ou par séparation d'opérateurs. 
% Pour pallier le problème de la large gamme d'échelles temporelles rencontrées, une approche hybride est ici étudiée: les méthodes implicites-explicites (ImEx) \cite{ASCHER1997151}.
% Ce travail vise donc principalement à comprendre comment la multirésolution adaptative interagit avec les différentes méthodes d'intégration temporelle.
% Il s'intéresse aux questions suivantes:
% \begin{itemize}
% \item[$\diamond$] {Comment les effets de la compression de maillage par multirésolution adaptative (MRA) sur les solutions numériques
% dépendent-ils du problème étudié et de la méthode numérique sur lesquelles elle se greffent ?}
% \item[$\diamond$] {Comment évoluent les propriétés des méthodes ImEx selon les caractéristiques des opérateurs des équations de diffusion-réaction}
%                 \footnote{Même si l'objectif est bien les équations d'advection-diffusion-réaction, l'étude s'est concentrée par simplicité sur l'interaction entre phénomènes de diffusion et de réactions.}
%                 { ?}
% \end{itemize}
Des moteurs aux batteries, des enjeux de défenses au nucléaire civil en passant par les problématiques de sécurité, de nombreux problèmes industriels 
font intervenir des systèmes physico-chimiques modélisés par un couplage entre la mécanique des fluides et des cinétiques chimiques complexes.
Ces modèles décrivent par exemple les problèmes de combustion \cite{Law2006,Echekki2009}, d'électrochimie \cite{Munteanu2024SPM}, de corrosion \cite{AliasgharMamaghani2025}
de propagation d'effluents \cite{exemple_effluent}.
Ces modèles font souvent intervenir des équations d'advection-diffusion-réaction (ADR), 
une gamme d'équations au dérivées partielles (EDP).
Elles comportent un opérateur d'\emph{advection} (transport par un flux), un opérateur de \emph{diffusion} (éparpillement de la matière par l'agitation thermique) 
et un opérateur de \emph{réaction} (modélisant une ou plusieurs réactions chimiques).\par 
Comme détaillé plus tard, ces équations sont difficiles à simuler avec précision (raideurs différentes, solution multi-échelles). L'objet du stage est donc de contribuer à l'élaboration
de méthodes de résolution haute-résolution de ces EDPs.\par 
Cette introduction présente d'abord les difficultés numériques liées aux équations d'ADR (\ref{par:difficulte_adr}), puis réalise un état de l'art des stratégies pour surmonter ces difficultés (\ref{par:intro_etat_art}),
vient enfin la problématique à laquelle les contributions du stage répondent (\ref{par:problematique}).
% Un choix équilibrant concision et précision au sein de cette introduction, toutefois si le lecteur ne trouve pas assez d'information à son goût,
% le chapitre \ref{par:tech} est un préambule mathématiques auquel il pourra se référer.

\subsection{Les difficultés des équations d'advection-diffusion-réaction}
    \label{par:difficulte_adr}
    Les équations d'ADR posent deux difficultés majeures aux numériciens:
    \begin{enumerate}[label=\Alph*.]
        \item des opérateurs aux propriétés différentes et opposées, antagonistes
        \item des solutions multi-échelles en temps et en espace
    \end{enumerate}
    \subsubsection{Des opérateurs aux propritétés antagonistes}
        Le terme antagoniste signifie que leurs propriétés sont très différentes et qu'alors, les méthodes numériques efficaces pour un opérateur sont inadaptées aux autres.
        (plus de détails en \ref{par:adv-diff-reaction})
        \begin{itemize}
            \item[$\diamond$] les \textit{méthodes explicites}\footnote{Définition en \ref{par:edo}} sont adaptées à l'opérateur d'advection, 
            mais présentent des problèmes de \textit{stabilité}\footnote{Définition de stabilité en \ref{par:edo}} pour l'opérateur de réaction.
            \item[$\diamond$] le caractère local et très \textit{raide}\footnote{Définition d'opérateur raide en \ref{par:edo}} de l'opérateur de réaction\footnote{
                Cette raideur viens des temps de relaxations très cours (quelques nanosecondes) des modèles de chimie complexe \cite{Wartha2021}}
            pousse à utiliser des \textit{méthodes implicites}\footnote{Définition en \ref{par:edo}}, cependant le caractère non-local 
            de la diffusion rend les coûts calculatoires prohibitifs en 3D.
            \item[$\diamond$] les méthodes explicites stabilisées comme \cite{AbdulleMedovikov2001} sont très adaptées à la résolution de l'opérateur de diffusion, mais 
            leurs propriétés ne sont pas adaptés au opérateurs de réactions décrivant la chimie complexe (très forte raideur, spectre imaginaire...).
        \end{itemize}
        Ainsi, une approche monolithique (qui traiterait tous les opérateurs d'un bloc) échouera souvent à simuler avec précision le système en 3D.
    \subsubsection{Des solutions multi-échelles}
    Les solutions des équations d'ADR sont souvent multi-échelles en temps et en espace \cite{duart2011} : pour obtenir une précision donnée, chaque zone spatio-temporelle 
    requiert des niveau de résolution différents. Par exemple pour un problème de combustion, tant que l'allumage n'est pas initié,
    un pas de temps et une résolution spatiale grossiers représentent fidèlement le système. En revanche après l'allumage,
    le pas de temps doit être réduit drastiquement pour rendre compte de la complexité des réactions de combustion déclenchées
    \footnote{Par exemple : création d'espèces intermédiaires à temps de vie court, couplage entre la réaction et la diffusion de chaleur...} et,
    proche de la flamme, une résolution spatiale élevées est nécessaire pour rendre compte de sa structure complexe, alors que 
    loin de la flamme une résolution plus grossière reste suffisante.
    Ainsi soit la résolution reste faible mais la précision médiocre ; 
    soit elle est élevée partout mais l'usage des ressources computationnelles et de la capacité mémoire est inefficace, 
    car une part significative de la simulation ne requiert pas une telle résolution.

        
    % (raideur, localités différentes - \textit{cf.} \ref{par:adv-diff-reaction}).\par


    % \textbf{La première difficulté} pousse à utiliser des stratégies d'adaptation de maillage pour disposer d'un maillage précis seulement où cela est nécessaire à la représentation fidèle de solution
    % \footnote{Si la solution est très régulière par endroit, il n'est pas nécessaire d'avoir des mailles trop fines, c'est un coût en calcul est en mémoire non-nécessaire.
    % A l'inverses là où la solution est complexe, ce coût est justifié puisque nécessaire à la representation fidèle de la solution.}.
    % Plusieurs stratégies d'adaptation existent et se déclinent en deux grandes catégories\cite{Vivarelli2025Fluids}:
    % \begin{enumerate}
    %     \item Les approches \textit{features-based} utilisent une quantité locale de la solution (par exemple la vitesse de l'écoulement) 
    %     et ses gradients pour inférer la \textit{complexité} de la solution en chaque point et adapter le maillage en conséquence.
    %     \item Les approches \textit{goal-based} utilisent une fonction de coût reflétant la qualité globale de la solution selon l'adaptation choisie.
    %     Généralement la fonction de coût est issue d'un problème adjoint sur une quantité d'intérêt.
    % \end{enumerate}
    % Le stage se centre sur la \textit{multi-résolution} adaptative \cite{harten1994}, une méthode d'adaptation s'appuyant sur une transformée en ondelette.
    % Elle ne tombe dans aucune des deux catégories précédente, étant fondée avant tout sur des arguments issus de la théorie de l'information 
    % (\textit{cf.} \ref{par:explication_MRA}).\par

    % \textbf{La seconde difficulté} décourage l'usage de schémas d'intégration en temps monolithiques (d'un seul bloc) qui traiteraient tous les opérateurs de la même manière.
    % Leurs propriétés étant très distinctes aucun schéma monolithique ne peut répondre aux besoins des trois opérateurs en même temps.
    % Cela motive l'emploi de schémas découplant les traitements subis par chaque opérateur. Parmi les méthodes les plus établies se trouvent 
    % les techniques de séparation d'opérateurs (\textit{splitting}) et notre objet d'étude: les méthodes ImEx.

\subsection{Les stratégies de simulation des équations d'advection-diffusion-réaction}
    \label{par:intro_etat_art}
    Pour parer ces difficultés trois familles de stratégies existent:
    \begin{enumerate}[label=\Alph*.]
        \item La séparation d'opérateurs, permet de développer des schémas non-monolithiques où chaque opérateur est traité indépendamment des autres avec une méthode adaptée à ses propriétés. 
        \item L'adaptation en espace, permet d'adapter localement la finesse du maillage, une résolution spatiale où la structure de la solution le demande (forts gradients, discontinuités...)
        et une résolution plus faible où la solution est lisse, simple, régulière.
        \item L'adaptation du pas de temps (non-abordé ici mais traité par exemple en \cite{duart2011}).
    \end{enumerate}
    Voici un bref état de l'art des stratégies de séparation d'opérateur et d'adaption en espace qui sont au coeur du stage.
    \subsubsection{Les stratégies de séparation d'opérateurs}
        Deux approches de séparation d'opérateurs existent : la séparation (ou \textit{splitting}) d'opérateurs classique et les méthodes implicites-explicites (ImEx).
        La frontière est poreuse entre ces deux concepts puisque certains schémas de \textit{splitting} peuvent être vus comme des ImEx. 
        Des détails mathématiques complémentaires sur les méthode ImEx sont présentés en \ref{par:ImEx_presentation}.\par
        \textbf{Le \textit{splitting}} repose sur un développement de Taylor de l'exponentielle de matrice.
        Il s'agit en pratique de simuler les opérateurs les un après les autres de sorte que le résultat soit ne défèrent d'un schéma monolithique
        qu'à une erreur de \textit{splitting} maîtrisée.
        Concrètement, la solution de l'EDP :
        \begin{align}
            \notag\text{Pour } U \text{ un espace fonctionnel sur }\mathbb R^d \text{ et A et B deux opérateurs de } U \text{ dans } U,\\\notag 
            \text{trouver }u \in \mathcal F (\mathbb R^+,U) \text{ tel que:}\\
            \dt{u} = Au + Bu,\\
            u(t=0,\cdot) = u_0(\cdot) \in U.
        \end{align}            
        s'écrit comme :
        \begin{align}
            u(t,\cdot) = e^{t(A+B)}u_0(\cdot).
        \end{align}
        où $e^{\cdot(A+B)} : \mathbb R^+ \rightarrow U$ est le semi-groupe correspondant à l'EDP (si $A$ et $B$ sont des matrices, c'est simplement l'exponentielle de matrices).
        Le schéma de \textit{splitting} de Lie, précis à l'ordre un, repose sur le développement suivant:
        \begin{align}
            e^{\Delta t(A+B)}u_0 = e^{\Delta tB} \underbrace{\circ e^{\Delta tA}u_0}_{\tilde{u_0}} + \underbrace{\mathcal O (\Delta t^2)}_{\text{Erreur de \textit{splitting}}}.
        \end{align}
        En acceptant cette erreur de splitting local $ \mathcal O (\Delta t^2)$, pour passer d'un état $u_0$ au temps $t$ à un état $u_1$ au $t+\Delta t$, 
        il suffit de simuler l'opérateur $A$ à partir de $u_0$ pendant $\Delta t$ pour obtenir un état intermédiaire $\tilde{u}_0$,
        puis de simuler l'opérateur $B$ à partir de $\tilde{u}_0$ pendant $\Delta t$ pour obtenir $u_1$.
        À aucun moment ne sont précisés les schémas utilisés pour simuler $A$ et $B$, chacun peut être choisi spécifiquement pour l'opérateur simulé.
        De même pour simuler chaque opérateur sur le pas de temps de \emph{splitting} $\Delta t$, chaque méthode est libre de faire du \emph{sous-cyclage},
        c'est à dire d'employer des sous-pas de temps adaptés à chaque opérateur
        \footnote{Par exemple, un seul pas peut suffire pour faire évoluer les opérateurs d'adection-diffusion
        de $\Delta t$ alors que plusieurs peuvent être nécessaire pour une réaction complexe.}.
        L'avantage est la liberté totale concernant le choix des schémas simulant chaque opérateur.
        Cependant, le \textit{splitting} vient au prix d'une solution entachée d'un terme d'erreur supplémentaire.
        Un des schémas de \textit{splitting} les plus utilisés est le schéma de Strang \cite{Strang1968}, 
        il permet de découpler deux opérateurs avec une précision d'ordre deux en temps.\par
        \textbf{Les méthodes ImEx} se distinguent du \textit{splitting} classique en imposant des conditions entre les schémas utilisés pour chaque opérateur, 
        espérant tirer de cette contrainte une erreur plus faible et potentiellement monter en ordre plus facilement.
        Les premières méthodes ImEx (hors \textit{splitting}) virent le jour à la fin des années 1990 
        avec les méthodes Runge et Kutta additives (ARK) par Ascher \textit{et al} \cite{ASCHER1997151},
        complétées dans les année 2000 par \cite{KENNEDY2003139} puis par \cite{pareschi2010implicitexplicitrungekuttaschemesapplications}.
        Ces méthode reposent sur une combinaison de plusieurs méthodes Runge et Kutta (une par opérateur) préservant un ordre de convergence global.
        Cela a pavé la voie à des méthodes ImEx plus complexes intégrant par exemples des méthodes stabilisées \cite{Abdulle2013} 
        ou encore à des méthodes ImEx couplées espace-temps, faites sur mesures pour une équation spécifique \cite{rebou2024}.\par 
        La promesse des méthodes ImEx est d'avoir les même avantages computationnels que le \emph{splitting} tout en s'affranchissant de l'erreur de découplage qu'il porte.
    % \subsubsection{Les stratégies d'adaptation en temps}
    %     L'objet de l'adaptation en temps est de choisir à chaque instant 
    \subsubsection{Les stratégies d'adaptation en espace}
        L'objectif est ici d'utiliser le maillage avec le moins de points possibles tout en préservant une précision choisie.
        \paragraph{Les stratégies \textit{features-based} et \textit{goal-based}}
        Plusieurs stratégies d'adaptation en espace existent et se déclinent en deux grandes catégories\cite{Vivarelli2025Fluids}:
        \begin{enumerate}
            \item Les approches \textit{features-based} utilisent une quantité locale de la solution (par exemple la vitesse de l'écoulement) 
            et ses gradients pour inférer localement le besoin de résolution spatiale et l'adapter en conséquence.
            \item Les approches \textit{goal-based} utilisent une fonction de coût reflétant la qualité globale de la solution selon l'adaptation choisie.
            Généralement la fonction de coût est issue d'un problème adjoint sur une quantité d'intérêt.
        \end{enumerate}
        \paragraph{La multi-résolution adaptative (MRA)}
        Le stage se centre sur la \textit{multi-résolution adaptative} \cite{harten1994} qui
        ne tombe dans aucune des deux catégories précédente. Puisqu'elle est fondée avant tout sur des arguments issus de la théorie de l'information.
        l’application de la multirésolution adaptative à la simulation numérique remonte aux travaux d'Ami Harten \cite{harten1994} dans les années 1990 qui adapte 
        une méthode de compression de donnée par analyse multi-échelle aux algorithmes de simulation.
        Ces travaux furent par la suite prolongés entre autre dans \cite{Kaibara2001,Cohen2003}.
        Cette technique permettant de réduire drastiquement les temps de simulation a été largement appliquée et 
        est une approche très compétitive \cite{compare_MRA_AMR} pouvant tirer parti des infrastructures de calcul moderne (multi-coeur, GPU) \cite{GPU_bench,duart2011}.\par
        Une présentation mathématique est proposée dans le préambule mathématique en \ref{par:explication_MRA}.
        Succinctement, avec un méthode MRA, la solution est représentée sur plusieurs grilles de finesses différentes.
        L'adaptation (compression) consiste à ignorer l'information superflue contenue sur les grilles les plus fines, là où la solution est assez régulière (la régularité est estimé grâce à la transformée en ondelette).
        À une erreur de compression/reconstruction près, un niveau de résolution maximal (grille fine) peut être retrouvé 
        grâce à une reconstruction polynomiale (décompression/prédiction).\par 
% 
    % \subsubsection{Motivation du sujet initial}
    % Le sujet initial était d'étudier l'usage conjoint des méthodes ImEx et de la MRA pour déterminer l'existence ou non de couplages délétères entre ces deux méthodes.\par 

    % \paragraph{Histoires et étude de la MRA}
    % L'origine de la multirésolution adaptative remonte aux travaux d'Ami Harten \cite{harten1994} dans les années 1990. Ils furent par la suite prolongés entre autre dans \cite{Kaibara2001,Cohen2003}.
    % Cette technique permettant de réduire drastiquement les temps de simulation a été largement appliquée et est une approche très compétitive \cite{compare_MRA_AMR} pouvant tirer parti des infrastructures de calcul moderne (multi-coeur, GPU) \cite{GPU_bench}.

    % \paragraph{Le découplage temporel des opérateurs, les méthodes ImEx}
    % La nécessité de résoudre des équations d'évolutions dont les opérateurs présentaient des propriétés antagonistes mena aux premières méthodes ImEx à la fin des années 1990 
    % avec notamment les méthodes Runge et Kutta additives (ARK) par Asher \textit{et al} \cite{ASCHER1997151} complétés dans les année 2000 par \cite{KENNEDY2003139} puis par \cite{FITZHUGH1961445}. 
    % L'objectif étant de dépasser la séparation d'opérateurs \cite{Strang1968} qui pose sérieux problèmes au delà l'ordre deux ou encore lorsque l'on cherche à découpler plus de deux opérateurs.
    % Cela a pavé la voie à des méthodes ImEx plus complexes intégrant par exemples des méthodes stabilisées \cite{Abdulle2013} 
    % ou encore à des méthodes ImEx couplées espace-temps, faites sur mesures pour une équation spécifique \cite{rebou2024}.

    % \paragraph{Interactions entre adaptation spatiale et le découplage des opérateurs}
    %     L'adaptation de maillage apporte de termes d'erreurs,
    %     le schéma de découplage temporel ajoute d'autres termes d'erreurs, 
    %     une \textit{interaction} se produirait si l'usage conjoint 
    %     de l'adaptation de maillage et du découplage temporel introduisait de nouveaux termes d'erreurs ne résultant que de l'\textit{interaction},
    %     du \textit{couplage}, d'\textit{interférences} entre les deux stratégies. L'idée est donc de déterminer si ces termes d'erreurs existent et si oui quelle est leur magnitude ? Portent-ils significativement préjudice à la qualité du schéma ? 
    %     L'interaction entre le \textit{splitting} et la MRA a été explicitement traité dans la thèse de Max Duarte \cite{duart2011}.    
    %     Toutefois cette problématique reste peu traitée dans la literature, alors même qu'elle transparaît dans de nombreux articles.
    %     Par exemple \cite{Zhang2025IMEXTSA} traite d'une méthode ImEx adapté en espace et en temps et l'on voit dans leur résultat qu'il y a un fort 
    %     couplage entre l'erreur liée à l'adaptation en temps et celle liée à l'adaptation en espace
    %     \footnote{C'est à dire que si l'on note $E_T$ l'erreur du schéma adapté en temps et $E_S$ l'erreur du schéma adapté en espace,
    %     alors l'erreur du schéma adapté en espace et en temps $E_{ST}$ ne vérifie pas du tout $E_{ST}\approx E_S + E_T$ mais plutôt $E_{ST} \gg E_S + E_T$.}. 
    %     On y distingue bien la problématique du couplage entre les traitements en temps et en espace mais ce n'est pas clairement étudié.
    %     \textbf{Ainsi l'étude d'éventuels mécanismes de couplage entre les méthodes ImEx et la MRA est très pertinent car elle comble un trou dans la literature et permet de justifier (ou non) une combinaison de méthodes très efficace sur le papier pour les équations d'ADR.}
    % \subsubsection{Une bifurcation en cours de recherche}
    %     Comme mentionné précédemment, mes recherches ont bifurqué vers un sujet voisin. Voici circonstances et motivations de ce changement de trajectoire:
    %     Puisque je devais étudier le couplage ImEx et MRA de façon fine, mes tuteurs m'ont conseillé 
    %     de d'abord m'entraîner sur une cas simplifié: une simple équation de diffusion résolu avec MRA.
    %     J'ai donc étudié théoriquement l'erreur sur \texttt{(I)} une MRA standard d'une part. Et d'autre part sur \texttt{(II)} schéma non-standard de MRA, où la solution compressée mais les valeurs servant à l'évaluation des flux numériques sont systématiquement reconstruites (décompressions locales); ce qui n'est pas le cas dans le schéma \texttt{I}.
    %     La différence entre les deux schéma n'est à ma connaissance pas étudiée par la littérature;
    %     c'est un sujet qui a récemment émergé au CMAP et qui a été étudié en \cite{belloti_et_al_2025} pour les problèmes d'advections 
    %     concluant que l'implémentation non-standard (avec reconstruction des flux) était plus précise.
    %     \textbf{Cependant} l'analyse pour la diffusion a mené à des conclusions opposés.
    %     Intrigué par ces résultats j'ai voulu les confirmer, les documenter et les expliquer.
    %     De fait cette découverte inopinée m'a fait bifurquer vers un autre objet d'étude: 
    %     l'impact du niveau d'évaluation du flux numériques sur les problèmes de diffusion et de diffusion-réaction.
    \newpage
    \section{Problématique}
    \label{par:problematique}
        L'adaptation spatiale est clé sur les problèmes advection-diffusion-réaction qui sont souvent multi-échelles.
        Cependant, cela introduit des perturbations susceptibles d'altérer les propriétés du schéma.
        Concernant l'adaptation par multirésolution adaptative, le travail \cite{belloti_et_al_2025} a récemment révélé et étudié 
        ces perturbations sur des problèmes d'advection. 
        Toutefois, ces interactions demeurent mal comprises d'un point de vue général.
        Cela dépend \emph{a priori} de nombreux facteurs comme l'EDP, le schéma ou encore la façon dont la MRA est mise en place. 
        En effet il existe plusieurs stratégies d'adaptation par multirésolution adaptative. Par exemple, ce stage se concentre sur les schémas volumes finis.
        Dans ce cadre, un \emph{flux} numérique doit être évalué à l'interface de chaque cellule ; cela mène à plusieurs stratégies d'adaptation par MRA (voir fig. \ref{fig:schema_algos_intro}): 
        \begin{itemize}
            \item Une approche \hlblue{MRA sans reconstruction des flux} où la solution évolue sur le maillage \underline{adapté} (compressé) grâce à un \emph{flux} évalué à partir des valeurs de la solution \underline{adaptée}.
            \item Une approche \hlperu{MRA avec reconstruction des flux} où la solution évolue sur le maillage \underline{adapté}, mais grâce à un \emph{flux} évalué à partir des valeurs de la solution \underline{reconstruite}, décompressée, plus précises.
        \end{itemize}
        Les différences subtiles entre ces deux approches de la MRA suffisent à modifier les interactions avec le schéma ;
        l'analyse devient d'autant plus difficile que les schémas d'ADR sont généralement complexes (ImEx, splitting, méthodes stabilisées, etc.).\\
    La problématique du stage est donc : \\\textbf{Révéler et comprendre les interactions entre l'adaptation spatiale par multirésolution adaptative et les schémas numériques 
    utilisés pour simuler les équations d'advection-diffusion-réaction.}
    Pour aborder cette vaste problématique, mon travail se concentre sur les interactions de la MRA avec les schémas de diffusion et de diffusion-réaction
    \footnote{Ce choix est fait car les problèmes d'advections ont déjà été étudiés en \cite{belloti_et_al_2025} et que l'objectif à terme est de proposer une analyse
    sur les trois opérateurs combinés : advection, diffusion, réaction.}. Il se focalisant sur différents points du problème:
    \begin{enumerate}
        \item[$\diamond$] Quelles sont les propriétés des approches ImEx par rapport au splitting et surtout, quelle approche est la plus impactée par l'adaptation spatiale ?
        \item[$\diamond$] D'où viennent les interactions entre l'adaptation spatiale et l'intégration en temps et comment les mitiger ?
    \end{enumerate}
\begin{figure}[h]
\label{fig:schema_algos_intro}
\begin{center}
\begin{tikzpicture}[scale = 0.7]
\foreach \i in {0,...,1}{
    \draw[fill=green!10] ({4*\i},0) rectangle ({4*(\i+1)},1);
}
\foreach \i in {0,...,3}{
    \draw[fill=black!10] ({2*\i},-1) rectangle ({2*(\i+1)},0);
}

\foreach \i in {0,...,7}{
    \draw[fill=black!10] ({\i},-2) rectangle ({(\i+1)},-1);
}

\foreach \i in {0,...,15}{
    \draw[fill=black!10] ({0.5*\i},-3) rectangle ({(.5*(\i+1))},-2);
}


\draw[navy, very thick, <-] (3.9,.9) -- (2,.5);
\draw[navy, very thick, <-] (4.1,.9) -- (6,.5);

% \draw[orange, very thick, <-] (3.9,.5) -- (3,-.5);
% \draw[orange, very thick, <-] (4.1,.5) -- (5,-.5);

\draw[peru, very thick, <-] (3.9,.2) -- (3.75,-2.5);
\draw[peru, very thick, <-] (4.1,.2) -- (4.25,-2.5);

\draw[black, very thick] (4,0) -- (4,1) node[pos=1, above] {interface d'intérêt};
\node[left] at (-.2,.5) {niveau $l$ (niveau courant)};
\node[left] at (-.2,-.5) {niveau $l+1$ };
\node[left] at (-.2,-1.5) {niveau $l+2$};
\node[left] at (-.2,-2.5) {niveau $l+3$ (ici, le niveau le plus fin)};

\draw[navy, very thick,->] (8.5,-3.2) -- (10,-3.2) node[pos=0,  left] {MRA sans reconstruction des flux.};
\draw[peru, very thick,->] (8.5,-3.7) -- (10,-3.7) node[pos=0,  left] {MRA avec reconstruction des flux au niveau le plus fin.};
% \draw[orange, very thick,->] (8.5,-3.5) -- (10,-3.5) node[pos=0,  left] {algo 3. : flux reconstruit au niveau inférieur};


\end{tikzpicture}


\end{center}
\caption{Illustration des deux approches MRA étudiées.}
\end{figure}
    % Avec un schéma MRA, la solution est donc représentée sur plusieurs grilles de finesses différentes.
    % L'adaptation (compression) consiste à ignorer l'information contenue sur les grilles les plus fines là où la solution est assez régulière;
    % un niveau de résolution maximal (grille fine) pourra être retrouvé (moyennant une erreur de compression) grâce à une reconstruction (décompression) par 
    % interpolation polynomiale.
    % Sur les schémas \texttt{volumes-finis + MRA} standards, les flux numériques sont évalués à partir de l'information sur le maillage adapté (compressé) alors
    % que les non-standards reconstruisent (décompressent) l'information nécessaire au calcul des flux à un niveau de représentation plus fin. Les algorithmes non-standards 
    % étant plus coûteux en calculs mais potentiellement plus précis.
    % La problématique dominante du stage est donc: \textbf{Quel est l'impact du niveau d'évaluation des flux numériques des schéma volumes finis avec multi-résolution adaptative
    % pour les problèmes de diffusion et de réaction diffusion.}
\newpage
\section{Organisation du rapport}
Le rapport est structuré comme suit.
\begin{itemize}
\item[$\circ$] Ce chapitre pose le contexte, la problématique et les objectifs.
\item[$\circ$] Le Chapitre \ref{par:tech} sert de préambule mathématique et algorithmique:
il rappelle les méthodes de discrétisation pour EDO et EDP, les équations d'advection-diffusion-réaction, 
quelques outils d'analyse de schémas numériques (stabilité, équations équivalentes), ainsi que les principes de la multirésolution adaptative; 
le lecteur expérimenté peut le parcourir rapidement et s'y référer au besoin.
\item[$\circ$] Le Chapitre \ref{par:contrib} rassemble trois contributions:
\begin{itemize}
    \item[$\diamond$] (\ref{par:contrib_1}): Une étude de la stabilité de deux méthodes ImEx ARK;
    suivie d'une comparaison avec un schéma de splitting de leur stabilité et convergence pour l'équation-test de Nagumo (réaction-diffusion).
    \item[$\diamond$] (\ref{par:contrib_2}): Une étude théorique, via le calcul d'équations équivalentes, 
    du comportement de l'erreur d'une méthode des lignes (discrétisation spatiale par volumes finis et intégration temporelle par méthode Runge et Kutta) pour un problème de diffusion linéaire.
    Cela dans trois contextes: 
    \texttt{(I)} sans multirésolution adaptative, 
    \texttt{(II)} avec une multirésolution adaptative MRA \hlblue{\textit{standard}},
    \texttt{(III)} avec approche multirésolution adaptative \hlperu{\textit{non-standard}}, (reconstruction des flux).
    \item[$\diamond$] (\ref{par:contrib_3}): Une étude expérimentale des différences entre les trois schéma de MRA mentionné précédemment 
    permettant de mettre en relation les résultats théoriques de la contributions \ref{par:contrib_2} avec les observations expérimentales.
\end{itemize}
\item[$\circ$] Le dernier chapitre (chapitre \ref{par:cc}) conclut en trois volets : scientifique, technique et personnel.
\end{itemize}

\newpage
\chapter{Préambule mathématique}
    \label{par:tech}
    Cette partie présente les objectifs du stage et les méthodes employées.
Elle introduit également le lecteur au sujet, à ses problématiques et comprend un préambule mathématique 
présentant un bref état de l'art et les notions élémentaires des différents domaines convoqués.
    \newpage
    % \section{Quelques notions techniques}
    \subsection{Intégrations des EDOs}
Les techniques d'approximation d'EDPs d'évolution comportent souvent une étape nécessitant la résolution d'une équation différentielle ordinaire
\footnote{On utilisera aussi le terme \textit{système dynamique}, même si en toute rigueur ce concept est un peu plus large.} (EDO),
c'est à dire une équation différentielle ne faisant intervenir qu'une seule variable différenciée, généralement le temps. Cette section rappelle 
quelques notions d'analyse et de simulation des EDOs du premier ordre.\par

\begin{definition}[Équation différentielle ordinaire]
    Une équation différentielle ordinaire (du premier ordre) est une équation de la forme :
    \begin{align}\label{def:def_ODE}
        u' &= A(u,t)\quad u : t \in \mathbb{R}^+ \mapsto u(t) \in \mathbb{R}^d\\\notag
        u(0)&=u_0 \in \mathbb{R}^d.
    \end{align}
\end{definition}
%%%%%%%%%%%%%%%%%%%%%%%%%%%%%%%%%%%%%%%%%%%%%%%%%%%%%%%%%%%%%%%%%%%%%%%%    SCHEMA EXPLICITES ET IMPLICITES
\subsection{Schémas explicites et implicites.}
L'approximation des EDO se fait grâce à des schémas numériques, c'est à dire une suite d'élément $(u^n)$ de $\mathbb{R}^d$.
Donnés une EDO et un pas de discrétisation temporel $\Delta t$, $u^n \in \mathbb{R}^d$ est approxime la solution de l'EDO au temps $t^n = n \Delta t$.
C'est à dire que la suite $(u^n)_{n\in \mathbb{N}} \in (\mathbb R^d)^\mathbb{N}$ définie par un schéma numérique cherche à avoir $u^n \approx u(t=n\Delta t)$.
Deux catégories de schémas numériques existent: les schémas explicites et les schémas implicites
% Seuls les schémas à un pas sont ici présentés et non pas les schémas multi-pas. Ce choix est fait en raison de la barrière de Dahlquist
% \footnote{\url{https://fr.wikipedia.org/wiki/Methode_lineaire_a_pas_multiples\#Premiere_et_deuxieme_limites_de_Dahlquist}}.


\begin{definition}[Schéma explicite]
    Un schéma numérique est dit explicite si la solution au pas de temps $n+1$ est obtenue uniquement grâce à la solution au pas de temps précédent $n$. Cela se formule usuellement sous la forme:
    \begin{align}
        u^{n+1} = u^n + f(u^n ,\Delta t ).
    \end{align}
\end{definition}

\begin{definition}[Schéma implicite]
    Un schéma numérique est dit implicite si la solution au pas de temps $n+1$ est obtenue au moins en partie grâce à la solution au pas de temps $n+1$. 
    Cela peut s'écrire écrit comme:
    \begin{align}
        u^{n+1} = u^n + f(u^{n+1} ,\Delta t ).
    \end{align}
    
\end{definition}

\textbf{Comparaison entre ces deux classes de schémas:}
Une itération d'un schéma implicite nécessite donc l'inversion d'un système linéaire ou non linéaire sur $\mathbb{R}^d$.
De fait, une itération implicite est généralement plus coûteuse qu'une itération d'un schéma explicite
\footnote{En particulier si la dimension de la solution $d$ est grande.}. 
Cependant pour des raisons de stabilité (voir \ref{par:stabilite_edo}) les méthodes explicites peuvent nécessiter des pas de temps bien plus fins, et donc bien plus d'itérations.
Le choix entre méthode explicite et implicite dépend de bien des facteurs (du problème, du niveau de précision voulu, de la difficulté d'implémentation etc...)
c'est un enjeu central de la simulation numérique.
%%%%%%%%%%%%%%%%%%%%%%%%%%%%%%%%%%%%%%%%%%%%%%%%%%%%%%%%%%%%%%%%%%%%%%%%
\subsection{Ordre de convergence d'un schéma}
L'ordre de convergence permet de lier l'erreur des solutions numérique au pas de temps, c'est-à-dire de quantifier l'efficacité d'un schéma numérique.
Pour définir la notion d'ordre de convergence d'un schéma numérique, il faut d'abord définir son erreur.
\begin{definition}[Erreur locale de troncature d'un schéma]
    L'erreur locale d'un schéma numérique de résolution d'une EDO est l'erreur que commet le schéma sur un pas de temps.
    Autrement dit si l'on note $u$ la solution de l'EDO à partir d'un état initial $u_0 = u(t=0)$ et $u_1$ l'approximation numérique proposée par la schéma 
    pour un pas de temps $\Delta t$ à partir de l'état $u_0$, l'erreur locale est: 
    \begin{align}
        e(\Delta t) = \Vert u(\Delta t) - u_1 \Vert_{\mathbb{R}^d}.
    \end{align}
\end{definition}

\begin{definition}[Erreur globale d'un schéma]
    L'erreur globale d'un schéma est l'erreur que commet le schéma sur plusieurs pas de temps. 
    Si l'on note $u^n$ la solution numérique au temps $t^n = n \Delta t$, alors l'erreur globale du schéma jusqu'au temps final $T = N \Delta t$ peut être définie comme:
    \begin{align}
        E(\Delta t) = \max_{0\leq n \leq N} \Vert u^n - u(t^n) \Vert_{\mathbb{R}^d}
    \end{align}
    ou plus simplement :
    \begin{align}
        E(\Delta t)=\Vert u^n - u(T) \Vert_{\mathbb{R}^d}.
    \end{align}
\end{definition}

\begin{definition}[Ordre de convergence]
    Un schéma numérique de résolution d'une EDO est dit d'ordre $p$ si l'erreur globale vérifie : $E(\Delta t) = O(\Delta t^{p})$
\end{definition}
Pour montrer qu'un schéma converge à un ordre $p$, on a recourt au théorème de Lax-Richtmyer.
Ce dernier requiert la notion de \emph{stabilité au perturbations} : 
\begin{definition}[Stabilité aux perturbations]
    Soit le schéma définit par un état initial $u_0$ et la relation : 
    $$u^{n+1} = u^n + \Delta t f(u^n,t^n,\Delta t),\quad n \in \{0 \dots N-1\}$$
    Soit un schéma perturbé perturbé partant de $\xi_0$ et défini par la relation : 
    $$\xi^{n+1} = \xi^n + \Delta t f(u^n,t^n,\Delta t) + \eta_n,\quad \text{où $(\eta_n)$ est une suite de perturbations}$$
    Alors le schéma est stable aux perturbations s'il ne les amplifie par de manière incontrôlée. C'est-à-dire qu'il existe $M>0$ indépendant de $\Delta t$ tel que :
    $$\max_{n} \vert u_n - \xi_n \leq M \left(\vert u_0 - \xi_0 \vert + \sum_{n=0}^{N-1}\right)$$
\end{definition}
\begin{theorem}[Lax-Richtmyer \cite{MassotSeries2023}]
Le théorème de Lax-Richtmyer affirme que pour un problème linéaire, 
le schéma converge à l'ordre $p$ si l'erreur locale vérifie 
$e(\Delta t) = O(\Delta t^{p+1})$ et que le schéma est stable aux perturbations.
\end{theorem}

%%%%%%%%%%%%%%%%%%%%%%%%%%%%%%%%%%%%%%%%%%%%%%%%%%%%%%%%%%%%%%%%%%%%%%%%


\subsection{Stabilité linéaire et raideur}\label{par:stabilite_edo}

Un schéma numérique d'ordre $p$ converge asymptotiquement en $O(\Delta t^p)$ vers la solution exacte d'une EDO lorsque $\Delta t$ est assez petit.
Cependant, cette convergence n'est effective que si le schéma est \textbf{linéairement stable}.  
Un schéma instable conduit à une divergence de la solution numérique : en pratique, au-delà d'un pas de temps critique $\Delta t_0$, la norme $\Vert u^n \Vert$ croît sans borne%
\footnote{Phénomène souvent appelé \og explosion numérique\fg.}, comme illustré figure~\ref{fig:stabilite_schema}.
Ainsi, pour entrer dans le régime asymptotique de convergence, il faut respecter une contrainte de stabilité de type $\Delta t \leq \Delta t_0$.  
Lorsque ce seuil est très faible, la simulation devient coûteuse car elle nécessite un grand nombre d'itérations $T_\mathrm{final}/\Delta t$.  
De manière générale, les schémas explicites sont plus sensibles à ces contraintes que les schémas implicites.

\begin{figure}[htbp]
    \centering
    \includegraphics[width=0.8\textwidth]{media/3_/2_/exemple_satabilite.pdf}
    \caption{Illustration du comportement attendu de l'erreur d'un schéma d'ordre deux dont le seuil d'instabilité est $\Delta t > 10^{-3}$.}
    \label{fig:stabilite_schema}
\end{figure}
Pour définir la notion de stabilité linéaire, il faut introduire l'équation-test de Dahlquist dépendante d'un paramètre $\lambda \in \mathbb{C}$ :
\begin{align}y' = \lambda y.\end{align}
Si $\Re (\lambda) < 0$, alors la dynamique de la solution est dissipative c'est à dire $y(t) \rightarrow 0.$
Un schéma numérique $(u^n)$ est stable sur cette équation si et seulement si $\left\vert u^{n+1}/u^n\right\vert < 1$ (sinon la solution numérique diverge alors qu'elle est sensée tendre vers $0$).
Cette condition dépend de \emph{l'indice spectral} $z = \Delta t \, \lambda$. Par exemple, le schéma 
un schéma d'Euler explicite s'écrit en fonction de $z$ : 
\begin{align}
    u^{n+1} = u^n + \Delta t \, \lambda u^n = (1+z)u^n.
\end{align}
Et donc la stabilité linéaire est assurée $\iff \left\vert \frac{u^{n+1}}{u^n}\right\vert \leq 1 \iff \vert 1+z \vert \leq 1$. 
Cela permet de définir la fonction de stabilité d'un schéma.
\begin{definition}[Fonction de stabilité]
    Pour un schéma numérique, la fonction de stabilité $S$ est définie comme :
    \begin{align}
        S(z) = u^1/u^0.
    \end{align}
    Où $u^1$ est l'évolution par le schéma de l'équation de Dhalquist à partir de $u_0 \in \mathbb{C}$ pour $\lambda$ et $\Delta t$ tels que $z = \lambda \, \Delta t$.
\end{definition}
\begin{definition}[Stabilité linéaire d'un schéma numérique]
    À pas de temps $\Delta t$ fixé, un schéma numérique est linéairement stable pour intégrer le système dynamique
    \begin{align}
    \frac{\text d u}{\text d t} = A(u,t), \qquad u(t)\in\mathbb{C}^d
    \end{align}
    si pour toute
    valeur propre $\lambda \in Sp(A)$ telle que $\Re(\lambda) < 0$: 
    \begin{align}
        \left\vert S(z = \lambda \, \Delta t) \right\vert \leq 1.
    \end{align}
\end{definition}
Pour aller plus loin, on peut introduire des notions plus fines (A-stabilité, L-stabilité), développées par exemple dans \cite{HairerAndWanner1,MassotSeries2023}.
La stabilité d'un schéma dépend directement du spectre de l'opérateur $A$ dans l'équation différentielle.
Lorsque $A$ possède des valeurs propres à grande partie réelle négative, les méthodes explicites imposent des pas $\Delta t$ extrêmement petits pour rester stables.  
Dans ce cas, le problème est qualifié de \emph{raide}.

\begin{definition}[Problème raide]
Un système dynamique est raide si les méthode explicites ne sont pas adéquates pour le résoudre \cite{HairerAndWanner1} ; ou de façon plus formelle le système dynamique :
\begin{align}
    \frac{\text d u}{\text d t} = A(u,t), \qquad u(t)\in\mathbb{C}^d
\end{align}
est dit \emph{raide} si la jacobienne $J_A$ admet des valeurs propres très négatives en valeur absolue, entraînant une condition de stabilité tellement restrictive que les méthodes explicites deviennent inutilisables en pratique.
\end{definition}




% \subsection{Stabilité et raideur}\label{par:stabilite_edo}
% Un schéma numérique d'ordre $p$ converge asymptotiquement en $O(\Delta t^p)$ vers la solution exacte de l'EDO lorsque $\Delta t$ est assez petit.
% Cependant, cette convergence n'est possible que si le schéma est \textbf{stable}.
% L'instabilité d'un schéma désigne la divergence de la solution numérique. 
% En pratique, au-delà d'un pas de temps critique $\Delta t_0$, 
% la norme de la solution numérique $\Vert u^n \Vert$ tend vers l'infini\footnote{Phénomène communément appelé "explosion" de la solution numérique.} (voir \ref{fig:stabilite_schema}).
% Cela impose une respecter une contrainte de stabilité du type $\Delta t \leq \Delta t_0$ pour entre dans le régime asymptotique de convergence du schéma.
% Si ce seuil $\Delta t_0$ très faible, la résolution de l'EDO nécessite un grand nombre d'itérations $T_{\text{final}}/\Delta t$, augmentant le coût calculatoire. 
% Les schémas explicites sont généralement plus prompts aux instabilités que les méthodes implicites.
% \par
% Cette instabilité peut s'interpréter de deux manières complémentaires : d'un point de vue mathématique, le schéma se comporte comme une suite géométrique de raison $|r| > 1$ ; 
% d'un point de vue physique, le schéma introduit artificiellement de l'énergie dans le système à chaque itération.
% \begin{figure}[htbp]
%     \centering
%     \includegraphics[width=0.8\textwidth]{media/3_/2_/exemple_satabilite.pdf}
%     \caption{Illustration du comportement attendu de l'erreur d'un schéma d'ordre deux dont le seuil d'instabilité serait $\Delta t > 10^{-1}$.}
%     \label{fig:stabilite_schema}
% \end{figure}
% \begin{definition}[Stabilité d'un schéma numérique]
% Un schéma numérique $\bigl(u^n\bigr)_{n\in\mathbb{N}} \in (\mathbb{R}^d)^{\mathbb{N}}$ est dit stable si la norme de la solution ne croît pas au fil des itérations, c'est-à-dire si
% \begin{align}
%     \Vert u^{n+1} \Vert \leq \Vert u^n \Vert.
% \end{align}
% En pratique, cette condition n'est satisfaite que si le pas de temps $\Delta t$ reste inférieur à un seuil $\Delta t_0$ dépendant à la fois du problème et du schéma utilisé.
% \end{definition}

% La stabilité d'un schéma dépend fortement du spectre de l'opérateur $A$ apparaissant dans l'équation différentielle (cf. \eqref{def:def_ODE}).  
% Lorsque certaines valeurs propres de $A$ sont très grandes en valeur absolue et négatives, les méthodes explicites deviennent inadaptées : le problème est dit \emph{raide}.

% \begin{definition}[Problème raide]
% Un système dynamique
% \begin{align}
%     \frac{\text d u}{\text{d}t} = A(u,t), \qquad u(t)\in\mathbb{R}^d
% \end{align}
% est dit \emph{raide} si la jacobienne $J_A$ possède des valeurs propres négatives de grande amplitude, imposant des échelles de temps très différentes.  
% Dans ce cas, la condition de stabilité des méthodes explicites devient tellement restrictive qu'elles sont inutilisables en pratique.
% \end{definition}

% La stabilité d'un schéma se caractérise de façon pratique à l'aide de la notion de fonction de stabilité.  
% Pour une équation linéaire $u'=\lambda u$, on définit l'indice spectral $z=\lambda \Delta t$ et une fonction $S$ telle que
% \[
% U^{n+1} = S(z)\,U^n.
% \]

% \begin{definition}[Fonction de stabilité]
% Un schéma numérique est stable pour un pas de temps $\Delta t$ donné si et seulement si
% \begin{align}
%     |S(z)| \leq 1 \qquad \text{avec } z=\lambda\Delta t.
% \end{align}
% \end{definition}

% \begin{exemple}[Équation de Dahlquist]
% Considérons l'équation test
% \begin{align}
%     u'(t) &= -\lambda u(t), \quad \lambda>0, \\
%     u(0) &= u_0.
% \end{align}
% Sa solution exacte est $u(t)=u_0 e^{-\lambda t}$.

% \medskip
% \textbf{Raideur.}  
% Si $\lambda$ est grand, la solution décroît très vite. Un schéma explicite impose alors un pas $\Delta t$ extrêmement petit pour rester stable : le problème est raide, au sens où l'explicite échoue par instabilité.

% \medskip
% \textbf{Fonction de stabilité.}
% \begin{itemize}
%     \item \emph{Euler explicite :} $U^{n+1}=(1-z)U^n \Rightarrow S(z)=1-z$.  
%         Stabilité $\iff 0\leq z\leq 2$, soit $\Delta t \leq 2/\lambda$. Pour $\lambda\gg 1$, cette contrainte est prohibitive.

%     \item \emph{Euler implicite :} $U^{n+1}=\tfrac{1}{1+z}U^n \Rightarrow S(z)=\tfrac{1}{1+z}$.  
%         Ici $|S(z)|\leq 1$ pour tout $z\geq 0$, donc aucune restriction sévère même pour des $\lambda$ très grands.
% \end{itemize}

% Cet exemple illustre la notion de raideur : la stabilité interdit l'usage d'un explicite, tandis qu'un implicite reste stable.
% \end{exemple}

% Pour aller plus loin, on introduit des notions plus fines comme la A-stabilité (stabilité pour tout $z$ à partie réelle négative) ou la L-stabilité (amortissement des modes très raides). Ces définitions, classiques en analyse numérique, sont détaillées par exemple dans \cite{HairerAndWanner1}.

% \begin{definition}[Stabilité d'un schéma numérique]
%     Un schéma numérique $\bigl( u^n \bigr)_{n \in \mathbb{N}} \in\bigl(\mathbb{R}^d \bigr)^{\mathbb{N}}$ est stable si et seulement si :
%     \begin{align}
%         \Vert u^{n+1} \Vert \leq \Vert u^n \Vert.
%     \end{align}
%     Cette condition est souvent vérifiée à la condition que le pas de discrétisation $\Delta t$ n'excède pas un seuil de stabilité $\Delta_0$ fonction de l'ODE et le schéma d'intégration.
% \end{definition}
% La stabilité d'une méthode d'intégration d'EDO dépend entre autres de l'opérateur intervenant dans l'équation (le $A$ dans l'équation \ref{def:def_ODE}).
% Un opérateur tendant à poser des problèmes de stabilité est dit raide.
% \begin{definition}[Problème raide]
%     Un système dynamique, est dit raide si les méthodes explicites ne sont pas adaptées à sa résolution.
%     En termes plus mathématiques le système:
%     \begin{align}
%     \frac{\text d u}{\text{d}t} = A(u,t), \quad u(t) \in \mathbb{R}^d, \forall t\geq 0.
%     \end{align}
%     est dit raide si la jacobienne de $A$, $J_A$ possède des valeurs propres négatives de grande amplitudes devant les autres valeurs propres.
%     Si tel est le cas, plusieurs relaxations sont mises en jeu mais chacune avec des temps caractéristiques d'ordres de grandeur différents.
%     Pour les méthodes explicites, si le pas de temps n'est pas assez petit, la relaxation rapide est mal résolue et
%     impose un gradient fort trop longtemps (le gradient devrait s'atténuer, mais à une échelle trop rapide pour être captée pour le pas de temps du schéma) ce qui 
%     déstabilise la méthode.
% \end{definition}
% En simplifiant, si un opérateur est raide, il impose une condition de stabilité très restrictive aux méthodes explicites et 
% force à choisir des méthodes implicites\footnote{La réalité est plus nuancée, nous le verrons.}.

% La limite de stabilité dépend du spectre de l'opérateur intervenant dans l'équation différentielle. 
% Pour une équation linéaire, cette limite dépend de chaque valeur propre de l'opérateur et peut être exprimé grâce à une \textit{fonction de stabilisé} fonction 
% d'un \textit{indice spectral}, produit de la valeur propre par la pas de temps: $z = \lambda \, \Delta t$.
% \begin{definition}[Fonction de stabilité]
%     Pour un schéma numérique donné, la fonction de stabilité $S$ est une fonction qui vérifie :
%     \begin{align}
%         (\text{Le schéma est stabile pour le pas de temps }\Delta t \iff \vert S(z) \vert < 1.)
%     \end{align}
% \end{definition}


% \begin{exemple}[Équation de Dahlquist]
% On considère l'équation test
% \begin{align}
%     \frac{\text d u}{\text d t} &= - \lambda u, \quad \lambda > 0, \\
%     u(0) &= u_0.
% \end{align}
% La solution exacte est $u(t) = u_0 e^{-\lambda t}$.

% \textbf{Raideur et stabilité.}
% Lorsque $\lambda$ est grand, la décroissance de la solution est très rapide.  
% Pour un schéma explicite, la stabilité impose en plus de prendre un pas $\Delta t$ extrêmement petit.  
% Autrement dit, la méthode explicite échoue par instabilité : c'est bien \emph{problème raide}.  
% Un problème est dit raide précisément quand les méthodes explicites deviennent inutilisables à cause de la condition de stabilité.

% \textbf{Fonction de stabilité.}
% On introduit l'indice spectral $z=\lambda \Delta t$ et la fonction de stabilité $S(z)$ définie par
% \[
% U^{n+1} = S(z)\,U^n.
% \]

% - \emph{Euler explicite :} $U^{n+1} = (1-z)U^n \;\Rightarrow\; S(z)=1-z$.  
% La condition $|S(z)|\leq 1$ impose $0 \leq z \leq 2$.  
% Donc si $\lambda$ est très grand, $\Delta t$ doit être minuscule : l'explicite devient inutilisable.

% - \emph{Euler implicite :} $U^{n+1} = \tfrac{1}{1+z} U^n \;\Rightarrow\; S(z)=\tfrac{1}{1+z}$.  
% On a $|S(z)|\leq 1$ pour tout $z \geq 0$, donc aucune contrainte de stabilité sévère, même si $\lambda$ est grand.

% Cet exemple montre comment la raideur se manifeste : l'explicite explose, l'implicite reste stable.
% \end{exemple}


% Il existe différents types de stabilité comme la A-stabilité (méthode stable indépendamment de la raideur du problème), la L-stabilité (schéma amortissant les hautes fréquences),
% par souci de concision cette partie s'achève ici mais ces notions sont développées par exemple dans \cite{HairerAndWanner1}.
\subsection{Les équations d'advection-diffusion-réaction}
Plaçons nous dans le contexte physique naturel des équations d'advection, diffusion, réaction:
Des particules sont placées dans un milieu fluide où elles \textbf{diffusent}, ce milieu fluide
est en mouvement, cet écoulement déplace les particules, il les \textbf{advecte}.
Enfin les particules \textbf{réagissent} entre-elles et ces réactions modifient les grandeurs thermodynamiques (température, pression) et \textit{in fine} les propriétés
du milieu fluide.
Les équations d'advection, diffusion, réaction modélisent donc ces trois phénomènes et leurs couplages respectifs.

\subsection{Les trois opérateurs}
\subsubsection{Advection}
    L'advection désigne le transport d'une quantité par un flot. L'opérateur d'advection le plus simple est l'opérateur
    de transport $c \frac{\partial}{\partial x}$:
    \begin{align}\frac{\partial u}{\partial t} = c \frac{\partial u}{\partial x}\end{align}
    De manière générale un opérateur d'advcection d'une quantité $u$ par un flot $\underline a$ s'écrit $\underline a \cdot \underline{\nabla} u$.
    Par exemple dans les équations de Navier-Stokes, l'opérateur $\underline{v} \cdot \underline{\underline \nabla} \, \underline{v}$ représente 
    la vitesse $\underline v$ qui est transportée par elle même. Une version simplifiée de ce phénomène est l'équation bien connue de Bürgers.\par 
    Les opérateurs d'advections sont généralement à valeurs propres imaginaires\footnote{Par abus, s'il s'agit d'un opréteur non-linéiare on lui associera les valeurs propres de sa Jacobienne.}.
    Ainsi ils sont peu raides mais résonnants. Les méthodes explicites sont généralements les plus adaptées pour les traiter.
    %       AFAIRE ---> trouver une référence pour justifier que les opérateurs d'advections sont à valeurs propres imaginaires.

\subsubsection{Diffusion}
    La diffusion désigne l'\textit{éparpillement} de particules au sein d'un milieu fluide
    \footnote{En théorie de l'information cela décrit la tendance de l'entropie augmenter et l'information à se moyenner, se flouter.}.
    Ce phénomène est la limite macroscopique du déplacement microscopiques 
    des particules à cause de l'agitation thermique. L'opérateur de diffusion le plus classique est celui de l'équation de la chaleur:
    \begin{align} \frac{\partial u}{\partial t} = D \Delta u.\end{align}
    Le spectre de cet opérateur est $\mathbb R^-$, il est donc infiniement raide. Lorsqu'il est discrétisé seul une partie de sa raideur est captée,
    en pratique la raideur de l'opérateur augmente quadratiquement avce la finesse de la discrétisation spatiale.\par
    %       AFAIRE ---> trouver une référence (ou faire la démo) pour le spectre.
    Cet opérateur est donc moyennement raide. Ainsi on pourrait penser qu'une méthode implcite est adéquate. Cependant ce n'est généralement pas le cas.
    En effet le coefficient de diffusion est généralement fonciton de la températeur, et donc l'éoprateur $D(T) \times \Delta(\cdot)$ varie généalement dans le temps et l'espace. 
    Ainsi il faut inverser à chaque itération l'opérateur implcite, et comme c'est un opérateur non local
    \footnote{Si l'opérateur de diffusion était local on pourrait résoudre plusieurs petit systèmes, potentiellements en parallèle ce qui est bien moins couteux qu'inverser 
    un grand système. Pour se convaincre, inverser un matrice pleine de taille $10^6$ coute au moins $10^{18}$ opérations, alors qu'inverser 100 systèmes de taille $10^4$
    coute $100 \times 10^{12} = 10^{14}$ soit dix mille fois moins, et si ces résolution étaient parallélisé ce serait un million de fois moins.},
    il faut inverser une matrice de taille $d >> 1$ dont la structure
    peut être très hétérogène.
    (car le coefficient de diffusion dépend de $T$ et du milieu, donc \textit{in fine} de $\underline x$). Aujourd'hui il est d'usage d'utiliser 
    des méthodes explicites stabilisées qui parviennent à gérer la raideur moyenne\footnote{Nous reviendrons sur ce qualificatif au prochain paragraphe.} comme les méthodes 
    ROK2 et ROK4\cite{abdulle2002fourth}.

\subsubsection{Réaction}

\subsection{Difficultés mathématiques intrinsèques}

\subsection{Les stratégies de simulation}
\subsubsection{L'adaptation de maillage}
    \paragraph{La multi-résolution adaptative}
    \paragraph{Autres méthodes}
\subsubsection{Les techniques d'intégration}
    \paragraph{Les méthodes ImEx}
    \paragraph{La séparation d'opérateurs}
\subsection{Simulation des EDPs d'évolution}
\begin{definition}[Méthode des lignes]
    Une méthode des lignes est une famille de méthodes numériques pour approximer les EDP d'évolutions
    Elle consiste à discrétiser les opérateurs spatiaux de l'équation afin d'obtenir une équation semi-discrétisée en espace,
    puis à utiliser une technique d'intégration en temps, pour obtenir la discrétisation complète de l'équation.
\end{definition}
\begin{tikzpicture}[node distance=4.3cm,box/.style={rectangle, draw, thick, minimum width=2.5cm, minimum height=1cm, align=center},arrow/.style={-{Stealth[length=3mm]}, thick},label/.style={font=\small, align=center}]
\node[box, fill=blue!20] (edp) {EDP};
\node[box, fill=green!20, right=of edp] (edo) {Équation\\semi-discrétisée\\(EDO)};
\node[box, fill=orange!20, right=of edo] (schema) {Schéma\\final};
\draw[arrow] (edp) -- node[above, label] {Discrétisation des\\opérateurs spatiaux} (edo);
\draw[arrow] (edo) -- node[above, label] {Méthode d'intégration\\temporelle} (schema);
\node[above=.1cm of edo, font=\bfseries] {Méthode des lignes};
\end{tikzpicture}

\begin{definition}[Méthodes d'intégration espace temps]
\end{definition}

\begin{definition}[Volumes finis]
\end{definition}
\subsection{Analyse de schéma numériques}
Staiblité... Convergence...
\begin{definition}[Procédure de Cauchy-Kovaleskaya]
\end{definition}
\begin{definition}[Équation modifiée]
\end{definition}

\subsection{La Multirésolution Adaptative}
%       INTRO
%       TRANSFORMÉE MULTI-ÉCHELLE
%       HEURISTIQUES D'ADAPTATION
%       ALGORITHMES
%           CALCULS NIVEAU COURANT
%           CALCUL NIVEAU FIN
%       SOURCES D'ERREURS
%       IMPLÉMENTATIONS
%           MÉTHODES CLASSIQUES 
%           SAMURAI

La multi-résolution adaptative (MRA) est une méthode très efficace pour les problèmes multi-échelles. 
L'objectif est de concentrer les efforts computationnels là où ils sont nécessaires. 
Concrètement cela consiste à augmenter la résolution de la grille de calcul où la solution est complexe et la diminuer où la solution est simple à décrire.
La MRA est donc une méthode de HPC (\textit{high performance computing}) puisqu'elle vise à optimiser l'allocation des ressources de calcul.\par
Cette partie introduit le lecteur à cette méthode en présentant d'abord le concept mathématique de transformée multi-échelle (ou transformée en ondelette)
qui est à la base de la MRA. 
Puis il est expliqué comment la transformée multi-échelle permet d'adapter le maillage pour optimiser la charge computationnelle.
Une fois ces prérequis établis, l'algorithme typique de mise en oeuvre de la multi-résolution adaptative est décrit.
S’ensuit alors naturellement une présentation des différentes implémentations de la MRA, avec une attention particulière sur celle développée au CMAP au travers 
du logiciel Samurai.
Enfin l'impact de la multi-résolution sur la qualité des solutions numériques est abordé. 

\subsubsection{La transformée multi-échelle}
    Cette partie présente la transformée multi-échelle discrète. La transformée multi-échelle continue en simulation numérique,
    c'est bien sûr la version discrète qui est utile.
    Elle se veut avant tout introductive et omets ou simplifie certaines notions; plus de détails sont donnés en \cite{postePoly}.

    \paragraph{Définition mathématique}
        Les explications sont développées en dimensions un à des fins pédagogiques, la plupart des concepts s'entendent naturellement aux dimensions supérieures. 
        De plus, la discrétisation de l'espace se fait selon une grille dyadique, d'autre choix pourraient être fait mais c'est un choix simple, naturel et standard.
        Il faut détailler cette notion.
        \begin{definition}[Grille dyadique]
            Une grille dyadique ou discrétisation dyadique d'un intervalle $I \subset \mathbb R$ est une série de partitions de $I$ indexées 
            par des entiers $j \in J \subset \mathbb N^*$.
            La discrétisation de niveau $j$ correspond à une partitions de $I$ en $2^j$ intervalles (voir fig. \ref{fig:schema_sdyadique}). Ainsi à chaque changement de niveau, du niveau $j$ vers le niveau $j+1$,
            la résolution de la discrétisation est doublée. Les cases de cette partition dyadique sont indexées par deux entiers: $j$ le niveau de résolution de la grille
            et $k$ l'index de la case au sein de ce niveau. En particulier les cases $2k$ et $2k+1$ du niveau $j+1$ correspondent à la case $k$ du niveau $j$.
        \end{definition}
        \begin{figure}[htbp]
    \centering
\begin{tikzpicture}


\node[anchor=west] at (-8.5, -.25) {Niveau de résolution $j=3$};
\node[anchor=west] at (-8.5, .25) {Niveau de résolution $j=2$};
\node[anchor=west] at (-8.5, .75) {Niveau de résolution $j=1$};
% Niveau j-1

\draw (-4,.5) rectangle (0,1);
\draw (0,.5) rectangle (4,1);

    % Niveau j
\draw (-4,0) rectangle (-2,.5);
\draw (-2,0) rectangle (0,.5);
\draw (0,0) rectangle (2,.5);
\draw (2,0) rectangle (4,.5);
%\node at (1,1.75) {$k=2$};

% Niveau j+1

\draw (-4,-.5) rectangle (-3,0);
\draw (-3,-.5) rectangle (-2,0);
\draw (-2,-.5) rectangle (-1,0);
\draw (-1,-.5) rectangle (0,0);
\draw (0,-.5) rectangle (1,0);
\draw (1,-.5) rectangle (2,0);
\draw (2,-.5) rectangle (3,0);
\draw (3,-.5) rectangle (4,0);

% Relations
\node at (-2, .75) {$k=0$};
\node at (+2, .75) {$k=1$};

\node at (-3, .25) {$k=0$};
\node at (-1, .25) {$k=1$};
\node at (+1, .25) {$k=2$};
\node at (+3, .25) {$k=3$};

\node at (-3.5, -.25) {$k=0$};
\node at (-2.5, -.25) {$k=1$};
\node at (-1.5, -.25) {$k=2$};
\node at (-0.5, -.25) {$k=3$};

\node at (+0.5, -.25) {$k=4$};
\node at (+1.5, -.25) {$k=5$};
\node at (+2.5, -.25) {$k=6$};
\node at (+3.5, -.25) {$k=7$};


\end{tikzpicture}
\caption{Exemple de grille dyadique}
\label{fig:schema_sdyadique}
\end{figure}
        Dans ce qui suit il est supposé sans perte de généralité que la discrétisation se fait sur l'intervalle $[0,1]$,
        ainsi le niveau $j$ correspond à des cellules de tailles $1/{2^j}$ et la cellule $k$ du niveau $j$ est centrée en
        $x_k^j = \frac{k+(k+1)}{2} \frac{1}{2^j} = \frac{2k+1}{2^j}.$
        La notion d'ondelette se définit de la manière suivante:
        \begin{definition}[Ondelette]
            Une ondelette est une fonction $\Phi \in L^2(\mathbb R)$ à support compact de moyenne nulle.
            Pour qu'une ondelette soit pertinente dans le cas de la transformée en multi-échelle il est requis
            que la famille $\Bigl(x \mapsto \Phi( 2^j k - x ) \bigr)_{ (j,k)\in \mathbb{Z} \times \mathbb{Z} }$ forme une base de $L^2(\mathbb{R})$.
            En effet la transformée en ondelette sera une projection sur cette base, un peu comme la transformée de Fourrier est une projection sur les 
            fonctions trigonométriques.
        \end{definition}
        Alors la transformée en ondelette discrète peut être définie:
        \begin{definition}[Transformée en ondelette discrète - \textit{Discrete Wavelet Transform, DWT}] Donnée une fonction $f$,
            le coefficient $\gamma_k^j$ de sa DWT sur la cellule $k$ au niveau de résolution $j$ est:
            \begin{align}
                &\gamma_k^j = \frac{1}{N_j} \int_\mathbb{R} \Phi(2^j\cdot k - t)f(t) \text{d} t,\\\notag
                &\text{Où $N_j$ est un coefficient normalisation dépendant du niveau $j$.}
            \end{align}
            Contrairement à une transformée de Fourier, les coefficients ne dépendent pas d'une mais de deux variables. En effet, 
            la transformée en ondelette est plus riche d'informations. Là où la transformée de Fourier ne donne qu'une information 
            sur le contenu fréquentiel d'un signal, la transformée en ondelette donne une information sur le contenu en fréquence \underline{et}
            sur la localisation de ce contenu fréquentiel.
        \end{definition}

    \paragraph{La notion de détails}
        La multi-résolution adaptative se sert de la transformée multi-échelle pour adapter le maillage, c'est à dire compresser l'information (voir \ref{par:adaptation}).
        Cela requiert l'introduction de la notion de détail. Ce concept permet de ne pas utiliser la transformée en ondelette pour quantifier le contenu absolu porté par une échelle particulière,
        mais plutôt à comprendre en quoi ce contenu s'éloigne de ce que les échelles supérieures pourraient laisser supposer, en quoi il est \textit{inattendu}.
        Pour résumer à partir d'un niveau de résolution $j$, on définit un \textbf{prédicteur} polynomial, qui tente d'inférer l'allure de la fonction au niveau $j+1$.
        Puis le \textbf{détail} ne cherche pas naïvement à quantifier et localiser l'information contenue aux échelles du niveau $j+1$ mais plutôt, 
        à quantifier l'écart à la prédiction polynomiale.
        \subparagraph{Le prédicteur}
            Donné un point central $(x_0,y_0)$, et $2s$ voisins $(x_{-s},y_{-s}),(x_{-s+1},y_{-s+1})\dots (x_{s-1},y_{s-1}),(x_s,y_s)$, un prédicteur polynomial 
            ponctuel cherche le polynôme $P$ de degré $2s$ passant par ces $2s+1$ points. Cela permet d'inférer des valeurs pour $y$ en tout point $x$.
            Pour trouver $P(X) = \sum_{k=0}^{2s} a_k X^k$ revient à résoudre le système linéaire:
            \begin{align}
                \forall j \in \{-s,\dots 0 ,\dots,s\}: y_j = \sum_{k=0}^{2s} a_k x_j^k
            \end{align}
            Ce stage se focalise sur les volumes finis, donc ce n'est pas un prédicteur ponctuel, adapté au différences finies (voir \ref{def:vol_finis}), qui est utilisé mais un prédicteur sur la valeur moyenne. 
            Il ne cherche à imposer les valeurs en chaque point mais à fixer la valeurs moyenne sur chaque cellule,
            cela ajoute peu de complexité puisqu'il suffit d'ajouter une intégration lors de l'établissement du système linéaire pour travailler sur les valeurs moyennes.
            Pour l'usage que souhaité ici, il s'agit en d'évaluer la solution sur la cellule $k$ du niveau de résolution $j$ 
            (ce qui correspond à la cellule de centre $x_k^j = (k+1/2) 2^{-j}$) au niveau de résolution supérieure $j+1$,
            il faut donc appliquer le correcteur linéaire centré sur $x_k^j$ en $x_{\pm} = \pm 2^{-(j+1)}$. En pratique cela revient à faire une combinaison linéaire des $2s$ voisins
            qui vient corriger la valeur en $x_k^j$.
            Le prédicteur dépend donc du nombre de voisins pris de part et d'autre, ce nombre noté $s$ et appelé le \textit{stencil} du prédicteur.
            Plus $s$ est grand plus l'opération de prédiction est précise (mais peut éventuellement devenir bruitée\footnote{Ce n'est pas détaillé ici mais le lecteur se réfèrera à la théorie de l'interpollation...})
            et plus elle est coûteuse. Le coût exact n'est pas évident à estimer puisque qu'une combinaison linéaire
            quelques termes se fait en $O(1)$ sur les machines modernes, toutefois quelques subtilités détaillé en \ref{par:adaptation} interviennent.
            Ainsi les valeurs usuelles tu stencil sont généralement $s=1$ ou $s=2$

        \subparagraph{Les détails}
            À présent que le prédicteur à été décrit le concept de détail peut enfin être abordé. On suppose que l'on dispose d'une fonction $\tilde u^j$
            qui soit une approximation de la fonction $u$, au niveau de résolution $j$. Comme vu précédemment, le prédicteur permet d'obtenir une approximation 
            de $u$ au niveau $j+1$ grâce à $\tilde u^j$. Cette prédiction est noté $\hat u^{j+1}$. Les détails à la résolution $j+1$ sont alors définis comme les 
            coefficient de la transformée multi-échelle de la différence entre cette prédiction et la vraie fonction: $u-\hat u^{j+1}$.
            Alors les coefficients de détails n'encode que ce qui n'était pas prédictible par l'interpolateur polynomial. 
            Ce concept de détail est essentiel. 
            En pratique on ne réalise pas l'opération comme expliqué plus haut puisque à prédicteur et ondelette fixés $\Phi$, il existe une \textit{ondelette duale} $\Psi$
            qui permet directement d'obtenir les détails pour la DWT sur $\Phi$ en réalisant une DWT sur $\Psi$ ce qui accélère considérablement les calculs.\par
            Une autre façon de voir la notion de détail est la suivante:
            pour un niveau de résolution $j$ on note $V_j = \text{Vect}\Bigl((Phi_k^j)_k\Bigr)$, c'est à dire l'ensemble de fonction 
            représentables par les ondelettes de niveau $j$.
            Pour les ondelettes classiques\footnote{C'est en tout cas vrai pour les ondelettes de Haar\cite{postePoly}.} la relation suivante est vérifiée
            $V_0 \subset V_1 \subset V_2 \subset \ldots \subset V_N$. Et bien alors l'espace des détails, celui accessible par l'ondelette duale est $Q_{j+1}$
            le supplémentaire de $V_j$ dans $V_{j+1}$, en d'autre terme, il représente toutes les informations, les \textit{détails} contenues dans 
            le niveau d'approximation $V_{j+1}$ qui n'étaient pas prise en compte par $V_j$ (à l'échelle $j$ ce n'était que des détails). 
            Les coefficients de la décomposition par l'ondelette duale sont  Ainsi grâce à l'ondelette duale, 
            il est possible de calculer les coefficients de détails $d_k^j$ qui à chaque montée en résolution n'encode que l'information qui n'était 
            pas contenue dans la décomposition en ondelette au niveau précédent.\par
            Les deux visions ne sont pas rigoureusement les mêmes, la première représente ce qui est réalisé en pratique lors de la MRA, la seconde 
            est la vision standard de la théorie des ondelettes. Toutefois les deux approches ont la même motivation, ne calculer et ne mettre en valeur à chaque
            niveau que ce qui est nouveau, ce qui n'était pas contenu dans les niveaux précédents.
        
        \subparagraph{Intuition sur les détails}
            Lorsque l'on s'intéresse aux coefficients de détails $d_k^j$,
            l'indice $j$ de \textit{dilatation} fixe l'échelle analysée, c'est à dire la longueur d'onde analysée. Par exemple si $j=5$, 
            les coefficients $d^5_k$ donne une information sur l'information portée par les longueurs d'onde de l'ordre de $2^{-5} = 1/32$.
            La variable $k$ précise l'indice de la cellule analysée.
            Par exemple $\Vert d^5_10 \Vert > \Vert d^5_5 \Vert$ signifie que l'information portée par l'échelle $1/32$
            est plus importante au voisinage de la case $10$ qu'au voisinage de la case $5$.
            De même si $\Vert d^j_7 \Vert > \Vert d^{j+1}_{14} \Vert$ cela signifie qu'au voisinage de $x=\frac{7}{2^j}$
            les longueurs d'ondes $\frac{1}{2^j}$ sont plus présentes que les longueurs d'ondes $\frac{1}{2^{j+1}}.$
            Pour se fixer les idées, c'est comme si la transformée de Fourrier n'avait qu'une vision globale du contenu en fréquence, 
            quelle ne voyait que la moyenne sur le domaine de la transformée en ondelette pour chaque longueur d'onde.
            \begin{align}
                \Vert TF\bigl[ f \bigr](\omega = 2^j) \Vert^2 \sim \Vert \sum_{k} d^j_k \Vert^2.
            \end{align}
            \par 
            Grâce à cette notion de détaille la décomposition multi-échelle permet une description de la solution physique où l'apport à la solution de chaque échelle,
            de chaque distance typique est quantifié, les coefficients $d_k^j$ décrivant l'information contenue dans les échelles de l'ordre de $2^{-j}$.
\subsubsection{L'adaptation}\label{par:adaptation}
    \paragraph{La compression par décomposition multi-échelle}
            Pour compresser une fonction (ou une image comme dans le processus \texttt{jpg}), le processus est très simple, il suffit de fixer un seuil de compression $\varepsilon > 0$,
            de calculer les coefficients de détails de la fonction, 
            puis d'omettre (en pratique de retirer de la mémoire) les coefficient dont la norme est inférieure à $\varepsilon$.
            En pratique le coefficient de compression dépend du niveau étudié puisque le volume des cellules mise en jeu chute avec le niveau $j$
            (un détail de $10^{-2}$ à moins d'importance s'il porte sur des cellules de taille $10$ que sur des cellules de taille $10^{-5}$).
            En pratique l'algorithme serait le suivant:
            \begin{enumerate}  
                \item Calculer les coefficients $d_k^j$.
                \item Pour chaque niveau $j$ et chaque coefficient $d_k^j$: Si $\vert d_k^j \vert < \varepsilon_j = 2^{-j}\varepsilon$, alors $d_k^j\leftarrow 0$, les coefficients sont seuillés.
                \item[] Pour reconstruire au niveau de résolution souhaité: utiliser les détails jusqu'au niveau $j$ le plus fin conservé puis utiliser le prédicteur 
                pour interpoler jusqu'au niveau désiré. Le vocabulaire est heureusement choisit, pour compresser on omet les détails négligeables l'on conserve les détails importants.
            \end{enumerate}
    \paragraph{L'adaptation de maillage et l'heuristique d'Harten}
            L'adaptation de maillage, est une variation de la compression par décomposition multi-échelles précédemment décrite. 
            C'est est une opération de compression de solutions physiques \textit{prudente}.
            Les coefficients y sont seuillés de manière moins impitoyable: certains coefficients qui devraient être écartés
            par la compression sont malgré tout préservés. Ce choix se fait sur la base d'intuitions physiques, 
            la plus connue étant \textot{l'heuristique d'Harten}, introduite par Ami Harten \cite{harten1994}, le père de la MRA.
            Elle stipule que même si un coefficient de détail $d_k^j$ devrait être supprimé, si le niveau de détail du niveau supérieur 
            (c'est à dire $d^{j-1}_{\lfloor k/2 \rfloor}$) est particulièrement élevé, par exemple qu'il est par deux fois supérieur au seuil $\varepsilon_{j-1}$,
            alors le coefficient doit être conservé. En d'autre terme, même si la compression considère l'information à l'échelle $j$ négligeable,
            l'intuition physique pose son véto puisque les échelles supérieurs sont de grandes magnitudes et que cela présage que dans 
            les pas de temps a venir les échelles seront nécessaires à la fidèle capture des phénomènes physiques simulés. Par exemple cela peut signifier qu'un front
            d'onde est en train d'arriver dans les zone étudiée, il faut donc que la simulation puisse en capter toute la richesse.
\subsubsection{Algorithmes de simulation numérique}
    \paragraph{Algorithme général}
            L'intégration de la MRA un algorithme de simulation physique se fait de la manière suivante: 
            tous les $n_{MRA}$ pas de temps (éventuellement être à chaque pas de temps), 
            la solution physique est adaptée selon le procédé explicité ci-dessus. Les détails précédemment omis
            devenant nécessaires sont obtenus grâce au prédicteur. Puis la simulation se poursuite à partir de cette grille de calcul adapté.
            L'économie de mémoire se fait puisque tous les détails ne sont pas stockés et l'économie de calculs puisque moins de cellules sont mises en 
            jeu lors du déroulement de l’algorithme de simulation.\par
            Par exemple, en dimension une, si une zone est adapté avec un niveau de détail de niveau 5, la densité de cellule sur lequel il faut réaliser des est de $2^5$
            et la densité mémoire est de $\sum_{j=0}^5 2^j$. Si le niveau le plus fin de la grille est par exemple 9, il y a un gain computationnel théorique de l'ordre de 
            $2^{9-5}=16$ et en terme de mémoire une économie de l'ordre de $\frac{\sum_{j=0}^5}{2^8} > 10$. Ce gain est exponentiel avec la dimension du problème.
    \paragraph{Le cas volumes finis}
            Il faut détailler ce qui signifie \textit{poursuivre la simulation sur la grille de calcul adaptée} dans le cadre des volumes finis.
            
    \paragraph{Le débat sur la reconstruction}
\subsubsection{Impact sur les solutions}
    \paragraph{}
    \paragraph{}
\subsubsection{Implémentations de la multi-résolution}
    \paragraph{Méthodes usuelles}
    \paragraph{Le logiciel Samurai}
    % \newpage
    % \section{Objectifs}
    % Ce stage vise à évaluer des stratégies numériques modernes pour les équations d'advection-diffusion-réaction (ADR).
Équations portent deux difficultés majeures: \textit{des opérateurs de raideurs multiples} et \textit{une ample variété d'échelles spatiales} (voir \ref{par:adv-diff-reaction}).
\subparagraph{Premier objectif: éprouver les méthodes ImEx comme alternative au \textit{splitting} pour la gestion des raideurs multiples}
    Le \textit{splitting} est une méthode très populaire pour intégrer des opérateurs de raideur différentes. 
    Cependant des méthodes alternatives dites implicites-explicites présentent des avantages tangibles, comme une meilleure montée en ordre et 
    une gestion intrinsèque du couplage des opérateurs entre opérateurs. 
    Les méthodes ImEx ont donc été comparées sur un cas particulier; sur le pan théorique (zone de stabilité etc...) et expérimental (étude de convergence...).
\subparagraph{Second objectif: étude de l'erreur apportée par la multi-résolution adaptative lorsque utilisée pour gérer les différentes échelles spatiales}
    La multi-résolution adaptative est très étudiée dans la communauté scientifique et par l'équipe comme outils pour palier le caractère multi-échelles des solutions des équations d'ADR.
    Une analyse d'erreur théorique a été conduite sur une équation de diffusion résolue par une méthode numérique usuelle à laquelle s'ajoute une étape de multi-résolution.
    La décision de travailler sur l'équation de diffusion a été prise pour compléter le travail réalisé par l'équipe en \cite{belloti_et_al_2025} qui se centrait sur les opérateurs d'advection.
    L'analyse d'erreur met en lumière un mécanisme apporté par la multirésolution-adaptative pouvant dégrader l'ordre de convergence d'une méthode numérique classique.
    Une expérience numérique a par la suite été menée pour entreprendre d'observer ce phénomène dans un contexte pratique.
Ces deux objectifs permettent d'apporter différents éléments de compréhension sur le comportement des stratégies de simulation des équations d'ADR.
    % \newpage
    % \section{Méthode de travail et outils}
    % \subsection{Méthode et démarche}
    Les deux contributions présentées au chapitre suivant (\ref{par:contrib_1} et \ref{par:contrib_2}) on été réalisée 
    en s'imprégnant de l'état de l'art, en identifiant un point mal compris ou mal étudié, puis
    en réalisant une étude théorique sur un cas particulier pour déceler et révéler des comportements potentiellement généraux
    pour enfin chercher des observations expérimentales des mécanismes mis à jour.
\subsection{Outils numériques}
    Divers outils numériques ont appuyés les travaux réalisés au cours du stage aidant à des des tâches diverses comme 
    l'évaluation de fonction, la visualisation, le calcul formel et bien sûr la mise en oeuvre d'expériences numériques.
    \subparagraph{Expérimentation numérique}
        Le logiciel Samurai développé par l'équipe du CMAP a été utilisée pour exécuter toutes les expérimentations numériques nécessaires 
        au travaux réalisés au cours du stage. Il s'agit d'une bibliothèque écrite en \texttt{C++} permettant de mettre en oeuvre 
        rapidement et efficacement des méthodes numériques avec ou sans MRA.
    \subparagraph{Calcul et réprésentation}
        Les librairies Python usuelles: Numpy et Matplotlib on permis d'évaluer et de représenter des fonctions en différents points et de 
        les représenter. Une bonne représentation est essentielle car elle permet d'appréhender rapidement l'information et aide l'esprit 
        a développer une intuition riche capable d'orienter les recherches et d'affiner la compréhension. 
        Ces outils ont été cruciaux pour évaluer les fonctions de stabilité des méthodes étudiées (voir \ref{}) et pour réaliser 
        des post-traitements (visualisation, calcul d'erreur...) sur les solutions issues des expériences numériques  
    \subparagraph{Calcul formel}
        La librairie Python de calcul formel Sympy a rendu possible le calcul des équations équivalentes (voir \ref{}) qui n'aurait pas 
        été réalisable autrement. 

\newpage
\chapter{Contribution}
    \label{par:contrib}
    Ce chapitre présente les travaux du stage sur la simulation numérique des EDPs d'advection-diffusion-réaction grâce
aux méthodes modernes de compression de maillage (multirésolution adaptative) et de découplage des opérateurs (ImEx et splitting).\par
La première contribution analyse deux méthode ImEx appliquée à l'équation de Nagumo (une équation de diffusion-réaction).
Elles y sont comparée à une méthode de \textit{splitting} sur les questions de stabilité et de convergence.
La seconde contribution étudie l'impact de la multirésolution adaptative sur la convergence d'une méthode des lignes sur un problème de diffusion.
Cette étude comprend un pan théorique (obtention de l'équation équivalente du schéma avec multirésolution adaptative) et un pan expérimental (étude de convergence grâce au logiciel Samurai).
    \newpage
    \section{Étude de méthodes ImEx sur une équation de diffusion-réaction}
        \label{par:contrib_1}
        \label{par:contrib_imex}
L'objectif est de comparer la pertinence des méthodes RK implicites-explicites au \textit{splitting} d'opérateurs traditionnel.
Pour introduire cette première étude l'équation de Nagumo est d'abord présentée comme un excellent cas
test pour éprouver les méthodes de résolution des équations d'advection-diffusion-réaction. 
Dans un second temps les méthodes ImEx utilisées sont détaillées. Par la suite leur stabilité
est évaluée dans un contexte général; puis en se focalisant sur l'équation de Nagumo, où elles sont comparée 
à une méthode de séparation d'opérateur classique (splitting de Strang).
Ceci permet de valider la pertinence \textit{a priori} de ces méthodes sur les équations de réaction-diffusion, 
et mène naturellement à une étude de convergence expérimentale.
        \newpage
        \subsection{L'équation de Nagumo}L'équation de Nagumo (ou FitzHugh-Nagumo) est issue de modèles de transmission de l'information nerveuse \cite{FITZHUGH1961445}.
Nous utilisons la forme spatiale de l'équation \cite{keener1998mathematical} avec un terme de réaction cubique pour ajouter de la non-linéarité:
\begin{align}
    \dt{u} = D \dxx{u} - ku(1-u^2).
\end{align}
        \newpage
        \subsection{Les méthodes ImEx}La classe de méthodes ImEx étudiée sont les méthodes de Runge et Kutta additives (RK-ImEx ou RK-additive).
Ces méthodes consistent à sommer plusieurs méthodes de Runge et Kutta appliquées chacune à un opérateur différent.
L'objectif est d'employer des RK explicites (RKE) et des RK implicites (RKI), en adéquation avec les besoin de chaque opérateur.
\subsubsection{Un exemple}
    Pour introduire aux méthodes de Runge et Kutta additives, commençons par un exemple simple et usons d'une méthode RK-ImEx
    d'ordre un, résultant de la somme de deux méthodes RK à un étages (RK1). Nous notons cette méthode ImEx111 \cite{ASCHER1997151}. 
    Les méthodes RK1 servant de briques élémentaires à la RK111 sont: un schéma d'Euler explicite et un schéma d'Euler implicite.
    Supposons que l'on cherche à approcher une équation d'évolution faisant intervenir deux opérateurs: $A^E$ se prêtant à des méthodes explicites\footnote{Par exemple, un opérateur peu raide mais non local.}
    et $A^I$ se prêtant aux méthodes implicites\footnote{Par exemple un opérateur raide mais local.}. L'équation cible serait de la forme: 
    \begin{align}
        \dt{u} = A^E u + A^I u.
    \end{align}
    \paragraph{Résolution par approche monolithique}
        Rappelons d'abord comme le problème serait résolu en n'utilisant qu'une seule RK1 pour tout le problème (approche monolithique).
        \subparagraph{Euler explicite}
            En résolvant avec Euler explicite, le schéma s'écrit: 
            \begin{align}
                u^{n+1} = u^n + \Delta t (A^E + A^I) u^n.
            \end{align}
            Mais si l'opérateur $A^I$ est très raide, la stabilité risque d'imposer un pas de temps très restrictif risquant de rendre la méthode non viable.
        \subparagraph{Euler implicite}
            En résolvant avec Euler implicite, le schéma s'écrit:
            \begin{align}
                u^{n+1} = \bigl(Id - \Delta t (A^E + A^I)\bigr)^{-1} u^n.
            \end{align}
            Mais si l'opérateur $A^E$ est rend l'inversion coûteuse;
            par exemple s'il est non-local (impliquant la résolution d'un gros système au lieu de plusieurs petits systèmes), 
            ou s'il est non linéaire (nécessite d'être réinverser à chaque pas de temps);
            alors cette méthode ne sera pas viable non plus.
    \paragraph{Résolution par une méthode ImEx: une Runge et Kutta Additive}
            Mettons en oeuvre la méthode ImEx111. 
            L’approximation au pas de temps $n+1$ s'écrit en sommant une contribution issue de la méthode Euler explicite (RKE1)
            et une contribution issue de la méthode Euler implicite (RKI1):
            \begin{align}
                u^{n+1} = u^n + \Delta t (\underbrace{k_1}_{\text{RKE1}} + \underbrace{k_1'}_{\text{RKI1}})
            \end{align}
            La contribution RKE1 s'écrit:
            \begin{align}
                k_1 = A^E u^n.
            \end{align}
            La contribution RKI1 s'écrit:
            \begin{align}
                k_1' = A^I u^{n+1}
            \end{align}
            Ainsi: 
            \begin{align}
                &u^{n+1} = u^{n+1} = u^n + \Delta t ( A^E u^n +  A^I u^{n+1}),\\\notag
                \text{donc: }&u^{n+1} - \Delta t  A^I u^{n+1} = u^n + \Delta t  A^E u^n,\\\notag
                \text{et donc: }&u^{n+1} = (Id - \Delta t A^I)^{-1} \circ (Id + \Delta t A^E) u^n.
            \end{align}
            Ainsi dans cette méthode seul l’opérateur $Id- \Delta t A^I$ doit être inversé. Ce qui était bien l’objectif. Les opérateurs ont été 
            découplés lors de la résolution.
\subsubsection{Cadre mathématique général}
    Pour construire des méthodes plus complexes et d'ordres supérieurs introduisons le formalisme de \cite{ASCHER1997151} pour traiter les méthodes RK-additives. 
    Ici, nous travaillons uniquement sur méthodes ImEx pour deux opérateurs mais théoriquement, il est possible de construire des méthodes ImEx pour traiter 
    autant d'opérateurs que l'on le souhaite \cite{KENNEDY2003139}.
    \paragraph{Notation}%AFAIRE -> ajouter une référence interne vers la notion de méthode DIRK et SDIRK
        Une méthode ImEx additive est construite à partir d'une méthode implicite à $s$ étages (une méthode DIRK et si possible SDIRK) et d'une méthode explicite à $s+1$ étages
        \footnote{Au besoin, la méthode explicite peut être à $s$ étages, qui est un cas particulier d'une méthode à $s+1$ étages.}.
        Pour uniformiser, le tableau de Butcher de la méthode implicite est complété par une ligne et une colonne de zéros afin que les deux méthodes
        s'écrivent comme si elles avaient le même nombre d'étages.
        Les tableaux de Butcher des deux méthodes s'écrivent alors :
        
        \subparagraph{Méthode RKE, $s+1$ étages:}
        \begin{align}
        \text{RKE} : \quad
        \begin{array}{c|c}
        \tilde{c} & \tilde{A} \\
        \hline
        & \tilde{b}^T
        \end{array}
        =
        \begin{array}{c|ccccc}
        0 & 0 & 0 & 0 & \cdots & 0 \\
        \tilde{c}_1 & \tilde{a}_{10} & 0 & 0 & \cdots & 0 \\
        \tilde{c}_2 & \tilde{a}_{20} & \tilde{a}_{21} & 0 & \cdots & 0 \\
        \vdots & \vdots & \vdots & \ddots & \ddots & \vdots \\
        \tilde{c}_s & \tilde{a}_{s0} & \tilde{a}_{s1} & \tilde{a}_{s2} & \cdots & 0 \\
        \hline
        & \tilde{b}_0 & \tilde{b}_1 & \tilde{b}_2 & \cdots & \tilde{b}_s
        \end{array}
        \end{align}
        
        \subparagraph{Méthode RKI (DIRK) $s$ étages:}
        \begin{align}
        \text{RKI} : \quad
        \begin{array}{c|c}
        c & A \\
        \hline
        & b^T
        \end{array}
        =
        \begin{array}{c|ccccc}
        0 & 0 & 0 & 0 & \cdots & 0 \\
        c_1 & 0 & a_{11} & 0 & \cdots & 0 \\
        c_2 & 0 & a_{21} & a_{22} & \cdots & 0 \\
        \vdots & \vdots & \vdots & \ddots & \ddots & \vdots \\
        c_s & 0 & a_{s1} & a_{s2} & \cdots & a_{ss} \\
        \hline
        & 0 & b_1 & b_2 & \cdots & b_s
        \end{array}
        \end{align}
        
        où les coefficients $\tilde{a}_{ij}$, $\tilde{b}_i$, $\tilde{c}_i$ définissent la méthode explicite et 
        les coefficients $a_{ij}$, $b_i$, $c_i$ définissent la méthode implicite DIRK. 
        
    \paragraph{Schéma général d'une méthode RK-additive}
        Une étape de la méthode RK-additive appliquée au système 
        $\frac{du}{dt} = A^E u + A^I u$ s'écrit :
        
        \subparagraph{Calcul des étages :}
        En initialisant $u_0 = u^n$,
        \begin{align}
        u_i &= u^n + \Delta t \sum_{j=0}^{i-1} \tilde{a}_{ij} A^E u_j + \Delta t \sum_{j=1}^{i} a_{ij} A^I u_j, \quad &i = 0, 1, \ldots, s
        \end{align}
        Soit en mettant en lumière le caractère implicite de la méthode sur $A^I$:
        \begin{align}
        (Id - \Delta t a_{ii} A^I) u_i &= u^n + \Delta t \sum_{j=0}^{i-1} \tilde{a}_{ij} A^E u_j + a_{ij} A^I u_j, \quad &i= 0, 1, \ldots, s
        \end{align}
        
        \subparagraph{Solution à l'étape suivante :}
        \begin{align}
        u^{n+1} &= u^n + \Delta t \sum_{i=0}^{s} \tilde{b}_i A^E u_i + \Delta t \sum_{i=1}^{s} b_i A^I u_i
        \end{align}
        
        Cette formulation générale permet de construire des méthodes d'ordre élevé. 
    \paragraph{Ordre de convergence}
        L'ordre d'une méthode RK-additive est bien sur borné par l'ordre le plus faible des méthodes RK individuelles qu'elle convoque.
        Naturellement, cette borne n'est pas nécessairement atteintes pour toute méthodes RK additionnée. Des conditions d'ordre liant
        les coefficients des méthodes individuelles entre eux doivent être respectées. Le nombre de ses conditions augmente (très) rapidement avec 
        l'ordre de la méthode et le nombre d'opérateurs à résoudre \cite{KENNEDY2003139}, le lecteur motivé se référera par exemple à \cite{Hairer1981}.
        \newpage
        \subsection{Analyse de stabilité}\label{par:contrib1:stab}
L'objectif est d'appréhender la viabilité des ImEx ARK sur l'équation de Nagumo. Dans ce but, leur stabilité est étudiée.
Dans un premier temps, une étude générale de la stabilité des ImEx ARK est menée.
Puis, l'étude de stabilité se centre sur l'application à l'équation de Nagumo. L'ensemble des codes utilisés pour évaluer numériquement et afficher les domaines de stabilités
sont disponibles à l'adresse: \href{https://github.com/Ocelot-Pale/ImEx_stability_Nagumo}{https://github.com/Ocelot-Pale/ImEx\_stability\_Nagumo}.

\subsubsection{Présentation des méthodes ImEx étudiées}
Les deux méthodes utilisées sont les ImEx222 et ImEx232 d’ordre 2, issues de \cite{ASCHER1997151}.  
Elles associent un schéma explicite de type ERK et un schéma implicite DIRK.

\paragraph{ImEx232}  
Schéma à trois étages explicites et deux étages implicites, avec $\gamma = \tfrac{2-\sqrt{2}}{2}$ et $\delta = -\tfrac{2\sqrt{2}}{3}$ :
\[
\text{Explicite:}\;
\begin{array}{c|ccc}
0 & 0 & 0 & 0\\
\gamma & \gamma & 0 & 0\\
1 & \delta & 1-\delta & 0\\ \hline
 & 0 & 1-\gamma & \gamma
\end{array}
\quad
\text{Implicite:}\;
\begin{array}{c|cc}
\gamma & \gamma & 0\\
1 & 1-\gamma & \gamma\\ \hline
 & 1-\gamma & \gamma
\end{array}
\]

\paragraph{ImEx222}  
Schéma à deux étages explicites et deux étages implicites, avec $\gamma = \tfrac{2-\sqrt{2}}{2}$ et $\delta = 1 - \tfrac{1}{2\gamma}$ :
\[
\text{Explicite:}\;
\begin{array}{c|ccc}
0 & 0 & 0 & 0\\
\gamma & \gamma & 0 & 0\\
1 & 1-\delta & \delta & 0\\ \hline
 & 1-\delta & \delta & 0
\end{array}
\quad
\text{Implicite:}\;
\begin{array}{c|cc}
\gamma & \gamma & 0\\
1 & 1-\gamma & \gamma\\ \hline
 & 1-\gamma & \gamma
\end{array}
\]


\subsubsection{Étude de stabilité générale des RK-ImEx}
    Avec une méthode ImEx, les deux opérateurs de l'EDP sont découplés, c'est là l'intérêt.
    Cependant cela complexifie l'analyse usuelle de stabilité. 
    En effet la fonction de stabilité attend alors deux variables, 
    le coefficient spectral $Z_E$ associé à l'opérateur traité explicitement et
    le coefficient spectral $Z_I$ associé à l'opérateur traité implicitement.
    Ainsi, pour chaque couple $(Z_E,Z_I)$ d'indices spectraux, la fonction de stabilité prend une valeur différente, et comme les coefficients spectraux sont des nombres complexes, 
    on ne peut plus visualiser d'un simple coup d'oeil le domaine de stabilité, puisque celui-ci se trouve dans un espace de dimension quatre $\mathbb{C}^2$.
    % \footnote{En effet la fonction de stabilité $R \mathbb C \times \mathbb C \rightarrow \mathbb R$ et $\dim  \mathbb C \times \mathbb C=4$.}.
    \paragraph{Calcul des fonctions d'amplification }
    Afin d'étudier la stabilité linéaire des méthodes, les fonctions d'amplifications ont été évaluées numériquement.
    L'algorithme est le suivante:
    \begin{enumerate}
        \item Entrer les valeurs de $(Z_E,Z_I)$ pour lesquelles la fonction de stabilité doit être évaluée
        \item Simuler un pas du schéma en partant de $u_0 = 1$ appliqué à une équation du type Dahlquist :\\$\dt{u} = \lambda_E u + \lambda_I u$
        \begin{enumerate}
            \item Construire toutes les approximations intermédiaires avec les valeurs 
            \item Construire l'approximation finale $u_1$
        \end{enumerate}
        \item Évaluer la norme de $u_1$
    \end{enumerate}
    Cette l'étude générale, c'est à dire pour tout $(Z_E,Z_I) \in \mathbb{C}^2$ n'est pas détaillée ici, toutefois
    le lecteur intéressé pourra trouver les graphiques représentants les domaines de stabilité sur le \href{https://github.com/Ocelot-Pale/ImEx_stability_Nagumo}{Notebook en ligne}.


\subsubsection{Étude de stabilité linéaire appliquée à l'équation de Nagumo}
    La démarche précédente est particularisée en se centrant sur l'équation de Nagumo ; 
    permettant d'étudier la stabilité des méthodes ImEx sur ce problème particulier.
    \paragraph{Valeurs propres mises en jeu}
        Comme expliqué en \ref{par:analyser_operateurs_nagumo} l'équation présente deux opérateurs : 
        \begin{itemize}
            \item[$\diamond$] La diffusion dont le spectre s'étend de $\frac{-1}{L^2}$ à $\frac{-1}{\Delta x^2}$ (où $L$ est la taille du domaine discrétisé).
            \item[$\diamond$] La réaction dont le spectre balaie continûment $-k$ jusqu'à $2k$
        \end{itemize}
        Pour particulariser l'analyse de stabilité il faut donc tracer le diagramme de stabilité des méthodes étudiées en prenant $Z_I \in \mathbb{R}^-$ 
        et $Z_E \in [-k;2k] \subset \mathbb{R}$ ce qui donne un espace à deux dimensions. Il est ensuite pertinent placer des couples $(Z_E,Z_I)$ correspondant.
        Cela donne les diagrammes en fig. \ref{fig:stabilite_nagumo}
    \paragraph{Résultats}
    \begin{figure}[htbp]
        \centering
        
        \begin{subfigure}{\textwidth}
            \centering
            \includegraphics[width=0.8\textwidth]{media/4_travail/2_nagumo/stabilite/STABILITE_D1_k1_dt1.0e-02_dx4.9e-03.png}
            \caption{Cas standard\\D=1, k=1, dt=1.0e-03, dx=2.4e-03}
            \label{fig:stabilite_nagumo_a}
        \end{subfigure}
        
        \vspace{0.5cm} % Espacement entre les sous-figures
        
        \begin{subfigure}{\textwidth}
            \centering
            \includegraphics[width=0.8\textwidth]{media/4_travail/2_nagumo/stabilite/STABILITE_D10_k0.1_dt1.0e-02_dx4.9e-03.png}
            \caption{Cas diffusion plus raide, réaction moins raide\\D=10, k=0.1, dt=1.0e-02, dx=4.9e-03}
            \label{fig:stabilite_nagumo_b}
        \end{subfigure}

        \begin{subfigure}{\textwidth}
            \centering
            \includegraphics[width=0.8\textwidth]{media/4_travail/2_nagumo/stabilite/STABILITE_D0.0002_k500_dt1.0e-02_dx4.9e-03.png}
            \caption{Cas diffusion moins raide, réaction plus raide\\D=2e-4, k=500, dt=1.0e-02, dx=4.9e-03}
            \label{fig:stabilite_nagumo_c}
        \end{subfigure}
        
        \caption{Diagrammes de stabilité des méthodes ImEx comparés à ceux d'une méthode explicite à un schéma de splitting sur l'équation de Nagumo, pour différents couples $D$ et $k$.}
        \label{fig:stabilite_nagumo}
    \end{figure}
        Ces diagrammes permettent d'analyser respectivement la stabilité de la méthode \emph{ImEx222}, de la méthode \emph{ImEx232},
        et, à titre de comparaison, la stabilité d'une méthode \emph{Runge et Kutta explicite} d'ordre 2\footnote{Celle apparaissant dans ImEx222.} 
        et d'un \emph{schéma de splitting de Strang} utilisant une RK explicite pour la réaction et une RK implicite pour la diffusion.
        Chaque colonne représente l'analyse d'une méthode différente.
        \begin{itemize}
        \item[$\diamond$] La première ligne présente le domaine de stabilité en fonction des indices spectraux $Z_E \in \mathbb{R}$ et $Z_I \in \mathbb{R}^-$.
        \item[$\diamond$] La seconde ligne est un zoom  autour de ces indices spectraux.
        \item[$\diamond$] Les points bleus représentent les couples d'indices spectraux intervenant dans la résolution de l'équation de Nagumo
        pour les paramètres d'équation choisis ($D$ et $k$) et les paramètres de discrétisation retenus ($\Delta t$ et $\Delta x$).
        
        \item[$\diamond$] La dernière colonne (splitting) présente une disposition différente, puisque les opérateurs sont totalement découplés.
            \begin{itemize}
                \item[$\diamond$] La première ligne correspond alors à la fonction de stabilité de la méthode explicite (avec un zoom autour des indices spectraux de la réaction) et
                \item[$\diamond$] la seconde ligne représente la fonction de stabilité de la méthode implicite. 
                \item[$\diamond$] Dans les deux cas, l'intervalle tracé en bleu représente la plage de valeurs d'indices spectraux balayés par chaque opérateur.
            \end{itemize}
        \end{itemize}
        \paragraph{Analyse}
            \subparagraph{Analyse générale}\label{par:analyse_generale_stab_nagumo}
                Analyse des domaines de stabilité (fig.\figref{fig:stabilite_nagumo}) :
                \begin{itemize}
                    \item[$\diamond$]\textbf{Méthode explicite:} En troisième colonne, le diagramme de stabilité d'une méthode explicite RK explicite d'ordre deux, sert de référence. 
                        Le domaine de stabilité s'étend pour des indices spectraux négatifs jusqu'à $-2$ (résultat classique des méthodes EKR2).
                        Le schéma traitant conjointement les deux opérateurs, l'indice spectral résultant est $Z=Z_E+Z_I$. 
                        De fait domaine de stabilité s'étend jusqu'à $-2$ selon l'axe $Z_E$ tant que $Z_I$ est négligeable et de même,
                        le domaine de stabilité s'étend jusqu'à $-2$ selon $Z_I$ tant que $Z_E$ est négligeable. 
                        Enfin il y a une zone intermédiaire où $Z_E$ et $Z_I$ sont tous les deux de l'ordre de l'unité\footnote{Attention à l'échelle logarithmique.}.

                    \item[$\diamond$]\textbf{Méthode ImEx232:} La seconde colonne montre que la méthode ImEx232 maintient un domaine de stabilité restreint (jusqu'à $-2$) selon l'axe $Z_E$,
                        mais présente un domaine de stabilité bien plus étendu selon l'axe $Z_I$.
                        C'est logique puisque la valeurs propre $Z_E$ est explicitée,
                        sont domaine pris seul n'a évolué, et la valeur propre $Z_I$ peut être très raide (très négative) puisque la méthode traite explicitement l'opérateur lié à $Z_I$.
                        % Voir que le domaine de stabilité est légèrement décalé vers la droite quand $Z_I$ est grand

                    \item[$\diamond$]\textbf{Méthode ImEx222:} Passant à la première colonne, le domaine de stabilité ImEx222 resemble beaucoup à celui de l'ImEx232.
                        Néanmoins, le domaine de stabilité de l’épateur explicité est considérablement élargit 
                        là où $Z_I$ est assez grand. Cette propriété est remarquable : cela signifie que la stabilité de méthode ImEx résulte d'un couplage des raideurs ;
                        contrairement au splitting qui, par nature, découple totalement les problématiques stabilités.
                        Plus précisément, plus l'opérateur implicité est raide, plus l'opérateur explicité peut être raide.
                        % AFAIRE :
                        % Pourquoi ? Qu'est ce qui change fondamentalement ? 
                        % ---> certaines VP positives ont une fonction d'amplification négatives pour la réaction, ca pourrait poser problème (perte d'ordre ?)
                        % justement pour le splitting c'est pas le cas, et lui il ne perd pas l'ordre ??? ... 
                \end{itemize}
            \subparagraph{Dépendance de la stabilité aux paramètres de l'équation $k$ et $D$}
                Grace au graphiques en \figref{fig:stabilite_nagumo} la disposition couples de valeurs 
                propres mis en jeu par l'équation de Nagumo peut être analysées selon les paramètre $k$ et $D$.\\
                \textbf{Contexte : }
                Les paramètres de simulation: $\Delta t$ et $\Delta x$ sont fixés.
                Les couples de paramètres choisis sont : $(k,D)=(1,1)$, $(k,D)=(0.1,10)$, $(k,D)=(500,2\, 10^{-4})$. 
                Le produit $kD$ est maintenu égal à un, ainsi la vitesse de propagation est toujours la même (\emph{cf.} \ref{par:analyser_operateurs_nagumo}).
                Ces couples de valeurs propres $Z_E,Z_I$ sont en effet tracés en bleus sur le graphique
                \footnote{Pour $Z_E$, le spectre est continu, il a donc fallut échantillonnés le long de l'axe $Z_E$}.\\
                \textbf{Analyse : }
                \begin{itemize}
                    \item[$\diamond$]\textbf{Cas standard, $(k,D)=(1,1)$} - \figref{fig:stabilite_nagumo_a}:\\
                        Dans ce cas, la raideur de la diffusion déstabilise la méthode RKE2 (de nombreux indices spectraux tombent dans le zones rouges à cause des grandes valeurs de $Z_I$).
                        Pour ces valeurs de $(\Delta x, \Delta t)$ cette méthode n'est donc pas stable.
                        C'est ce qui est attendu, les méthodes explicite imposent des pas de temps très restrictifs sur les problèmes de diffusion.

                        En revanche, les méthodes ImEx sont stables puisque le domaine de stabilité est infini vers $Z_I \rightarrow -\infty$.
                        Certains couples de valeurs propres tombent malgré tout dans une zone instable (en bas à gauche). En réalité ce n'est pas un problème car il s'agit 
                        de couples où l'indice spectral $Z_E$ associé à l'opérateur de réaction est positif. Donc ce n'est est pas vraiment une instabilité, la méthode elle reflète simplement
                        la dynamique explosive de la réaction. D'ailleurs le graphique de la méthode explicite du splitting, il y a une zone 
                        où la fonction d'amplification est d'amplitude supérieure à un, le splitting reproduit donc fidèlement la dynamique de la réaction.
                        En revanche,
                        pour les méthodes ImEx, il existe des couples de d'indice spectraux propres où $Z_E$ est positif alors que la fonction d'amplification est d'amplitude inférieure à un. 
                        Cela pourrait être un frein pour reproduire fidèlement la dynamique explosive de la réaction dans les zones concernées
                        \footnote{Il n'est pas évident d'avoir \textit{a priori} la bonne intuition car peut être que la diffusion calme en quelque sorte 
                        le caractère explosif de la réaction et qu'alors une fonction d'amplification d'amplitude $< 1$ est normal...}.
                    \item[$\diamond$]\textbf{Cas diffusion raide, réaction peu raide, $(k,D)=(0.1,10)$}  - \figref{fig:stabilite_nagumo_b}:\\
                        Ici, $D=10$ donc toutes les valeurs propres liées à la diffusion sont multipliées par 10 par rapport au cas précédent. 
                        De fait la méthode RK2E de référence présente des instabilités pour encore plus de couples de valeurs propres est n'est pas pas viable.
                        Concernant les méthodes ImEx222 et ImEx232 elles restent stables, et cette fois-ci tous les couples d'indices spectraux liés à la 
                        dynamique explosive de la réaction sont amortie ce qui n'est pas forcément incohérent puisque la diffusion domine.
                        % ---> AFAIRE voir expérimentalement ce que ça donne... 
                    \item[$\diamond$]\textbf{Cas diffusion peu raide, réaction très raide$(k,D)=(500,2\, 10^{-4})$} - \figref{fig:stabilite_nagumo_c}\\
                        Dans ce cas de figure, $k=500$. La grande valeur du coefficient de réaction rend cette dernière très raide. 
                        Cela a pour effet de dilater les couples d'indices spectraux selon l'axe des abscisse puisque $Z_E \in [- 500 \Delta t, + 1000 \Delta t]$
                        alors que pour $k=1$: $Z_E \in [- \Delta t , 2\Delta t]$.
                        Ici la méthode explicite au sein des ImEx n'est plus stable pour la réaction, ainsi toutes les méthodes deviennent instables. 
                        Le splitting également devient instable car il utilise aussi la méthode RK2E pour la réaction. 
                        Le fait que la méthode explicite de l'ImEx soit instable pour l'opérateur explicité peu sembler un obstacle infranchissable,
                        cependant ce n'est pas si simple.
                        Pour illustrer ce point, étendons l'analyse avec le cas spécial en \figref{fig:stabilite_nagumo_cas_special}, dans ce cas la réaction est toujours raide $k=500$ mais la diffusion est également très raide car $D=500$
                        \footnote{Jusqu'ici, la vitesse de propagation était la même dans tous les scénarios puisque $kD$ était maintenu constant. Dans le scénario présenté ici, ce n'est plus le cas}
                        Alors la méthode ImEx222 devient stable, comme vu en \ref{par:analyse_generale_stab_nagumo}, plus l'opérateur traité implicitement est raide, 
                        plus la méthode permet à l'opérateur traité explicitement d'être raide. C'est un cas remarquable ou le couplage intervenant au sein de la méthode ImEx
                        la rend plus stable que le splitting !
                \end{itemize}
                \begin{figure}[htbp]
                    \centering
                    \includegraphics[width=0.8\textwidth]{media/4_travail/2_nagumo/stabilite/STABILITE_D500_k500_dt1.0e-02_dx4.9e-03.png}
                    \caption{Pour $k=500$ et $D=500$: diagrammes de stabilité des méthodes ImEx et de référence sur l'équation de Nagumo.}
                    \label{fig:stabilite_nagumo_cas_special}
                \end{figure}
            % \subparagraph{Analyse selon les paramètres de simulation $\Delta t$ et $\Delta x$}
                
        \newpage
        \subsection{Étude de la convergence}À présent que la stabilité des deux méthodes ImEx ont été comparées au \textit{splitting} d'opérateur, 
il est naturel de poursuivre par une expérience numérique pour qualifier la convergence de chaque méthode et 
d'évaluer la pertinence des méthodes ImEx face au \textit{splitting}.\par
\textbf{Présentation de l'expérience: }
    L'expérience est réalisée sur l'équation de Nagumo 1D à partir d'une solution initiale correspondant au profil de l'onde propagative de l'équation (voir \ref{par:analyser_operateurs_nagumo}).
    Succinctement, la simulation à lieu sur le domaine spatial $[-20,+20]$ entre $t=0$ et $t=3$.
    La grille spatial est divisée en $2^{13}$ cellules ce qui équivaut à un pas d'espace $\Delta t \approx 4.8 \, 10^{-3}$.
    Des conditions de Neumann homogènes et une vitesse de propagation adaptées permettent de maintenir le front d'onde au centre du domaine et 
    de négliger les effets de bords afin de comparer à la solution analytique exacte d'onde propagative. Les erreurs sont calculés sur le domaine $[-5,+5]$ pour ce centrer sur l'étude du front d'onde.\par
\textbf{Résultats: }
    Les résultats de l'expérience sont présentés en \ref{fig:imex_vs_splitting}. Il apparaît que si le schéma de splitting et ImEx222 ont des performances similaires, la méthode ImEx232
    à clairement une constante de convergence plus faible. De fait on voit que sur un maillage clanique (non-adapté), les méthodes ImEx peuvent apporter une cohérence globale qui mène a 
    une meilleure précision. Dans la section suivante, nous allons réitérer l'expérience sur un maillage adapté par multi-résolution adaptative pour étudier si l'adaptation en espace 
    interagit avec les méthodes ImEx et la méthode de splitting.
    \begin{figure}[hb!]
        \centering
        \includegraphics[width=\linewidth]{media/4_travail/2_nagumo/convergence/ImEx_vs_splitting_k10_D0.1.pdf}
    \caption{Comparaison de la convergence du schéma de splitting avec celle des méthodes ImEx222 et ImEx232
    sur l'équation de Nagumo avec $D=0.1$ et $k=10$}
    \label{fig:imex_vs_splitting}
    \end{figure}
\newpage
\subsection{Mise en lumière expérimental de couplages entre la méthode en temps et l'adaptation spatiale}
\label{par:couplage_temps_adaptation}
\textbf{Objectifs et contexte de l'étude: }
L'objectif est d'observer d'éventuelles interactions entre la méthode de découplage des opérateurs (ImEx/splitting) et l'adaptation en espace par multi-résolution adaptative.
Pour ce faire la comparaison entre ImEx222, ImEx2332 et splitting a été refaite (fig \ref{fig:couplage-MRA-temps}) en adaptant spatialement chaque schéma par MRA.
Si des différences par rapport à l'étude précédente (fig. \ref{fig:imex_vs_splitting}) apparaissent, ils résultent nécessairement de couplages entre la méthode d'intégration en temps
et l'adaptation spatiale.\par
\begin{figure}[htbp!]
    \centering
    \includegraphics[width=\linewidth]{media/4_travail/2_nagumo/couplage/couplage_MRA_temps.pdf}
    \caption{Convergence des schémas ImEx et de splitting, adaptés en espace par MRA, sur l'équation de Nagumo pour $k=10$, $D=0.1$.
    Les flux sont évalués au niveau courant (\textit{cf.} \ref{par:contrib_2}), la prédiction/reconstruction est assurée par un prédicteur à trois points et l'erreur est comparé à une solution convergé en temps.}
    \label{fig:couplage-MRA-temps}
\end{figure}
\textbf{Analyse des résultats:} pour les grands pas de temps, la convergence est similaire au cas non-adapté (fig. \ref{fig:imex_vs_splitting}). En particulier la méthode
ImEx232 exhibe une constante de convergence plus faible que le splitting. 
En revanche, lorsque l'erreur sature les méthodes ImEx offrent systématiquement des performances moins bonnes que le splitting.
La solution est comparée à une méthode quasi-exacte en temps, la convergence s'infléchie donc lorsque les erreurs de liée à la MRA sont de l'ordre des erreurs en temps.
Il semble donc que l'erreur plateau soit composée d'un terme lié à la compression, au $\varepsilon$ choisit, et d'une erreur de couplage entre l'adaptation spatiale
et la méthode en temps. Plus précisément il apparaît que les méthodes ImEx interagissent avec l'adaptation spatiale d'une manière plus néfaste que le splitting.\\
% \textbf{Limites de l'étude:} ces résultats expérimentaux peuvent être impactés par de nombreux paramètres numériques. Par exemple,
% il n'est pas garantis que les résultats soient les mêmes pour un prédicteur à 5 points au lieu de 3.
        \newpage
        \subsection{Conclusion}Cette première contribution a comparé un schéma de splitting ERK2+IRK2 à deux schémas ImEx ARK sur des questions de stabilités de convergence. 
L'étude de convergence à été réalisées dans deux contextes différents : sans adaptation spatiale et avec adaptation spatiale par MRA.
Les principaux résultats sont : 
\begin{itemize}
    \item[$\diamond$] \textbf{Stabilité:} Tandis que par nature le splitting 
    découple les problématiques de stabilités, celle des méthodes ARK résulte d'un couplage entre le spectre des deux opérateurs.  
    Cela pourrait être exploité astucieusement puisque dans certains cas (\textit{cf.} \ref{par:analyse_generale_stab_nagumo})
    un opérateur implicité très raide peut stabiliser la méthode explicite. Ce serait particulièrement intéressement pour des problèmes de diffusion-réaction
    où une réaction implicite très raide pourrait étendre la stabilité d'une diffusion explicite.
    \item[$\diamond$] \textbf{Convergence:} Les deux approches, ImEx ARK et splitting sont toutes les deux viables sur le maillage non-adapté.
    Il semble empiriquement que bien choisies, l'approche ImEx peut en effet offrir des constantes de convergences meilleures que le simple spitting.
    \item[$\diamond$] \textbf{Couplage avec l'adaptation spatiale:} Il a été monté empiriquement
    que les méthode ImEx peuvent interagir de manière néfaste avec l'adaptation spatiale par MRA.
    Il est probable que les méthode de splitting interagissent également, mais il semble que cette interaction soit plus limités.
    Comme montré au chapitre suivant (chapitre \ref{par:contrib_2}) ce type d’interaction (méthode en temps - adaptation spatiale)
    sont complexes et dépendent énormément des schémas temporelles et des modalités de mises en oeuvre de la MRA. 
    Donc cette étude ne suffit pas à tirer des conclusions générales,
    en revanche elle confirme l'existence de tels couplage et montre qu'ils peuvent être important car  (\textit{cf} \ref{par:couplage_temps_adaptation}) une méthode ImEx meilleure que le splitting sur un schéma non-adapté 
    devient moins bonne que le splitting pour un schéma adapté (dès lors que les erreurs de MRA ne sont plus négligeables).
\end{itemize}
    
    \newpage
    \section{Obtention de l'équation équivalente d'une méthode de lignes avec multirésolution adaptative sur un problème de diffusion.}
        \label{par:contrib_2}
        La multirésolution adaptative (MRA) a démontré une grande efficacité expérimentalement.\label{par:intro_contrib_eq_equiv}
Cependant, son impact sur la qualité des solutions obtenues n'est pas encore totalement compris mathématiquement.
En \cite{belloti_et_al_2025}, une étude de l'erreur introduite par la multirésolution adaptative a été menée.
Cette étude se concentre sur les équations d'advection résolues par des schémas de type \textit{one-step} \cite{DARU2004563} et
compare l'équation équivalente des schémas avec MRA et celle des schémas sans MRA pour mettre en lumière les différences introduites par la multirésolution.
La présente étude se place dans la continuité de cette démarche\footnote{À un niveau plus modeste.} et suit un cheminement similaire
pour déterminer, grâce aux équations équivalentes, l'impact de la MRA sur une équation de diffusion résolue par une méthode des lignes d'ordre deux.
La différence est donc double: d'une part la nature de l'opérateur étudié est différent (diffusion et non advection), d'autre part le type de schéma 
utilisés est également différent (méthode des lignes, ne couplant pas les erreurs en espace et en temps contre les schémas \textit{one-step} 
traitant d'un coup d'un seul les erreurs en temps et en espace.). L'objectif est d'évaluer l'impact de la MRA dans des contextes variés.
D'abord, le problème cible est présenté ainsi que le schéma de référence.
Par la suite, les équations équivalentes du schéma de référence et du schéma avec multirésolution adaptative sont évaluées puis analysées.
Enfin, les résultats théoriques obtenus sont éprouvés expérimentalement.
        \newpage
        \subsection{Cadre de l'étude}                   \subsubsection{Problème cible}
    Nous cherchons à résoudre le problème de diffusion suivant :
    \begin{align}
        \dt{u} = D \dxx{u}.
    \end{align}
    Nous ignorons les problématiques de conditions de bords.
        \subsubsection{Méthode des lignes utilisée}
            Pour résoudre cette équation aux dérivées partielles, nous utilisons une méthode des lignes. 
            D'abord un schéma volume fini pour la discrétisation spatiale menant à l'équation semi-discrétisée suivante : 
            \begin{align}
                dt{U}(t) = \frac{D}{\underbrace{\Delta x}_{\text{cellule}}} \Bigl(\frac{U_{k+1} - 2 U_{k} + U_{k-1}}{\underbrace{\Delta x}_{\text{approx. gradients}}}\Bigr)
            \end{align}
            Puis une méthode de Runge und Kutta explicite d'ordre deux sur l’opérateur linéaire donne :
            \begin{align}
                U_k^{n+1} &= U_k^n\\ \notag
                &+ D \frac{\Delta t}{\underbrace{\Delta x}_{\text{cellule}}} \Bigl(\frac{U_{k+1} - 2 U_{k} + U_{k-1}}{\underbrace{\Delta x}_{\text{approx. gradients}}}\Bigr)\\ \notag
                &+\frac{1}{2} \,D^2 \frac{\Delta t^2}{\underbrace{\Delta x^2}_{\text{cellule}}} \Bigl(\frac{U_{k+2} -4 U_{k+1}  +6 U_{k} -4 U_{k-1} + U_{k-2}}{\underbrace{\Delta x^2}_{\text{approx. gradients}}}\Bigr).
            \end{align}
            Cela s'écrit sous la forme conservative suivante:
            \begin{align}
                u^{n+1}_k = u^n_k +  D \frac{\Delta t}{\Delta x} \Bigl( \Phi^n_{k+1/2} - \Phi^n_{k-1/2} \Bigr)  + \left( D \frac{\Delta t}{\Delta x} \right)^2 \Bigl( \Psi^n_{k+1/2} - \Psi^n_{k-1/2} \Bigr) 
            \end{align}
            Avec:

            \begin{align}
                \Phi^n_{k+1/2} &= \frac{1}{\Delta x}(u^n_{k+1} - u^n_{k}),\\
                \Phi^n_{k-1/2} &= \frac{1}{\Delta x}(u^n_{k} - u^n_{k-1}),\\\notag
                \Psi^n_{k+1/2} &= \frac{1}{\Delta x^2} \left(\frac{1}{2} u^n_{k+2} -  \frac{3}{2}  u^n_{k+1} +  \frac{3}{2} u^n_{k} -  \frac{1}{2} u^n_{k-1}\right),\\\notag
                \Psi^n_{k-1/2} &= \frac{1}{\Delta x^2} \left(\frac{1}{2} u^n_{k+1} -  \frac{3}{2}  u^n_{k}   +  \frac{3}{2}u^n_{k-1} -  \frac{1}{2}u^n_{k-2}\right).\notag
            \end{align}

        \subsubsection{multirésolution adaptative}
        % AFAIRE --> déplacer la partie purement descriptive de la MRA dans une section du préambule technique 
            La multirésolution adaptative consiste à compresser le maillage, puis a effectuer les calculs sur le maillage compressé.
            Le schéma classique est le suivant : 
            \begin{enumerate}
                \item Partir d'un état compressé au pas de temps $n$.
                \item Calculer la solution au pas de temps $n+1$
                \item Compresser de nouveau selon un seuil de compression $\varepsilon$ grâce à une transformée multiéchelle.
            \end{enumerate}
            Lors de la compression, la transformée multiéchelle représente la solution sur plusieurs niveaux de détails, du plus global, au plus local.
            Plus le niveau est profond, c'est à dire plus il est local, moins les détails associés portent d'information. 
            L'opération de compression est réalisée en supprimant en chaque cellules, les niveaux dont la valeur des détails
            passent sous un certain seuil \footnote{Typiquement $2^{\Delta l} \varepsilon$ où ${\Delta l}$} \cite{postelApprox} .

            Ce seuil $\varepsilon$ n'est pas l'unique juge lors la compression, des heuristiques reposant sur la quantité d'information des détails de niveau supérieur
            sont utilisées pour ne pas seuiller systématiquement. L'objectif est en quelque sorte d'anticiper le besoin de détails\footnote{Même si la quantité d'information laisse entendre que 
            certains détails pourraient être ignorés, l'intuition physique pose sont veto et force certains détails à être conservés par précaution, par exemple si un front front d'onde arrive.}.
            La plus connue est l'heuristique d'Ami Harten \cite{harten1994}.

            Plusieurs stratégies existent pour réaliser le calcul d'un pas de temps à l'autre. Généralement, on estime les quantités au temps $n+1$ aux niveaux courants, 
            à partir des quantité au niveau courant au temps $n$. Ensuite une opération de reconstruction-prédiction détermine le niveau de finesse requis de la solution au temps $n+1$.
            Il est également possible de calculer les quantités du temps $n+1$ au niveau courant, à partir 
            des quantités au temps $n$ \textbf{reconstruites à un niveau plus fin}. Bien que cela aie une faible efficacité computationnelle, cela réduirait les erreurs liés à la multirésolution
            selon la qualité du prédicteur employé comme discuté en \cite{belloti_et_al_2025}. Ici nous allons étudier théoriquement les erreurs dans un contexte similaire. 
            Nous nous plaçons sur une cellule à un niveau de détail fixé, les flux sont calculés à partir de quantités reconstruites à un niveau de détails $\Delta l$ plus fin.
            Le raisonnement et les ressources de calcul formel de \cite{belloti_et_al_2025} on été d'une aide précieuse.

            \textbf{Calcul du flux au travers de $\Delta l$ niveaux:\\}
            Lorsque l'on applique le procédé de multirésolution, étant donné une cellule à un niveau de détail donné $l$, on cherche à faire évoluer la valeur à l'étape $n$ vers la valeur à l'étape $n+1$. 
            Pour ce faire, il faut évaluer les flux à partir les cellules voisines. Dès lors plusieurs choix s'offrent à nous. Où bien on utilise les cellules voisines à leurs niveaux courants, où bien on use de l'opérateur 
            de reconstruction afin d'estimer les cellules voisines à des niveaux plus fins.\par
            Dans un premier temps le stencil est choisi égal à 1. L'opérateur de prédiction d'un niveau à l'autre s'écrit alors : 
            \begin{align}
                \hat u^{l+1}_{2k} &= +\frac{1}{8} u^l_{k-1} + u^l_k - \frac{1}{8} u^l_{k+1},\\
                \hat u^{l+1}_{2k+1} &= -\frac{1}{8} u^l_{k-1} + u^l_k + \frac{1}{8} u^l_{k+1}.
            \end{align}
            Puis en notant $\doublehat{u}^{l+\Delta l}_{(\cdot)}$ cet opérateur de prédiction itéré au travers de $\Delta l$ niveaux\footnote{
                Au sens où l'on applique le prédicteur à des données déjà issues d'une prédiction.
            } : 
            \begin{align}
                \begin{bmatrix}
                    \doublehat{u}^{(l+\Delta l)}_{2^{\Delta l}k-2}\\
                    \doublehat{u}^{(l+\Delta l)}_{2^{\Delta l}k-1}\\
                    \doublehat{u}^{(l+\Delta l)}_{2^{\Delta l}k}\\
                    \doublehat{u}^{(l+\Delta l)}_{2^{\Delta l}k+1}\\
                \end{bmatrix}
                    =
                \underbrace{
                \begin{bmatrix}
                    +1/8 & 1 & -1/8 & 0 \\
                    -1/8 & 1 & +1/8 & 0 \\
                    0 & +1/8 & 1 & -1/8 \\
                    0 & -1/8 & 1 & +1/8 
                \end{bmatrix}^{\Delta l}}_{\text{Matrice de passage } P \text{ pour }s=1.}
                \cdot
                \begin{bmatrix}
                    u^l_{k-2}\\
                    u^l_{k-1}\\
                    u^l_{k}\\
                    u^l_{k+1}\\
                \end{bmatrix}
            \end{align}
\paragraph{Flux calculés au niveau le plus fin.}
On travaille sur une cellule au niveau courant $l$ (cellules de tailles $\tilde{\Delta x}=2^{\Delta l}\Delta x$) et l’on reconstruit les états au niveau $l+\Delta l$ grâce à des flux
flux au niveau fin, dont les gradients sont approximé par un pas $\Delta x$. La mise à jour conservative utilisée est donc
\begin{align}
u_k^{n+1}
= u_k^{n}
+ \frac{D\,\Delta t}{\underbrace{\Delta x\,2^{\Delta l}}_{\text{cellule}}}\Bigl(\doublehat{\Phi}_{k+\frac12}^n-\doublehat{\Phi}_{k-\frac12}^n\Bigr)
+ \left(\frac{D\,\Delta t}{\underbrace{\Delta x\,2^{\Delta l}}_{\text{cellule}}}\right)^2 \Bigl(\doublehat{\Psi}_{k+\frac12}^n-\doublehat{\Psi}_{k-\frac12}^n\Bigr).
\end{align}
Les flux sont évalués au \emph{niveau fin} (facteurs $1/\Delta x$ et $1/\Delta x^2$ portés par les flux) à partir d’états reconstruits $\doublehat{u}^{\,l+\Delta l}$ :
\begin{align}
\doublehat{\Phi}_{k-\frac12}^n
&= \frac{1}{\Delta x}\Bigl(
\doublehat{u}^{\,l+\Delta l}_{2^{\Delta l}k}
-\doublehat{u}^{\,l+\Delta l}_{2^{\Delta l}k-1}\Bigr),\\
\doublehat{\Phi}_{k+\frac12}^n
&= \frac{1}{\Delta x}\Bigl(
\doublehat{u}^{\,l+\Delta l}_{2^{\Delta l}(k+1)}
-\doublehat{u}^{\,l+\Delta l}_{2^{\Delta l}(k+1)-1}\Bigr),\\
\doublehat{\Psi}_{k-\frac12}^n
&= \frac{1}{\Delta x^2}\Bigl(
\tfrac12\,\doublehat{u}^{\,l+\Delta l}_{2^{\Delta l}k+1}
-\tfrac32\,\doublehat{u}^{\,l+\Delta l}_{2^{\Delta l}k}
+\tfrac32\,\doublehat{u}^{\,l+\Delta l}_{2^{\Delta l}k-1}
-\tfrac12\,\doublehat{u}^{\,l+\Delta l}_{2^{\Delta l}k-2}\Bigr),\\
\doublehat{\Psi}_{k+\frac12}^n
&= \frac{1}{\Delta x^2}\Bigl(
\tfrac12\,\doublehat{u}^{\,l+\Delta l}_{2^{\Delta l}(k+1)+1}
-\tfrac32\,\doublehat{u}^{\,l+\Delta l}_{2^{\Delta l}(k+1)}
+\tfrac32\,\doublehat{u}^{\,l+\Delta l}_{2^{\Delta l}(k+1)-1}
-\tfrac12\,\doublehat{u}^{\,l+\Delta l}_{2^{\Delta l}(k+1)-2}\Bigr).
\end{align}

\paragraph{Écriture matricielle.}
Pour simplifier l'implémentation des calculs dans les codes de calculs formel, il est pertinent d'écrire se qui précède sous forme matricielle.\\
Pour les flux gauches:
\begin{align}
\doublehat{\Phi}_{k-\frac12}^n
&= \frac{1}{\Delta x}\,
\begin{bmatrix}0 \\ -1 \\ +1 \\ 0\end{bmatrix}^\top \cdot
\begin{bmatrix}
\frac{1}{8} & 1 & -\frac{1}{8} & 0\\
-\frac{1}{8} & 1 & +\frac{1}{8} & 0\\
0 & +\frac{1}{8} & 1 & -\frac{1}{8}\\
0 & -\frac{1}{8} & 1 & +\frac{1}{8}
\end{bmatrix}^{\!\Delta l} \cdot
\begin{bmatrix}
u^l_{k-2}\\ u^l_{k-1}\\ u^l_{k}\\ u^l_{k+1}
\end{bmatrix},\\
\doublehat{\Psi}_{k-\frac12}^n
&= \frac{1}{\Delta x^2}\,
\begin{bmatrix}-\tfrac12 \\ +\tfrac32 \\ -\tfrac32 \\ +\tfrac12\end{bmatrix}^\top \cdot
\begin{bmatrix}
\frac{1}{8} & 1 & -\frac{1}{8} & 0\\
-\frac{1}{8} & 1 & +\frac{1}{8} & 0\\
0 & +\frac{1}{8} & 1 & -\frac{1}{8}\\
0 & -\frac{1}{8} & 1 & +\frac{1}{8}
\end{bmatrix}^{\!\Delta l} \cdot
\begin{bmatrix}
u^l_{k-2}\\ u^l_{k-1}\\ u^l_{k}\\ u^l_{k+1}
\end{bmatrix}.
\end{align}
Pour les flux droits:
\begin{align}
\doublehat{\Phi}_{k+\frac12}^n
&= \frac{1}{\Delta x}
\begin{bmatrix}0 \\ -1 \\ +1 \\ 0\end{bmatrix}^\top \cdot
\begin{bmatrix}
\frac{1}{8} & 1 & -\frac{1}{8} & 0\\
-\frac{1}{8} & 1 & +\frac{1}{8} & 0\\
0 & +\frac{1}{8} & 1 & -\frac{1}{8}\\
0 & -\frac{1}{8} & 1 & +\frac{1}{8}
\end{bmatrix}^{\!\Delta l} \cdot
\begin{bmatrix}
u^l_{k-1}\\ u^l_{k}\\ u^l_{k+1}\\ u^l_{k+2}
\end{bmatrix},\\
\doublehat{\Psi}_{k+\frac12}^n
&= \frac{1}{\Delta x^2}
\begin{bmatrix}-\tfrac12 \\ +\tfrac32 \\ -\tfrac32 \\ +\tfrac12\end{bmatrix}^\top \cdot
\begin{bmatrix}
\frac{1}{8} & 1 & -\frac{1}{8} & 0\\
-\frac{1}{8} & 1 & +\frac{1}{8} & 0\\
0 & +\frac{1}{8} & 1 & -\frac{1}{8}\\
0 & -\frac{1}{8} & 1 & +\frac{1}{8}
\end{bmatrix}^{\!\Delta l} \cdot
\begin{bmatrix}
u^l_{k-1}\\ u^l_{k}\\ u^l_{k+1}\\ u^l_{k+2}
\end{bmatrix}.
\end{align}


            % En particulier, si la cellule étudiée est au niveau courant $l$ alors on choisira d'aller approximer le flux au niveau le plus fin, c'est à dire avec $\dlbar = \bar l - l$.
            % Dès lors les flux approximés au niveau fins sont : 
            % \begin{align}
            %     \doublehat{\Phi}_{k-1/2} &= \doublehat{u}^{l+\dlbar}_{2^{\dlbar} k} -  \doublehat{u}^{l+\dlbar}_{2^{\dlbar} k-1} + \frac{1}{2} \lambda 
            %     \Bigl(
            %         \doublehat{u}^{l+\dlbar}_{2^{\dlbar} k+1}
            %         - 3 \doublehat{u}^{l+\dlbar}_{2^{\dlbar} k}
            %         + 3 \doublehat{u}^{l+\dlbar}_{2^{\dlbar} k-1}
            %         - \doublehat{u}^{l+\dlbar}_{2^{\dlbar} k-2}
            %     \Bigr),\\
            %     \doublehat{\Phi}_{k+1/2} &=  \doublehat{u}^{l+\dlbar}_{2^{\dlbar} (k+1)} -  \doublehat{u}^{l+\dlbar}_{2^{\dlbar} (k+1)-1} + \frac{1}{2} \lambda \Bigl(
            %         \doublehat{u}^{l+\dlbar}_{2^{\dlbar} (k+1)+1}
            %         - 3 \doublehat{u}^{l+\dlbar}_{2^{\dlbar} (k+1)}
            %         + 3 \doublehat{u}^{l+\dlbar}_{2^{\dlbar} (k+1)-1}
            %         - \doublehat{u}^{l+\dlbar}_{2^{\dlbar} (k+1)-2}
            %     \Bigr)
            % \end{align}
            
            % Cela s'écrit sous la forme matricielle suivante (utile pour utiliser les outils de calcul formel).
            % \begin{align}
            %     \doublehat{\Phi}_{k-1/2}
            %         &=
            %     \begin{bmatrix}
            %         -\frac{\lambda}{2}&
            %         (\frac{3}{2} \lambda - 1)&
            %         (1 - \frac{3}{2} \lambda)&
            %         \frac{\lambda}{2}&
            %     \end{bmatrix}
            %     \begin{bmatrix}
            %         +1/8 & 1 & -1/8 & 0\\
            %         -1/8 & 1 & +1/8 & 0\\
            %         0 & +1/8 & 1 & -1/8\\
            %         0 & -1/8 & 1 & +1/8\\
            %     \end{bmatrix}^{\dlbar}
            %     \begin{bmatrix}
            %         u^l_{k-2}\\
            %         u^l_{k-1}\\
            %         u^l_{k}\\
            %         u^l_{k+1}\\
            %     \end{bmatrix}
            % \end{align}
            % \begin{align}
            %     \doublehat{\Phi}_{k+1/2}
            %         &=
            %     \begin{bmatrix}
            %         -\frac{\lambda}{2}&
            %         (\frac{3}{2} \lambda - 1)&
            %         (1 - \frac{3}{2} \lambda)&
            %         \frac{\lambda}{2}&
            %     \end{bmatrix}
            %     \begin{bmatrix}
            %         +1/8 & 1 & -1/8 & 0\\
            %         -1/8 & 1 & +1/8 & 0\\
            %         0 & +1/8 & 1 & -1/8\\
            %         0 & -1/8 & 1 & +1/8\\
            %     \end{bmatrix}^{\dlbar}
            %     \begin{bmatrix}
            %         u^l_{k-1}\\
            %         u^l_{k}\\
            %         u^l_{k+1}\\
            %         u^l_{k+2}\\
            %     \end{bmatrix}.
            % \end{align}

            % Attention le schéma final est légèrement différent car il fait ici intervenir deux pas d'espace: $\Delta x$ le pas au niveau le plus fin
            % et $\Tilde {\Delta x} = 2^{\Delta l} \Delta x$ le pas du niveau courrant. Ainsi le schéma final est :
            % \begin{align}
            %     {u}^{n+1}_k = {u}^n_k + \frac{\lambda}{2^{\Delta l}} \Bigl( \doublehat{\Phi}^n_{k+1/2} - \doublehat{\Phi}^n_{k-1/2} \Bigr)
            % \end{align}
        \subsection{Les équations équivalentes}
            \label{par:contrib_2:resultats}
            \subsubsection{Calcul des équations équivalentes}
Tout le calculs ont été réalisés grâce à la bibliothèque de calcul formel \texttt{Sympy} et les codes 
sont disponibles à l'adresse: \href{https://github.com/Ocelot-Pale/etude_MR_RK2}{\nolinkurl{https://github.com/Ocelot-Pale/etude_MR_RK2}}.
\paragraph{Équation équivalente du schéma \texttt{I} - non adapté}
    Le calcul de l'équation équivalente sans MRA donne:
    \begin{align}
        \frac{\partial u}{\partial t}  =&D \frac{\partial^{2}u}{\partial x^{2}}
        + \Delta x^{2} \frac{D}{12}             \frac{\partial^{4}u}{\partial x^{4}} 
        -  \Delta t^{2} \frac{D^{3}}{6}          \frac{\partial^{6}u}{\partial x^{6}} 
        -  \Delta t^{3} \frac{D^{4}}{24}        \frac{\partial^{8}u}{\partial x^{8}}  + \mathcal{O}(\Delta x^4 , \Delta t^4).
    \end{align}
    
    Le schéma de référence est donc bien d'ordre deux en espace et en temps.
    En utilisant la constante constante de Von Neumann : $\lambda = D \frac{\Delta t}{\Delta x^2}$ : 
    \begin{align}\label{eq:ref:cfl}
        \frac{\partial u}{\partial t}  =&D \frac{\partial^{2}u}{\partial x^{2}}
        + \Delta x^{2} \frac{D}{12}             \frac{\partial^{4}u}{\partial x^{4}} 
        - \lambda^2 \Delta  x^{4} \frac{D}{6}          \frac{\partial^{6}u}{\partial x^{6}} 
        - \lambda^3 \Delta x^{6} \frac{D}{24}        \frac{\partial^{8}u}{\partial x^{8}}  + \mathcal{O}(\Delta x^7).
    \end{align}
\paragraph{Équation équivalente du schéma \texttt{II} - adapté \emph{sans} reconstruction des flux}
    Lorsque le schéma est adapté sans reconstruction des flux sur $\Delta l$ niveaux de détails,
    l'équation du équivalente est :
    \begin{align}\label{eq:sansRecons:brute}
        \frac{\partial}{\partial t} u=
            D \frac{\partial^{2}u}{\partial x^{2}}
            + (2^{\Delta l} \Delta x)^{2}  \frac{D}{12} \frac{\partial^{4}u}{\partial x^{4}}
            -\Delta t^{2} \frac{D^{3}}{6}   \frac{\partial^{6}u}{\partial x^{6}}
            -\Delta t^{3} \frac{D^{4} }{24} \frac{\partial^{8}u}{\partial x^{8}}
            + \mathcal{O}(\Delta x^4 , \Delta t^4).
    \end{align}
    En somme, sans reconstruction des flux, le schéma avec MRA se comporte comme le schéma de référence mais sur un maillage plus grossier. 
    La constante d'erreur en espace est effet de l'ordre de $2^{2\Delta l} \frac{D}{12} \frac{\partial^{4}u}{\partial x^{4}}$ 
    au lieu de $\frac{D}{12}\frac{\partial^{4}u}{\partial x^{4}}$.
    En injectant la constante de Von Neumann dans l'équation \eqref{eq:sansRecons:brute} :
    \begin{align}\label{eq:sansRecons:cfl}
        \frac{\partial}{\partial t} u=
            D \frac{\partial^{2}u}{\partial x^{2}}
            + (2^{\Delta l} \Delta x)^{2}  \frac{D}{12} \frac{\partial^{4}u}{\partial x^{4}}
            -\lambda^2 \Delta x^{4} \frac{D}{6}   \frac{\partial^{6}u}{\partial x^{6}}
            -\lambda^3 \Delta x^{6} \frac{D}{24} \frac{\partial^{8}u}{\partial x^{8}} + \mathcal{O}(\Delta x^7)
    \end{align}
\paragraph{Équation équivalente du schéma \texttt{III} - adapté \emph{avec} reconstruction des flux}
    En évaluant les flux à partir des cellules reconstruites au niveau le plus fin, l'équation équivalente est:
    \begin{align}\label{eq:equiv_brute_recons}
        \frac{\partial u}{\partial t} =&\; D \frac{\partial^2 u}{\partial x^2} \\\notag
        &- \frac{\Delta t}{2} D^2\, \bigl( 2^{2\Delta l}- 1 \bigr)          \frac{\partial^4 u}{\partial x^4}
        - \Delta t^2\, \frac{D^3}{6}          \frac{\partial^6 u}{\partial x^6}
        - \Delta t^3\, \frac{D^4}{24}         \frac{\partial^8 u}{\partial x^8} \\\notag
        &+ 2^{2\Delta l}\, \frac{D\, \Delta x^2}{12}    \frac{\partial^4 u}{\partial x^4}
        - 2^{2\Delta l}\, \frac{D\, \Delta l\, \Delta x^2}{4} \frac{\partial^4 u}{\partial x^4} + \mathcal{O}(\Delta x^4 , \Delta t^4).
    \end{align}
    Ce schéma est d'ordre un en temps contrastant avec l'ordre deux des deux schémas précédents.
    Ainsi, théoriquement la reconstruction au plus fin des flux réduit l'ordre de convergence temporelle de la méthode des lignes.
    % Cependant en pratique le caractère explicite de la méthode ERK2 impose la contrainte de stabilité $\lambda = \frac{D \Delta t } {\Delta x^2} < 1/2$
    % masquant la perte d'ordre puisque cela mène à l’élution équivalente:
    En utilisant la constante constante de Von Neumann, l'équation \eqref{eq:equiv_brute_recons} se réécrit : 
    \begin{align}\label{eq:equiv_cfl_recons}
        \frac{\partial u}{\partial t}
        =&+ D \frac{\partial^{2}u}{\partial x^{2}}\\\notag
        &+ \Delta x^{2}\, D \, \Bigl( 
        \frac{\lambda}{2} (2^{2 \Delta l} - 1) + \frac{2^{2 \Delta l} }{12} (1 - 3 \Delta l)
        \Bigr)\frac{\partial^{4}u}{\partial x^{4}}\\\notag
        & - \Delta x^{4} \frac{D \lambda^{2} \frac{\partial^{6}u}{\partial x^{6}}}{6} - \Delta x^{6} \frac{D \lambda^{3} \frac{\partial^{8}u}{\partial x^{8}}}{24}
        + \mathcal{O}(\Delta x^7). 
    \end{align}
\subsubsection{Comparaison}
    Pour comprendre le mécanisme menant à cette perte d'ordre,
    l'équation équivalente du schéma avec AMR et reconstruction des flux est comparée à celle du schéma de référence, 
    \textbf{avant} d'appliquer la procédure de Cauchy-Kovaleskaya; c'est à dire sans exploiter $\dt{u}=D \dxx{u}$.
    \paragraph{Sans multirésolution (schéma de référence)}
        L'équation modifiée sans multirésolution, avant procédure de Cauchy-Kovaleskaya est:
        \begin{align}
            \frac{\partial u}{\partial t}  &= D \frac{\partial^{2} u}{\partial x^{2}} \\\notag
                &+ \frac{1}{2} \underbrace{\Bigl(D^{2}\frac{\partial^{4} u}{\partial x^{4}} - \frac{\partial^{2} u}{\partial t^{2}} \Bigr)}_{\substack{\text{Se compense par} \\ \text{la procédure de} \\ \text{Cauchy-Kovaleskaya}}} \Delta t
                + \frac{D}{12} \frac{\partial^{4} u}{\partial x^{4}}  \Delta x^{2}
                - \frac{1}{24} \frac{\partial^{4} u}{\partial t^{4}}  \Delta t^{3} 
                - \frac{1}{6}  \frac{\partial^{3} u}{\partial t^{3}}  \Delta t^{2}
                + \mathcal{O}(\Delta x^4 , \Delta t^4)..
        \end{align}
        La méthode est bien d'ordre un, car à l'ordre un : $\frac{\partial u}{\partial t}  = D \frac{\partial^{2} u}{\partial x^{2}}$ et donc le terme $D^{2}\frac{\partial^{4} u}{\partial x^{4}} - \frac{\partial^{2} u}{\partial t^{2}}$
        se compense au cours de la procédure de Cauchy-Kovaleskaya.
    \paragraph{multirésolution avec reconstruction des flux}
        L'équation modifiée avec multirésolution et reconstruction des flux, sans appliquer procédure de Cauchy-Kovaleskaya est:
        \begin{align}
            \frac{\partial u}{\partial  t} =&D \frac{\partial^{2} u}{\partial x^{2}}\\\notag
            &+ \frac{\Delta t}{2} \underbrace{\Bigl(2^{2 \Delta l} D^{2}           \frac{\partial^{4} u}{\partial x^{4}} -\frac{\partial^{2} u}{\partial t^{2}} \Bigr)}_{\text{Ne se compensent plus.}}
            -\frac{\Delta t^{3}}{24}                          \frac{\partial^{4} u}{\partial t^{4}} 
            - \frac{\Delta t^{2}}{6}                           \frac{\partial^{3} u}{\partial t^{3}}
            +\frac{\Delta x^{2}}{12} (1 - 3\Delta l)    2^{2 \Delta l} D \frac{\partial^{4} u}{\partial x^{4}}
            + \mathcal{O}(\Delta x^4 , \Delta t^4).
        \end{align}
        Dans ce cas le terme en facteur du $\Delta t$ ne se s'annule plus. En effet le terme $D^{2}\frac{\partial^{4} u}{\partial x^{4}}$ est devenu au cours de la reconstruction
        $2^{2 \Delta l} D^{2}\frac{\partial^{4} u}{\partial x^{4}}$. En conséquence, la méthode perds un ordre de convergence temporel.\par
        Ce mécanisme s'explique de la manière suivante: dans l'équation équivalente, le terme $\frac{\partial^{2} u}{\partial t^{2}}$ apparaît indépendamment de la discrétisation spatiale
        \footnote{Il emerge de la différence $u_k^{n+1} - u_k^{n}$ à $k$ fixé.}. La méthode des lignes initiale crée un terme \textit{sur mesure} pour le compenser en approximant le terme spatial
        $D^{2}\frac{\partial^{4} u}{\partial x^{4}}$. Cependant au cours du processus de compression-reconstruction, cette approximation est entachée d'un facteur $2^{2 \Delta l}$.
        En d'autres termes le terme spatial construit pour compenser un terme temporel a été modifié par la multi-résolution, alors que le terme temporel lui n'est pas affecté par la multirésolution.
        Ainsi, les deux termes ne se compensent plus et l'ordre est perdu.
\subsubsection{Conclusion sur le résultat obtenu grâce aux équations équivalentes}
    Il a été ici mis en lumière que la reconstruction des flux au plus fin sur appliquée à méthode des lignes très simple
    peut théoriquement mener à un couplage des erreurs espace-temps polluant l'ordre initial de la méthode.
    Alors que cela n'arrive pas lorsque les flux sont évaluées plus grossièrement.
    En particulier l'étape de reconstruction-reconstruction altère des termes spatiaux qui ne compensent plus certaines erreurs temporelles et perturbent l'ordre de la méthode Runge et Kutta.
        \newpage
        \subsection{Complément expérimental}
            \label{par:contrib_2:exp}
            \subsubsection{Des défis expérimentaux}
    L'observation du mécanisme de perte d'ordre mis théoriquement à jour précédemment 
    est une tâche ardue.
    \paragraph{Une expérience exacte impossible à reproduire}
        Il n'est pas possible d'utiliser la méthode numérique utilisée dans l'étude théorique pour essayer de la valider expérimentalement. 
        En effet, la méthode est une RK2 explicite, elle impose sur ce problème de diffusion une condition de stabilité du type $\Delta t \propto \Delta x^2$,
        ce qui en pratique donne $\Delta t \ll \Delta x$. De fait, la majorité des erreurs sont liés au pas d'espace "grand" devant le pas de temps, et donc
        si l'on fixe le pas d'espace pour faire converger la méthode en temps, l'erreur est déjà saturée en temps et on n'observe rien. 
        C'est un classique de l'analyse numérique.
    \paragraph{Une tentative infructueuse}
        Suite à cette limite, l'expérience a été retentée avec une méthode voisine, une RK2, mais sans contrainte de stabilité. 
        Le code de calcul Samurai a donc été relancé avec une méthode RK implicite d'ordre deux, plus précisément une SDIRK.
        Avec cette méthode, l'ordre deux est observé et ce qu'importe les paramètre de l'AMR.
        Plusieurs hypothèses peuvent expliquer ce résultat:
        \begin{enumerate}
            \item Le phénomène n'est pas présent sur cette méthode implicite.
            \item D'autres problèmes biaisent l'expérience (voir paragraphe suivant).
            \item Les calculs ou les prémisses du développement théorique précédent des équations équivalentes sont faux.
        \end{enumerate}
    \paragraph{Des biais multiples}
        De nombreux biais expérimentaux persistent et peuvent expliquer l'invisibilité du phénomène.
        Par exemple l'étude précédente, ne prend pas en compte les conditions de bord, ou peut être que le maillage n'est compressé que localement 
        ce qui n'altère que peu la convergence. Enfin, le plus grave, dans les calculs précédents, il a été fait l'hypothèse que l'évaluation est faite au niveau le plus fin 
        (que la solution est entièrement reconstruite pour l'évaluation des flux), ce qui n'est pas fait en pratique. Cette idée viens du fait qu'intuitivement, 
        si l'on reconstruite jusqu'au niveau le plus fin le flux, l'erreur devrait diminuer et c'est ce que suggère \cite{belloti_et_al_2025}.
        Cependant cette fonctionnalité n'étant pas encore disponible dans le logiciel de calcul, l'expérience à été réalisée sans reconstruire le flux au niveau 
        le plus fin, mais en prenant la valeur du niveau courant de compression. Il semble peu probable que ce soit la cause de la non-observation du phénomène 
        de perte d'ordre mais cela reste un biais potentiel. Enfin peut être qu'un bug s'est glissé dans mon implémentation mais cela semble peu probable 
        puisque ce serait une erreur d'implémentation qu'il "améliore" l'ordre de convergence... 
        l'ordre.
\subsection{... DEPEND DE LA SUITE ...}
        \subsection{Conclusion}
            En conclusion, en résolvant le problème de diffusion grâce à la méthode des lignes proposée, la multi-résolution adaptative usuelle (sans reconstruction des flux)
préserve l'ordre du schéma. En revanche et contre toute attente, lorsque les flux sont reconstruits au niveau le plus fin, l'ordre deux en temps du semble formellement réduit à un.
Malheureusement ce phénomène n'a pas pu être mis en lumière expérimentalement, notamment à cause de la contrainte de stabilité.
Face à ces difficultés des expériences numériques plus ambitieuses ont été entreprises 
(méthodes stabilisées, différents niveaux de reconstruction, extension à d'autre cas que la diffusion "pure" etc...).
C'est l'object de la contribution suivante qui étudie empiriquement l'impact de la reconstruction ou non du flux sur les problèmes de diffusion puis de diffusion-réaction.
    
    \newpage
    \section{Impact de la qualité de reconstruction des flux pour les problèmes diffusion avec AMR.}
        \label{par:contrib_3}
        Cette troisième contribution prolonge empiriquement la précédente,
étudiant \emph{expérimentalement} l'impact de différentes approches de multirésolution adaptatives sur les solution numériques des problèmes diffusifs. 
Elle étudie trois manières de mettre en place l'adaptation spatiale par MRA:
celle des schémas \texttt{II} (sans reconstruction des flux) et
\texttt{III} (avec reconstruction des flux) de la contribution précédente
ainsi qu'une approche \textit{intermédiaire}.\par
Ce travail s'articule de la manière suivante:
\begin{itemize}
\item[$\diamond$]\ref{par:contrib_3:pres_algos} Présentation des paradigmes MRA comparés.
\item[$\diamond$]\ref{par:contrib_3:etude_ERK2} Première expérience numériques comparant l'erreur en fonction du paradigme d'AMR choisi
sur le problème de diffusion résolu par le schéma numériques de la contribution antérieure (\ref{par:contrib_2}) - méthode des lignes, Volumes Finis + Runge et Kutta explicite.
\item[$\diamond$]\ref{par:contrib_3:stab_MRA} Les résultats, inattendus
ont conduit à formuler l'hypothèse que les schéma \emph{avec} reconstruction présenterai des problèmes de stabilité pour certains modes que le schéma \emph{sans} reconstruction n'aurait pas.
Cependant cette hypothèse est invalidée par une étude de stabilité linéaire.
\item[$\diamond$]\ref{par:contrib_3:ROCK2} La principale limite de l'étude antérieure est
la contrainte de stabilités de la méthode explicite. Cela ne permet d'observer que des solutions convergées en temps (erreur spatiale dominante car stabilité $\Rightarrow \Delta t \ll \Delta x$).
Pour poursuivre la comparaison dans un contexte où les erreurs temporelles ne sont pas négligeables, une seconde expérience 
est réalisée remplaçant la méthode ERK2 par la méthode \emph{stabilisée} ROCK2 \cite{abdulle2002fourth} permettant d'accéder à 
une plus large gamme de pas de temps.
\item[$\diamond$]\ref{par:contrib_3:eq_equiv} Les résultats numériques pour des pas de temps moins restreints sont encore plus inattendus que les précédents.
Cependant en les reliant aux travaux théoriques précédents  (\ref{par:contrib_2}), l'ensemble des comportements obtenus est finalement compris et expliqué.
\end{itemize}
        \subsection{Les schémas paradigmes d'AMR comparés}
            \label{par:contrib_3:pres_algos}
            % La différence entre les trois algorithmes étudiés réside dans la manière de conjuguer les volumes finis \cite{LeVeque1990} et la multirésolution adaptative.
% Le calcul des \textit{flux}, central dans les méthodes volumes finis, 
% requiert l'évaluation de termes dépendant de la solution aux \textit{interfaces} amont et avales des cellules.
% Les volumes finis n'approximant que les valeurs moyennes sur les cellules et non les valeurs ponctuelles aux interfaces, 
% les termes de flux sont alors évalués comme fonction des valeurs moyennes sur les cellules voisines de l'interface.\par
% La MRA rend la définition des flux numériques non-univoque - c'est ce qui est ici étudié.\par 
% Comme la MRA défini plusieurs grilles de pas $\Delta x,2 \Delta x, 4 \Delta x,...$ il est ambigu de choisir sur quelle grille évaluer les voisins de chaque interface.
% En effet, à niveau de détail $l$ fixé, le flux numérique doit-il être déterminé à partir des cellules voisines du niveau $l,l+1,l+2...$ (voir le schéma en fig. \ref{fig:schema_algos}) ? 
% Le premier algorithme étudié (la référence en MRA), consiste à évaluer les flux à partir des voisins du même niveau que la cellule étudiée. C'est à dire que 
% si flux concerne une cellule de la grille de niveau $l$, les voisines de l'interface sont choisit également au niveau $l$. Cela revient à résoudre localement l'EDP au niveau courant de la grille.
% Cet algorithme est la norme en MRA car ne ne requiert aucun calcul supplémentaire, les valeurs sur la grille au niveau $l$ sont directement accessibles.
% Le second algorithme consiste à systématiquement les choisir les cellules de la grille la plus fine. Intuitivement c'est le plus précis, mais plus coûteux car 
% la grille plus fine n'est pas directement accessible (voir la partie sur la reconstruction en \ref{par:explication_MRA}). 
% Enfin le troisième algorithme est un compromis entre les deux approches précédentes. Il consiste à calculer les flux à partir des valeurs un niveau en deçà du niveau courant, 
% pour gagner un peu en précision sans pour autant s'exposer à des coûts computationnels prohibitifs.
% En \cite{belloti_et_al_2025}, la différence théorique entre les deux premiers algorithmes a été étudiée sur des problèmes d'advection linéaires et
% une comparaison expérimentale entre les trois algorithmes a été réalisée sur des problèmes d'advections linéaires et non-linéaires.
Les schémas comparés dans cette étude sont ceux de l'étude précédente (\ref{par:contrib_2}) : 
le schéma \texttt{I} non adapté (référence),
le schéma \texttt{II} adapté sans reconstruction des flux,
le schéma \texttt{III} adapté avec reconstruction des flux au niveau le plus fin.
À ceux-ci s'ajoute une approche intermédiaire entre les schémas \texttt{II} et \texttt{III} : le schéma \texttt{IV}
qui est adapté et qui reconstruit les flux avec \underline{un} niveau de détails supplémentaire.
C'est à dire que si en un point, la multirésolution adaptative représente la solution au niveau de détail $l$, et si le 
maillage propose un niveau de détail maximal $l^{\max} > l$ ; alors le schéma \texttt{II} calcule les flux à partir des données au niveau $l$,
le schéma \texttt{III} calcule les flux à partir de données reconstruites du niveau $l$ jusqu’au niveau $l^{\max}$ (nécessitant $\Delta l=l^{\max}-l$ reconstructions) 
et le schéma \texttt{IV} calcul les flux à partir de données reconstruites du niveau $l$ jusqu'au niveau $l+1$ (nécessitant une seule reconstruction). 
L'objectif du schéma \texttt{IV} est d'être un bon compromis entre précision et coût computationnel.
        \subsection{Expérience numérique avec une méthode Runge et Kutta explicite}
            \label{par:contrib_3:etude_ERK2}
            La première expérience numérique porte sur l'équation de diffusion résolue par le même schéma numérique que dans l'étude théorique précédente (\textit{cf:} \ref{par:intro_contrib_eq_equiv}).
c'est à dire une discrétisation spatiale d'ordre deux du Laplacien intégré en temps par une méthode Runge et Kutta explicite d'ordre 2. 
Une méthode implicite pourrait paraître plus appropriée pour s'affranchir des problèmes de stabilité; cependant l'inversion d'un système linéaire couplé à 
la reconstruction des flux à des niveaux plus fin (non-standard, algos 2 et 3) est difficile à implémenter informatiquement et aurait ralenti l'étude.
À cause de la contrainte de stabilité $\Delta t \propto \Delta x^2$,
seules des solutions convergées en temps (erreur spatiale dominante puisque $\Delta t \ll \Delta x$) sont observées. Cette contrainte est levée dans l'expérience suivante en \ref{par:etude_diff_rock4}.\par 
\subsubsection{Résultats numériques}
Les résultats sont étonnants, l'algorithme 1 (le plus grossier) offre la plus faible erreur et plus l'algorithme reconstruit finement les flux plus l'erreur augmente. 
A titre d'exemple, pour un état initial en forme de courbe de Gauss, une seuil de compression $\varepsilon = 10^{-4}$ et un maillage présentant jusqu'à 4 niveau de finesse, les erreurs $L^2$ au temps final $T_f=1$ sont les suivantes:\par
\begin{center}\begin{tabular}{|c|c|c|}
\hline
Algorithme $n^o$ & Niveau d'évaluation des flux & Erreur $L^2$ \\
\hline
1 & Courant           & $1 \times 10^{-4}$ \\
2 & Inférieur direct  & $2 \times 10^{-4}$ \\
3 & Plus fin          & $3 \times 10^{-4}$ \\
référence & Sans AMR& $2 \times 10^{-5}$ \\
\hline
\end{tabular}\end{center}
Bien sûr, l'algorithme sans AMR reste celui avec la plus petite erreur.
\subsubsection{Analyse et hypothèses}
Ce résultat est assez surprenant puisqu'on s'attendrait à ce qu'une reconstruction plus précise du flux donne de meilleurs résultats.
Face à ces résultats, plusieurs hypothèses sont émises:
\begin{enumerate}
    \item Le fait de reconstruire \textit{déstabilise} la méthode.
    \item L'usage de cellules plus grandes pour évaluer les flux augmente le caractère diffusif du schéma et réduit "accidentellement" l'erreur.
    \item Peut-être que cette tendance ne vaut que sur des solutions où l'erreur spatiale domine et que les résultats seraient différents lorsque l'erreur temporelle reste dominante.
\end{enumerate}
La première hypothèse est éprouvée dans la prochaine section \ref{par:stab_amr} et une méthode stabilisée est utilisée en \ref{etude_diff_rock4} pour explorer numériquement la troisième hypothèse.
        \newpage
        \subsection{Analyse de stabilité}
            \label{par:contrib_3:stab_MRA}
            \label{par:stab_amr}
Le caractère inattendu du résultat précédent - le schéma AMR sans reconstruction des flux est meilleur que celui avec reconstruction - a fait émergé l'hypothèse 
que le schéma \textit{avec reconstruction} présenterait des problèmes de stabilités pour certains modes de l'équation.
Une étude comparative de la stabilité des deux méthodes a donc été réalisée par analyse de Fourrier - Von Neumann en adaptant le code de calcul formel ayant fourni les équations équivalente en \ref{par:contrib_2}.
Une interface permettant de visualiser ces résultats est disponible à l'adresse : \href{https://github.com/Ocelot-Pale/etude\_MR\_RK2}{https://github.com/Ocelot-Pale/etude\_MR\_RK2}.\\
\textbf{La principale conclusion est que la méthode \textit{avec} reconstruction ne présente pas plus de problème de stabilité que la méthode \textit{sans} reconstruction.}



        \newpage
        \subsection{Expérience numérique avec une méthode explicite stabilisée}
            \label{par:contrib_3:ROCK2}
            \label{par:etude_diff_rock4}
Pour observer ce qui se produit lorsque l'erreur n'est pas saturée en espace mais que l'erreur temporelle intervient également, 
le logiciel Ponio \footnote{\href{https://github.com/hpc-maths/ponio}{https://github.com/hpc-maths/ponio}} a été couplé à Samurai.
Il permet d'utiliser facilement des méthodes d'intégration en temps complexe. 
Grâce Ponio, l’expérience précédent à été réitérée en remplaçant la méthode ERK2 par la méthode stabilisée ROCK4 \cite{AbdulleMedovikov2001}. 
Cette méthode reste explicite (ce qui permet grâce à Samurai d'étudier facilement les différentes façons d'évaluer les flux) 
tout en assouplissant significativement la contrainte de stabilité.
\subsubsection{Résultats numériques}
    Les résultats sont présentés en fig. \ref{fig:convergence_diffusion_rok4}. Chaque graphique correspond à un paramétrage différent de l'AMR. 
    Les colonnes correspondent à différents niveau de compression: $\varepsilon \in \{10^{-5}, 10^{-4}, 10^{-3}\}$.
    Les lignes correspondent à différents niveau de profondeurs du maillage: $\Delta l^{max} \in \{1,2,4\}$.
    Sur chaque graphique l'erreur $L^2$ tracée en fonction du pas de temps pour chaque méthode de calcul du flux numérique.
    La courbe blanche correspond à la solution sans AMR sur la grille la plus fine, elle sert de référence.
    La courbe marron ($\texttt{flux\_recons\_lvl=0}$) est celle de l'algorithme 1, standard en AMR - reconstruction des flux avec les cellules du niveau courant.
    La courbe verte  ($\texttt{flux\_recons\_lvl=-1}$) représente l’algorithme 2 où les cellules servant à l'évaluation des flux sont systématiquement reconstruites au niveau le plus fin.
    Enfin la courbe bleu ($\texttt{flux\_recons\_lvl=1}$) représente l'algorithme 3 où la solution est reconstruite d'un niveau pour évaluer le flux numérique.

\begin{figure}[h!]
    \centering
    \includegraphics[width=\textwidth]{media/4_travail/3/flux_reconstruction_method_diffusion.pdf}
    \caption{THE CAPTION}
    \label{fig:convergence_diffusion_rok4}
\end{figure}

\clearpage
\begin{figure}[p]
\centering
\rotatebox{90}{%
\begin{minipage}{\textheight}
\centering
\begin{subfigure}{0.32\textwidth}
    \centering
    \includegraphics[width=\linewidth]{media/4_travail/3/01_error_profile_big_cfl.png}
    \subcaption{Constante CFL Élevée}
\end{subfigure}\hfill
\begin{subfigure}{0.32\textwidth}
    \centering
    \includegraphics[width=\linewidth]{media/4_travail/3/02_error_profile_medium_cfl.png}
    \subcaption{Constante CFL Moyenne}
\end{subfigure}\hfill
\begin{subfigure}{0.32\textwidth}
    \centering
    \includegraphics[width=\linewidth]{media/4_travail/3/03_error_profile_small_cfl.png}
    \subcaption{Constante CFL Petite}
\end{subfigure}

\caption{Profils d'erreur pour différentes valeurs de la constante CFL. C'est le pas d'espace est fixé, 
cela revient simplement à changer le pas de temps.}
\label{fig:error_profiles_landscape}
\end{minipage}%
}
\end{figure}
\clearpage


    
    Si le paramètres de compression est très restrictif $\varepsilon = 10^{-5}$ (colonne de gauche), les 4 algorithmes sont équivalents, en effet il n'y a presque jamais d'adaptation de maillage, car le seuil de compression est trop restrictif.
    Pour des paramètres de compression plus raisonnables ($\varepsilon = 10^{-4}$ et $\varepsilon = 10^{-3}$), les observations sont plus riches.
    Lorsque les erreurs temporelles de dominent pas, les trois algorithmes d'AMR améliorent la convergence - ce qui est surprenant. Cependant leur convergence s'arrête plus tôt, du fait des 
    erreurs spatiales liées à l'AMR et pour des pas de temps assez petit, le schéma sans AMR demeure meilleur. Pour un pas de temps où l'erreur en temps domine, 
    en figure \ref{fig:flux_reconstruction_error} est tracé l'erreur au temps final pour chacune des méthodes est tracé et il semble effectivement que les méthodes avec AMR lissent
    "mieux" que la méthode sans AMR qui semble un peu plus bruité. 


    Pour des pas de temps menant à une erreur spatiale dominante, les résultats sont similaires à ce qui avait été observées, plus les flux sont évaluées à partir de reconstructions fines, 
    plus le plateau de saturation en erreur spatiales est important.
    \subsubsection{Conclusion}
    Deux conclusions sont à tirer de cette expérience:
    \begin{enumerate}
        \item L'AMR tend à faire chuter plus rapidement l'erreur temporelle qu'avec une méthode non adaptée spatialement, et cela indépendamment de la méthode d'évaluation du flux numérique.
        \item L'AMR mène à un plateau de convergence plus élevée que sans AMR (l'erreur spatiale domine). Ce qui est étonnant est que cette saturation de l'erreur arrive d'autant plus vite que les flux sont évaluées à partir de reconstructions fines; même observation qu'avec la méthode ERK2.
    \end{enumerate}
    Ces résultats ne valent a priori que sur une équation de diffusion "pure" et la relation d'ordre entre les erreurs qui parait incohérente n'est peut être que le résultat d'un 
    "heureux effet lissant" qu'apporte l'AMR et qui est plus prononcé encore si l'on évalue les flux au niveau courant (algorithme 1).
    La prochaine section se propose donc de reprendre l'expérimentation sur une équation de diffusion-réaction, présentant une onde progressive ce qui s'approche un peu plus
    d'un contexte de simulation industrielle réaliste.
    Dès lors, il ne s'agit plus pour le schéma de simplement lisser mais également suivre la dynamique d'un front d'onde ! Peut-être que les résultats seront différents. 
        \newpage
        \subsection{Lien avec les équations équivalentes}
            \label{par:contrib_3:eq_equiv}
            \label{par:lien_rock_equiv}
Pour comprendre les résultats de convergence précédents (\ref{par:etude_diff_rock4}), le profil des erreurs a été tracé (voir \ref{fig:error_profiles_landscape}) pour chaque méthode numérique et pour différents régimes de CFL ; puis 
ces observations ont été reliées aux équations équivalentes développées en \ref{par:contrib_2}. 
Cette mise en relation est cohérente car les équations équivalentes ont été calculées pour un schéma d'intégration en temps ERK2 et l'expérience a été réalisée avec 
ROCK2 qui est une ERK2 stabilisée.\\
Les résultats précédents sont confirmés:
\begin{enumerate}
    \item Pour les petites CFL, les méthodes avec reconstruction sont moins précisés que les méthodes sans reconstruction.
    \item Il existe une gamme de CFL très précise pour laquelle les schémas avec reconstruction sur-performent les autres méthodes.
\end{enumerate}
Munis de ces nouveaux résultats, une première hypothèse est formulée puis invalidée, avant qu'une seconde ne soit proposée et validée.
\subsubsection{Première Hypothèse}
\textbf{L'observation} suivante a de plus été faite: \textit{il semble que le profil de l'erreur avec reconstruction ressemble à la dérivée spatiale d'ordre quatre
de la solution et que le signe de cette erreur change avec la CFL}.\\
\textbf{Une première hypothèse} a d'abord été proposée: \textit{l'évolution du profil d'erreur selon la CFL s'explique par le terme $\Delta x^{2}\, D \, \Bigl( 
        \frac{\lambda}{2} (2^{2 \Delta l} - 1) + \frac{2^{2 \Delta l} }{12} (1 - 3 \Delta l)
        \Bigr)\frac{\partial^{4}u}{\partial x^{4}}$ dans l'équation équivalente \eqref{eq:equiv_cfl_recons}.}
En effet pour les grandes CFL le coefficient $\frac{\lambda}{2} (2^{2 \Delta l} - 1) + \frac{2^{2 \Delta l} }{12} (1 - 3 \Delta l)$ est positif et pour les petites CFL il est négatif, pour $\lambda = \frac{4^{\Delta l}}{6}\,\frac{3 \Delta l - 1 }{4^{\Delta l}-1} \underbrace{>}_{\Delta l >0} 0$ ce coefficient serait null expliquant 
chute de l'erreur pour ces CFL, l'erreur s'allégeant du terme d'ordre deux.\\
\textbf{Validation expérimentale:} Pour valider cette hypothèse, une régression linéaire entre l'erreur numérique et la dérivée $4^e$ de la solution a été réalisée par 
une méthode des moindres carrées : $$\min_{\alpha \in \mathbb R} \Vert \alpha \partial_x^4 u - \text{err} \Vert^2.$$
Ce modèle (fig. \ref{fig:derive4_vs_err}) explique bien l'erreur pour les petites CFL ($R^2>0.9$) mais mal pour les grandes CFL ($R^2 \sim 0.2$).
\begin{figure}[htpb]
    \centering
    \includegraphics[width=0.75\textwidth]{media/4_travail/3/erreur_vs_deriv4.pdf}
    \caption{Régression entre l'erreur numérique expérimentale (AMR + reconstruction fine) et la dérivées $4^e$ de la solution.}
    \label{fig:derive4_vs_err}
\end{figure}
\newpage
\subsubsection{Seconde Hypothèse}
\textbf{Une seconde observation} a alors révisé la première: \textit{à grande CFL, l'erreur ne ressemble pas à l'opposé de $\partial_x^4 u$ mais à 
$\partial_x^6 u$.}\\
\textbf{Une seconde hypothèse} a alors été émise :
    \begin{itemize}
        \item[$\diamond$] Pour les grandes CFL, le terme $- \Delta x^4 \frac{D}{6} \lambda^2 \partial_x^6 u$, quadratique en $\lambda$ domine dans \eqref{eq:equiv_cfl_recons},
        \item[$\diamond$] Pour les petites CFL, le terme $\Delta x^{2}\, D \, \Bigl(\frac{\lambda}{2} (2^{2 \Delta l} - 1) + \frac{2^{2 \Delta l} }{12} (1 - 3 \Delta l)\Bigr)\frac{\partial^{4}u}{\partial x^{4}}$, affine en $\lambda$ domine.
    \end{itemize}
    Le fait que pour les petites CFL le schéma \texttt{II} (sans reconstruction) soit plus précis que le schéma \texttt{III} (avec reconstruction),
    s'explique simplement par le fait que la constante d'erreur pondérant terme d'erreur dominant $\Delta x^2 \partial_x^4 u$ est plus grand pour le schéma \texttt{III}:
\begin{center}
    \renewcommand{\arraystretch}{1}
    \begin{tabular}{@{}clc@{}}
        \toprule
        \textbf{Schéma n\textsuperscript{o}} & \textbf{Évaluation des flux} &
        \makecell[c]{\textbf{Constante pondérant l'erreur en $\Delta x^{2}\,\partial_{x}^{4}u$}\\
                    \textbf{(dominante quand $\lambda$ est petite)}} \\
        \midrule
        I   & $\varnothing$ AMR          & $\dfrac{D}{12}$ \\[1mm]
        II  & Sans reconstruction         & $2^{\Delta l}\,\dfrac{D}{12}$ \\[1mm]
        III & Avec reconstruction         &
            $D\,\Bigl(\dfrac{\lambda}{2}\,(2^{2\Delta l}-1)
            + \dfrac{2^{2\Delta l}}{12}\,(1-3\Delta l)\Bigr)$ \\[1mm]
        \bottomrule
    \end{tabular}
\end{center}
\textbf{Validation expérimentale: } pour valider empiriquement cette nouvelle hypothèse, l'erreurs a été modélisée par moindre carré comme $\text{err} \approx \alpha \partial_x^4 u + \beta \partial_x^6u$. 
Ce modèle explique très bien l'erreur pour tous les régimes de CFL (voir \ref{fig:derivees_vs_err}).
\begin{figure}[h!]
    \centering
    \includegraphics[width=0.75\textwidth]{media/4_travail/3/derivees_spatiales_VS_err_num.pdf}
    \caption{Régression entre l'erreur numérique expérimentale (AMR + reconstruction fine) et une combinaison linéaire des dérivées $4^e$ et $6^e$ de la solution.}
    \label{fig:derivees_vs_err}
\end{figure}\\
\textbf{Remarque:} la plage de CFL où la méthode avec reconstruction est plus la précise correspond en fait au cas ou les profils de $ \alpha \partial_x^4 u$
de $\beta \partial_x^6 u$ se compensent. C'est donc un comportement tout à fait accidentel lié au fait que les dérives de la courbe de Gauss sont en quelque sort "en opposition de phase".\\
\textbf{Analyse retrospective:} Le comportement observé s'explique par une \emph{incohérence précision} entre le schéma et la reconstruction.
Une prédiction à trois points n'apporte pas d'information supplémentaire par rapport au schéma spatial d'ordre deux :
elle reste du même ordre de précision et ne réduit donc pas le terme d'erreur dominant.
Dans ces conditions, la reconstruction ajoute du bruit au lieu d'améliorer la solution.\\
\textbf{Complément important:} Ce n'est pas présenté ici, mais le travail à été refait avec un prédicteur à 5 points, permettant d'approximer correctement la dérivée d'ordre quatre.
Alors la reconstruction apporte un gain notable, les solutions numériques avec et sans MRA sont dans ce cas très proches. 
        \newpage
        \subsection{Extension sur une équation de diffusion-réaction}     Cette expérience permet d'observer l'impact du niveau de reconstruction des flux quand l'opérateur de diffusion est couplé à un opérateur de diffusion et faisant émerger des dynamiques de couplage.
L'équation de diffusion est donc remplacée par l'équation de Nagumo (voir \ref{par:analyser_operateurs_nagumo}).
Des ondes progressives sont solution de cette équation, le schéma numérique donc doit être en mesure de suivre la dynamique du front d'onde.

\subsubsection{Résumé de l'expérience}
L'expérience à été réalisée dans des conditions similaires à l'expérience numérique \ref{par:contrib_imex}. 
C'est à dire :
\begin{itemize}
    \item[$\diamond$] Profil de l'onde propagative comme état initial.
    \item[$\diamond$] Domaine étendu avec conditions de Neumann aux limites pour limiter les effets de bord
\end{itemize}
L'unique différence majeur est le remplacement de la méthode Runge et Kutta ImEx par un schéma de séparation d'opérateur avec une méthode stabilisée explicite pour la diffusion (ROCK 2 \cite{AbdulleMedovikov2001}).
Ce choix découle, comme précédemment expliqué, de la nécessité d'éviter toute inversion de système linéaire lorsque les flux sont reconstruits à partir de reconstructions
fines.


\subsubsection{Résultats de l'expérience}
Les résultats de l'expérience sont donnés en figure \ref{fig:flux_reconstruction_nagumo}, selon le pas de temps les observation sont:
\begin{itemize}
    \item[$\diamond$] \textbf{Pas de temps "petits" - erreur spatiale dominante:} L'erreur se comporte pareillement au cas de la diffusion pure, plus les flux sont reconstruit finement, plus l'erreur augmente.
    \item[$\diamond$] \textbf{Pas de temps "grands" - erreur temporelle dominante:} L'erreur se comporte encore comme dans le cas de la diffusion pure, la méthode de reconstruction n'induite pas de variations sensibles de l'erreur.
    \item[$\diamond$] \textbf{Pas de temps 'intermédiaires" - erreur spatio-temporelle: } C'est l'observation la plus intéressante: plus les flux sont finement évalués, plus l'erreur chute brutalement. 
            Comme si \textit{la reconstruction fine des flux de diffusion prévenait un couplage des erreurs en temps et en espace}. Plus l'onde est raide, plus l'effondrement de l'erreur est marqué
            ce qui suggère que cette amélioration vient avant tout d'un meilleur suivi du front d'onde. Enfin, quand le pas de temps continue de diminuer, l'erreur remonte comme si 
            une sorte d'instabilité polluait la solution (voir figure AFAIRE). 
\end{itemize}
\begin{figure}[h!]
    \centering
    \includegraphics[width=\textwidth]{media/4_travail/3/flux_reconstruction_method_nagumo.pdf}
    \caption{Courbes de convergence de chaque méthode d'AMR pour différents paramètres de l'équation. Plus $k$ est élevé, plus le profil de l'onde est raide et plus la réaction domine. La célérité de l'onde est néanmoins identique pour chaque jeu de paramètres puisque le produit $kD$ reste constant d'une expérience à l'autre.}
    \label{fig:flux_reconstruction_nagumo}
\end{figure} 
        \newpage
        \subsection{Conclusion}
            Le résultat principal est que, sur le schéma méthode des lignes étudié, une reconstruction systématique des flux n'est bénéfique que si la reconstruction se fait par un
prédicteur polynomial à au moins cinq points. Sinon la reconstruction des flux apporte avant tout du bruit et dégrade la qualité de la solution numérique.
\textbf{Il semble raisonnable de conjecturer que de manière générale, la reconstruction des flux sera positive si elle est capable de capter les dérivées spatiales dominantes dans les termes d'erreur.}
En somme : si les équations équivalentes du schéma non-adapté montrent que l'erreur dominante porte sur une dérivée spatiale d'ordre $k$. 
Alors sur le schéma adapté gagnera à reconstruire les flux au niveau le plus fin uniquement si le prédicteur possède au moins $k+1$ points.
Sinon, il présume d'une précision dont il ne dispose pas et utilise du bruit comme de l'information de qualité. Si le prédicteur 
possède moins de $k+1$ points, alors il est préférable de ne pas reconstruire car alors le schéma dispose d'une 
information fiable de la dérivée $k$ au niveau $l$ de résolution courant de l'adaptation et évite que schéma n'"hallucine" en quelque sorte les 
valeur de la dérivée $k^e$ à un niveau de résolution élevé $l^{\max}$.
\chapter{Conclusion}
    \label{par:cc}
\paragraph*{Conclusion scientifique}\label{par:cc1}
% \subparagraph*{Résumé des travaux -} Une analyse fine de la stabilité 
\subparagraph*{Résultats —}  
Le stage propose une analyse de stabilité et de convergence de deux méthodes ImEx ARK comparées au splitting sur une équation de diffusion-réaction. 
Concernant la stabilité, les méthodes ImEx ont des domaines de stabilité complexes et la raideur d'un opérateur peut influencer la stabilité de la méthode vis-à-vis de l'autre opérateur. 
Bien identifiée, ces propriétés pourraient être exploitées et permettre, dans certains cas, des contraintes de stabilité moins restrictives que les approches par \textit{splitting}. Concernant la
convergence, il a été confirmé expérimentalement que les méthodes ARK, ne présentent pas d'erreur de \emph{splitting}, peuvent avoir une constante de convergence plus faible 
que le \emph{splitting}. En revanche, en \ref{par:couplagetempsadaptation}, on constate expérimentalement qu'une méthode ImEx et un \emph{splitting} de même ordre interagissent différemment avec l'adaptation spatiale.
Sur le cas particulier étudié, le \textit{splitting} d'opérateur semble moins impacté que la méthode ARK (mais il n'est absolument pas certain que cela se généralise).\par

De plus, les équations équivalentes d'une méthode des lignes pour un problème de diffusion pure ont été calculées 
pour différentes stratégies d'adaptation de maillage (avec et sans reconstruction des flux). 
Cela permet de montrer théoriquement que la reconstruction des flux, telle que réalisée dans cette approche, dégrade la qualité des solutions numériques.
Ces équations équivalentes ont été mise en perspectives des expériences numériques qui ont confirmé les prédictions théoriques.
L'analyse mène au résultat principal suivant : la reconstruction des flux au niveau le plus fin dégrade la solution numérique si l'erreur de reconstruction n'est pas négligeable devant l'erreur du schéma initial.

\medskip
\subparagraph*{Perspectives —}  
Ce travail met en évidence l'importance d'une compréhension fine des interactions 
entre la multirésolution adaptative et le schéma numérique pour exploiter pleinement le potentiel de cette méthode.  
Il laisse penser que reconstruction des flux au niveau le plus fin améliorerait la qualité des solutions numériques à la condition que la prédiction polynomiale 
se fasse à partir d'au moins \(k+2\) points, avec \(k\) l'ordre de discrétisation spatiale.
Il faudrait donc reprendre l'analyse théorique (section~\ref{par:contrib_2}) 
et expérimentale (section~\ref{par:contrib_3}) en utilisant un prédicteur à cinq points pour la reconstruction des flux (au lieu d'un prédicteur à trois points ici).
De plus une étude des coûts calculs de chaque approche devrait être réalisée
pour évaluer le ratio coût/bénéfice d'une reconstruction des flux selon le stencil nécessaire.
À plus long terme, une analyse théorique systématique du couplage entre méthodes ImEx et MRA serait nécessaire : les observations présentées en \ref{par:couplagetempsadaptation} suggèrent un comportement non trivial, propre à chaque intégrateur temporel.

\paragraph*{Conclusion sur ma progression technique}\label{par:cc2}
Sur le plan théorique, j'ai approfondi des domaines variés : méthodes de volumes finis \cite{LeVeque1990}, systèmes dynamiques et méthodes de Runge Kutta additives \cite{HairerAndWanner1}, ainsi que la théorie des ondelettes, en résonance avec mes cours antérieurs.
Sur le plan pratique, je me suis familiarisé avec des outils puissants d'analyse (équations équivalentes, analyse de Von Neumann), avec des codes de recherche avancés tels que \texttt{Samurai}, et avec les exigences de la simulation numérique (estimation rigoureuse des erreurs, gestion de grilles de taille différente, instabilités).
J'ai aussi développé mes compétences de programmation scientifique en Python et en \texttt{C++}, ainsi que ma maîtrise de l'environnement Unix (terminal, git, bash).
Enfin, j'ai renforcé mes capacités de communication scientifique, en diffusant mes codes pour en favoriser la reproductibilité et en produisant des graphiques complexes mais lisibles, parfois interactifs.
\paragraph*{Conclusion personnelle}\label{par:cc3}
Ce stage a renforcé mon intérêt pour les mathématiques appliquées, 
en me confrontant à la fois aux exigences théoriques de l'analyse et aux réalités concrètes de la mise en œuvre computationnelle.
Cela a révélé mon épanouissement au sein des environnements où les méthodes mathématiques contribuent directement à résoudre des problématiques complexes, 
qu'elles relèvent de la recherche scientifique ou d'applications technologiques.
Les gains rendus possibles par la diversité de l'équipe, 
la réutilisation des outils développés et leur interopérabilité m'ont fait prendre conscience
de l'importance du travail collectif et de la mise en commun des savoir-faire dans la réussite d'un projet scientifique.
        

\newpage

% Bibliographie
\printbibliography
\addcontentsline{toc}{chapter}{Bibliographie}

% \newpage

% % Annexes
% \appendix
% \chapter*{Annexes}
% \addcontentsline{toc}{chapter}{Annexes}
% \section*{Annexe A : Évaluation numérique des fonctions de stabilité}
% \lstinputlisting{chapitres/9_Annexes/stab_fnc.py}\label{annexe:A}


% Contenu de la première annexe.

% \section{Annexe B : Titre de l'annexe}
% Contenu de la deuxième annexe.

\end{document}