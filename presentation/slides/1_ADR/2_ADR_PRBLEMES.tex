\begin{frame}{Difficultés intrinsèques}{Équations d'ADR}
% \begin{columns}[c,onlytextwidth]

    % ---- Colonne 1 : Opérateurs ----
    % \column{0.67\textwidth}
    \noindent\color{Primary}\rule{\linewidth}{0.6pt}\color{black}\\
    % ---- Colonne 2 : Multi-échelle ----
    % \column{0.30\textwidth}
    \textbf{Des solutions multi-échelles}\\[0.5em]
    \begin{itemize}
        \item plusieurs échelles de temps :\\
        \begin{itemize}
            \item chimie complexe $\tau \sim 50\mathrm{ns}$,\\
            \item advection $\mathrm{Mach }2$ sur $50\mathrm{cm}$ : $\tau \sim 10\mathrm{ms}$.
        \end{itemize}
        \item plusieurs échelles d'espace
    \end{itemize}\pause
    \noindent\color{Primary}\rule{\linewidth}{0.6pt}\color{black}\\
    \textbf{L'intégration conjointe des opérateurs}\\[0.5em]
    Les trois opérateurs ont des propriétés très différentes :
    \begin{itemize}
        \item Advection : Peu raide, raisonnant (spectre autour de $i\mathbb R$).
        \item Diffusion : Moyennement raide, spectre autour de $\mathbb R^-$.
        \item Réaction :  Très raide, hautement non-linéaire, local.
    \end{itemize}
    $\implies$ les approches monolithiques peinent.
\end{frame}
