\begin{frame}{Conclusion}{Contribution 1 | Comparaison ImEx - Splitting}
    \noindent\color{Primary}\rule{\linewidth}{0.6pt}\color{black}\\
    \textbf{Apprentissage personnel : }\\
    \begin{itemize}
        \item Étude de la théorie des ImEx \cite{Hairer1981,ASCHER1997151,KENNEDY2003139} et des schémas pour lois de conservation \cite{LeVeque1990}.
        \item Familiarisation avec l'étude de stabilité linéaire pour les ImEx ($(z_E,z_I) \in \mathbb{C}^2$).
        \item Code par volume fini en C++ grâce à la libraire Samurai.
    \end{itemize}
    \noindent\color{Primary}\rule{\linewidth}{0.6pt}\color{black}\\
    \textbf{Résultat empirique}\\
        Le schéma de \emph{splitting} est plus robuste à la \emph{multirésolution adaptative} que les ImEx.\\
        \scriptsize{Cela n'est pas forcément généralisale, c'est un cas particulier.}
\end{frame}