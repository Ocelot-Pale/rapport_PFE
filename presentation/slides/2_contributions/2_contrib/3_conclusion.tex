\begin{frame}{Contribution 2 | Équations Équivalentes, diffusion et MRA}{Conclusion}
    \textbf{Apprentissage personnel : }\\
    \begin{itemize}
        \item Calcul formel grâce à la libraire Sympy.
        \item Développement de la compréhension de l'algorithme d'adaptation.
        \item Usage des \emph{équations équivalentes} comme puissant outil d'analyse, dans la continuité de l'équipe hpc@maths \cite{Massot2025_meshAdaptation,belloti_et_al_2025}.
    \end{itemize}
    \noindent\color{Primary}\rule{\linewidth}{0.6pt}\color{black}\\
    \textbf{Résultats}\\
        \begin{itemize}
            \item \textbf{Développement d'équations équivalente} pour un schéma diffusif avec plusieurs types d'adaptation
            permettant une mise en lumière de l'erreur pour chaque contexte.\\
            \scriptsize{(sans MRA, MRA sans reconstruction des flux et MRA avec reconstruction des flux)}
            \item \textbf{La reconstruction semble ajouter plus de termes d'erreurs}\\\scriptsize{+ potentiellement perte d'ordre (à confirmer).}
        \end{itemize}
\end{frame}