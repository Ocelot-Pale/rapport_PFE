\begin{frame}{Conclusion}{Contribution 3 | Étude numérique, diffusion et MRA}

    \textbf{Apprentissage personnel : }\\
    \begin{itemize}
        \item Mise en place du calcul des flux au niveaux le plus fin grâce à l'équipe Samurai.
        \item Utilisation capacité de Ponio pour l'utilisation de la méthode stabilisée. 
    \end{itemize}
    \noindent\color{Primary}\rule{\linewidth}{0.6pt}\color{black}\\
    \textbf{Résultats}\\
        \begin{itemize}
        \item Mise en relation des résultats numériques et du travail sur les équations équivalentes.
        \item La reconstruction semble ici poser problème : prédicteur polynomial précis à l'ordre 3, permet une évaluation des 
            gradients à l'ordre 2. Or le schéma est d'ordre 2, donc erreur de prédiction du même ordre et s'accumule.
        \item \underline{Conjecture} il faut que le prédicteur permette d'évaluer les flux de sorte à ce que 
        l'erreur de reconstruction portant sur les flux soit négligeable devant l'ordre du schéma. 
        \end{itemize}
\end{frame}