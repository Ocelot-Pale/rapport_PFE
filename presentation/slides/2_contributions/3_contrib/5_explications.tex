\begin{frame}{Explications}{Contribution 3 | Étude numérique, diffusion et MRA}
    \noindent\color{Primary}\rule{\linewidth}{0.6pt}\color{black}\\
        \textbf{Chute prématurée de l'erreur: }\\
        Les dérivées $\partial_x^{4}$ et  $\partial_x^{6}$ ont des poids différents dans l’erreur selon la constante de Von Neumann $\lambda$.
        Leurs profils se "compensent" quand elles ont un poids comparable.
    \noindent\color{Primary}\rule{\linewidth}{0.6pt}\color{black}\\
        \textbf{Moins bonne performances quand l'erreur temporelle est faible}\\
            Plus de termes d'erreur dans la contribution dominante de l'erreur : 
            \begin{center}
                \renewcommand{\arraystretch}{1}
                \begin{tabular}{@{}clc@{}}
                    \toprule
                    \textbf{Schéma n\textsuperscript{o}} & \textbf{Évaluation des flux} &
                    \makecell[c]{\textbf{Constante pondérant l'erreur en $\Delta x^{2}\,\partial_{x}^{4}u$}\\
                                \textbf{(dominante quand $\lambda$ est petite)}} \\
                    \midrule
                    I   & $\varnothing$ AMR          & $\dfrac{D}{12}$ \\[1mm]
                    II  & Sans reconstruction         & $2^{\Delta l}\,\dfrac{D}{12}$ \\[1mm]
                    III & Avec reconstruction         &
                        $D\,\Bigl(\dfrac{\lambda}{2}\,(2^{2\Delta l}-1)
                        + \dfrac{2^{2\Delta l}}{12}\,(1-3\Delta l)\Bigr)$ \\[1mm]
                    \bottomrule
                \end{tabular}
            \end{center}
\end{frame}