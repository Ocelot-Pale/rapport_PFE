\documentclass[11pt,a4paper]{article}

% Packages essentiels
\usepackage[utf8]{inputenc}
\usepackage[french]{babel}
\usepackage[T1]{fontenc}
\usepackage{textcomp} % pour certains symboles
\usepackage{helvet} % Police Arial-like
\renewcommand{\familydefault}{\sfdefault}

% Mise en page
\usepackage[top=2.5cm, bottom=2.5cm, left=2.5cm, right=2.5cm]{geometry}
\usepackage{setspace}
\singlespacing

% Packages utiles
\usepackage{graphicx}
\usepackage{hyperref}
\usepackage{apacite} % Citations APA
\usepackage{fancyhdr}
\usepackage{titlesec}
\usepackage{enumitem}
\usepackage{csquotes}

% Configuration des en-têtes et pieds de page
\pagestyle{fancy}
\fancyhf{}
\fancyhead[L]{\textit{Réflexion TES - PFE 2025}}
\fancyhead[R]{\textit{\nom\ \prenom}}
\fancyfoot[C]{\thepage}

% Configuration des titres
\titleformat{\section}
  {\normalfont\large\bfseries}
  {\thesection}{1em}{}
\titleformat{\subsection}
  {\normalfont\normalsize\bfseries}
  {\thesubsection}{1em}{}

% Configuration des liens
\hypersetup{
    colorlinks=true,
    linkcolor=black,
    filecolor=magenta,      
    urlcolor=blue,
}

% Variables à personnaliser
\newcommand{\nom}{EDELINE}
\newcommand{\prenom}{Alexandre}
\newcommand{\formation}{ENSTA Paris - Promotion 2025}
\newcommand{\annee}{2024-2025}
\newcommand{\entreprise}{CMAP - Équipe HPC}
\newcommand{\titreMission}{Étude de méthodes ImEx et méthodes des lignes pour la multi-résolution adaptative}

\begin{document}

% Page de garde
\begin{titlepage}
    \centering
    \vspace*{2cm}
    
    {\LARGE \textbf{Réflexion sur les enjeux de transition écologique et sociale}}\\[0.5cm]
    {\Large \textbf{dans le cadre du Projet de Fin d'Études}}\\[2cm]
    
    {\large Mission : \textit{\titreMission}}\\[0.5cm]
    {\large laboratoires : \textbf{\entreprise}}\\[2cm]
    
    {\large \textbf{\prenom\ \nom}}\\[0.5cm]
    {\large \formation}\\[0.5cm]
    {\large Année universitaire \annee}\\[4cm]
    
\end{titlepage}

\newpage

% Table des matières (optionnelle pour un document court)
% \tableofcontents
% \newpage

\section*{Introduction.}
  Ce document atteste de ma prise en compte des enjeux de la transition écologique et sociale lors de mon entrée dans le monde professionnel au travers de mon projet de fin d'étude.
\section{Constat.}
  \subsection{Description de la mission.}
  Mon projet de fin d'étude à pris place au Centre de Mathématiques Appliquées de l'école Polytechnique (CMAP). 
  J'ai travaillé dans un contexte HPC (calcul haute performance) sur des méthodes numériques pour simuler efficacement les réaction d'advection-diffusion-réaction (ADR).
  Ces équations décrivent les systèmes physiques où interagissent la mécanique des fluides et des réactions chimiques.
  Les enjeux écologiques sont alors évidents car il s'agit de simuler la combustion au sein d'un moteur d'avion, 
  le comportement du pétrole dans le sous-sol, mais aussi les systèmes innovants autour de l'hydrogènes ou des batteries...
  \subsection{Enjeux et impacts}
    \paragraph{Pan écologique.}
      L'analyse porte sur trois thématiques, les secteur applications, les activités de calcul haute performance (HPC) et les pratiques de la recherche.
      \begin{itemize}
      \item[$\diamond$]\textbf{Applications des équations d'ADR:} la recherche sur les équation d'ADR est au centre des enjeux écologiques actuels, en effet la discipline est d'abord 
      liée à des secteurs responsables d'une part majeure des émissions de gaz à effets de serres et de polluants puisqu'elle sert traditionnellement
      les besoins des industries pétrolières, automobiles et aéronautiques. 
      Cependant les liens écologique de la discipline ne se limitent pas à des secteurs parfois qualifiés d'écocides puisqu’elle résolument tournée vers leur avenir en transition. 
      En effet les réactions d'advection-diffusion-réaction permettent également de simuler le comportement des accumulateurs électrochimiques (batteries) et
      d'éprouver la sécurité des installations nucléaires étant jugés par de nombreux observateurs comme des technologies essentielles à une électrification des usages grâce à un réseau électrique intégrant les énergies renouvelables.
      De même les simulations sur les combustion autre que celles des hydrocarbures traditionnels comme celle des \textit{neo-fuels} pour la mobilité ou de l'hydrogène pour l'industrie lourde
      parachève d'en faire des équations centrales de la recherche et développement pour la transition écologique.
      \item[$\diamond$]\textbf{Les activités de HPC:} La consommation énergétique des activités numériques modernes représente environ [REF] . Avec le développement exponentiel
      des applications grand publique des algorithmes d'intelligence artificiels cette part n'a de cesse de croître et est prévue pour atteindre [REF] d'ici [REF] .
      Or mon stage est en lien directe cette consommation, car si le sujet d'étude sont les équations d'ADR, le paradigme numérique utilisé est le calcul haute performance. 
      Cela désigne l'usage optimal des ressources de calculs disponibles ainsi que des architectures existantes. Cette discipline peut être mise au service d'une amélioration des performances
      mais elle peut également être mise au service d'une réduction des ressources de calcul utilisées ainsi que l'intensité énergétiques. Ainsi 
      le HPC est également au centre de la transition écologiques puisqu'il permet théoriquement une réduction de l'emprunte énergétique de nos activités numériques bien que
      l'apparition quasi systématique d'un effet rebond (les gains d'efficacité se traduisent par un accroissement de la consommation et non par des économies d'énergies).
      Enfin, les HPC est une clé dans de nombreux domaines d'innovation, les simulations physiques, l'aide à la synthèse de nouvelles molécules,
      la mise au points de procédés plus économes dans l'industrie ou encore une meilleure allocation des ressources de mobilité, de logistique.
      \item[$\diamond$]\textbf{Les pratiques de la recherche en mathématiques:} au sein de la communauté de recherche en mathématiques appliquées,
      si l'on exclue la question des applications (précédemment traitée), le principal enjeux est celui des déplacements par avions pour assister à des conférences scientifiques.
      En effet le transport aérien est résolument un mode de transport hautement émetteur en gaz à effets de serre, puisqu'un aller-retour Paris-Tokyo emmets $4.0t$ de $CO_2$, 
      soit 2 fois plus que le budget carbone annuel de $2.0t$ par habitant proposé par l'état au travers de la stratégie nationale bas carbone (Ministère de la transition, SNBC, 2025).
      \end{itemize}
    \paragraph{Pan social.}
      Sur le plan social, la recherche française est traditionnellement jugée comme un milieu ouvert, marqué par de fortes collaborations 
      internationales où la recherche de savoir prime souvent sur les biais culturels. La question de l'intégration interne 
      me semble secondaire et devrait le devenir encore davantage avec les nouvelles générations. 
      En revanche, l'accès au monde scientifique dès le plus jeune âge reste un enjeu majeur : les inégalités scolaires et sociales 
      limitent encore la diversité des parcours menant à la recherche. En effet l'accès aux études supérieures scientifiques reste très 
      corrélé au terreau socio-économique des individus (Bonneau et Grobon, 2022).
      Un autre point sensible est la situation des jeunes chercheurs (doctorants, post-doctorants), dont les contrats courts 
      et l'absence de CDI rendent l’accès au logement et à la stabilité de vie plus difficile, même si leur rémunération n’est pas 
      faible en valeur absolue. Cela pose un problème d’attractivité et de pérennité des carrières scientifiques, alors même que 
      la société a besoin de compétences solides pour relever les défis de la transition écologique et sociale.
\section{Blocages} 
      Mon analyse se poursuit en étudiant les points bloquants la transition écologique et sociale sur les thématiques précédemment évoquées.
      \begin{itemize}
        \item[$\diamond$]\textbf{Les applications rentables sont celles qui financent: }
          Ce qui limite principalement la réorientation des activités de simulations des industries "polluantes" vers les industries et usages "verts"
          est le financement. Cela est lié au défis sous-jacent de la transition, à son essence même: les activités dites vertes ont ujourdhui un ROI bien plus faible
          que les activités carbonées. Ce document est trop court pour en détailler les raisons (thermodynamique, carbon lock-in (Unruh, G. C. 2000)...) mais le résultat 
          est que sans \textit{incentives} faussant le marché (subventions, taxes, régulation...) et sans vision long terme dans la gestion des entreprises et de leur innovation, le tout en bonne intelligence avec organismes de recherches publiques (CEA, CNRS, ANR...), 
          la logique financière occidentale (rendement à court terme) peinera à faire financer des travaux alignés avec les enjeux écologiques, là où des visions long termes comme les Keiretsu Japonaises
          ou une volonté politique ferme et stable accompagnée d'investissements publics ciblés, d'ampleurs et contrôlés (développement nucléaire en France d'après-guerre, subventions chinoises dans le secteur ENR et nucléaire...) 
          semblent plus propices à l’émergence de financements de recherches alignés avec la transition écologique.
        \item[$\diamond$]\textbf{Le HPC, l'effet rebond domine :}
          L’amélioration de l’efficacité énergétique ou algorithmique dans le calcul haute performance 
          s’accompagne presque systématiquement d’un effet rebond : les gains réalisés ne se traduisent pas par une 
          baisse absolue de consommation mais par une massification des usages (plus de simulations, plus de données traitées, 
          délais raccourcis mais budgets énergétiques globaux constants ou croissants). 
          De plus, rien ne garantit que le développement futur du HPC soit orienté vers des usages écologiquement bénéfiques. 
          Au contraire, la logique de retour sur investissement pousse à financer en priorité des applications à forte valeur 
          stratégique ou économique — la simulation militaire, la finance à haute fréquence, l’ingénierie sociale ou les modèles 
          d’intelligence artificielle générative — dont la contribution nette à la transition écologique est loin d’être évidente. 
          Ainsi, l’effet rebond et la logique de rentabilité court terme constituent un double blocage majeur pour aligner le HPC 
          avec les objectifs de durabilité.
        \item[$\diamond$]\textbf{Les voyages pour les \textit{confs.}, au centre de la culture des chercheurs :}
          Les déplacements internationaux font partie intégrante du statut social et symbolique du chercheur. 
          Accepter une rémunération inférieure à celle du privé est, pour certains, compensé par la possibilité de voyager, 
          de rencontrer des collègues du monde entier et de participer à une communauté scientifique globale. 
          Cette pratique est ancrée dans la culture académique et la supprimer totalement pourrait paraître déconnecté.
          De plus l’absence physique à des des collaborations stratégiques avec des partenaires situés dans des 
          pays moins "écolo-friendly" pourrait ternir l'image relationnelle de la recherche Française dans les pays concernés. 
          Par exemple, mon équipe s’est rendue aux États-Unis pour une conférence organisée avec la NASA : un cadre d’échanges qui 
          aurait été difficile à reproduire à distance sans pertes scientifiques et relationnelles. 
          Le blocage tient donc au rôle identitaire et professionnel que représentent ces voyages.
      \end{itemize}
  \section{Leviers}
  Au regard des blocages identifiés, plusieurs leviers peuvent être envisagés afin de mieux intégrer les enjeux de transition 
  écologique et sociale dans les activités de recherche et d’innovation.

  \begin{itemize}
    \item[$\diamond$]\textbf{Réorienter les financements par la régulation :} 
    Pour orienter les financements verts de applications "vertes", un levier me semble être appliquer de manière effective les réglementations environnementales et sociales, 
    en particulier dans les filières d’extraction et d’approvisionnement (cobalt en RDC, cuivre au Pérou,le pétrole au Cameroun etc.). 
    Aujourd’hui, l’absence de contrôle strict permet à ces activités de rester extrêmement rentables malgré des impacts 
    écologiques et humains considérables. Une loi ferme et mondiale, sans échappatoires, conduirait mécaniquement, par la logique 
    capitaliste habituelle, les entreprises à investir davantage en recherche et développement pour trouver des alternatives plus durables. 
    Je reste très sceptique sur l’efficacité des subventions massives comme outil principal, car elles génèrent souvent des effets d’aubaine 
    et des risques de corruption. En revanche, des investissements publics stratégiques, s’ils sont stables, contrôlés et ciblés 
    (par exemple via des commandes industrielles sur des infrastructures ou technologies précises), peuvent orienter efficacement l’innovation. 
    Une réglementation claire, avec moins d’exceptions et davantage de responsabilité pour les acteurs financiers, 
    limiterait le capitalisme de connivence et la logique de court terme. Elle favoriserait une planification plus durable, où la solvabilité 
    à long terme d’une entreprise dépendrait réellement de la soutenabilité effective de ses activités, plutôt que de manipulations politiques, 
    de lobbying ou d’habileté réglementaire.
    \item[$\diamond$]\textbf{Limiter l’effet rebond numérique et HPC :} 
    Chaque gain d’efficacité entraîne un élargissement des usages 
    plutôt qu’une réduction absolue. Pour y faire face, il faudrait combiner deux leviers : d’une part, une meilleure éducation 
    à la sobriété numérique, afin que les usagers comme les décideurs soient conscients du coût énergétique caché des usages 
    (streaming, IA générative, simulations HPC). D’autre part, le développement d’outils de mesure standardisés de l’empreinte carbone 
    des calculs (au niveau des centres de données, mais aussi des projets de recherche financés) permettrait de fixer des plafonds 
    et d’introduire une forme de comptabilité énergétique obligatoire. Enfin, des ruptures d’ordre de grandeur sont possibles 
    par l’innovation technologique : par exemple, certains modèles d’IA (Gemini) sont beaucoup moins coûteux en entraînement 
    que GPT-4 pour des performances proches, car ici, la logique capitaliste s'applique tout de même, moins de calculs, moins de coût énergétique, meilleure rentabilité.
    \item[$\diamond$]\textbf{Réinventer les conférences scientifiques :} 
    Les déplacements internationaux sont un point sensible car ils appartiennent à la culture et au statut social du chercheur. 
    Les supprimer totalement paraît illusoire, mais plusieurs leviers existent. 
    D’abord, développer à l’échelle européenne un outil numérique robuste et pérenne, conçu pour les besoins académiques 
    (interactions en direct, sessions hybrides, outils collaboratifs), pourrait créer un standard régional crédible et ensuite 
    s’imposer comme modèle mondial. Ensuite, la reconnaissance institutionnelle des présentations à distance doit être renforcée, 
    pour que les jeunes chercheurs n’aient pas à choisir entre visibilité scientifique et sobriété carbone. 
    Enfin, il semble que les nouvelles générations de chercheurs soient déjà plus sensibles aux enjeux climatiques : 
    une dynamique d’auto-limitation volontaire est donc probable, au moins en Europe, et pourrait être accompagnée 
    par des incitations douces (budgets voyages plafonnés, priorité aux collaborations locales).

    \item[$\diamond$]\textbf{Éducation et accès équitable aux carrières scientifiques :} 
    La transition écologique et sociale exige un vivier diversifié de chercheurs et d’ingénieurs. Or l’accès aux filières scientifiques 
    reste aujourd’hui très corrélé au milieu socio-économique. Dans certaines zones, le passage par un lycée privé est quasiment 
    devenu obligatoire pour espérer des études supérieures exigeantes, ce qui crée une inégalité structurelle.
    Une politique éducative ambitieuse devrait viser à réduire la taille des classes, relever les standards dans les écoles publiques 
    et rendre géographiquement plus homogène le niveau de formation. 
    Elle devrait aussi assumer un certain rôle de « propagande positive » : rendre socialement valorisant,
    le fait de réussir à l’école et de s’orienter vers des filières scientifiques. 
    Enfin, plus d’interactions directes avec les lycéens seraient bienvenues — visites de laboratoires, rencontres avec des chercheurs, 
    ateliers pratiques — sans se limiter aux élèves déjà sensibilisés (souvent enfants d’enseignants ou de scientifiques). 
    Des initiatives comme les « cordées de la réussite » ont montré l’efficacité de ces approches pour ouvrir des perspectives 
    nouvelles à des élèves qui n’auraient autrement jamais envisagé ce type de parcours.
  \end{itemize}
\section{Réflexions personnelles}
Ce travail de réflexion m’a surtout permis de clarifier l’ambivalence de mon domaine d’étude. 
D’un côté, les méthodes numériques et le calcul haute performance sont indispensables pour simuler 
des systèmes physiques complexes et accélérer des innovations nécessaires à la transition énergétique 
(hydrogène, batteries, sûreté nucléaire, procédés industriels sobres). 
De l’autre, ces mêmes outils peuvent tout aussi bien être mobilisés pour prolonger l’exploitation 
des hydrocarbures ou pour des applications éloignées de tout objectif écologique (finance, armement, 
marketing de masse). Cette dualité m’a marqué : la technologie en elle-même n’est ni “bonne” ni “mauvaise”, 
tout dépend de l’orientation que lui donnent les financements et les institutions.

J’ai également pris conscience du poids énergétique du numérique. Avant ce stage, je considérais le HPC 
surtout comme une question de performance scientifique. Désormais, je l’associe aussi à un coût écologique réel.
Je continue à aimer la HPC car je trouve cela beau de faire plus avec moins, que ce soit dans une logique de rentabilité 
ou d'écologie.

Enfin, cette réflexion m’a conduit à m’interroger sur le rôle social du chercheur et de l’ingénieur. 
La recherche doit rester ouverte et attractive, mais elle doit aussi se réformer pour être plus accessible. 
À titre personnel, je retiens qu’il n’est pas nécessaire d’opposer radicalement exigence scientifique et 
préoccupation écologique : l’un des défis de ma génération d’ingénieurs sera justement de faire coïncider les deux. 
Cela influence déjà ma vision professionnelle : j’aimerais mettre mes compétences en mathématiques appliquées 
au service de projets où la performance numérique est indissociable d’une exigence de durabilité.

Enfin, puisqu’on me demande d’être sincère, je dois ajouter que je n’apprécie pas ce type d’exercice. 
Il donne parfois le sentiment qu’il s’agit avant tout de vérifier que l’étudiant sait dépasser la technique, 
ce qui peut sembler un peu infantilisant. 
Par ailleurs, nous faire disserter dans un cadre académique sur des questions qui relèvent aussi du champ politique 
peut paraître discutable : il existe déjà des lieux d’action et de réflexion collectifs (associations, syndicats, institutions démocratiques) 
qui ont pour rôle d’aborder ces débats. 
Néanmoins, cet exercice m’a tout de même permis de formaliser certains constats que je n’aurais sans doute pas pris le temps d’écrire, 
et d’enrichir ma manière d’articuler mes préoccupations scientifiques et écologiques.


        \end{document}
