\label{par:cc}
\paragraph*{Conclusion scientifique}\label{par:cc1}
% \subparagraph*{Résumé des travaux -} Une analyse fine de la stabilité 
\subparagraph*{Résultats —}  
Le stage propose une analyse de stabilité et de convergence de deux méthode ImEx ARK comparées au splitting sur une équation de diffusion-réaction. 
Concernant la stabilité, les méthodes ImEx ont des domaines de stabilités complexes et la raideur d'un opérateur peut influencer la stabilité de la méthode vis à vis de l'autre opérateur. 
Bien identifié, ces propriétés pourraient être exploitées et permettre sur certains cas des contraintes de stabilités moins restrictives que les approches par \textit{splitting}. Concernant la
convergence, il a été confirmé expérimentalement que les méthodes ARK, ne présentation pas d'erreur de \emph{splitting}, peuvent avoir une constante de convergence plus faible 
que le \emph{splitting}. En revanche en \ref{par:couplagetempsadaptation}, on constate expérimentalement qu'une méthode ImEx et un \emph{splitting} de même ordre interagissent différemment avec l'adaptation spatiale.
Sur le cas particulier étudié, le \textit{splitting} d'opérateur semble moins impacté que la méthode ARK (mais il n'est absolument pas certain que cela se généralise).\par

De plus, les équations équivalentes d'une méthode des lignes pour un problème de diffusion pure ont été calculées 
pour différentes stratégies d'adaptation de maillage (avec et sans reconstruction des flux). 
Cela permet de montrer théoriquement que la reconstruction des flux, telle que réalisée dans cette approche, dégrade la qualité des solutions numériques.
Les équations équivalentes ont été mis en perspectives des expériences numériques qui ont confirmé les prédictions théoriques.
L'analyse mène au résultat principal étant que : la reconstruction des flux au niveau le plus fin dégrade la solution numérique si l'erreur de reconstruction n'est pas négligeable devant l'erreur du schéma initial.

\medskip
\subparagraph*{Perspectives —}  
Ce travail met en évidence l'importance d'une compréhension fine des interactions 
entre la multirésolution adaptative et le schéma numérique pour exploiter pleinement le potentiel de cette méthode.  
Il laisse penser que reconstruction des flux au niveau le plus fin améliorerait la qualité des solutions numériques à la condition que la prédiction polynomiale 
se fasse à partir d'au moins \(k+2\) points, avec \(k\) l'ordre de discrétisation spatiale.
Il faudrait donc reprendre l'analyse théorique (section~\ref{par:contrib_2}) 
et expérimentale (section~\ref{par:contrib_3}) en utilisant un prédicteur à cinq points pour la reconstruction des flux (au lieu d'un prédicteur à trois points ici).
De plus une étude des coûts calculs de chaque approche devrait être réalisée
pour évaluer le ratio coût/bénéfice d'une reconstruction des flux selon le stencil nécessaire.
À plus long terme, une analyse théorique systématique du couplage entre méthodes ImEx et MRA serait nécessaire : les observations présentées en \ref{par:couplagetempsadaptation} suggèrent un comportement non trivial, propre à chaque intégrateur temporel.

\paragraph*{Conclusion sur ma progression technique}\label{par:cc2}
Sur le plan théorique, j'ai approfondi des domaines variés : méthodes de volumes finis \cite{LeVeque1990}, systèmes dynamiques et méthodes de Runge Kutta additives \cite{HairerAndWanner1}, ainsi que la théorie des ondelettes, en résonance avec mes cours antérieurs.
Sur le plan pratique, je me suis familiarisé avec des outils puissants d'analyse (équations équivalentes, analyse de Von Neumann), avec des codes de recherche avancés tels que \texttt{Samurai}, et avec les exigences de la simulation numérique (estimation rigoureuse des erreurs, gestion de grilles de taille différente, instabilités).
J'ai aussi développé mes compétences de programmation scientifique en Python et en \texttt{C++}, ainsi que ma maîtrise de l'environnement Unix (terminal, git, bash).
Enfin, j'ai renforcé mes capacités de communication scientifique, en diffusant mes codes pour en favoriser la reproductibilité et en produisant des graphiques complexes mais lisibles, parfois interactifs.
\paragraph*{Conclusion personnelle}\label{par:cc3}
Ce stage a renforcé mon intérêt pour les mathématiques appliquées, 
en me confrontant à la fois aux exigences théoriques de l'analyse et aux réalités concrètes de la mise en œuvre computationnelle.
Cela à révélé mon épanouissement au sein des environnements où les méthodes mathématiques contribuent directement à résoudre des problématiques complexes, 
qu'elles relèvent de la recherche scientifique ou d'applications technologiques.
Les gains rendus possibles par la diversité de l'équipe, 
la réutilisation des outils développés et leur interopérabilité m'ont fait prendre conscience
de l'importance du travail collectif et de la mise en commun des savoir-faire dans la réussite d'un projet scientifique.