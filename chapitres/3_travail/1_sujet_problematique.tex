Ce travail participe à l'élaboration de méthodes numériques pour l'approximation des équations aux dérivées partielles d'évolution.
En particulier les équations d'advection-diffusion-réaction (présentation en \ref{par:adv-diff-reaction}). Elles décrivent par exemple les systèmes physiques couplant
mécanique des fluides, thermodynamique et réactions chimiques\footnote{Typiquement des problèmes de combustion.}.
Ces équations sont difficiles à simuler du fait de leur caractère multi-échelle
\footnote{Une réaction chimique a des temps et distances typiques généralement plusieurs ordres de grandeur plus faibles que les temps et distances typiques de la mécanique des fluides.}.
Pour gérer les différentes échelles spatiales, des méthodes de compression de maillage sont souvent mises en oeuvre. 
La méthode de compression utilisée et étudiée ici est la multirésolution adaptative \cite{harten1994}.
Les différentes échelles temporelles\footnote{En termes techniques, les différents termes des équations étudiées ont des raideurs très différentes.}
sont usuellement gérées par force brute ou par séparation d'opérateurs. 
Pour pallier le problème de la large gamme d'échelles temporelles rencontrées, une approche hybride est ici étudiée: les méthodes implicites-explicites (ImEx) \cite{ASCHER1997151}.
Ce travail vise donc principalement à comprendre comment la multirésolution adaptative interagit avec les différentes méthodes d'intégration temporelle.
Il s'intéresse aux questions suivantes:
\begin{itemize}
\item[$\diamond$] {Comment les effets de la compression de maillage par multirésolution adaptative (MRA) sur les solutions numériques
dépendent-ils du problème étudié et de la méthode numérique sur lesquelles elle se greffent ?}
\item[$\diamond$] {Comment évoluent les propriétés des méthodes ImEx selon les caractéristiques des opérateurs des équations de diffusion-réaction}
                \footnote{Même si l'objectif est bien les équations d'advection-diffusion-réaction, l'étude s'est concentrée par simplicité sur l'interaction entre phénomènes de diffusion et de réactions.}
                { ?}
\end{itemize}