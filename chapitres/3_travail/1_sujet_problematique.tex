Mon travail participe à l'élaboration de méthodes numériques pour l'approximations des équations au dérivées partielles d'évolution.
En particulier, j'ai travaillé sur les équations d'advections-diffusion-réaction. Elles représentent typiquement des systèmes physiques couplant
mécanique des fluides, thermodynamique et réactions chimiques\footnote{Typiquement des problèmes de combustion.}.
Ces équations sont difficiles à simuler du fait de leur caractère multi-échelle
\footnote{Une réaction chimique a des temps et distnaces typiques généralement plusieurs ordres de grandeurs plus faibles que les temps et distances typiques de la mécanique des fluides.}.
Pour gérer les différentes échelles spatiales, des méthodes de compression de maillage ont été mises en oeuvre. 
La méthode de compression utilisée ici est la multirésolution adaptative, 
Les différentes échelles temporelles\footnote{En terme technique, les différents termes des équations étudiées ont des raideurs très différentes.} sont 
usuellement gérées par force brute ou par séparation d'opérateurs. Ici nous allons égalmeent étudier des méthodes hybrides: les méthodes implcites-explicites (ImEx).
Mon travaille vise principalement à comprendre comment s'agence la multirésolution adaptative avec les différentes méthodes d'intégrations temporelles.