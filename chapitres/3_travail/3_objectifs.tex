Mon premier objectif a été de comparer des performances des méthodes ImEx avec une méthode de séparation d'opérateur.
J'ai mis en lumière leurs différents domaines de stabilité et réalisé des tests de convergence. Cette étude a été réalisée sur une équation simple mais très représentative des équations de réaction-diffusion,
l'équation de Nagumo.\par%AFAIRE : AJOUTER UNE REF
% EN FAIT J AI TROUVE UN RESULTAT BIZARRE ET IL M ONT PAS AIDE A DEBUGGER NI A L ANALYSER JE SUIS UN PEU DEMUNI JE SAIS PAS SI C'EST MON BUG, UN BUG SAMURAI OU UN VRAI RESULTAT...
Mon second objectif était d'explorer l'impact théorique de la multi-résolution adaptative (MRA) sur un schéma numérique classique.
J'ai mené une analyse d'erreur sur une équation de diffusion résolue par une méthode des lignes d'ordre deux, et j'ai découvert % AFAIRE: VALIDER CE RESULTAT PARCEQUE C'EST BIZARRE...
que la multirésolution-adaptative peut dégarder l'ordre de convergence d'une méthode. J'ai par la suite tenté (sans succès) de mettre en 
évidence ce résultat expérimentalement grâce au code Samurai. J'ai chosit de travailler sur la diffusion pour compléter le travail 
réalisé par l'équipe en \cite{belloti_et_al_2025} qui se centrait sur la diffusion.