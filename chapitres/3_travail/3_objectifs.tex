Ce stage vise à évaluer des stratégies numériques modernes pour les équations d'advection-diffusion-réaction (ADR).
Ces stratégies cherchent à contrecarrer les difficultés majeures des équations d'ADR: \textit{des opérateurs de raideurs multiples} et \textit{une ample variété d'échelles spatiales} (voir \ref{par:adv-diff-reaction}).
\subparagraph{Premier objectif: éprouver les méthodes ImEx comme alternative au \textit{splitting} pour la gestion des raideurs multiples}
    Le \textit{splitting} est une méthode très populaire pour intégrer des opérateurs de raideurs différentes. 
    Cependant des méthodes alternatives dites implicites-explicites (ImEx) présentent des avantages tangibles, comme une meilleure montée en ordre et 
    une gestion intrinsèque du couplage des erreurs.
    Ces méthodes ImEx ont donc été comparées sur un cas particulier; sur le pan théorique (domaine de stabilité) et expérimental (étude de convergence).
\subparagraph{Second objectif: étude de l'erreur apportée par la multi-résolution adaptative lorsque utilisée pour gérer les différentes échelles spatiales}
    La multi-résolution adaptative est très étudiée dans la communauté scientifique et par l'équipe comme outils pour palier le caractère multi-échelles des solutions des équations d'ADR.
    Une analyse d'erreur théorique a été conduite sur une équation de diffusion résolue par une méthode numérique usuelle à laquelle s'ajoute une étape de multi-résolution.
    La décision de travailler sur l'équation de diffusion a été prise pour compléter le travail réalisé par l'équipe en \cite{belloti_et_al_2025} qui se centrait sur les opérateurs d'advection.
    L'analyse d'erreur met en lumière un mécanisme apporté par la multirésolution-adaptative pouvant dégrader l'ordre de convergence d'une méthode numérique classique.
    Une expérience numérique a par la suite été menée pour entreprendre d'observer ce phénomène dans un contexte pratique.
\subparagraph{}
Ces deux objectifs permettent d'apporter différents éléments de compréhension sur le comportement des stratégies de simulation des équations d'ADR.