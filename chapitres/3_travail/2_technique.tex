\subsubsection{Définitions}
\begin{definition}[Méthode des lignes]
    Une méthode des lignes 
\end{definition}

Introduisons d'abord la notion de raideur d'un sysètme dynamqiue. 
\begin{definition}[Problème raide]
    Un système dynamqiue, est dit raide si les méthodes explicites ne sont pas adaptées à sa résolution.
    En termes plus mathématiques le système 
    \begin{align}
    \frac{\text d u}{\text{d}t} = A u \quad u(t) \in \mathbb{R}^d \: \forall t\geq 0.
    \end{align}
    est dit raide si l'opérateur $A$ possède de \text{grandes} valeurs propres négatives
    \footnote{Ici \textit{grand} est à comprendre au sens de \textit{grande aplitude devant d'autres valeurs propres}.}.
\end{definition}

\begin{exemple}[Équation de Dhalquist]
    Pour saisir de manière plus intuitive le concept de raideur, prenons le cas symple de l'équation de Dhalquist définissant le système suivant
    \footnote{C'est le cas le plus simple d'une valeur propre négative}:
    \begin{align}
        \frac{\text d u }{\text d t} = - \lambda u\quad \lambda > 0.
    \end{align}
    La solution est classiquement : $u(t) = u(0)e^{-\lambda t}$. Ainsi passé quelque $\lambda$ la dynamqiue du système est au point mort. 
    Grossièrement la dynamqiue digne d'intérêt du système se concentre entre $t=0$ et $t=\frac{\lambda}{10}$. Au delà, $u(t>\frac{\lambda}{10}) = o(u(t=0))$, la dynamqiue est terminée.
    Ainsi le lecteur comprend aisément que si l'on souhaite simuler le comportement d'un tel système, il faut prendre des pas de temps petits devant $\vert \lambda \vert^{-1}$.
    Si $\lambda$ est de grande amplitude cela peut devenir très contraignant... Si l'on souhaite utiliser des méthodes explicites, c'est encore pire car la raideur du sysème 
    n'est plus un simple contrainte de précision mais de stabilité. En effet si l'on cherche à approximer le sysètme par un schéma d'Euler explicite, alors : 
    $U^{n+1} = U^n (1 - \lambda \Delta t)$ alors la contrainte de stabilité est $\Delta t \, \lambda < 1$ ce qui est contraignant si $\lambda$ est grand. 
    Si $\lambda = 10^5$ alors il faut avoir $\Delta t /approx 10^{-5}$ donc pour simuler le système entre $t=0$ et $t=1$ il faut cent-milles points !
    On comprend mieux la définition précédente \textit{Un système dynamqiue, est dit raide si les méthodes explicites ne sont pas adaptées à sa résolution.}
\end{exemple}