%       INTRO
%       TRANSFORMÉE MULTI-ÉCHELLE
%       HEURISTIQUES D'ADAPTATION
%       ALGORITHMES
%           CALCULS NIVEAU COURANT
%           CALCUL NIVEAU FIN
%       SOURCES D'ERREURS
%       IMPLÉMENTATIONS
%           MÉTHODES CLASSIQUES 
%           SAMURAI
La multi-résolution adaptative (MRA) est une méthode très efficace pour les problèmes multi-échelles. 
L'objectif est de concentrer les efforts computationnels là où ils sont nécessaires. 
Concrètement cela consiste à augmenter la résolution de la grille de calcul où la solution est complexe et la diminuer où la solution est simple à décrire.
La MRA est donc une méthode de HPC (\textit{high performance computing}) puisqu'elle vise à optimiser l'allocation des ressources de calcul.\par
Cette partie introduit le lecteur à cette méthode en présentant d'abord le concept mathématique de transformée multi-échelle qui est à la base de la MRA. 
Puis il est expliqué comment la transformée multi-échelle permet d'adapter le maillage pour optimiser la charge computationnelle.
Une fois ces prérequis établis, l'algorithme typique de mise en oeuvre de la multi-résolution adaptative est décrit.
S’ensuit alors naturellement une présentation des différentes implémentations de la MRA, avec une attention particulière sur celle développée au CMAP au travers 
du logiciel Samurai.
Enfin l'impact de la multi-résolution sur la qualité des solutions numériques est abordé. 
\subsubsection{La transfomée multi-échelle}
    \paragraph{}
    \paragraph{}
\subsubsection{L'adaptation}
    \paragraph{}
    \paragraph{}
\subsubsection{Algorithmes de simunlation numérique}
    \paragraph{}
    \paragraph{}
\subsubsection{Impact sur les solutions}
    \paragraph{}
    \paragraph{}
\subsubsection{Implémentations de la multi-résolution}
    \paragraph{Méthodes usuelles}
    \paragraph{Le logiciel Samurai}