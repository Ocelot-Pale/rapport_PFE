Plaçons nous dans le contexte physique naturel des équations d'advections, diffsuion, réaction: 
Des particules sont placées dans un milieu fluide où elles \textbf{diffusent}, ce milieu fluide 
est en mouvement, cet écoulement déplace les particules, il les \textbf{advecte}.
Enfin les particules \textbf{réagissent} entre-elles et ces réactions modifient les grandeurs themordynmaiques (tempoérature, préssion) et \textit{in fine} les propriétés
du mileu fluide.
Les équations d'advection, diffusion, réaction modélisent donc ces trois phénomènes et leurs couplages respectifs
\subsection{Les trois opérateurs}
    \subsubsection{Advection}
    \subsubsection{Diffusion}
    \subsubsection{Réaction}
\subsection{Difficultés mathématiques intrinsèques}

\subsection{Les stratégies de simulation}
    \subsubsection{L'adaptation de maillage}
        \paragraph{La multi-résolution adaptative}
        \paragraph{Autres méthodes}
    \subsubsection{Les techniques d'intégration}
        \paragraph{Les méthodes ImEx}
        \paragraph{La séparation d'opérateurs}