Plaçons nous dans le contexte physique naturel des équations d'advection, diffusion, réaction:
Des particules sont placées dans un milieu fluide où elles \textbf{diffusent}, ce milieu fluide
est en mouvement, cet écoulement déplace les particules, il les \textbf{advecte}.
Enfin les particules \textbf{réagissent} entre-elles et ces réactions modifient les grandeurs thermodynamiques (température, pression) et \textit{in fine} les propriétés
du milieu fluide.
Les équations d'advection, diffusion, réaction modélisent donc ces trois phénomènes et leurs couplages respectifs.

\subsection{Les trois opérateurs}
\subsubsection{Advection}
    L'advection désigne le transport d'une quantité par un flot. L'opérateur d'advection le plus simple est l'opérateur
    de transport $c \frac{\partial}{\partial x}$:
    \begin{align}\frac{\partial u}{\partial t} = c \frac{\partial u}{\partial x}\end{align}
    De manière générale un opérateur d'advcection d'une quantité $u$ par un flot $\underline a$ s'écrit $\underline a \cdot \underline{\nabla} u$.
    Par exemple dans les équations de Navier-Stokes, l'opérateur $\underline{v} \cdot \underline{\underline \nabla} \, \underline{v}$ représente 
    la vitesse $\underline v$ qui est transportée par elle même. Une version simplifiée de ce phénomène est l'équation bien connue de Bürgers.\par 
    Les opérateurs d'advections sont généralement à valeurs propres imaginaires\footnote{Par abus, s'il s'agit d'un opréteur non-linéiare on lui associera les valeurs propres de sa Jacobienne.}.
    Ainsi ils sont peu raides mais résonnants. Les méthodes explicites sont généralements les plus adaptées pour les traiter.
    %       AFAIRE ---> trouver une référence pour justifier que les opérateurs d'advections sont à valeurs propres imaginaires.

\subsubsection{Diffusion}
    La diffusion désigne l'\textit{éparpillement} de particules au sein d'un milieu fluide
    \footnote{En théorie de l'information cela décrit la tendance de l'entropie augmenter et l'information à se moyenner, se flouter.}.
    Ce phénomène est la limite macroscopique du déplacement microscopiques 
    des particules à cause de l'agitation thermique. L'opérateur de diffusion le plus classique est celui de l'équation de la chaleur:
    \begin{align} \frac{\partial u}{\partial t} = D \Delta u.\end{align}
    Le spectre de cet opérateur est $\mathbb R^-$, il est donc infiniement raide. Lorsqu'il est discrétisé seul une partie de sa raideur est captée,
    en pratique la raideur de l'opérateur augmente quadratiquement avce la finesse de la discrétisation spatiale.\par
    %       AFAIRE ---> trouver une référence (ou faire la démo) pour le spectre.
    Cet opérateur est donc moyennement raide. Ainsi on pourrait penser qu'une méthode implcite est adéquate. Cependant ce n'est généralement pas le cas.
    En effet le coefficient de diffusion est généralement fonciton de la températeur, et donc l'éoprateur $D(T) \times \Delta(\cdot)$ varie généalement dans le temps et l'espace. 
    Ainsi il faut inverser à chaque itération l'opérateur implcite, et comme c'est un opérateur non local
    \footnote{Si l'opérateur de diffusion était local on pourrait résoudre plusieurs petit systèmes, potentiellements en parallèle ce qui est bien moins couteux qu'inverser 
    un grand système. Pour se convaincre, inverser un matrice pleine de taille $10^6$ coute au moins $10^{18}$ opérations, alors qu'inverser 100 systèmes de taille $10^4$
    coute $100 \times 10^{12} = 10^{14}$ soit dix mille fois moins, et si ces résolution étaient parallélisé ce serait un million de fois moins.},
    il faut inverser une matrice de taille $d >> 1$ dont la structure
    peut être très hétérogène.
    (car le coefficient de diffusion dépend de $T$ et du milieu, donc \textit{in fine} de $\underline x$). Aujourd'hui il est d'usage d'utiliser 
    des méthodes explicites stabilisées qui parviennent à gérer la raideur moyenne\footnote{Nous reviendrons sur ce qualificatif au prochain paragraphe.} comme les méthodes 
    ROK2 et ROK4\cite{abdulle2002fourth}.

\subsubsection{Réaction}
    Les phénomène sont en général bien adaptés aux méthodes implcites car extrêmements raides et locaux.
    En effet,les temps typiques d'une réaction chimique\footnote{
    En réalité une réaction chimique simple (une combustion $H_2/O_2$ fait intervenir une dizaine de composés et reactions intermédaires, dont les temps typiuqes sont très faibles.)} sont de l'ordre de la nano-seconde.
    %AFAIRE AJOUTER UNE JUSTIFICATION
    De fait, les réactions chimiques sont très diffciles à simuler par des méthodes explicites.
    Et les méthodes implcites ne sont pas très chères dans ce contexte, en effet comme les réactions sont locales 
    (à chaque pas de temps les particules les particules au sein d'une cellules ne réagissent qu'avec les autres particules de la même cellule)
    les méthodes explicites peuvent se paralléliser. En d'autres termes il est possible de mettre en oeuvre une méthode implcite par cellule,
    ce qui revient à inverser un opérateur de petite dimension en chaque cellule, et il n'est pas nécessaire d'inverser un énorme système.

\subsection{Difficultés mathématiques intrinsèques}
    Les développements précédents auront convaincu que résoudre chaque phénomène individuellement, n'est pas insurmontable. 
    Cependant, les résoudre tous en même temps, c'est à dire les coupler, est en pratique très difficile.
    En effet, lorsque l'on couple les trois opérateurs, il en résulte un unique opérateur qui doit être traité par une méthode numérique.
    Et là, les difficultés arrive, si la méthode est explicite ou explicite stabilisée, la raideur de la réaction va imposer des pas de temps extrêments restricitfs,
    à l'inverse si l'on choisit une méthode implicite, la non-localité de la diffusion nécessite l'inversion d'un système de taille déraisonnable. 
    Cette approche naïve, monolithique, n'est donc pas adaptée. Il faut trouver d'autres stratégies de pour simuler ces équations d'advction-réaction-diffusion.

\subsection{Les stratégies de simulation}
\subsubsection{L'adaptation de maillage}
    \paragraph{La multi-résolution adaptative}
    \paragraph{Autres méthodes}
\subsubsection{Les techniques d'intégration}
    \paragraph{Les méthodes ImEx}
    \paragraph{La séparation d'opérateurs}