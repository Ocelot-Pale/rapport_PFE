\subsubsection{Méthodes des lignes et schéma couplé espaces temps.}
    \begin{definition}[Méthode des lignes]
        Une méthode des lignes est une famille de méthodes numériques pour approximer les EDP d'évolutions
        Elle consiste à discrétiser les opérateurs spatiaux de l'équation afin d'obtenir une équation semi-discrétisée en espace,
        puis à utiliser une technique d'intégration en temps, pour obtenir la discrétisation complète de l'équation.
    \end{definition}
    \begin{tikzpicture}[node distance=4.3cm,box/.style={rectangle, draw, thick, minimum width=2.5cm, minimum height=1cm, align=center},arrow/.style={-{Stealth[length=3mm]}, thick},label/.style={font=\small, align=center}]
    \node[box, fill=blue!20] (edp) {EDP};
    \node[box, fill=green!20, right=of edp] (edo) {Équation\\semi-discrétisée\\(EDO)};
    \node[box, fill=orange!20, right=of edo] (schema) {Schéma\\final};
    \draw[arrow] (edp) -- node[above, label] {Discrétisation des\\opérateurs spatiaux} (edo);
    \draw[arrow] (edo) -- node[above, label] {Méthode d'intégration\\temporelle} (schema);  
    \node[above=.1cm of edo, font=\bfseries] {Méthode des lignes};
    \end{tikzpicture}

    \begin{definition}[Méthodes d'intégration espace temps]
    \end{definition}
\subsubsection{Les volumes finis}
    Le paradigme numérique utilisé dans le stage est celui des volumes finis, particulièrement adaptés au lois de conservations.
    Les volumes finis discrétisent la valeur moyenne sur les mailles, là où les différences finies discrétisent la valeur au nœuds du maillage et 
    les éléments finis discrétisent l'espace fonctionnel lui même. Les explications suivantes sont présentés avec plus de détails dans \cite{LeVeque1990}.
    \begin{definition}[Volumes finis]
        Donné un maillage de pas de discrétisation $\Delta x$ d'un domaine $\Omega$ en cellules $(C_j)_{j\in J}$ de volume de l'ordre $\Delta x^d$ (ou $d$ est la dimension), 
        la discrétisation par volume fini approxime les quantités:
        \begin{align}
            U_j = \frac{1}{\vert C_j \vert} \int_{C_j} u(x) d\Omega.
        \end{align}
    \end{definition}
    Les volumes finis brillent lors de la simulation les lois de conservations, c'est à dire les EDPs de la forme :
    \begin{align}
        \dt{u} = div(f(u)).
    \end{align}
    En effet lorsque cette relation est moyennée sur une cellule $C_j$ du maillage, cela donne: 
    \begin{align}
        \int_{C_j} \dt{u}&= \int_{C_j}  div(f(u)),\\
        \dt{U_j} &= \int_{\partial C_j} f(u).\label{eq:apar_flux}
    \end{align}
    \begin{definition}[Flux physique]
        Dans l'équation \ref{eq:apar_flux}, le terme $\int_{\partial C_j} f(u)$ est appelé $\Phi_j$ le terme de \textit{flux} de la cellule $j$;
        physiquement il quantifie "l'entrée de $f(u)$" au sein de la cellule $C_j$. En une dimension $\Phi_j = f(u^+_j) - f(u^-_j)$;
        le flux dépend simplement des valeurs à l'interface de la cellule.
    \end{definition}
    La définition précédent est une intégrale de bord faisant intervenir les valeurs exactes de $u$ le long de l'interface.
    Malheureusement, paradigme des volumes n'offre qu'un accès au valeurs moyennes de $u$ sur chaque cellule.
    Ainsi, il faut approximer $\Phi_j$ à partir des valeurs moyennes.
    \begin{definition}[Flux numérique]
        Un \textit{flux numérique} $\Psi$ est fonction permettant d’approximer $\Phi_j$ à partir des valeurs moyennes sur la cellule $j$ et les cellules voisines:
        $\Psi(U_{j-s} , \ldots , U_j ,\ldots , U_{j+s})$. Le nombre de cellules intervenant dans le calcul $s$ est appelé le \textit{stencil}.
    \end{definition}

    \begin{definition}[Flux numérique d'ordre $p$]
        Pour être exploitable, un flux numérique doit être \textit{consistant} à un ordre $p\geq 1$ avec la loi de conservation étudiée.
        Donné une solution $u$ de régularité $C^{p+1}$ un flux numérique est dit consistant à l'ordre $p$ si, $\forall j$:
        \begin{align}
            \vert \Psi \bigl( \tilde U_{j-s}(t) , \ldots ,\tilde  U_j(t), \ldots ,  \tilde U_{j+s}(t)\bigr) -  \Phi_j \vert = O(\Delta x^p).
        \end{align}
        Où $\tilde U_j$ désigne la moyenne exacte (et non une approximation) de $u$ sur la cellule $C_j$.
    \end{definition}
    D'un point de vue méthode des lignes, le paradigme des volumes finis donne l'équation semi-discrétisée en espace suivante:
    \begin{align}
        \dt{U_j(t)} = \Psi \bigl( U_{j-s}(t) , \ldots , U_j(t), \ldots ,  U_{j+s}(t)\bigr),
    \end{align}
    il ne reste qu'à l'intégrer grâce à une méthode d'intégration des EDOs.


\subsection{Analyse de schéma numériques}
    Pour qualifier une schéma numérique, sur différents sujets comme 
    la faisabilité de sa mise en oeuvre, la qualité de la solution qu'il fourni ou encore la dynamique de l'erreur qu'il introduit, 
    divers concepts ont étés développés. Ce qui suit introduit les notions pertinentes pour comprendre les travaux du stage. 

    \subsubsection{Stabilité d'un schéma numérique}
        Les schéma simulant les équations au dérivées partielles ont les mêmes nécessités de stabilité, que ceux simulant les EDOs. 
        D'ailleurs les méthodes des lignes transforment précisément l'EDP en EDO. 
        Pour éviter les redondance, le lecteur se référera à la partie \ref{par:stabilite_edo}.
    \subsubsection{Analyse de l'erreure}
        L'analyse de l'ordre de convergence d'un schéma permet de quantifier l'erreur asymptotique 
        (c'est à dire quand les pas de temps et d'espace sont "assez petits")que commet le schéma.
        L'équation équivalente du schéma permet de comprendre ce qui résout simule "vraiment" le schéma et notamment la dynamique de l'erreur.
        \begin{definition}[Équation modifiée]
            ...
        \end{definition}
        L'équation modifiée du schéma peut devenir plus interprétable si dérivées temporelles d'ordres supérieurs à un sont remplacés par 
        des dérivées spatiales, cela est permis grâce à la procédure dite de Cauchy-Kovaleskaya.
        \begin{definition}[Procédure de Cauchy-Kovaleskaya]
            ...
        \end{definition}
    