    Les équations aux dérivées partielles d'évolutions sont des EDPs dont une des variables différenciée est le temps.
    Cette section introduit divers éléments d'analyse et simulation pour ces équations.
    Elle introduit d'abord la notion de méthode des lignes, une approche classique d'élaboration de schéma numériques pour les EDPs d'évolutions.
    Puis elle présente la notion de volume fini qui est le paradigme de discrétisation spatiale utilisé dans ce stage.
    Enfin elle détaille quelques outils d'analyse des schémas numériques pour les EDPs dévolution (convergence, stabilité, analyse d'erreur).
    \subsection{Les méthodes des lignes.}
    \begin{definition}[Méthode des lignes]
        Une méthode des lignes est une famille de méthodes numériques pour approximer les EDP d'évolutions
        Elle consiste à discrétiser les opérateurs spatiaux de l'équation afin d'obtenir une équation semi-discrétisée en espace,
        puis à utiliser une technique d'intégration en temps, pour obtenir la discrétisation complète de l'équation.
    \end{definition}
    \begin{tikzpicture}[node distance=4.3cm,box/.style={rectangle, draw, thick, minimum width=2.5cm, minimum height=1cm, align=center},arrow/.style={-{Stealth[length=3mm]}, thick},label/.style={font=\small, align=center}]
    \node[box, fill=blue!20] (edp) {EDP};
    \node[box, fill=green!20, right=of edp] (edo) {Équation\\semi-discrétisée\\(EDO)};
    \node[box, fill=orange!20, right=of edo] (schema) {Schéma\\final};
    \draw[arrow] (edp) -- node[above, label] {Discrétisation des\\opérateurs spatiaux} (edo);
    \draw[arrow] (edo) -- node[above, label] {Méthode d'intégration\\temporelle} (schema);  
    \node[above=.1cm of edo, font=\bfseries] {Méthode des lignes};
    \end{tikzpicture}

    Il existe également des approches plus sophistiquées, comme les méthodes couplées espace-temps \cite{DARU2004563}, mais ce stage se concentre sur les méthodes des lignes.
\subsection{Les volumes finis}
    La technique de discrétisation spatiale utilisée dans ce travaille est celle des \textit{volumes finis}\cite{LeVeque1990}.
    Cette approche discrétise la valeur moyenne sur les cellules du maillage, là où les approches par \textit{différences finies} \cite{LeVeque2007} discrétisent la valeur au noeuds du maillage et 
    celles par \textit{éléments finis} \cite{Ciarlet1982} discrétisent l'espace fonctionnel lui même.
    \begin{definition}[Volumes finis]
        Donné une discrétisation d'un domaine $\Omega$ par un. maillage en cellules $(C_j)_{j\in J}$ dont le volume de l'ordre $\Delta x^d$ (ou $d$ est la dimension), 
        la discrétisation par volume fini approxime les quantités:
        \begin{align}
            U_j = \frac{1}{\vert C_j \vert} \int_{C_j} u(x) d\Omega.
        \end{align}
    \end{definition}
    Les volumes finis brillent lors de la simulation les lois de conservations, c'est à dire les EDPs de la forme :
    \begin{align}
        \dt{u} = div(f(u)).
    \end{align}
    En effet lorsque cette relation est moyennée sur une cellule $C_j$ du maillage, cela donne: 
    \begin{align}
        \int_{C_j} \dt{u}&= \int_{C_j}  div(f(u)),\\
        \dt{U_j} &= \int_{\partial C_j} f(u).\label{eq:apar_flux}
    \end{align}
    La notions de flux est centrale dans l'approche par volume finie et très importante pour comprendre le travail réalisée en \ref{par:contrib_2}.
    \begin{definition}[Flux physique]
        Dans l'équation \ref{eq:apar_flux}, le terme $\int_{\partial C_j} f(u)$ est appelé $\Phi_j$ le terme de \textit{flux} de la cellule $j$;
        physiquement il quantifie "l'entrée de $f(u)$" au sein de la cellule $C_j$. En une dimension $\Phi_j = f(u^+_j) - f(u^-_j)$;
        le flux dépend simplement des valeurs à l'interface de la cellule.
    \end{definition}
    La définition précédente est une intégrale de bord faisant intervenir les valeurs exactes de $u$ le long de l'interface.
    Malheureusement, paradigme des volumes n'offre qu'un accès au valeurs moyennes de $u$ sur chaque cellule.
    Ainsi, il faut approximer $\Phi_j$ à partir des valeurs moyennes.
    \begin{definition}[Flux numérique]
        Un \textit{flux numérique} $\Psi$ est fonction permettant d’approximer $\Phi_j$ à partir des valeurs moyennes sur la cellule $j$ et les cellules voisines:
        $\Psi(U_{j-s} , \ldots , U_j ,\ldots , U_{j+s})$. Le nombre de cellules intervenant dans le calcul $s$ est appelé le \textit{stencil}.
    \end{definition}

    \begin{definition}[Flux numérique d'ordre $p$]
        Pour être exploitable, un flux numérique doit être \textit{consistant} à un ordre $p\geq 1$ avec la loi de conservation étudiée.
        Donné une solution $u$ de régularité $C^{p+1}$ un flux numérique est dit consistant à l'ordre $p$ si, $\forall j$:
        \begin{align}
            \vert \Psi \bigl( \tilde U_{j-s}(t) , \ldots ,\tilde  U_j(t), \ldots ,  \tilde U_{j+s}(t)\bigr) -  \Phi_j \vert = O(\Delta x^p).
        \end{align}
        Où $\tilde U_j$ désigne la moyenne exacte (et non une approximation) de $u$ sur la cellule $C_j$.
    \end{definition}
    D'un point de vue méthode des lignes, le paradigme des volumes finis donne l'équation semi-discrétisée en espace suivante:
    \begin{align}
        \dt{U_j(t)} = \Psi \bigl( U_{j-s}(t) , \ldots , U_j(t), \ldots ,  U_{j+s}(t)\bigr),
    \end{align}
    il ne reste qu'à l'intégrer grâce à une méthode d'intégration des EDOs.


\subsection{Analyse de schéma numériques}
    Pour qualifier une schéma numérique, sur différents sujets comme 
    la faisabilité de sa mise en oeuvre, la qualité de la solution qu'il fourni ou encore la dynamique de l'erreur qu'il introduit, 
    divers concepts ont étés développés. Ce qui suit introduit les notions pertinentes pour comprendre les travaux du stage: 
    \begin{itemize}
        \item[$\diamond$] \ref{par:stab_edp} L'analyse de stabilité.
        \item[$\diamond$] \ref{par:conv_edp} La quantification de l'erreur grâce à l'étude de la convergence.
        \item[$\diamond$] \ref{par:eq_equiv} La qualification de la dynamique de l'erreur grâce à la notion d'équation équivalente.
    \end{itemize}

    \subsubsection{Stabilité d'un schéma numérique}\label{par:stab_edp}
        Les schémas simulant les équations au dérivées partielles ont les mêmes nécessités de stabilité, que ceux simulant les EDOs. 
        D'ailleurs les méthodes des lignes transforment précisément une EDP en EDO. 
        Pour éviter les redondances, le lecteur se référera à la partie \ref{par:edo}. La seule différence entre la stabilité des schémas pour EDP et pour EDO est que,
        lorsque q'une EDP est résolue numériquement, la raideur de l'opérateur spatial discret dépend généralement explicitement du pas d'espace $\Delta x$.
        De fait condition de stabilité ne pas réellement EDP mais plutôt de l'EDO résultant de la discrétisation en espace. Cela mène à 
        des conditions du type : $\Delta t \leq \lambda(\Delta x)$. La constante $\lambda$ est appelée la constante de stabilité,
        ou de manière parfois incorrecte la constante de Courant-Friedrich-Levy (CFL).
    \subsubsection{Analyse de l'erreure}
        L'analyse de l'ordre de convergence d'un schéma permet de quantifier l'erreur asymptotique 
        (c'est à dire lorsque les pas de temps et d'espace sont "assez petits") que commet le schéma.
        \paragraph{Ordre de convergence d'un schéma}\label{par:conv_edp}
        Les schémas numériques sont désignés par $\bigl(U^n_j\bigr)_{n,j}$ où $n$ représente le pas de temps 
        et $j$ la cellule. Ainsi, si la grille contient $c$ cellules, 
        pour tout $n$, $U^n$ est un élément de l'espace vectoriel $\mathbb{R}^c$.
        Le vecteur $\bigl(u(x_j,t^n)\bigr)_{j \in \{1,\ldots,c\}}$ est noté $u(\cdot,t^n)$.

        \begin{definition}[Erreur globale]
        L'erreur globale $E_G$ d'un schéma $\bigl(U^n_j\bigr)_{n,j}$ sur un temps $T_f = N_f \Delta t$ peut être définie comme l'erreur au temps final :
        \begin{align}
        E_G = \Delta x^d \Vert U^{N_f} - u(\cdot , T_f) \Vert
        \end{align}
        où $\Vert \cdot \Vert$ désigne une norme sur $\mathbb{R}^c$.
        Ou bien comme l'intégrale de l'erreur sur tous les pas de temps :
        \begin{align}
        E_G = \sum_{k=0}^{N_f}\Delta x^d \Vert U^{k} - u(\cdot , t^k) \Vert
        \end{align}
        \end{definition}
        \begin{definition}[Ordre de convergence d'un schéma]
            Un schéma $\Bigl(U^n_j\Bigr)_{n,j}$ est dit convergent à l'ordre $p$ en temps et $q$ en espace si 
            son erreur de globale s'écrit: $E_G = O(\Delta t^{p} + \Delta x^{q})$.
        \end{definition}
        \paragraph{Equation équivalente}\label{par:eq_equiv}
        \begin{definition}[Équation équivalente]
        L'équation équivalente d'un schéma est l'EDP dont la solution satisfait le schéma.
        Elle est calculée par des développements de Taylor en temps et en espace.
        Comparer l'équation équivalente et l'équation cible fait alors naturellement apparaître les termes d'erreur et leur dynamique.
        \end{definition}

        Ce qui suit décrit une méthode générique pour obtenir l'équation équivalente d'un schéma, la notion d'équation équivalente est clé dans la contribution en \ref{par:intro_contrib_eq_equiv}.

        \subparagraph{Première étape : développement de Taylor}

        L'existence d'une fonction assez régulière vérifiant le schéma est supposée.
        Dans le schéma numérique les termes $u^{n+\delta_t}_{k+\delta_x}$ sont remplacés par leur pendant
        continu : $u(x+\delta_x \Delta x, t + \delta_t \Delta t)$. Le développement de Taylor suivant est alors réalisé :
        \begin{align}
        u(x+\delta_x \Delta x, t + \delta_t \Delta t) = u(x,t)
        &+ \sum_{i=1}^{\infty} \frac{(\delta_x \Delta x)^i}{i!}
        \frac{\partial^i u}{\partial x^i}(x,t) \notag \\
        &+ \sum_{j=1}^{\infty} \frac{(\delta_t \Delta t)^j}{j!}
        \frac{\partial^j u}{\partial t^j}(x,t) \notag \\
        &+ \sum_{i,j \geq 1} \frac{(\delta_x \Delta x)^i (\delta_t \Delta t)^j}{i! \cdot j!}
        \frac{\partial^{i+j} u}{\partial x^i \partial t^j}(x,t)
        \end{align}

        L'équation aux dérivées partielles qui apparaît est l'équation équivalente. Elle peut être tronquée à l'ordre voulu.

        \subparagraph{Deuxième étape : procédure de Cauchy-Kovaleskaya (optionnelle)}

        Afin d'enrichir l'analyse, et permettre le développement de schémas couplés espace-temps, l'étape précédente peut être suivie d'une procédure de Cauchy-Kovaleskaya.
        La procédure de Cauchy-Kovaleskaya consiste à utiliser la relation entre les dérivées en espace et en temps données par l'équation cible et remplacer les dérivées en temps par des dérivées en espace dans l'équation équivalente.
        Par exemple, pour un schéma ayant pour équation cible la diffusion $\frac{\partial u}{\partial t} = D \frac{\partial^2 u}{\partial x^2}$, cela consiste à écrire de manière itérée :
        \begin{align}
        \frac{\partial^2 u}{\partial t^2} &= D^2 \frac{\partial^4 u}{\partial x^4} \notag \\
        \frac{\partial^3 u}{\partial t^3} &= D^3 \frac{\partial^6 u}{\partial x^6} \notag \\
        &\vdots \notag \\
        \frac{\partial^n u}{\partial t^n} &= D^n \frac{\partial^{2n} u}{\partial x^{2n}}
        \end{align}

        \subparagraph{Dernière étape : relation entre le pas de temps et d'espace (optionnelle)}

        Lorsque le schéma est utilisé en pratique, il est courant d'imposer une relation entre les pas d'espace et de temps, par exemple une condition de stabilité du type
        $\Delta t \propto \Delta x$ ou $\Delta t \propto \Delta x^2$.
        Il est donc utile d'injecter cette relation dans l'équation équivalente pour comprendre le comportement du schéma en conditions réelles.

        Ces outils d'analyse numérique constituent les fondements nécessaires pour aborder la simulation numérique des EDPs. La section suivante présente la multi-résolution adaptative qui permet de réduire le coût en calcul et en mémoire de ces méthodes.