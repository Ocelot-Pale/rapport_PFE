\begin{definition}[Méthode des lignes]
    Une méthode des lignes est une famille de méthodes numériques pour approximer les EDP d'évolutions
    Elle consiste à discrétiser les opérateurs spatiaux de l'équation afin d'obtenir une équation semi-discrétisée en espace,
    puis à utiliser une technique d'intégration en temps, pour obtenir la discrétisation complète de l'équation.
\end{definition}
\begin{tikzpicture}[node distance=4.3cm,box/.style={rectangle, draw, thick, minimum width=2.5cm, minimum height=1cm, align=center},arrow/.style={-{Stealth[length=3mm]}, thick},label/.style={font=\small, align=center}]
\node[box, fill=blue!20] (edp) {EDP};
\node[box, fill=green!20, right=of edp] (edo) {Équation\\semi-discrétisée\\(EDO)};
\node[box, fill=orange!20, right=of edo] (schema) {Schéma\\final};
\draw[arrow] (edp) -- node[above, label] {Discrétisation des\\opérateurs spatiaux} (edo);
\draw[arrow] (edo) -- node[above, label] {Méthode d'intégration\\temporelle} (schema);  
\node[above=.1cm of edo, font=\bfseries] {Méthode des lignes};
\end{tikzpicture}

\begin{definition}[Méthodes d'intégration espace temps]
\end{definition}

Dans la suite de notre étude nous allons utiliser la paradigme des volumes finis.
Les volumes finis sont particulièrement adaptés au lois de conservations.
Les volumes finis discrétisent la valeur moyenne sur les mailles, alors que les différences finies discrétisent la valeur au noeuds du maillage et 
les éléments finis discrétisent l'espace fonctionnel lui même.
\begin{definition}[Volumes finis]
    Donné un maillage $(C_j)_{j\in J}$ d'un domaine $\Omega$, la discrétisation par volume fini approxime les quantités:
    \begin{align}
        U_j = \frac{1}{\vert C_j \vert} \int_{C_j} u(x) d\Omega.
    \end{align}
\end{definition}
\subsection{Analyse de schéma numériques}
Staiblité... Convergence...
\begin{definition}[Procédure de Cauchy-Kovaleskaya]
\end{definition}
\begin{definition}[Équation modifiée]
\end{definition}