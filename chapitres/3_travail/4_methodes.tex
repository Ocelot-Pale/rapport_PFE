\subsection{Méthode et démarche}
    Les deux contributions présentées au chapitre suivant (\ref{par:contrib_1} et \ref{par:contrib_2}) on été réalisée 
    en s'imprégnant de l'état de l'art, en identifiant un point mal compris ou mal étudié, puis
    en réalisant une étude théorique sur un cas particulier pour déceler et révéler des comportements potentiellement généraux
    pour enfin chercher des observations expérimentales des mécanismes mis à jour.
\subsection{Outils numériques}
    Divers outils numériques ont appuyés les travaux réalisés au cours du stage aidant à des des tâches diverses comme 
    l'évaluation de fonction, la visualisation, le calcul formel et bien sûr la mise en oeuvre d'expériences numériques.
    \subparagraph{Expérimentation numérique}
        Le logiciel Samurai développé par l'équipe du CMAP a été utilisée pour exécuter toutes les expérimentations numériques nécessaires 
        au travaux réalisés au cours du stage. Il s'agit d'une bibliothèque écrite en \texttt{C++} permettant de mettre en oeuvre 
        rapidement et efficacement des méthodes numériques avec ou sans MRA.
    \subparagraph{Calcul et réprésentation}
        Les librairies Python usuelles: Numpy et Matplotlib on permis d'évaluer et de représenter des fonctions en différents points et de 
        les représenter. Une bonne représentation est essentielle car elle permet d'appréhender rapidement l'information et aide l'esprit 
        a développer une intuition riche capable d'orienter les recherches et d'affiner la compréhension. 
        Ces outils ont été cruciaux pour évaluer les fonctions de stabilité des méthodes étudiées (voir \ref{}) et pour réaliser 
        des post-traitements (visualisation, calcul d'erreur...) sur les solutions issues des expériences numériques  
    \subparagraph{Calcul formel}
        La librairie Python de calcul formel Sympy a rendu possible le calcul des équations équivalentes (voir \ref{}) qui n'aurait pas 
        été réalisable autrement. 