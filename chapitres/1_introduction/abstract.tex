\section*{Abstracts}\addcontentsline{toc}{section}{Abstracts}
\subsection*{Résumé en Français}
    \textbf{Mots-clés :} Schémas Numériques, Simulation des EDP d'Évolution, Multirésolution Adaptative, Méthodes ImEx, 
    Advection-Diffusion-Réaction, Analyse d'erreur numérique, Analyse de stabilité.\par
    \noindent\rule{\textwidth}{0.4pt}
    % \vspace{0.3cm}
    % Ce papier documente mon projet de fin d'études qui a pris place au laboratoire du Centre de Mathématiques Appliquées de l'École Polytechnique (CMAP).
    % Cette expérience en recherche académique a été une opportunité exceptionnelle car elle m'a permis de mieux comprendre les rouages de la recherche,
    % de mettre en application et en relation les concepts et savoirs-faire acquis au cours de mes études, d'améliorer la communication et le partage de mon travail,
    % d'échanger avec des chercheurs d'horizons divers
    % et découvrir des thèmes et des problématiques scientifiques qu'y m'étaient inconnues, en somme de parfaire mon parcours académique et assurer une heureuse transition avec le monde professionnel.
    % Mon travail de recherche porte sur des méthodes modernes pour la simulation des équations d'advection-diffusion-réaction (ADR), des EDP assez capricieuses,
    % aux applications multiples, régissants entre autres les phénomènes de combustions. Ce rapport contient une introduction aux défis que portent ces équations et 
    % une introduction aux stratégies imaginées pour relever ses défis, le tout accompagnés de quelques rappels mathématiques bienvenus.
    % Il inclut bien sûr une présentation de mes contributions: deux études, une portant sur \textbf{la multi-résolution adaptative},
    % j'y présente une étude théorique de l'erreur qu'elle apporte sur un cas particulier;
    % et l'autre sur \textbf{les méthodes Runge et Kutta Implicites-Explicites (ImEx)},
    % je présente une analyse sur les équations d'ADR de ces méthodes et les compare a un autre méthode plus standard. 
    Les systèmes couplant mécanique des fluides et chimie complexe se modélisent par les équations d'advection-diffusion-réaction (ADR), 
    une classe d'équations aux dérivées partielles dont la résolution numérique requiert à la fois des stratégies d'adaptation de maillage 
    (par exemple la multirésolution adaptative, MRA) et des méthodes d'intégration temporelle spécifiques (splitting, schémas ImEx).
    Ce travail étudie les interactions entre ces deux composantes (adaptation spatiale et intégration temporelle).
    Il apporte (i) une comparaison empirique de l'effet de la MRA sur les schémas de splitting et les schémas ImEx, 
    ainsi qu'une une analyse (ii) théorique et (iii) numérique des interactions entre schéma numérique et MRA sur des problèmes de diffusion et diffusion réaction.
\subsection*{English abstract}
    \textbf{Keywords  :} Numerical Schemes, Evolution PDE Simulation, Adaptive Multiresolution, ImEx Methods,
    Advection-Diffusion-Reaction, Numerical Error Analysis, Stability Analysis.\par
    \noindent\rule{\textwidth}{0.4pt}
    Systems coupling fluid mechanics and complex chemistry are modeled by advection-diffusion-reaction (ADR) equations, 
    a class of partial differential equations whose numerical resolution requires both spatial mesh adaptation strategies 
    (such as adaptive multiresolution, MRA) and dedicated time integration methods (splitting, ImEx schemes).
    This work investigates the interactions between these two components (spatial adaptation and temporal integration).
    It provides (i) an empirical comparison of the effect of MRA on splitting and ImEx schemes, 
    and (ii) a theoretical and (iii) numerical analysis of the interaction that emerges between the numerical scheme and MRA adaptation on a diffusion problem and a diffusion-reaction problem.
\newpage