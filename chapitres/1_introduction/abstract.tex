\section*{Abstracts}\addcontentsline{toc}{section}{Abstracts}
\subsection*{Résumé en Français}
    \textbf{Mots-clés :} Schémas Numériques, Simulation des EDP d'Évolution, Multirésolution Adaptative, Méthodes ImEx, 
    Advection-Diffusion-Réaction, Analyse d'erreur numérique, Analyse de stabilité.\par
    \noindent\rule{\textwidth}{0.4pt}
    % \vspace{0.3cm}
    Ce papier documente mon projet de fin d'études qui a pris place au laboratoire du Centre de Mathématiques Appliquées de l'École Polytechnique (CMAP).
    Cette expérience en recherche académique a été une opportunité exceptionnelle car elle m'a permis de mieux comprendre les rouages de la recherche,
    de mettre en application et en relation les concepts et savoirs-faire acquis au cours de mes études, d'améliorer la communication et le partage de mon travail,
    d'échanger avec des chercheurs d'horizons divers
    et découvrir des thèmes et des problématiques scientifiques qu'y m'étaient inconnues, en somme de parfaire mon parcours académique et assurer une heureuse transition avec le monde professionnel.
    Mon travail de recherche porte sur des méthodes modernes pour la simulation des équations d'advection-diffusion-réaction (ADR), des EDP assez capricieuses,
    aux applications multiples, régissants entre autres les phénomènes de combustions. Ce rapport contient une introduction aux défis que portent ces équations et 
    une introduction aux stratégies imaginées pour relever ses défis, le tout accompagnés de quelques rappels mathématiques bienvenus.
    Il inclut bien sûr une présentation de mes contributions: deux études, une portant sur \textbf{la multi-résolution adaptative},
    j'y présente une étude théorique de l'erreur qu'elle apporte sur un cas particulier;
    et l'autre sur \textbf{les méthodes Runge et Kutta Implicites-Explicites (ImEx)},
    je présente une analyse sur les équations d'ADR de ces méthodes et les compare a un autre méthode plus standard. 
\subsection*{English abstract}
    \textbf{Keywords  :} Numerical Schemes, Evolution PDE Simulation, Adaptive Multiresolution, ImEx Methods,
    Advection-Diffusion-Reaction, Numerical Error Analysis, Stability Analysis.\par
    \noindent\rule{\textwidth}{0.4pt}
    This paper documents my final year project conducted at the Applied Mathematics laboratory of École Polytechnique (CMAP).
    This academic research experience has been an exceptional opportunity as it allowed me to better understand the workings of research,
    to apply and connect the concepts and skills acquired during my studies, to improve communication and sharing of my work,
    to interact with researchers from diverse backgrounds,
    and to discover scientific themes and problems that were previously unknown to me, in short, to complete my academic journey and ensure a smooth transition to the professional world.
    My research work focuses on modern methods for simulating advection-diffusion-reaction (ADR) equations, rather challenging PDEs
    with multiple applications, governing among others combustion phenomena. This report contains an introduction to the challenges posed by these equations and
    an introduction to the strategies devised to address these challenges, all accompanied by some welcome mathematical reminders.
    It naturally includes a presentation of my contributions: two studies, one focusing on \textbf{adaptive multiresolution},
    where I present a theoretical study of the error it introduces in a particular case,
    and the other on \textbf{Implicit-Explicit Runge-Kutta methods (ImEx)},
    where I present an analysis of these methods on ADR equations and compare them to another more standard method.
\newpage