Le rapport est structuré comme suit.
\begin{itemize}
\item[$\circ$] Ce chapitre pose le contexte, la problématique et les objectifs.
\item[$\circ$] Le Chapitre \ref{par:tech} sert de préambule mathématique et algorithmique:
il rappelle les méthodes de discrétisation pour EDO et EDP, les équations d'advection-diffusion-réaction, 
quelques outils d'analyse de schémas numériques (stabilité, équations équivalentes), ainsi que les principes de la multirésolution adaptative; 
le lecteur expérimenté peut le parcourir rapidement et s'y référer au besoin.
\item[$\circ$] Le Chapitre \ref{par:contrib} rassemble trois contributions:
\begin{itemize}
    \item[$\diamond$] (\ref{par:contrib_1}): Une étude de la stabilité de deux méthodes ImEx ARK;
    suivie d'une comparaison avec un schéma de splitting de leur stabilité et convergence pour l'équation-test de Nagumo (réaction-diffusion).
    \item[$\diamond$] (\ref{par:contrib_2}): Une étude théorique, via le calcul d'équations équivalentes, 
    du comportement de l'erreur d'une schéma méthode des lignes pour un problème de diffusion. 
    Cela dans trois contextes: 
    \texttt{(I)} sans multirésolution adaptative, 
    \texttt{(II)} avec une multirésolution adaptative MRA \hlblue{\textit{standard}},
    \texttt{(III)} avec approche multirésolution adaptative \hlperu{\textit{non-standard}}, (reconstruction des flux).
    \item[$\diamond$] (\ref{par:contrib_3}): Une étude expérimentale des différences entre les trois schéma de MRA mentionné précédemment 
    permettant de mettre en relation les résultats théoriques de la contributions \ref{par:contrib_2} avec les observations expérimentales.
\end{itemize}
\item[$\circ$] Le dernier chapitre (chapitre \ref{par:cc}) conclut en trois volets : scientifique, technique et personnel.
\end{itemize}