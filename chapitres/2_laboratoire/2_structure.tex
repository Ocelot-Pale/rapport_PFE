Le rapport est structuré de la manière suivante: le second chapitre (\ref{par:tech}) est un préambule mathématique et algorithmique qui détailles plusieurs notions utiles 
pour comprendre mes contributions. Le lecteur expérimenté peut sauter ce chapitre et s'y référer au cours de sa lecture si certain terme lui semblent flou.
Le troisième chapitre (\ref{par:contrib}) détaille les principales réalisations de mon stage et l'ultime chapitre (\ref{par:cc}) conclu ce rapport.\par 
\textbf{Le second chapitre }présente entre autre: les méthodes de discrétisation des EDOs et EDPs, 
les équations d'advections-diffusion-réactions, 
quelques notions et outils pour analyser les schémas numériques
et enfin la multi-résolution adaptative.\par
\textbf{Le troisième chapitre } propose trois contributions :
\begin{enumerate}
    \item Une étude (en \ref{par:contrib_1}) de la stabilité de deux méthodes ImEx ARK suivi d'une comparaison avec un schéma de splitting de leur stabilité et convergence pour 
    l'équation-test dite de Nagumo.
    \item Une étude théorique (en \ref{par:contrib_2}), via le calcul d'équations équivalentes, du comportement de l'erreur d'une schéma méthode des lignes pour un problème de diffusion.
    Cela dans trois contextes: \texttt{(I)} sans multi-résolution adaptative, \texttt{(II)} avec MRA standard, \texttt{(III)} avec MRA non-standard (reconstruction des flux).
    \item Une étude expérimentale (en \ref{par:contrib_3}) des différences entre les trois schéma de MRA mentionné précédemment et une mise en relation des résultats théoriques de la contributions précédentes 
    et des observations expérimentales.
\end{enumerate}\par
\textbf{La conclusion } se décline en trois parties thématiques: scientifique, technique et personnelle.