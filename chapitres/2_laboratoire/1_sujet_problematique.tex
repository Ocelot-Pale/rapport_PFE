% Ce travail participe à l'élaboration de méthodes numériques pour l'approximation des équations aux dérivées partielles d'évolution.
% En particulier les équations d'advection-diffusion-réaction (présentation en \ref{par:adv-diff-reaction}). Elles décrivent par exemple les systèmes physiques couplant
% mécanique des fluides, thermodynamique et réactions chimiques\footnote{Typiquement des problèmes de combustion.}.
% Ces équations sont difficiles à simuler du fait de leur caractère multi-échelle
% \footnote{Une réaction chimique a des temps et distances typiques généralement plusieurs ordres de grandeur plus faibles que les temps et distances typiques de la mécanique des fluides.}.
% Pour gérer les différentes échelles spatiales, des méthodes de compression de maillage sont souvent mises en oeuvre. 
% La méthode de compression utilisée et étudiée ici est la multirésolution adaptative \cite{harten1994}.
% Les différentes échelles temporelles\footnote{En termes techniques, les différents termes des équations étudiées ont des raideurs très différentes.}
% sont usuellement gérées par force brute ou par séparation d'opérateurs. 
% Pour pallier le problème de la large gamme d'échelles temporelles rencontrées, une approche hybride est ici étudiée: les méthodes implicites-explicites (ImEx) \cite{ASCHER1997151}.
% Ce travail vise donc principalement à comprendre comment la multirésolution adaptative interagit avec les différentes méthodes d'intégration temporelle.
% Il s'intéresse aux questions suivantes:
% \begin{itemize}
% \item[$\diamond$] {Comment les effets de la compression de maillage par multirésolution adaptative (MRA) sur les solutions numériques
% dépendent-ils du problème étudié et de la méthode numérique sur lesquelles elle se greffent ?}
% \item[$\diamond$] {Comment évoluent les propriétés des méthodes ImEx selon les caractéristiques des opérateurs des équations de diffusion-réaction}
%                 \footnote{Même si l'objectif est bien les équations d'advection-diffusion-réaction, l'étude s'est concentrée par simplicité sur l'interaction entre phénomènes de diffusion et de réactions.}
%                 { ?}
% \end{itemize}
L'objet initial du stage était d'étudier l'usage conjoint de méthodes implicites-explicites (ImEx) et de l'adaptation de maillage par 
multirésolution adaptative (MRA) pour la simulation des équations d'advections-diffusion-réaction.
Cependant, mes explorations scientifique menèrent mes recherches plutôt vers l'étude de l'erreur liée à la MRA sur l'opérateur de diffusion.   
\subsection{Les équations d'advections-réaction-diffusion}
    \subsubsection{Application des équations d'ADR}
    Les equations d'ADR sont une gamme d'équations au dérivées partielles (EDP) comportant un opérateur d'advection (transport par un flux),
    un opérateur de diffusion (éparpillement de la matière par l'agitation thermique) et un opérateur de réaction (modélisant une ou plusieurs réactions chimiques).
    Ces EDPs modélisent l'interaction entre la mécanique des fluides et des réactions chimiques: les différentes espèces chimiques sont advectées par l'écoulement et diffusent au sein du fluide;
    selon de leur répartition elles réagissent entre elles, ce qui modifie en retour les propriétés du fluide (température, pression, masse volumique...). 
    Ces EDPs sont entre-autre au coeur des problèmes de combustion (moteurs), d'explosion (militaire, sécurité industrielle), d’électrochimie (batteries, industrie), 
    et défluents (industrie polluantes).
    \subsubsection{Les difficultés des équations d'ADR et stratégies numériques}
    Les équations d'ADR présente deux difficultés majeures aux numériciens: le caractère multi-échelle des solutions et les propriétés antagonists des opérateurs quelles mets en jeu
    (raideur, localités différentes). Plus de détails sont donnés en \ref{}.\par

    \textbf{La première difficulté} pousse à utiliser des stratégies d'adaptation de maillage. L'objectif est de disposer d'un maillage fin seulement là ou la complexité de la solution le requiert.
    Plusieurs techniques existent et se déclinent en deux grandes catégories\cite{}:
    \begin{enumerate}
        \item Les méthodes \textit{features-based} qui utilise une quantité locale de la solution et ses gradients pour prédire la \textit{complexité} de la solution en chaque point 
        et adapter en conséquence.
        \item Les méthodes \textit{goal-based} qui cherchent à optimiser une fonction de coût reflétant la qualité générale de la solution selon l'adaptation choisie.
        Généralement la fonction de coût est issue d'un problème adjoint sur une quantité d’intérêt.
    \end{enumerate}
    Le stage ce centre sur la multi-résolution adaptative \cite{}, une méthode d'adaptation s'appuyant sur une transformée en ondelette.
    Elle ne tombe dans aucune des deux catégories précédente, étant basée avant tout sur des arguments issus de la théorie de l'information.\par

    \textbf{La seconde difficulté} décourage l'usage de schémas monolithiques qui traiteraient tous les opérateurs de la même manière 
    (leur properties étant très distinctes aucun une schéma monolithique ne répondra aux besoins des trois opérateurs en même temps).
    Cela motive l'emploi de schémas découplant les traitements subis par chaque opérateur. Parmi les méthodes les plus établies se trouvent 
    les techniques de séparation d'opérateurs et notre objet d'étude: les méthodes ImEx.

\subsection{Motivation du sujet initial et de l'orientation effective des recherches}
    \subsubsection{Motivation du sujet initial}
    Le sujet initial était d'étudier l'usage conjoint des méthodes ImEx et de la MRA pour trancher l’existence ou non de couplages délétères entre ces deux méthodes.\par 

    \paragraph{L'histoire de la MRA}
    L'origine de la multirésolution adaptative remonte aux années 1990 et les travaux d'Ami Harten \cite{harten1994} et a été prolongé entre autre dans \cite{Kaibara2001,Cohen2003}.
    Cette technique permettant de réduire drastiquement les temps de simulation a été largement appliquée \cite{}...

    \paragraph{Le découplage des opérateurs, les méthodes ImEx}
    La nécessité de résoudre des équations d'évolutions dont les opérateurs présentaient des propriétés antagonistes mena aux premières méthodes ImEx dans la fin des années 1990 
    avec notamment les méthodes Runge et Kutta additives (ARK) par Asher \textit{et al} \cite{ASCHER1997151} complétés dans les année 2000 par \cite{KENNEDY2003139} puis par \cite{FITZHUGH1961445}. 
    L'objectif étant de dépasser la séparation d'opérateurs \cite{Strang1968} qui se généralise mal au delà de l'ordre deux et deux deux opérateurs \cite{}% trouver des justifications...
    Cela a pavé la voie à des méthodes ImEx plus complexes intégrant par exemples des méthodes stabilisées \cite{Abdulle2013} 
    ou encore a des méthodes ImEx, couplées espace-temps, faites sur mesures pour une équation spécifique \cite{rebou2024}.

    \paragraph{Interactions entre adaptation spatiale et le découplage des opérateurs}

    \subsubsection{Une bifurcation en cours de recherche}
        Comme mentionné précédemment,j'ai bifurqué vers un autre sujet au cours de mes recherches. Je détaille donc les circonstances et motivations de ce changement de trajectoire.\par 

        J'ai commencé par me familiariser avec méthodes ImEx (ARK) en les comparants à la séparation d'opérateurs en terme de stabilité et de convergence sur une équation simple.
        
        Puis, comme je devais étudier le couplage ImEx et MRA de façon fine, mes tuteurs m'ont conseillé pour m’entraîner, 
        d'étudier d'abord l'impact théorique de la MRA sur une simple équation de diffusion, sans schéma en temps complexe (pas d'ImEx).

        J'ai étudié donc ce cas avec \texttt{(I)} une MRA standard mais j'ai également exploré \texttt{(II)} une implémentation non-standard de la MRA, où bien que la solution soit compressée, 
        les valeurs servant à l'évaluation des flux numériques sont systématiquement reconstruites (décompressions locales).
        La différence n'est à ma connaissance pas étudiée par la littérature, 
        c'est un sujet qui a récemment été soulevé au CMAP et étudiée sur les problèmes d’advections \cite{belloti_et_al_2025} 
        concluant que l'implémentation non-standard (avec reconstruction des flux) était plus précise.
        \textbf{Cependant} mon analyse théorique sur l'équation de diffusion a donnés des résultats opposés à ceux obtenus sur les problems d'advection.
        Intrigué par ces résultats j'ai voulu les confirmer, les documenter et les expliquer. De fait cette découverte inopinée m'a fait bifurquer vers un autre objet d'étude: 
        l'impact du niveau d'évaluation du flux numériques sur les problèmes de diffusion et de diffusion-réaction. 
