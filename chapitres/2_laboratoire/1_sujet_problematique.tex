% Ce travail participe à l'élaboration de méthodes numériques pour l'approximation des équations aux dérivées partielles d'évolution.
% En particulier les équations d'advection-diffusion-réaction (présentation en \ref{par:adv-diff-reaction}). Elles décrivent par exemple les systèmes physiques couplant
% mécanique des fluides, thermodynamique et réactions chimiques\footnote{Typiquement des problèmes de combustion.}.
% Ces équations sont difficiles à simuler du fait de leur caractère multi-échelle
% \footnote{Une réaction chimique a des temps et distances typiques généralement plusieurs ordres de grandeur plus faibles que les temps et distances typiques de la mécanique des fluides.}.
% Pour gérer les différentes échelles spatiales, des méthodes de compression de maillage sont souvent mises en oeuvre. 
% La méthode de compression utilisée et étudiée ici est la multirésolution adaptative \cite{harten1994}.
% Les différentes échelles temporelles\footnote{En termes techniques, les différents termes des équations étudiées ont des raideurs très différentes.}
% sont usuellement gérées par force brute ou par séparation d'opérateurs. 
% Pour pallier le problème de la large gamme d'échelles temporelles rencontrées, une approche hybride est ici étudiée: les méthodes implicites-explicites (ImEx) \cite{ASCHER1997151}.
% Ce travail vise donc principalement à comprendre comment la multirésolution adaptative interagit avec les différentes méthodes d'intégration temporelle.
% Il s'intéresse aux questions suivantes:
% \begin{itemize}
% \item[$\diamond$] {Comment les effets de la compression de maillage par multirésolution adaptative (MRA) sur les solutions numériques
% dépendent-ils du problème étudié et de la méthode numérique sur lesquelles elle se greffent ?}
% \item[$\diamond$] {Comment évoluent les propriétés des méthodes ImEx selon les caractéristiques des opérateurs des équations de diffusion-réaction}
%                 \footnote{Même si l'objectif est bien les équations d'advection-diffusion-réaction, l'étude s'est concentrée par simplicité sur l'interaction entre phénomènes de diffusion et de réactions.}
%                 { ?}
% \end{itemize}
L'objet initial du stage était d'étudier l'usage conjoint de méthodes implicites-explicites (ImEx) et de l'adaptation de maillage par 
multirésolution adaptative (MRA) pour la simulation des équations d'advections-diffusion-réaction.
Cependant, mes explorations scientifique menèrent mes recherches plutôt vers l'étude de l'erreur liée à la MRA sur l'opérateur de diffusion.   
\subsection{Les équations d'advections-réaction-diffusion}
    \subsubsection{Application des équations d'ADR}
    Les equations d'ADR sont une gamme d'équations au dérivées partielles (EDP) comportant un opérateur d'advection (transport par un flux),
    un opérateur de diffusion (éparpillement de la matière par l'agitation thermique) et un opérateur de réaction (modélisant une ou plusieurs réactions chimiques).
    Ces EDPs modélisent l'interaction entre la mécanique des fluides et des réactions chimiques: les différentes espèces chimiques sont advectées par l'écoulement et diffusent au sein du fluide;
    selon de leur répartition elles réagissent entre elles, ce qui modifie en retour les propriétés du fluide (température, pression, masse volumique...). 
    Ces EDPs sont entre-autre au coeur des problèmes de combustion (moteurs), d'explosion (militaire, sécurité industrielle), d'électrochimie (batteries, industrie), 
    et d'effluents (industrie polluantes).
    \subsubsection{Les difficultés des équations d'ADR et stratégies numériques}
    Les équations d'ADR présente deux difficultés majeures aux numériciens: le caractère multi-échelle des solutions et les propriétés antagonistes des opérateurs quelles mets en jeu
    (raideur, localités différentes - \textit{cf.} \ref{par:adv-diff-reaction}).\par

    \textbf{La première difficulté} pousse à utiliser des stratégies d'adaptation de maillage pour disposer d'un maillage précis seulement où cela est nécessaire à la représentation fidèle de solution
    \footnote{Si la solution est très régulière par endroit, il n'est pas nécessaire d'avoir des mailles trop fines, c'est un coût en calcul est en mémoire non-nécessaire.}.
    Plusieurs techniques d'adaptation existent et se déclinent en deux grandes catégories\cite{Vivarelli2025Fluids}:
    \begin{enumerate}
        \item Les méthodes \textit{features-based} utilisant une quantité locale de la solution (par exemple la vitesse de l'écoulement) 
        et ses gradients pour inférer la \textit{complexité} de la solution en chaque point et adapter le maillage en conséquence.
        \item Les méthodes \textit{goal-based} optimisant une fonction de coût reflétant la qualité globale de la solution selon l'adaptation choisie.
        Généralement la fonction de coût est issue d'un problème adjoint sur une quantité d'intérêt.
    \end{enumerate}
    Le stage se centre sur la multi-résolution adaptative \cite{harten1994}, une méthode d'adaptation s'appuyant sur une transformée en ondelette.
    Elle ne tombe dans aucune des deux catégories précédente, étant surtout fondée avant tout sur des arguments issus de la théorie de l'information 
    (\textit{cf.} \ref{par:explication_MRA}).\par

    \textbf{La seconde difficulté} décourage l'usage de schémas d'intégration en temps monolithiques (d'un seul bloc) qui traiteraient tous les opérateurs de la même manière.
    Leurs propriétés étant très distinctes aucun schéma monolithique ne répondra aux besoins des trois opérateurs en même temps.
    Cela motive l'emploi de schémas découplant les traitements subis par chaque opérateur. Parmi les méthodes les plus établies se trouvent 
    les techniques de séparation d'opérateurs (\textit{splitting}) et notre objet d'étude: les méthodes ImEx.

\subsection{Motivation du sujet initial et orientation effective des recherches}
    \subsubsection{Motivation du sujet initial}
    Le sujet initial était d'étudier l'usage conjoint des méthodes ImEx et de la MRA pour déterminer l'existence ou non de couplages délétères entre ces deux méthodes.\par 

    \paragraph{Histoires et étude de la MRA}
    L'origine de la multirésolution adaptative remonte aux années 1990 et les travaux d'Ami Harten \cite{harten1994} et a été prolongé entre autre dans \cite{Kaibara2001,Cohen2003}.
    Cette technique permettant de réduire drastiquement les temps de simulation a été largement appliquée \cite{}...

    \paragraph{Le découplage des opérateurs, les méthodes ImEx}
    La nécessité de résoudre des équations d'évolutions dont les opérateurs présentaient des propriétés antagonistes mena aux premières méthodes ImEx dans la fin des années 1990 
    avec notamment les méthodes Runge et Kutta additives (ARK) par Asher \textit{et al} \cite{ASCHER1997151} complétés dans les année 2000 par \cite{KENNEDY2003139} puis par \cite{FITZHUGH1961445}. 
    L'objectif étant de dépasser la séparation d'opérateurs \cite{Strang1968} qui se généralise mal au delà de l'ordre deux et deux deux opérateurs \cite{}% trouver des justifications...
    Cela a pavé la voie à des méthodes ImEx plus complexes intégrant par exemples des méthodes stabilisées \cite{Abdulle2013} 
    ou encore a des méthodes ImEx, couplées espace-temps, faites sur mesures pour une équation spécifique \cite{rebou2024}.

    \paragraph{Interactions entre adaptation spatiale et le découplage des opérateurs}
        L'interaction entre le \textit{splitting} et la MRA a été explicitement traité dans la thèse de Max Duarte \ref{duart2011}.    
        Toutefois l'interaction entre les méthodes d'adaptation en espace et les schémas découplant les traitements infligés à chaque opérateur (ImEx, \textit{splitting})
        reste peu traité dans la literature, alors même que cette problématique transparaît dans de nombreux articles.
        Par exemple \cite{Zhang2025IMEXTSA} traite d'une méthode ImEx adapté en espace et en temps et l'on voit dans leur résultat qu'il y a un fort 
        couplage entre l'erreur liée à l'adaptation en temps et celle liée à l'adaptation en espace
        \footnote{C'est à dire que si l'on note $E_T$ l'erreur du schéma adapté en temps et $E_S$ l'erreur du schéma adapté en espace,
        alors l'erreur du schéma adapté en espace et en temps $E_{ST}$ ne vérifie pas du tout $E_{ST}\approx E_S + E_T$ mais plutôt $E_{ST} \gg E_S + E_T$.}. 
        On y distingue bien la problématique du couplage entre les traitements en temps et en espace mais ce n'est pas clairement étudié.
        \textbf{Ainsi l'étude d'éventuel mécanismes de couplages entre les méthodes ImEx et la MRA est très pertinent car elle comble un trou dans la literature et permet de justifier (ou non) une combinaison de méthodes très efficace sur le papier pour les équations d'ADR.}
    \subsubsection{Une bifurcation en cours de recherche}
        Comme mentionné précédemment, mes recherches ont bifurqué vers un sujet voisin. Voici circonstances et motivations de ce changement de trajectoire:
        Puisque je devais étudier le couplage ImEx et MRA de façon fine, mes tuteurs m'ont conseillé 
        de d'abord m'entraîner sur une cas simplifié: une simple équation de diffusion résolu avec MRA.
        J'ai donc étudié théoriquement l'erreur sur \texttt{(I)} une MRA standard d'une part. Et d'autre part sur \texttt{(II)} schéma non-standard de MRA, où la solution compressée mais les valeurs servant à l'évaluation des flux numériques sont systématiquement reconstruites (décompressions locales); ce qui n'est pas le cas dans le schéma \texttt{I}.
        La différence entre les deux schéma n'est à ma connaissance pas étudiée par la littérature;
        c'est un sujet qui a récemment émergé au CMAP et qui a été étudié en \cite{belloti_et_al_2025} pour les problèmes d'advections 
        concluant que l'implémentation non-standard (avec reconstruction des flux) était plus précise.
        \textbf{Cependant} l'analyse pour la diffusion a mené à des conclusions opposés.
        Intrigué par ces résultats j'ai voulu les confirmer, les documenter et les expliquer.
        De fait cette découverte inopinée m'a fait bifurquer vers un autre objet d'étude: 
        l'impact du niveau d'évaluation du flux numériques sur les problèmes de diffusion et de diffusion-réaction.
    \newpage
    \section{Problématique principale}
    Avec un schéma MRA, la solution est donc représentée sur plusieurs grilles de finesses différentes.
    L'adaptation (compression) consiste à ignorer l'information contenue sur les grilles les plus fines là où la solution est assez régulière;
    un niveau de résolution maximal (grille fine) pourra être retrouvé (moyennant une erreur de compression) grâce à une reconstruction (décompression) par 
    interpolation polynomiale.
    Sur les schémas \texttt{volumes-finis + MRA} standards, les flux numériques sont évalués à partir de l'information sur le maillage adapté (compressé) alors
    que les non-standards reconstruisent (décompressent) l'information nécessaire au calcul des flux à un niveau de représentation plus fin. Les algorithmes non-standards 
    étant plus coûteux en calculs mais potentiellement plus précis.
    La problématique dominante du stage est donc: \textbf{Quel est l'impact du niveau d'évaluation des flux numériques des schéma volumes finis avec multi-résolution adaptative
    pour les problèmes de diffusion et de réaction diffusion.}