\subsection{Le laboratoire.}
Le Centre de Mathématiques Appliquées de l'École Polytechnique\footnote{\href{https://cmap.ip-paris.fr}{https://cmap.ip-paris.fr}} (CMAP) a été créé en 1974.
Cette création répond au besoin émergent de mathématiques appliquées face au développement des méthodes de conception et de simulation par calcul numérique dans de nombreuses applications industrielles de l'époque (nucléaire, aéronautique, recherche pétrolière, spatial, automobile).
Le laboratoire fut fondé grâce à l'impulsion de trois professeurs : Laurent \textsc{Schwartz}, Jacques-Louis \textsc{Lions} et Jacques \textsc{Neveu}. Jean-Claude \textsc{Nédélec} en fut le premier directeur, et la première équipe de chercheurs associés comprenait P.A. \textsc{Raviart}, P. \textsc{Ciarlet}, R. \textsc{Glowinski}, R. \textsc{Temam}, J.M. \textsc{Thomas} et J.L. \textsc{Lions}. 
Les premières recherches se concentraient principalement sur l'analyse numérique des équations aux dérivées partielles.
Le CMAP s'est diversifié au fil des décennies, intégrant notamment les probabilités dès 1976, puis le traitement d'images dans les années 1990 et les mathématiques financières à partir de 1997. 

Le laboratoire a formé plus de 230 docteurs depuis sa création et a donné naissance à plusieurs \textit{startups} spécialisées dans les applications industrielles des mathématiques appliquées.
Le CMAP comprend trois pôles  de recherche: le pôle analyse, le pôle probabilités et le pôle décision et données.
J'ai intégré l'équipe \textbf{HPC@Maths} du \textbf{pole analyse}.
\subsection{L'équipe HPC@Math}
    L'équipe HPC@Math\footnote{\href{https://initiative-hpc-maths.gitlab.labos.polytechnique.fr/site/index.html}{https://initiative-hpc-maths.gitlab.labos.polytechnique.fr/site/index.html}} 
    travaille à l'interface des mathématiques, de l'informatique et de la physique (mécanique des fluides, thermodynamique) pour développer 
    des méthodes numériques efficaces pour la simulation des EDP. 
    L'équipe s'intéresse entre-autre aux problèmes multi-échelles, au travers des équations d'advection-réaction-diffusion.
    Les travaux de l'équipe s'inscrive toujours dans un contexte HPC (high performance computing). 
    Ce terme désigne l'usage optimal des ressources informatiques disponibles et de leur architectures. 
    Chaque méthode est développée en se demandant si elle pourra être facilement parallélisé, si elle pourra être déployée sur GPU ou sur un cluster de calcul de manière efficace. 