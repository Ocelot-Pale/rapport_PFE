\subsection{Historique et activités}
Le Centre de Mathématiques Appliquées de l'École Polytechnique\footnote{\href{https://cmap.ip-paris.fr}{https://cmap.ip-paris.fr}} (CMAP) a été créé en 1974 lors du déménagement de l'École Polytechnique vers Palaiseau. 
Cette création répond au besoin émergent de mathématiques appliquées face au développement des méthodes de conception et de simulation par calcul numérique dans de nombreuses applications industrielles de l'époque(nucléaire, aéronautique, recherche pétrolière, spatial, automobile).
Le laboratoire fut fondé grâce à l'impulsion de trois professeurs : Laurent \textsc{Schwartz}, Jacques-Louis \textsc{Lions} et Jacques \textsc{Neveu}. Jean-Claude \textsc{Nédélec} en fut le premier directeur, et la première équipe de chercheurs associés comprenait P.A. \textsc{Raviart}, P. \textsc{Ciarlet}, R. \textsc{Glowinski}, R. \textsc{Temam}, J.M. \textsc{Thomas} et J.L. \textsc{Lions}. 
Les premières recherches se concentraient principalement sur l'analyse numérique des équations aux dérivées partielles.
Le CMAP s'est diversifié au fil des décennies, intégrant notamment les probabilités dès 1976, puis le traitement d'images dans les années 1990 et les mathématiques financières à partir de 1997. 
Le laboratoire a formé plus de 230 docteurs depuis sa création et a donné naissance à plusieurs startups spécialisées dans les applications industrielles des mathématiques appliquées.

\subsection{La recherche au CMAP}
Le CMAP comprend trois pôles  de recherche: le pôle analyse, le pôle probabilités et le pôle décision et données. Chaque pôle acceuil en son sein plusieurs équipes :
\begin{enumerate}
    \item \textbf{Analyse}
        \begin{itemize}
            \item[$\diamond$] EDP pour la physique.
            \item[$\diamond$] Mécanique, Matériaux, Optimisation de Formes.
            \item[$\diamond$] HPC@Maths (calcul haute performance).
            \item[$\diamond$] PLATON (quantification des incertitudes en calcul scientifique), avec l'INRIA.
        \end{itemize}
    \item \textbf{Probabilités}
        \begin{itemize}
            \item[$\diamond$] Mathématiques financières.
            \item[$\diamond$] Population, système particules en interaction.
            \item[$\diamond$] ASCII (interactions stochastiques coopératives), avec l'INRIA.
            \item[$\diamond$] MERGE (évolution, reproduction, croissance et émergence), avec l'INRIA.
        \end{itemize}
    \item \textbf{Décision et données}
        \begin{itemize}
            \item[$\diamond$] Statistiques, apprentissage, simulation, image.
            \item[$\diamond$] RandOpt (optimisation aléatoire).
            \item[$\diamond$] Tropical (algèbre $(\max , +)$), avec l’INRIA.
        \end{itemize}
\end{enumerate}
J'ai intégré l'équipe \textbf{HPC@Maths} \textbf{pole analyse}.
De nombreuses équipe sont partagées entre le CMAP et l'INRIA ce qui démontre l'aspect appliqué du laboratoire.

\subsection{L'équipe HPC@Math et l'envrionnement de travail}
\paragraph{L'équipe HPC@Math}
    L'équipe HPC@Math
\paragraph{Envrionnement de travail}
