\subsubsection{Équation équivalente du schéma sans MRA}
    \begin{align}
        \frac{\partial u}{\partial t}  =&+ D \frac{\partial^{2}u}{\partial x^{2}} 
        + \Delta x^{2} \frac{D}{12}             \frac{\partial^{4}u}{\partial x^{4}} 
        -  \Delta t^{2} \frac{D^{3}}{6}          \frac{\partial^{6}u}{\partial x^{6}} 
        -  \Delta t^{3} \frac{D^{4}}{24}        \frac{\partial^{8}u}{\partial x^{8}}.
    \end{align}
    Le schéma de base est donc bien d'ordre deux en espace et en temps.
    En supposant une relation de stabilité du type $\lambda = \frac{D\, \Delta t }{\Delta x^2}$:
    \begin{align}
        \frac{\partial u}{\partial t}  =&\; D \frac{\partial^{2}u}{\partial x^{2}} 
        + \Delta x^{2} \frac{D}{12}             \frac{\partial^{4}u}{\partial x^{4}}
        - \Delta x^{4} \frac{D \lambda^2}{6}     \frac{\partial^{6}u}{\partial x^{6}}
        - \Delta x^{6} \frac{D \lambda^3}{24}    \frac{\partial^{8}u}{\partial x^{8}}.
    \end{align}

\subsubsection{Équation équivalente du schéma avec MRA}
        \begin{align}
        \frac{\partial u}{\partial t} =&\; D \frac{\partial^2 u}{\partial x^2} \\\notag
        &- \frac{\Delta t}{2} D^2\, \bigl( 2^{2\Delta l}- 1 \bigr)          \frac{\partial^4 u}{\partial x^4}
        - \Delta t^2\, \frac{D^3}{6}          \frac{\partial^6 u}{\partial x^6}
        - \Delta t^3\, \frac{D^4}{24}         \frac{\partial^8 u}{\partial x^8} \\\notag
        &+ 2^{2\Delta l}\, \frac{D\, \Delta x^2}{12}    \frac{\partial^4 u}{\partial x^4}
        - 2^{2\Delta l}\, \frac{D\, \Delta l\, \Delta x^2}{4} \frac{\partial^4 u}{\partial x^4}
    \end{align}
    Nous constatons que formellement le schéma est formellement d'ordre un.
    Cela suggère donc que théoriquement, la multirésolution devrait faire perdre l'ordre de convergence temporelle de la méthode des lignes.
    Cependant en pratique pour des raisons de stabilité on impose $\lambda= \frac{D \Delta t } {\Delta x^2}$.
    Cela masque la perte en ordre\footnote{Cela à été vérifié expérimentalement} puisque cela mène à l'éqution équivalente:

    \begin{align}
        \frac{\partial u}{\partial t}
        =&+ D \frac{\partial^{2}u}{\partial x^{2}}\\\notag
        &+ \Delta x^{2}\Bigl(\frac{2^{2 \Delta l} D  \lambda \frac{\partial^{4}u}{\partial x^{4}}}{2} 
        -  \frac{2^{2 \Delta l} D \Delta l \frac{\partial^{4}u}{\partial x^{4}}}{4} 
        -  \frac{D \lambda \frac{\partial^{4}u}{\partial x^{4}}}{2} 
        +  \frac{2^{2 \Delta l} D\frac{\partial^{4}u}{\partial x^{4}}}{12}\Bigr)\\\notag
        &- \Delta x^{6} \frac{D \lambda^{3} \frac{\partial^{8}u}{\partial x^{8}}}{24} 
        - \Delta x^{4} \frac{D \lambda^{2} \frac{\partial^{6}u}{\partial x^{6}}}{6} 
    \end{align}