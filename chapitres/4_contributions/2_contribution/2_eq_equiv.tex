\subsubsection{Calcul des équations équivalentes}
Tout le calculs ont été réalisés grâce à la bibliothèque de calcul formel \texttt{Sympy} et les codes 
sont disponibles à l'adresse: \href{https://github.com/Ocelot-Pale/etude_MR_RK2}{\nolinkurl{https://github.com/Ocelot-Pale/etude_MR_RK2}}.
\paragraph{Équation équivalente du schéma sans MRA}
    Le calcul de l'équation équivalente sans MRA donne:
    \begin{align}
        \frac{\partial u}{\partial t}  =&+ D \frac{\partial^{2}u}{\partial x^{2}} 
        + \Delta x^{2} \frac{D}{12}             \frac{\partial^{4}u}{\partial x^{4}} 
        -  \Delta t^{2} \frac{D^{3}}{6}          \frac{\partial^{6}u}{\partial x^{6}} 
        -  \Delta t^{3} \frac{D^{4}}{24}        \frac{\partial^{8}u}{\partial x^{8}}.
    \end{align}
    Le schéma de base est donc bien d'ordre deux en espace et en temps.
    En supposant une relation de stabilité du type $\lambda = \frac{D\, \Delta t }{\Delta x^2}$:
    \begin{align}
        \frac{\partial u}{\partial t}  =&\; D \frac{\partial^{2}u}{\partial x^{2}} 
        + \Delta x^{2} \frac{D}{12}             \frac{\partial^{4}u}{\partial x^{4}}
        - \Delta x^{4} \frac{D \lambda^2}{6}     \frac{\partial^{6}u}{\partial x^{6}}
        - \Delta x^{6} \frac{D \lambda^3}{24}    \frac{\partial^{8}u}{\partial x^{8}}.
    \end{align}

\paragraph{Équation équivalente du schéma avec MRA}
    Dans le cas avec MRA, le logiciel de calcul formel donne l'équation équivalente:
    \begin{align}
        \frac{\partial u}{\partial t} =&\; D \frac{\partial^2 u}{\partial x^2} \\\notag
        &- \frac{\Delta t}{2} D^2\, \bigl( 2^{2\Delta l}- 1 \bigr)          \frac{\partial^4 u}{\partial x^4}
        - \Delta t^2\, \frac{D^3}{6}          \frac{\partial^6 u}{\partial x^6}
        - \Delta t^3\, \frac{D^4}{24}         \frac{\partial^8 u}{\partial x^8} \\\notag
        &+ 2^{2\Delta l}\, \frac{D\, \Delta x^2}{12}    \frac{\partial^4 u}{\partial x^4}
        - 2^{2\Delta l}\, \frac{D\, \Delta l\, \Delta x^2}{4} \frac{\partial^4 u}{\partial x^4}
    \end{align}
    Nous constatons que formellement le schéma est formellement d'ordre un en temps.
    Cela suggère donc que théoriquement, la multirésolution devrait faire perdre l'ordre de convergence temporelle de la méthode des lignes.
    Cependant en pratique la contrainte de stabilité impose une relation type CFL $\lambda = \frac{D \Delta t } {\Delta x^2} < 1/2$.
    Cela masque la perte en ordre\footnote{Cela à été vérifié expérimentalement} puisque cela mène à l’élution équivalente:

    \begin{align}
        \frac{\partial u}{\partial t}
        =&+ D \frac{\partial^{2}u}{\partial x^{2}}\\\notag
        &+ \Delta x^{2}\Bigl(\frac{2^{2 \Delta l} D  \lambda \frac{\partial^{4}u}{\partial x^{4}}}{2} 
        -  \frac{2^{2 \Delta l} D \Delta l \frac{\partial^{4}u}{\partial x^{4}}}{4} 
        -  \frac{D \lambda \frac{\partial^{4}u}{\partial x^{4}}}{2} 
        +  \frac{2^{2 \Delta l} D\frac{\partial^{4}u}{\partial x^{4}}}{12}\Bigr)\\\notag
        &- \Delta x^{6} \frac{D \lambda^{3} \frac{\partial^{8}u}{\partial x^{8}}}{24} 
        - \Delta x^{4} \frac{D \lambda^{2} \frac{\partial^{6}u}{\partial x^{6}}}{6} 
    \end{align}
\subsubsection{Comparaison}
    Un ordre de précision a été perdu en temps. Nous essayons a présent de comprendre quel mécanisme mène à cette baisse de performances.
    Ma démarche consiste à comparer les équations équivalentes avec et sans multirésolution, \textbf{avant} d'appliquer la procéduer de Cauchy-Kovaleskaya.
    \paragraph{Sans MRA}
        L'équation modifiée sans multirésolution, avant procédure de Cauchy-Kovaleskaya est:
        \begin{align}
            \frac{\partial u}{\partial t}  &= D \frac{\partial^{2} u}{\partial x^{2}} \\\notag
                &+ \frac{1}{2} \underbrace{\Bigl(D^{2}\frac{\partial^{4} u}{\partial x^{4}} - \frac{\partial^{2} u}{\partial t^{2}} \Bigr)}_{\substack{\text{Se compense par} \\ \text{la procédure de} \\ \text{Cauchy-Kovaleskaya}}} \Delta t
                + \frac{D}{12} \frac{\partial^{4} u}{\partial x^{4}}  \Delta x^{2}
                - \frac{1}{24} \frac{\partial^{4} u}{\partial t^{4}}  \Delta t^{3} 
                - \frac{1}{6}  \frac{\partial^{3} u}{\partial t^{3}}  \Delta t^{2}.
        \end{align}
        La méthode est bien d'ordre un, car à l'ordre un : $\frac{\partial u}{\partial t}  = D \frac{\partial^{2} u}{\partial x^{2}}$ et donc le terme $D^{2}\frac{\partial^{4} u}{\partial x^{4}} - \frac{\partial^{2} u}{\partial t^{2}}$
        se compense au cours de la procédure de Cauchy-Kovaleskaya.
    \paragraph{Avec MRA}
        L'équation modifiée avec multirésolution, avant procédure de Cauchy-Kovaleskaya est:
        \begin{align}
            \frac{\partial u}{\partial  t} =&D \frac{\partial^{2} u}{\partial x^{2}}\\\notag
            &+ \frac{\Delta t}{2} \underbrace{\Bigl(2^{2 \Delta l} D^{2}           \frac{\partial^{4} u}{\partial x^{4}} -\frac{\partial^{2} u}{\partial t^{2}} \Bigr)}_{\text{Ne se compensent plus.}}
            -\frac{\Delta t^{3}}{24}                          \frac{\partial^{4} u}{\partial t^{4}} 
            - \frac{\Delta t^{2}}{6}                           \frac{\partial^{3} u}{\partial t^{3}}
            +\frac{\Delta x^{2}}{12} (1 - 3\Delta l)    2^{2 \Delta l} D \frac{\partial^{4} u}{\partial x^{4}}
        \end{align}
        Dans ce cas le terme en facteur du $\Delta t$ ne se s'annule plus. En effet le terme $D^{2}\frac{\partial^{4} u}{\partial x^{4}}$ est devenu au cours de la reconstruction
        $2^{2 \Delta l} D^{2}\frac{\partial^{4} u}{\partial x^{4}}$. En conséquence, la méthode perds un ordre de convergence temporel.\par
        Ce mécanisme s'explique de la manière suivante: dans l'équation équivalente, le terme $\frac{\partial^{2} u}{\partial t^{2}}$ apparaît indépendamment de la discrétisation spatiale
        \footnote{Il emerge de la différence $u_k^{n+1} - u_k^{n}$ à $k$ fixé.}. La méthode des lignes initiale crée un terme \textit{sur mesure} pour le compenser en approximant le terme spatial
        $D^{2}\frac{\partial^{4} u}{\partial x^{4}}$. Cependant au cours du processus de compression-reconstruction, cette approximation est entachée d'un facteur $2^{2 \Delta l}$.
        En d'autres termes le terme spatial construit pour compenser un terme temporel a été modifié par la multi-résolution, alors que le terme temporel lui n'est pas affecté par la multirésolution.
        Ainsi, les deux termes ne se compensent plus et l'ordre est perdu.
\subsubsection{Conclusion sur le résultat obtenu grâce aux équations équivalentes}
    Il a été ici mis en lumière que la multirésolution-adaptative appliquée à méthode des lignes très simple
    peut théoriquement mener à un couplage des erreurs espace-temps polluant l'ordre initial de la méthode.
    En particulier l'étape de reconstruction-reconstruction altère des termes spatiaux qui ne compensent plus certaines erreurs temporelles. 
    %AFAIRE -> FINIR LA CONCLUSION 
    % Pour l'heure il conviendrait d'étudier ce phénomène sur d'autre méthodes, d'autres équations et surtout de l'étudier expérimentalement.
    % Je ne parviens pas à mettre en lumière ce phénomène expérimentalement. 
    % D'abord ce n'est pas possible de reproduire fidèlement cette situation car en pratique la condition de stabilité force $\Delta t \propto \Delta x^2$. 
    % Ensuite j'ai essayé de mettre en lumière le phénomène sur une méthode implicite (RKI2 SDIRK) mais le phénomène n'a pas été observé. 
    % ...     à suivre    ... 
