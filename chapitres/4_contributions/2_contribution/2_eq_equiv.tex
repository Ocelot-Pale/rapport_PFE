\subsubsection{Calcul des équations équivalentes}
Tout le calculs ont été réalisés grâce à la bibliothèque de calcul formel \texttt{Sympy} et les codes 
sont disponibles à l'adresse: \href{https://github.com/Ocelot-Pale/etude_MR_RK2}{\nolinkurl{https://github.com/Ocelot-Pale/etude_MR_RK2}}.
\paragraph{Équation équivalente du schéma \texttt{I} - non adapté}
    Le calcul de l'équation équivalente sans MRA donne:
    \begin{align}
        \frac{\partial u}{\partial t}  =&D \frac{\partial^{2}u}{\partial x^{2}}
        + \Delta x^{2} \frac{D}{12}             \frac{\partial^{4}u}{\partial x^{4}} 
        -  \Delta t^{2} \frac{D^{3}}{6}          \frac{\partial^{6}u}{\partial x^{6}} 
        -  \Delta t^{3} \frac{D^{4}}{24}        \frac{\partial^{8}u}{\partial x^{8}}  + \mathcal{O}(\Delta x^4 , \Delta t^4).
    \end{align}
    
    Le schéma de référence est donc bien d'ordre deux en espace et en temps.
    En utilisant la constante constante de Von Neumann : $\lambda = D \frac{\Delta t}{\Delta x^2}$ : 
    \begin{align}\label{eq:ref:cfl}
        \frac{\partial u}{\partial t}  =&D \frac{\partial^{2}u}{\partial x^{2}}
        + \Delta x^{2} \frac{D}{12}             \frac{\partial^{4}u}{\partial x^{4}} 
        - \lambda^2 \Delta  x^{4} \frac{D}{6}          \frac{\partial^{6}u}{\partial x^{6}} 
        - \lambda^3 \Delta x^{6} \frac{D}{24}        \frac{\partial^{8}u}{\partial x^{8}}  + \mathcal{O}(\Delta x^7).
    \end{align}
\paragraph{Équation équivalente du schéma \texttt{II} - adapté \emph{sans} reconstruction des flux}
    Lorsque le schéma est adapté sans reconstruction des flux sur $\Delta l$ niveaux de détails,
    l'équation du équivalente est :
    \begin{align}\label{eq:sansRecons:brute}
        \frac{\partial}{\partial t} u=
            D \frac{\partial^{2}u}{\partial x^{2}}
            + (2^{\Delta l} \Delta x)^{2}  \frac{D}{12} \frac{\partial^{4}u}{\partial x^{4}}
            -\Delta t^{2} \frac{D^{3}}{6}   \frac{\partial^{6}u}{\partial x^{6}}
            -\Delta t^{3} \frac{D^{4} }{24} \frac{\partial^{8}u}{\partial x^{8}}
            + \mathcal{O}(\Delta x^4 , \Delta t^4).
    \end{align}
    En somme, sans reconstruction des flux, le schéma avec MRA se comporte comme le schéma de référence mais sur un maillage plus grossier. 
    La constante d'erreur en espace est effet de l'ordre de $2^{2\Delta l} \frac{D}{12} \frac{\partial^{4}u}{\partial x^{4}}$ 
    au lieu de $\frac{D}{12}\frac{\partial^{4}u}{\partial x^{4}}$.
    En injectant la constante de Von Neumann dans l'équation \eqref{eq:sansRecons:brute} :
    \begin{align}\label{eq:sansRecons:cfl}
        \frac{\partial}{\partial t} u=
            D \frac{\partial^{2}u}{\partial x^{2}}
            + (2^{\Delta l} \Delta x)^{2}  \frac{D}{12} \frac{\partial^{4}u}{\partial x^{4}}
            -\lambda^2 \Delta x^{4} \frac{D}{6}   \frac{\partial^{6}u}{\partial x^{6}}
            -\lambda^3 \Delta x^{6} \frac{D}{24} \frac{\partial^{8}u}{\partial x^{8}} + \mathcal{O}(\Delta x^7)
    \end{align}
\paragraph{Équation équivalente du schéma \texttt{III} - adapté \emph{avec} reconstruction des flux}
    En évaluant les flux à partir des cellules reconstruites au niveau le plus fin, l'équation équivalente est:
    \begin{align}\label{eq:equiv_brute_recons}
        \frac{\partial u}{\partial t} =&\; D \frac{\partial^2 u}{\partial x^2} \\\notag
        &- \frac{\Delta t}{2} D^2\, \bigl( 2^{2\Delta l}- 1 \bigr)          \frac{\partial^4 u}{\partial x^4}
        - \Delta t^2\, \frac{D^3}{6}          \frac{\partial^6 u}{\partial x^6}
        - \Delta t^3\, \frac{D^4}{24}         \frac{\partial^8 u}{\partial x^8} \\\notag
        &+ 2^{2\Delta l}\, \frac{D\, \Delta x^2}{12}    \frac{\partial^4 u}{\partial x^4}
        - 2^{2\Delta l}\, \frac{D\, \Delta l\, \Delta x^2}{4} \frac{\partial^4 u}{\partial x^4} + \mathcal{O}(\Delta x^4 , \Delta t^4).
    \end{align}
    Ce schéma est d'ordre un en temps contrastant avec l'ordre deux des deux schémas précédents.
    Ainsi, théoriquement la reconstruction au plus fin des flux réduit l'ordre de convergence temporelle de la méthode des lignes.
    % Cependant en pratique le caractère explicite de la méthode ERK2 impose la contrainte de stabilité $\lambda = \frac{D \Delta t } {\Delta x^2} < 1/2$
    % masquant la perte d'ordre puisque cela mène à l’élution équivalente:
    En utilisant la constante constante de Von Neumann, l'équation \eqref{eq:equiv_brute_recons} se réécrit : 
    \begin{align}\label{eq:equiv_cfl_recons}
        \frac{\partial u}{\partial t}
        =&+ D \frac{\partial^{2}u}{\partial x^{2}}\\\notag
        &+ \Delta x^{2}\, D \, \Bigl( 
        \frac{\lambda}{2} (2^{2 \Delta l} - 1) + \frac{2^{2 \Delta l} }{12} (1 - 3 \Delta l)
        \Bigr)\frac{\partial^{4}u}{\partial x^{4}}\\\notag
        & - \Delta x^{4} \frac{D \lambda^{2} \frac{\partial^{6}u}{\partial x^{6}}}{6} - \Delta x^{6} \frac{D \lambda^{3} \frac{\partial^{8}u}{\partial x^{8}}}{24}
        + \mathcal{O}(\Delta x^7). 
    \end{align}
\subsubsection{Comparaison}
    Pour comprendre le mécanisme menant à cette perte d'ordre,
    l'équation équivalente du schéma avec AMR et reconstruction des flux est comparée à celle du schéma de référence, 
    \textbf{avant} d'appliquer la procédure de Cauchy-Kovaleskaya; c'est à dire sans exploiter $\dt{u}=D \dxx{u}$.
    \paragraph{Sans multirésolution (schéma de référence)}
        L'équation modifiée sans multirésolution, avant procédure de Cauchy-Kovaleskaya est:
        \begin{align}
            \frac{\partial u}{\partial t}  &= D \frac{\partial^{2} u}{\partial x^{2}} \\\notag
                &+ \frac{1}{2} \underbrace{\Bigl(D^{2}\frac{\partial^{4} u}{\partial x^{4}} - \frac{\partial^{2} u}{\partial t^{2}} \Bigr)}_{\substack{\text{Se compense par} \\ \text{la procédure de} \\ \text{Cauchy-Kovaleskaya}}} \Delta t
                + \frac{D}{12} \frac{\partial^{4} u}{\partial x^{4}}  \Delta x^{2}
                - \frac{1}{24} \frac{\partial^{4} u}{\partial t^{4}}  \Delta t^{3} 
                - \frac{1}{6}  \frac{\partial^{3} u}{\partial t^{3}}  \Delta t^{2}
                + \mathcal{O}(\Delta x^4 , \Delta t^4)..
        \end{align}
        La méthode est bien d'ordre un, car à l'ordre un : $\frac{\partial u}{\partial t}  = D \frac{\partial^{2} u}{\partial x^{2}}$ et donc le terme $D^{2}\frac{\partial^{4} u}{\partial x^{4}} - \frac{\partial^{2} u}{\partial t^{2}}$
        se compense au cours de la procédure de Cauchy-Kovaleskaya.
    \paragraph{multirésolution avec reconstruction des flux}
        L'équation modifiée avec multirésolution et reconstruction des flux, sans appliquer procédure de Cauchy-Kovaleskaya est:
        \begin{align}
            \frac{\partial u}{\partial  t} =&D \frac{\partial^{2} u}{\partial x^{2}}\\\notag
            &+ \frac{\Delta t}{2} \underbrace{\Bigl(2^{2 \Delta l} D^{2}           \frac{\partial^{4} u}{\partial x^{4}} -\frac{\partial^{2} u}{\partial t^{2}} \Bigr)}_{\text{Ne se compensent plus.}}
            -\frac{\Delta t^{3}}{24}                          \frac{\partial^{4} u}{\partial t^{4}} 
            - \frac{\Delta t^{2}}{6}                           \frac{\partial^{3} u}{\partial t^{3}}
            +\frac{\Delta x^{2}}{12} (1 - 3\Delta l)    2^{2 \Delta l} D \frac{\partial^{4} u}{\partial x^{4}}
            + \mathcal{O}(\Delta x^4 , \Delta t^4).
        \end{align}
        Dans ce cas le terme en facteur du $\Delta t$ ne se s'annule plus. En effet le terme $D^{2}\frac{\partial^{4} u}{\partial x^{4}}$ est devenu au cours de la reconstruction
        $2^{2 \Delta l} D^{2}\frac{\partial^{4} u}{\partial x^{4}}$. En conséquence, la méthode perds un ordre de convergence temporel.\par
        Ce mécanisme s'explique de la manière suivante: dans l'équation équivalente, le terme $\frac{\partial^{2} u}{\partial t^{2}}$ apparaît indépendamment de la discrétisation spatiale
        \footnote{Il emerge de la différence $u_k^{n+1} - u_k^{n}$ à $k$ fixé.}. La méthode des lignes initiale crée un terme \textit{sur mesure} pour le compenser en approximant le terme spatial
        $D^{2}\frac{\partial^{4} u}{\partial x^{4}}$. Cependant au cours du processus de compression-reconstruction, cette approximation est entachée d'un facteur $2^{2 \Delta l}$.
        En d'autres termes le terme spatial construit pour compenser un terme temporel a été modifié par la multi-résolution, alors que le terme temporel lui n'est pas affecté par la multirésolution.
        Ainsi, les deux termes ne se compensent plus et l'ordre est perdu.
\subsubsection{Conclusion sur le résultat obtenu grâce aux équations équivalentes}
    Il a été ici mis en lumière que la reconstruction des flux au plus fin sur appliquée à méthode des lignes très simple
    peut théoriquement mener à un couplage des erreurs espace-temps polluant l'ordre initial de la méthode.
    Alors que cela n'arrive pas lorsque les flux sont évaluées plus grossièrement.
    En particulier l'étape de reconstruction-reconstruction altère des termes spatiaux qui ne compensent plus certaines erreurs temporelles et perturbent l'ordre de la méthode Runge et Kutta.