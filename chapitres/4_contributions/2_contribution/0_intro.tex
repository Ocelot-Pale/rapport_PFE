% La multirésolution adaptative (MRA) a démontré une grande efficacité expérimentalement.\label{par:intro_contrib_eq_equiv}
% Cependant, son impact sur la qualité des solutions obtenues n'est pas encore totalement compris mathématiquement.
% En \cite{belloti_et_al_2025}, une étude de l'erreur introduite par la multirésolution adaptative a été menée.
% Cette étude se concentre sur les équations d'advection résolues par des schémas de type \textit{one-step} \cite{DARU2004563} et
% compare l'équation équivalente des schémas avec MRA et celle des schémas sans MRA pour mettre en lumière les différences introduites par la multirésolution.
% La présente étude se place dans la continuité de cette démarche\footnote{À un niveau plus modeste.} et suit un cheminement similaire
% pour déterminer, grâce aux équations équivalentes, l'impact de la MRA sur une équation de diffusion résolue par une méthode des lignes d'ordre deux.
% La différence est donc double: d'une part la nature de l'opérateur étudié est différent (diffusion et non advection), d'autre part le type de schéma 
% utilisés est également différent (ici une méthode des lignes, ne couplant pas les erreurs en espace et en temps est utilisé).
% D'abord, le problème cible est présenté ainsi que le schéma de référence et le schéma avec AMR.
% Par la suite, les équations équivalentes du schéma de référence et du schéma avec multirésolution adaptative sont évaluées puis analysées.
% Enfin, une tentative de confirmation expérimentale est présentée.
Cette deuxième contribution étudie l'interaction entre la multirésolution adaptative (MRA) et le schéma d'intégration temporel, 
une interaction encore mal comprise. 
Une première analyse de ce type a été menée sur des problèmes d'advection en \cite{belloti_et_al_2025}. 
La présente étude se concentre sur les problèmes de diffusion. 
Pour analyser l'impact de la MRA, les équations équivalentes de trois schémas sont calculées (voir \ref{par:paradigme_MRA}) :

\begin{enumerate}
    \renewcommand{\labelenumi}{\Roman{enumi}.}
    \item Une méthode des lignes d'ordre deux servant de référence. Il s'agit d'une discrétisation spatiale d'ordre deux par volume finis, 
    puis une intégrations temporelle par une méthode Runge et Kutta explicite d'ordre deux.
    \item Une méthode correspondant au schéma \texttt{I} mais adapté en espace par MRA, \emph{où les valeurs nécessaires à l'évaluation des flux numériques 
    est évaluée au niveau de compression courant}.
    \item Une méthode correspondant au schéma \texttt{I} avec MRA, \emph{où les valeurs nécessaires à l'évaluation des flux numériques 
    sont systématiquement reconstruites au niveau de résolution le plus fin}.
\end{enumerate}
L'objectif est de comprendre, grâce aux équations équivalentes (\emph{cf} \ref{}), la nature, la dynamique et l’amplitude de l'erreur qu'introduit la MRA selon l'approche utilisée.
L'enjeux est de savoir si le gain de précision que le schéma \texttt{III} pourrait apporter justifierait le surcoût en calcul qu'il impose.
En \cite{belloti_et_al_2025}, il a été montré que cette approche améliore pourtant significativement la qualité des solutions numériques sur les problèmes d'advection.
En sera-t-il de même pour la diffusion ? La réponse est plus mitigée, cela motivera d'ailleurs la troisième contribution en \ref{par:contrib_3}.