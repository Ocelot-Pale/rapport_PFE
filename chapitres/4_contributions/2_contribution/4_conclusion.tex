En conclusion, en résolvant le problème de diffusion grâce à la méthode des lignes proposée, la multi-résolution adaptative usuelle (sans reconstruction des flux)
préserve l'ordre du schéma. En revanche et contre toute attente, lorsque les flux sont reconstruits au niveau le plus fin, l'ordre deux en temps du semble formellement réduit à un.
Malheureusement ce phénomène n'a pas pu être mis en lumière expérimentalement, notamment à cause de la contrainte de stabilité.
Face à ces difficultés des expériences numériques plus ambitieuses ont été entreprises 
(méthodes stabilisées, différents niveaux de reconstruction, extension à d'autre cas que la diffusion "pure" etc...).
C'est l'object de la contribution suivante qui étudie empiriquement l'impact de la reconstruction ou non du flux sur les problèmes de diffusion puis de diffusion-réaction.