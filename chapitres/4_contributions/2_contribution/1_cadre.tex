\subsubsection{Problème cible}
    Nous cherchons à résoudre le problème de diffusion suivant :
    \begin{align}
        \dt{u} = D \dxx{u}.
    \end{align}
    Nous ignorons les problématiques de conditions de bords.
        \subsubsection{Méthode des lignes utilisée}
            Pour résoudre cette équation aux dérivées partielles, nous utilisons une méthode des lignes. 
            D'abord un schéma volume fini pour la discrétisation spatiale menant à l'équation semi-discrétisée suivante : 
            \begin{align}
                dt{U}(t) = \frac{D}{\underbrace{\Delta x}_{\text{cellule}}} \Bigl(\frac{U_{k+1} - 2 U_{k} + U_{k-1}}{\underbrace{\Delta x}_{\text{approx. gradients}}}\Bigr)
            \end{align}
            Puis une méthode de Runge und Kutta explicite d'ordre deux sur l’opérateur linéaire donne :
            \begin{align}
                U_k^{n+1} &= U_k^n\\ \notag
                &+ D \frac{\Delta t}{\underbrace{\Delta x}_{\text{cellule}}} \Bigl(\frac{U_{k+1} - 2 U_{k} + U_{k-1}}{\underbrace{\Delta x}_{\text{approx. gradients}}}\Bigr)\\ \notag
                &+\frac{1}{2} \,D^2 \frac{\Delta t^2}{\underbrace{\Delta x^2}_{\text{cellule}}} \Bigl(\frac{U_{k+2} -4 U_{k+1}  +6 U_{k} -4 U_{k-1} + U_{k-2}}{\underbrace{\Delta x^2}_{\text{approx. gradients}}}\Bigr).
            \end{align}
            Cela s'écrit sous la forme conservative suivante:
            \begin{align}
                u^{n+1}_k = u^n_k +  D \frac{\Delta t}{\Delta x} \Bigl( \Phi^n_{k+1/2} - \Phi^n_{k-1/2} \Bigr)  + \left( D \frac{\Delta t}{\Delta x} \right)^2 \Bigl( \Psi^n_{k+1/2} - \Psi^n_{k-1/2} \Bigr) 
            \end{align}
            Avec:

            \begin{align}
                \Phi^n_{k+1/2} &= \frac{1}{\Delta x}(u^n_{k+1} - u^n_{k}),\\
                \Phi^n_{k-1/2} &= \frac{1}{\Delta x}(u^n_{k} - u^n_{k-1}),\\\notag
                \Psi^n_{k+1/2} &= \frac{1}{\Delta x^2} \left(\frac{1}{2} u^n_{k+2} -  \frac{3}{2}  u^n_{k+1} +  \frac{3}{2} u^n_{k} -  \frac{1}{2} u^n_{k-1}\right),\\\notag
                \Psi^n_{k-1/2} &= \frac{1}{\Delta x^2} \left(\frac{1}{2} u^n_{k+1} -  \frac{3}{2}  u^n_{k}   +  \frac{3}{2}u^n_{k-1} -  \frac{1}{2}u^n_{k-2}\right).\notag
            \end{align}
\newpage
        \subsubsection{La multirésolution adaptative, différents différents paradigmes ?}\label{par:paradigme_MRA}
            La multirésolution adaptative est présenté avec plus de détails dans le préambule mathématique en \ref{par:explication_MRA}.    
            Pour rappel, MRA consiste à compresser le maillage, puis a effectuer les calculs sur le maillage compressé.
            Le schéma classique est le suivant : 
            \begin{enumerate}
                \item Partir d'un état compressé au pas de temps $n$.
                \item Calculer la solution au pas de temps $n+1$
                \item Compresser de nouveau selon un seuil de compression $\varepsilon$ grâce à une transformée multiéchelle et à l'heuristique d'Harten.
            \end{enumerate}
            Plusieurs stratégies existent pour réaliser le calcul d'un pas de temps à l'autre. La valeur d'une cellule à un niveau de détail donné est approchée au temps pas de temps suivant
            grâce à un flux numérique évalué soit
            \begin{itemize}
                \item[$\diamond$] à partir des cellules voisines au même niveau de détails. C'est la méthode dite \textit{sans reconstruction des flux}. C'est la paradigme classique en MRA.
                \item[$\diamond$] à partir des cellules voisines au niveau le plus fin. C'est la méthode dite \textit{avec reconstruction des flux au niveau le plus fin}.
                Cette approche n'est pas standard en MRA, elle nécessite une reconstruction grâce à l'opérateur prédiction de la solution au travers de plusieurs niveau. 
                Toutefois il est attendu que cela réduise l'erreur puisqu'elle permet d'évaluer le flux numérique à partir de valeurs plus précises.
                Pour des problèmes d'advection, ces gains ont été établis théoriquement et validé expérimentalement en \cite{belloti_et_al_2025}.
                Ce travail réalise une étude similaire pour les problèmes de diffusion.
            \end{itemize}
            Cette contribution fournis les équations équivalentes des schémas sans MRA, avec MRA et sans reconstruction des flux et avec MRA et avec reconstruction des flux;
            à partir de quoi elle analyse les erreurs théoriques portées par chacune de ces approches.
            Dans la suite, seul est décrit le procédé de MRA avec reconstruction des flux car c'est le plus complexe et par soucis de concision, les calculs sans reconstruction des flux ne sont pas détaillés.
            \textbf{Calcul du flux au travers de $\Delta l$ niveaux:\\}
            Lorsque l'on applique le procédé de multirésolution, étant donné une cellule à un niveau de détail donné $l$, on cherche à faire évoluer la valeur à l'étape $n$ vers la valeur à l'étape $n+1$. 
            Pour ce faire, il faut évaluer les flux à partir les cellules voisines. Dès lors plusieurs choix s'offrent à nous. Où bien on utilise les cellules voisines à leurs niveaux courants, où bien on use de l'opérateur 
            de reconstruction afin d'estimer les cellules voisines à des niveaux plus fins.\par
            Dans un premier temps le stencil est choisi égal à 1. L'opérateur de prédiction d'un niveau à l'autre s'écrit alors : 
            \begin{align}
                \hat u^{l+1}_{2k} &= +\frac{1}{8} u^l_{k-1} + u^l_k - \frac{1}{8} u^l_{k+1},\\
                \hat u^{l+1}_{2k+1} &= -\frac{1}{8} u^l_{k-1} + u^l_k + \frac{1}{8} u^l_{k+1}.
            \end{align}
            Puis en notant $\doublehat{u}^{l+\Delta l}_{(\cdot)}$ cet opérateur de prédiction itéré au travers de $\Delta l$ niveaux\footnote{
                Au sens où l'on applique le prédicteur à des données déjà issues d'une prédiction.
            } : 
            \begin{align}
                \begin{bmatrix}
                    \doublehat{u}^{(l+\Delta l)}_{2^{\Delta l}k-2}\\
                    \doublehat{u}^{(l+\Delta l)}_{2^{\Delta l}k-1}\\
                    \doublehat{u}^{(l+\Delta l)}_{2^{\Delta l}k}\\
                    \doublehat{u}^{(l+\Delta l)}_{2^{\Delta l}k+1}\\
                \end{bmatrix}
                    =
                \underbrace{
                \begin{bmatrix}
                    +1/8 & 1 & -1/8 & 0 \\
                    -1/8 & 1 & +1/8 & 0 \\
                    0 & +1/8 & 1 & -1/8 \\
                    0 & -1/8 & 1 & +1/8 
                \end{bmatrix}^{\Delta l}}_{\text{Matrice de passage } P \text{ pour }s=1.}
                \cdot
                \begin{bmatrix}
                    u^l_{k-2}\\
                    u^l_{k-1}\\
                    u^l_{k}\\
                    u^l_{k+1}\\
                \end{bmatrix}
            \end{align}
\paragraph{Flux calculés au niveau le plus fin.}
On travaille sur une cellule au niveau courant $l$ (cellules de tailles $\tilde{\Delta x}=2^{\Delta l}\Delta x$) et l’on reconstruit les états au niveau $l+\Delta l$ grâce à des flux
flux au niveau fin, dont les gradients sont approximé par un pas $\Delta x$. La mise à jour conservative utilisée est donc
\begin{align}
u_k^{n+1}
= u_k^{n}
+ \frac{D\,\Delta t}{\underbrace{\Delta x\,2^{\Delta l}}_{\text{cellule}}}\Bigl(\doublehat{\Phi}_{k+\frac12}^n-\doublehat{\Phi}_{k-\frac12}^n\Bigr)
+ \left(\frac{D\,\Delta t}{\underbrace{\Delta x\,2^{\Delta l}}_{\text{cellule}}}\right)^2 \Bigl(\doublehat{\Psi}_{k+\frac12}^n-\doublehat{\Psi}_{k-\frac12}^n\Bigr).
\end{align}
Les flux sont évalués au \emph{niveau fin} (facteurs $1/\Delta x$ et $1/\Delta x^2$ portés par les flux) à partir d’états reconstruits $\doublehat{u}^{\,l+\Delta l}$ :
\begin{align}
\doublehat{\Phi}_{k-\frac12}^n
&= \frac{1}{\Delta x}\Bigl(
\doublehat{u}^{\,l+\Delta l}_{2^{\Delta l}k}
-\doublehat{u}^{\,l+\Delta l}_{2^{\Delta l}k-1}\Bigr),\\
\doublehat{\Phi}_{k+\frac12}^n
&= \frac{1}{\Delta x}\Bigl(
\doublehat{u}^{\,l+\Delta l}_{2^{\Delta l}(k+1)}
-\doublehat{u}^{\,l+\Delta l}_{2^{\Delta l}(k+1)-1}\Bigr),\\
\doublehat{\Psi}_{k-\frac12}^n
&= \frac{1}{\Delta x^2}\Bigl(
\tfrac12\,\doublehat{u}^{\,l+\Delta l}_{2^{\Delta l}k+1}
-\tfrac32\,\doublehat{u}^{\,l+\Delta l}_{2^{\Delta l}k}
+\tfrac32\,\doublehat{u}^{\,l+\Delta l}_{2^{\Delta l}k-1}
-\tfrac12\,\doublehat{u}^{\,l+\Delta l}_{2^{\Delta l}k-2}\Bigr),\\
\doublehat{\Psi}_{k+\frac12}^n
&= \frac{1}{\Delta x^2}\Bigl(
\tfrac12\,\doublehat{u}^{\,l+\Delta l}_{2^{\Delta l}(k+1)+1}
-\tfrac32\,\doublehat{u}^{\,l+\Delta l}_{2^{\Delta l}(k+1)}
+\tfrac32\,\doublehat{u}^{\,l+\Delta l}_{2^{\Delta l}(k+1)-1}
-\tfrac12\,\doublehat{u}^{\,l+\Delta l}_{2^{\Delta l}(k+1)-2}\Bigr).
\end{align}

\paragraph{Écriture matricielle.}
Pour simplifier l'implémentation des calculs dans les codes de calculs formel, il est pertinent d'écrire se qui précède sous forme matricielle.\\
Pour les flux gauches:
\begin{align}
\doublehat{\Phi}_{k-\frac12}^n
&= \frac{1}{\Delta x}\,
\begin{bmatrix}0 \\ -1 \\ +1 \\ 0\end{bmatrix}^\top \cdot
\begin{bmatrix}
\frac{1}{8} & 1 & -\frac{1}{8} & 0\\
-\frac{1}{8} & 1 & +\frac{1}{8} & 0\\
0 & +\frac{1}{8} & 1 & -\frac{1}{8}\\
0 & -\frac{1}{8} & 1 & +\frac{1}{8}
\end{bmatrix}^{\!\Delta l} \cdot
\begin{bmatrix}
u^l_{k-2}\\ u^l_{k-1}\\ u^l_{k}\\ u^l_{k+1}
\end{bmatrix},\\
\doublehat{\Psi}_{k-\frac12}^n
&= \frac{1}{\Delta x^2}\,
\begin{bmatrix}-\tfrac12 \\ +\tfrac32 \\ -\tfrac32 \\ +\tfrac12\end{bmatrix}^\top \cdot
\begin{bmatrix}
\frac{1}{8} & 1 & -\frac{1}{8} & 0\\
-\frac{1}{8} & 1 & +\frac{1}{8} & 0\\
0 & +\frac{1}{8} & 1 & -\frac{1}{8}\\
0 & -\frac{1}{8} & 1 & +\frac{1}{8}
\end{bmatrix}^{\!\Delta l} \cdot
\begin{bmatrix}
u^l_{k-2}\\ u^l_{k-1}\\ u^l_{k}\\ u^l_{k+1}
\end{bmatrix}.
\end{align}
Pour les flux droits:
\begin{align}
\doublehat{\Phi}_{k+\frac12}^n
&= \frac{1}{\Delta x}
\begin{bmatrix}0 \\ -1 \\ +1 \\ 0\end{bmatrix}^\top \cdot
\begin{bmatrix}
\frac{1}{8} & 1 & -\frac{1}{8} & 0\\
-\frac{1}{8} & 1 & +\frac{1}{8} & 0\\
0 & +\frac{1}{8} & 1 & -\frac{1}{8}\\
0 & -\frac{1}{8} & 1 & +\frac{1}{8}
\end{bmatrix}^{\!\Delta l} \cdot
\begin{bmatrix}
u^l_{k-1}\\ u^l_{k}\\ u^l_{k+1}\\ u^l_{k+2}
\end{bmatrix},\\
\doublehat{\Psi}_{k+\frac12}^n
&= \frac{1}{\Delta x^2}
\begin{bmatrix}-\tfrac12 \\ +\tfrac32 \\ -\tfrac32 \\ +\tfrac12\end{bmatrix}^\top \cdot
\begin{bmatrix}
\frac{1}{8} & 1 & -\frac{1}{8} & 0\\
-\frac{1}{8} & 1 & +\frac{1}{8} & 0\\
0 & +\frac{1}{8} & 1 & -\frac{1}{8}\\
0 & -\frac{1}{8} & 1 & +\frac{1}{8}
\end{bmatrix}^{\!\Delta l} \cdot
\begin{bmatrix}
u^l_{k-1}\\ u^l_{k}\\ u^l_{k+1}\\ u^l_{k+2}
\end{bmatrix}.
\end{align}


            % En particulier, si la cellule étudiée est au niveau courant $l$ alors on choisira d'aller approximer le flux au niveau le plus fin, c'est à dire avec $\dlbar = \bar l - l$.
            % Dès lors les flux approximés au niveau fins sont : 
            % \begin{align}
            %     \doublehat{\Phi}_{k-1/2} &= \doublehat{u}^{l+\dlbar}_{2^{\dlbar} k} -  \doublehat{u}^{l+\dlbar}_{2^{\dlbar} k-1} + \frac{1}{2} \lambda 
            %     \Bigl(
            %         \doublehat{u}^{l+\dlbar}_{2^{\dlbar} k+1}
            %         - 3 \doublehat{u}^{l+\dlbar}_{2^{\dlbar} k}
            %         + 3 \doublehat{u}^{l+\dlbar}_{2^{\dlbar} k-1}
            %         - \doublehat{u}^{l+\dlbar}_{2^{\dlbar} k-2}
            %     \Bigr),\\
            %     \doublehat{\Phi}_{k+1/2} &=  \doublehat{u}^{l+\dlbar}_{2^{\dlbar} (k+1)} -  \doublehat{u}^{l+\dlbar}_{2^{\dlbar} (k+1)-1} + \frac{1}{2} \lambda \Bigl(
            %         \doublehat{u}^{l+\dlbar}_{2^{\dlbar} (k+1)+1}
            %         - 3 \doublehat{u}^{l+\dlbar}_{2^{\dlbar} (k+1)}
            %         + 3 \doublehat{u}^{l+\dlbar}_{2^{\dlbar} (k+1)-1}
            %         - \doublehat{u}^{l+\dlbar}_{2^{\dlbar} (k+1)-2}
            %     \Bigr)
            % \end{align}
            
            % Cela s'écrit sous la forme matricielle suivante (utile pour utiliser les outils de calcul formel).
            % \begin{align}
            %     \doublehat{\Phi}_{k-1/2}
            %         &=
            %     \begin{bmatrix}
            %         -\frac{\lambda}{2}&
            %         (\frac{3}{2} \lambda - 1)&
            %         (1 - \frac{3}{2} \lambda)&
            %         \frac{\lambda}{2}&
            %     \end{bmatrix}
            %     \begin{bmatrix}
            %         +1/8 & 1 & -1/8 & 0\\
            %         -1/8 & 1 & +1/8 & 0\\
            %         0 & +1/8 & 1 & -1/8\\
            %         0 & -1/8 & 1 & +1/8\\
            %     \end{bmatrix}^{\dlbar}
            %     \begin{bmatrix}
            %         u^l_{k-2}\\
            %         u^l_{k-1}\\
            %         u^l_{k}\\
            %         u^l_{k+1}\\
            %     \end{bmatrix}
            % \end{align}
            % \begin{align}
            %     \doublehat{\Phi}_{k+1/2}
            %         &=
            %     \begin{bmatrix}
            %         -\frac{\lambda}{2}&
            %         (\frac{3}{2} \lambda - 1)&
            %         (1 - \frac{3}{2} \lambda)&
            %         \frac{\lambda}{2}&
            %     \end{bmatrix}
            %     \begin{bmatrix}
            %         +1/8 & 1 & -1/8 & 0\\
            %         -1/8 & 1 & +1/8 & 0\\
            %         0 & +1/8 & 1 & -1/8\\
            %         0 & -1/8 & 1 & +1/8\\
            %     \end{bmatrix}^{\dlbar}
            %     \begin{bmatrix}
            %         u^l_{k-1}\\
            %         u^l_{k}\\
            %         u^l_{k+1}\\
            %         u^l_{k+2}\\
            %     \end{bmatrix}.
            % \end{align}

            % Attention le schéma final est légèrement différent car il fait ici intervenir deux pas d'espace: $\Delta x$ le pas au niveau le plus fin
            % et $\Tilde {\Delta x} = 2^{\Delta l} \Delta x$ le pas du niveau courrant. Ainsi le schéma final est :
            % \begin{align}
            %     {u}^{n+1}_k = {u}^n_k + \frac{\lambda}{2^{\Delta l}} \Bigl( \doublehat{\Phi}^n_{k+1/2} - \doublehat{\Phi}^n_{k-1/2} \Bigr)
            % \end{align}