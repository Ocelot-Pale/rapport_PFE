\subsubsection{Des défis expérimentaux}
    L'observation du mécanisme de perte d'ordre mis théoriquement à jour précédemment 
    est une tâche ardue.
    \paragraph{Une expérience exacte impossible à reproduire}
        Il n'est pas possible d'utiliser la méthode numérique utilisée dans l'étude théorique pour essayer de la valider expérimentalement. 
        En effet, la méthode est une RK2 explicite, elle impose sur ce problème de diffusion une condition de stabilité du type $\Delta t \propto \Delta x^2$,
        ce qui en pratique donne $\Delta t \ll \Delta x$. De fait, la majorité des erreurs sont liés au pas d'espace "grand" devant le pas de temps, et donc
        si l'on fixe le pas d'espace pour faire converger la méthode en temps, l'erreur est déjà saturée en temps et on n'observe rien. 
        C'est un classique de l'analyse numérique.
    \paragraph{Une tentative infructueuse}
        Suite à cette limite, l'expérience a été retentée avec une méthode voisine, une RK2, mais sans contrainte de stabilité. 
        Le code de calcul Samurai a donc été relancé avec une méthode RK implicite d'ordre deux, plus précisément une SDIRK.
        Avec cette méthode, l'ordre deux est observé et ce qu'importe les paramètre de l'AMR.
        Plusieurs hypothèses peuvent expliquer ce résultat:
        \begin{enumerate}
            \item Le phénomène n'est pas présent sur cette méthode implicite.
            \item D'autres problèmes biaisent l'expérience (voir paragraphe suivant).
            \item Les calculs ou les prémisses du développement théorique précédent des équations équivalentes sont faux.
        \end{enumerate}
    \paragraph{Des biais multiples}
        De nombreux biais expérimentaux persistent et peuvent expliquer l'invisibilité du phénomène.
        Par exemple l'étude précédente, ne prend pas en compte les conditions de bord, ou peut être que le maillage n'est compressé que localement 
        ce qui n'altère que peu la convergence. Enfin, le plus grave, dans les calculs précédents, il a été fait l'hypothèse que l'évaluation est faite au niveau le plus fin 
        (que la solution est entièrement reconstruite pour l'évaluation des flux), ce qui n'est pas fait en pratique. Cette idée viens du fait qu'intuitivement, 
        si l'on reconstruite jusqu'au niveau le plus fin le flux, l'erreur devrait diminuer et c'est ce que suggère \cite{belloti_et_al_2025}.
        Cependant cette fonctionnalité n'étant pas encore disponible dans le logiciel de calcul, l'expérience à été réalisée sans reconstruire le flux au niveau 
        le plus fin, mais en prenant la valeur du niveau courant de compression. Il semble peu probable que ce soit la cause de la non-observation du phénomène 
        de perte d'ordre mais cela reste un biais potentiel. Enfin peut être qu'un bug s'est glissé dans mon implémentation mais cela semble peu probable 
        puisque ce serait une erreur d'implémentation qu'il "améliore" l'ordre de convergence... 
        l'ordre.
\subsection{... DEPEND DE LA SUITE ...}