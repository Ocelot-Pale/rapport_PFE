Pour comprendre les résultats de convergence précédents (\ref{par:etude_diff_rock4}), le profil des erreurs a été tracé (voir \ref{fig:error_profiles_landscape}) pour chaque méthode numérique et pour différents régimes de CFL ; puis 
ces observations ont été reliées aux équations équivalentes développées en \ref{par:contrib_2}. 
Cette mise en relation est cohérente car les équations équivalentes ont été calculées pour un schéma d'intégration en temps ERK2 et l'expérience a été réalisée avec 
ROCK2 qui est une ERK2 stabilisée.\\
Les résultats précédents ont d'abord été confirmés:
\begin{enumerate}
    \item Pour les petites CFL, les méthodes avec reconstruction sont moins précisés que les méthodes sans reconstruction.
    \item Il existe une gamme de CFL très précise pour laquelle les schémas avec reconstruction sur-performent les autres méthodes.
\end{enumerate}
Une première hypothèse a été formulée mais elle a été par la suite invalidée. Une seconde hypothèse à alors été proposées qui a elle été validée.
\subsubsection{Première Hypothèse}
\textbf{L'observation} suivante a de plus été faite: \textit{il semble que le profil de l'erreur avec reconstruction ressemble à la dérivée spatiale d'ordre quatre
de la solution et que le signe de cette erreur change avec la CFL}.\\
\textbf{Une première hypothèse} a d'abord été proposée: \textit{l'évolution du profil d'erreur selon la CFL s'explique par le terme ... dans l'équation équivalente \ref{}.
Pour les grandes CFL le coefficient ... est ... et pour les petites CFL il est ....} Enfin pour $\lambda = \dots$ ce coefficient serait null expliquant 
chute de l'erreur pour ces CFL, l'erreur s'allégeant du terme d'ordre deux.\\
\textbf{Validation expérimentale:} Pour valider cette hypothèse, une régression linéaire entre l'erreur numérique et la dérivée $4^e$ de la solution a été réalisée par 
une méthode des moindres carrées : $$\min_{\alpha \in \mathbb R} \Vert \alpha \partial_x^4 u - \text{err} \Vert^2.$$
Ce modèle explique bien l'erreur pour les petites CFL ($R^2>0.9$) mais mal pour les grandes CFL ($R^2 \sim 0.7$).
\subsubsection{Seconde Hypothèse}
\textbf{Une seconde observation} a alors complété la première: \textit{à grande CFL, l'erreur ne ressemble pas à l'opposé de $\partial_x^4 u$ mais à 
$\partial_x^6 u$.}\\
\textbf{Une seconde hypothèse} a alors été émise: \textit{l'erreur s'explique aux grandes CFL car dans l'équation équivalente le terme $\lambda^2 \Delta x^4 \partial_x^6 u $ 
domine et aux petites CFL car le terme $\lambda \delta x^2 \partial_x^4 u$ domine.} Les piètres performances de la méthode avec reconstruction pour de petites CFL 
s'explique alors simplement, il y a plus de coefficients devant le terme d'erreur $\Delta x^2 \partial_x^4 u$, dominant pour les petites CFL:
\begin{align}
    .\\
    .
\end{align}
\textbf{Pour valider} cette nouvelle hypothèse, l'erreurs a été modélisée comme $\text{err} \approx \alpha \partial_x^4 u + \beta \partial_x^6u$ par moindre carré. 
Ce modèle explique très bien l'erreur pour tous les régimes de CFL (voir \ref{fig:derivees_vs_err}).
\begin{figure}[h!]
    \centering
    \includegraphics[width=0.75\textwidth]{media/4_travail/3/derivees_spatiales_VS_err_num.pdf}
    \caption{Régression entre l'erreur numérique expérimentale (AMR + reconstruction fine) et une combinaison linéaire des dérivées $4^e$ et $6^e$ de la solution.}
    \label{fig:derivees_vs_err}
\end{figure}\\
\textbf{Remarque:} la plage de CFL où la méthode avec reconstruction est plus la précise correspond en fait au cas ou les profils de $ \alpha \partial_x^4 u$
de $\beta \partial_x^6 u$ se compensent. C'est donc un comportement tout à fait accidentel lié au fait que les dérives de la courbe de Gauss sont en quelque sort "en opposition de phase".
\textbf{Analyse:} avec le recul, les résultats s'expliquent comme suit. Le schéma d'intégration en temps nécessite l'approximation de la dérivée spatiale d'ordre quatre de la solution.
Ce qui ne peut être fait de manière satisfaisante à partir d'une prédiction polynomiale à trois point \cite{}/\ref{}.\\
\textbf{Complément:} Ce n'est pas présenté ici, mais le travail à été refait avec un prédicteur à 5 points, permettant d'approximer correctement la dérivée d'ordre quatre.
Alors la reconstruction apporte un gain notable, les solutions numériques avec et sans MRA sont dans ce cas très proches.