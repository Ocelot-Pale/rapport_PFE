Les équations équivalentes (obtenues en \ref{par:contrib_2}) offrent une piste pour comprendre le comportements différents des erreurs des méthodes reconstruisant le flux de celles ne le reconstruisant pas.
L'analyse théorique est toujours faite au travers des équation équivalente des schémas issu de la méthode ERK2 et non de ROCK2; 
toutefois, cela fournit un cadre de réflexion cohérent puisque ROCK2 reste une ERK2 avec des étages de stabilisation.
Lorsque l'on compare les équations équivalentes des méthodes de MRA avec et sans reconstruction, les termes qui diffèrent au premier ordre non nul sont : 
\begin{align}
    \notag
    &\text{MRA sans reconstruction: }& 2^{2 \Delta l} \frac{D}{12}\Delta x^2 \frac{\partial^4 u}{\partial x^4}
    \\\notag
    &\text{MRA avec reconstruction: }& \left( 2^{2 \Delta l}\frac{D}{12}(1-3\Delta l) + \frac{D}{2} \lambda (2^{\Delta l} - 1)\right) \Delta x^2 \frac{\partial^4 u}{\partial x^4}
\end{align}
Avec reconstruction un terme dépend de la constante CFL diffusive $\lambda = \frac{D \Delta t}{\Delta x^2}$. Ainsi le coefficient devant cette erreur en $\partial_x^4 u$ 
est positif pour grands $\lambda$ (à pas de temps fixé, cela correspond à de grands $\Delta t$) et négatif pour de petits pas de temps. Pour un certain pas de temps, le coefficient s'annule.
Ce terme modulerait donc le signe et l'amplitude de l'erreur porté par la dérivée quatrième. 
L'effondrement de l'erreur sur une courte gamme de pas de temps pour les méthode 
reconstruisant le flux ne serait que le résultat de l'annulation \textit{fortuite} de ce terme; lorsqu'il change continûment de signe. 
C'est pour l'instant la seule explication que je puisse apporter au observation expérimentales.
Elle semble crédible puisqu'elle permet de décrire le mécanisme observé expérimentalement et qu'il résulte de l'étude de la principale différence entre les deux méthodes. 
Un autre point qui pointe dans cette direction est que le profil de l'erreur ressemble fortement la dérivée spatiale d'ordre quatre de la solution comme peut le constater le lecteur:
\begin{figure}
    
\end{figure}