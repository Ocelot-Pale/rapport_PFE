Cette troisième contribution s'inscrit directement dans la continuité de la précédente.
L'objectif est ici d'étudier expérimentalement l'impact de la multirésolution et de ses modalités de mise en oeuvre 
sur les approximations numériques des problèmes diffusifs. Elle propose trois algorithmes d'AMR se distinguant
par la qualité de l’évolution des flux numériques.\par
D'abord les différences entre les trois algorithmes d'AMR sont mises en lumière, 
puis une première expérience numérique reprenant le schéma numérique de la contribution précédente (VF + ERK2) 
est réalisée. Face à des résultats surprenants, l'hypothèse de l'émergence d'instabilités résiduelles pour certains algorithmes a été émise. 
Cependant cette hypothèse semble être invalidée par une étude de stabilité linéaire réalisée grâce à Sympy.
La contrainte de stabilités imposées par la méthode ERK2 n'avait jusqu'ici permises d'observer que des solutions convergées en temps (erreur spatiale dominante).
Restait inconnu l'impact de la méthode d'évaluation des flux numériques dans un contexte ou les erreurs temporelles restent dominantes. Une seconde expérience 
à alors été réalisé avec la méthode explicite stabilité ROK4 au lieu de la méthode ERK2, permettant d'accéder à des pas de temps plus grands.