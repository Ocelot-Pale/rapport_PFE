Cette troisième contribution prolonge empiriquement la précédente
en étudiant \emph{expérimentalement} l'impact de différentes approches de multirésolution adaptatives sur les solution numériques des problèmes diffusifs. 
Elle étudie trois manières de mettre en place l'adaptation spatiale par MRA:
celle des schémas \texttt{II} (sans reconstruction des flux) et
\texttt{III} (avec reconstruction des flux) de la contribution précédente
ainsi qu'une approche \textit{intermédiaire}.\par
Ce travail s'articule de la manière suivante:
\begin{itemize}
\item[$\diamond$]\ref{par:contrib_3:pres_algos} Présentation des trois paradigmes d'AMR.
\item[$\diamond$]\ref{par:contrib_3:etude_ERK2} Première expérience numériques comparant l'erreur en fonction du paradigme d'AMR choisi
sur le problème de diffusion résolu par le schéma numériques de la contribution antérieure () - méthode des lignes, Volumes Finis + Runge et Kutta explicite.
\item[$\diamond$]\ref{par:contrib_3:stab_MRA} Les résultats, inattendus
ont conduit à formuler l'hypothèse que les schéma \emph{avec} reconstruction serait moins stables que le schéma \emph{sans} reconstruction.
Cependant cette hypothèse est invalidée par une étude de stabilité linéaire.
\item[$\diamond$]\ref{par:contrib_3:ROCK2} La principale limite de l'étude antérieure est
la contrainte de stabilités de la méthode explicite imposant . Cela ne permet d'observer que des solutions convergées en temps (erreur spatiale dominante car stabilité $\Rightarrow \Delta t \ll \Delta x$).
Pour poursuivre la comparaison dans un contexte où les erreurs temporelles ne sont pas négligeables, une seconde expérience 
est réalisée remplaçant la méthode ERK2 par la méthode \emph{stabilisée} ROCK2 \cite{abdulle2002fourth} permettant d'accéder à 
une plus large gamme de pas de temps.
\item[$\diamond$]\ref{par:contrib_3:eq_equiv} Les résultats numériques pour des pas de temps moins restreint sont encore plus inattendus que les précédents.
Cependant en les reliant aux travaux théoriques précédents (equations équivalentes \ref{}), l'ensemble des comportements obtenus sont finalement compris et expliqués.
\end{itemize}