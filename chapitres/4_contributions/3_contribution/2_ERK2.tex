La première expérience numérique a donc été faire sur l'équation de diffusion avec le même schéma numérique que dans l'étude théorique précédente (\textit{cf}\ref{par:intro_contrib_eq_equiv});
c'est à dire une discrétisation spatiale d'ordre deux du Laplacien intégré en temps avec une méthode Runge et Kutta explicite d'ordre 2. 
Il pourrait sembler plus astucieux d'utiliser une méthode implicite pour s'affranchir des problèmes de stabilité, cependant l'inversion d'un système linéaire couplé à 
la reconstruction des flux à des niveaux plus fin (non-standard, algos 2 et 3) est très difficile techniquement et aurait ralenti l'étude.
A cause de la contrainte de stabilité $\Delta t \propto \Delta x^2$, seules des solutions convergées en temps (erreur spatiale dominante) ont pu être observées (cette contrainte sera levée par la suite en \ref{}).\par 
\subsubsection{Résultats numériques}
Il résulte expérimentalement que dans ce contexte, étonnamment, l'algorithme un (le plus grossier) offre la plus faible erreur et que plus l'algorithme reconstruit finement les flux plus l'erreur est grande. 
A titre d'exemple, sur une solution initiale gaussienne, avec une seuil de compression $\varepsilon = 10^{-4}$ et un maillage présentant jusqu'à 4 niveau de finesse, les erreurs $L^2$ au temps final $T_f=1$ sont les suivantes:\par
\begin{center}\begin{tabular}{|c|c|c|}
\hline
Algo $n^o$ & Niveau d'évaluation des flux & Erreur $L^2$ \\
\hline
1 & Courant           & $1 \times 10^{-4}$ \\
2 & Inférieur direct  & $2 \times 10^{-4}$ \\
3 & Plus fin          & $3 \times 10^{-4}$ \\
référence & Sans AMR& $2 \times 10^{-5}$ \\
\hline
\end{tabular}\end{center}
Bien sûr, l'algorithme sans AMR reste celui avec la plus petite erreur.
\subsubsection{Analyse et hypothèses}
Ce résultat est assez surprenant puisqu'on s'attendrait à ce qu'une reconstruction plus précise du flux donne de meilleurs résultats.
Dès lors plusieurs hypothèses peuvent être émises:
\begin{enumerate}
    \item Le fait de reconstruire \textit{déstabilise} la méthode.
    \item L'usage de cellules plus grandes pour évaluer les flux augmente le caractère diffusif du schéma.
    \item Peut-être que cet effet n'a lieu que sur des solutions où l'erreur spatiale domine et que les résultats seraient différents lorsque l'erreur temporelle reste dominante.
\end{enumerate}
La première hypothèse est éprouvée dans la prochaine section \ref{par:stab_amr} et une méthode stabilisée est utilisée en \ref{} pour explorer numériquement la troisième hypothèse.