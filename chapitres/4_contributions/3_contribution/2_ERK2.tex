Cette première expérience numérique résout numériquement l'équation de diffusion par le schéma de la contribution \ref{par:contrib_2} (Volumes Finis + ERK2).
Une méthode implicite pourrait paraître plus appropriée pour s'affranchir des problèmes de stabilité; cependant l'inversion d'un système linéaire couplé à 
la reconstruction des flux à des niveaux plus fin (non-standard, algos \texttt{} et 3) est difficile à implémenter informatiquement et aurait ralenti l'étude.
À cause de la contrainte de stabilité $\Delta t \propto \Delta x^2$,
seules des solutions convergées en temps (erreur spatiale dominante puisque $\Delta t \ll \Delta x$) sont observées. Cette contrainte est levée dans l'expérience suivante en \ref{par:etude_diff_rock4}.\par 
\subsubsection{Résultats numériques}
Les résultats sont étonnants: parmi les schéma adaptés, le schéma \texttt{II} (le plus grossier) offre la plus faible erreur et plus l'algorithme reconstruit finement les flux plus l'erreur augmente. 
À titre d'exemple, pour la diffusion d'une Gaussienne, une seuil de compression $\varepsilon = 10^{-4}$ et un maillage présentant jusqu'à 4 niveau de finesse, les erreurs $L^2$ au temps final $T_f=1$ sont les suivantes:\par
\begin{center}\begin{tabular}{|c|c|c|}
\hline
Schéma $n^o$ & Niveau d'évaluation des flux & Erreur $L^2$ \\
\hline
\texttt{I} & $\emptyset$ MRA & $2 \times 10^{-5}$ \\
\texttt{II} & Courant & $1 \times 10^{-4}$ \\
\texttt{III} & Inférieur direct ($l+1$) & $2 \times 10^{-4}$ \\
\texttt{IV} & Plus fin $l^{max}$& $3 \times 10^{-4}$ \\

\hline
\end{tabular}\end{center}
L'erreur des différents schéma adaptés est de l'ordre du seuil de compression choisit $\varepsilon = 10^{-4}$ montrant que ce sont bien
les effets de la MRA qui sont dominants ici.
\subsubsection{Analyse et hypothèses}
Ce résultat est assez surprenant puisqu'on s'attendrait à ce qu'une reconstruction plus précise du flux donne de meilleurs résultats.
Face à ces résultats, plusieurs hypothèses sont émises:
\begin{enumerate}
    \item La reconstruction introduit des \textit{instabilités} dans le schéma.
    \item Peut-être que cette tendance ne vaut que sur des solutions où l'erreur spatiale domine et que les résultats seraient différents lorsque l'erreur temporelle reste dominante.
\end{enumerate}
La première hypothèse est éprouvée dans la prochaine section \ref{par:stab_amr} et une méthode stabilisée est utilisée en \ref{par:contrib_3:ROCK2} pour explorer numériquement la seconde.