Cette première expérience résout numériquement l'équation de diffusion par le schéma de la contribution \ref{par:contrib_2} (Volumes Finis + ERK2).
Une méthode implicite pourrait paraître plus appropriée pour s'affranchir des problèmes de stabilité; cependant l'inversion d'un système linéaire rend la reconstruction
des flux (schémas \texttt{III} et \texttt{IV}) trop complexes à implémenter.
Ainsi, à cause de la contrainte de stabilité $\Delta t \propto \Delta x^2$,
seules des solutions convergées en temps (erreur spatiale dominante puisque $\Delta t \ll \Delta x$) sont observées. Cette contrainte est levée dans l'expérience suivante en \ref{par:etude_diff_rock4}.\par 
\subsubsection{Résultats numériques}
Les résultats sont étonnants: parmi les schéma adaptés, le schéma \texttt{II} (le plus grossier) offre la plus faible erreur et plus l'algorithme reconstruit finement les flux plus l'erreur augmente. 
À titre d'exemple, pour la diffusion d'une Gaussienne (\textit{cf} \ref{par:contrib_2:exp}) et un seuil de compression $\varepsilon = 10^{-4}$, les erreurs $L^2$ au temps final $T_f=1$ sont les suivantes:\par
\begin{center}\begin{tabular}{|c|c|c|}
\hline
Schéma $n^o$ & Niveau d'évaluation des flux & Erreur $L^2$ \\
\hline
\texttt{I} & $\emptyset$ MRA & $2 \times 10^{-5}$ \\
\texttt{II} & Courant & $1 \times 10^{-4}$ \\
\texttt{III} & Plus fin $l^{\max}$& $3 \times 10^{-4}$ \\
\texttt{IV} & Inférieur direct ($l+1$) & $2 \times 10^{-4}$ \\


\hline
\end{tabular}\end{center}
L'erreur des différents schémas adaptés par MRA est de l'ordre du seuil de compression choisi $\varepsilon = 10^{-4}$ montrant que ce sont bien
les effets de la MRA qui dominent ici.
\subsubsection{Analyse et hypothèses}
Ce résultat est assez surprenant puisqu'on s'attendrait à ce qu'une reconstruction plus précise du flux donne de meilleurs résultats.
Toutefois c'est assez cohérent avec l'analyse théorique en \ref{par:contrib_2} qui prédit que reconstruire augmente le nombre de terme d'erreur (\emph{cf.} \ref{par:contrib_2:resultats}).
Face à ces résultats, plusieurs hypothèses sont émises:
\begin{enumerate}
    \item La reconstruction introduit des \textit{instabilités} dans le schéma.
    \item Peut-être que cette tendance ne vaut que sur des solutions où l'erreur temporelle pure (sans prendre en compte une éventuelle interaction avec la MRA) est négligeable.
    En effet, le caractère explicite de la méthode ERK2 force le choix $\Delta t \propto \Delta x^2$ et donc en pratique $\Delta t \ll \Delta x$.
    Et donc le régime $\Delta t \sim \Delta x$ ne peut pas être exploré expérimentalement.
\end{enumerate}
La première hypothèse est éprouvée dans la prochaine section \ref{par:stab_amr} et une méthode explicite stabilisée est utilisée\footnote{
On préfère une méthode explicite stabilisée à une méthode implicite, car la reconstruction systématique des flux au niveau fin est très complexe pour une méthode implicite.} en \ref{par:contrib_3:ROCK2} pour explorer numériquement la seconde.