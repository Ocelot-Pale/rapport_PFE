Le résultat principal est que, sur le schéma méthode des lignes étudié, une reconstruction systématique des flux n'est bénéfique que si la reconstruction se fait par un
prédicteur polynomial à au moins cinq points. Sinon la reconstruction des flux apporte avant tout du bruit et dégrade la qualité de la solution numérique.
\textbf{Il semble raisonnable de conjecturer que de manière générale, la reconstruction des flux sera positive si elle est capable de capter les dérivées spatiales dominantes dans les termes d'erreur.}
En somme : si les équations équivalentes du schéma non-adapté montrent que l'erreur dominante porte sur une dérivée spatiale d'ordre $k$. 
Alors sur le schéma adapté gagnera à reconstruire les flux au niveau le plus fin uniquement si le prédicteur possède au moins $k+1$ points.
Sinon, il présume d'une précision dont il ne dispose pas et utilise du bruit comme de l'information de qualité. Si le prédicteur 
possède moins de $k+1$ points, alors il est préférable de ne pas reconstruire car alors le schéma dispose d'une 
information fiable de la dérivée $k$ au niveau $l$ de résolution courant de l'adaptation et évite que schéma n'"hallucine" en quelque sorte les 
valeur de la dérivée $k^e$ à un niveau de résolution élevé $l^{\max}$.