Le résultat principal est que, sur le schéma méthode des lignes étudié, une reconstruction systématique des flux n'est bénéfique que si la reconstruction est faite par un
prédicteur polynomial à plus de trois points. Dans le cas étudié la reconstruction des flux à partir d'un prédicteur à trois points apporte avant tout du bruit et dégrade la qualité de la solution numérique.
\textbf{
    Il semble raisonnable de conjecturer que de manière générale,
    la reconstruction des flux est bénéfique si et seulement si 
    sa précision est strictement supérieure à l'ordre spatial du schéma.}
En somme pour un discrétisation spatiale est d'ordre $k$, la prédiction polynomiale doit au moins être d'ordre $k+1$ et donc se baser $k+2$ points, 
c'est à dire un stencil $s = \left\lceil{\frac{k+1}{2}}\right\rceil$.\par 
Sur notre cas d'étude, le prédicteur prend trois points alors que la règle précédente en demanderait au moins quatre.
Ainsi la prédiction polynomiale ajoute des erreurs du même ordre que le schéma, dégradant nettement la solution.
De plus cela explique qu'avec un prédicteur à cinq points (cas présenté dans la soutenance de stage),
la reconstruction mitige considérablement les perturbations induites par la multi-résolution adaptative.