Nous étudions la viabilité des RK-ImEx sur l'équation de Nagumo. Pour cela nous étudions sa stabilité.
Dans un premier temps, une étude générale de la stabilité de des RK-ImEx est menée.
Dans un second temps, l'étude se centre sur l'application à l'équation de Nagumo.

\subsubsection{Étude de stabilité générale des RK-ImEx}
    Lorsque l'on use d'une méthode ImEx, les deux (ou plus) opérateurs sont découplés, c'est bien là l'intérêt.
    Cependant cela complique légèrement l'analyse usuelle de stabilisé. En effet la fonction de stabilité attend alors deux variables, %AFAIRE -> ajouter les renvois vers la partie introduction à l'étude de stabilité etc...
    le coefficient spectral associé à l'opérateur traité explicitement et le coefficient spectral associé à l'opérateur traité implicitement.
    Ainsi, pour chaque couple de valeur propre, la fonction de stabilité prend une valeur différente et comme les coefficients spectraux sont complexes, 
    on ne peut plus visualiser d'un simple coup d'œil le domaine de stabilité (comme en ... AFAIRE), puisque celui-ci est de dimension quatre
    \footnote{En effet la fonction de stabilité $R \mathbb C \times \mathbb C \rightarrow \mathbb R$ et $\dim  \mathbb C \times \mathbb C=4$.}.
    \paragraph{Calcul des fonctions de stabilité}

\subsubsection{Étude de stabilité appliquée à l'équation de Nagumo}
    Nous allons particulariser la démarche suivante en la centrant sur l'équation de Nagumo. Cela vas nous permettre de comprendre 
    comme se comportent les méthodes ImEx sur ce problème particulier.
    \paragraph{Valeurs propres mises en jeu}
        Comme expliqué en \ref{par:analyser_operateurs_nagumo} l'équation présente deux opérateurs : 
        \begin{itemize}
            \item[$\diamond$] La diffusion dont le spectre d'étend de $\frac{-1}{L^2}$ à $\frac{-1}{\Delta x^2}$ (où $L$ est la taille du domaine discrétisé).
            \item[$\diamond$] La réaction dont le spectre balaie continûment $-k$ jusqu'à $2k$
        \end{itemize}
        Pour restreindre l'analyse de stabilité il faut donc tracer le diagramme de stabilité des méthodes étudiées en prenant $Z_I \in \mathbb{R}^-$ 
        et $Z_E \in [-k;2k] \subset \mathbb{R}$ ce qui nous donne un espace à deux dimensions. Il faut ensuite placer des couples $(Z_E,Z_I)$ correspondant.
        Lorsque l'on réalise se travaille nous trouvons les diagrammes suivant: 
        
        \begin{figure}[htbp]
            \centering
            \includegraphics[width=\textwidth]{media/4_travail/2_nagumo/stabilite/STABILITE_D10_k0.1_dt1.0e-03_dx2.4e-03.png}
            \caption{Diagrammes de stabilité des méthodes ImEx et de référence sur l'équation de Nagumo.}
            \label{fig:stabilite_nagumo}
        \end{figure}

        J'invite le lecteur à prendre un peu de temps pour comprendre la logique de ces graphiques car ils sont très éclairants. 
        Ces diagrammes permettent d'analyser respectivement la stabilité de la méthode ImEx222, de la méthode ImEx232, 
        ainsi qu'à titre de comparaison, la stabilité d'une méthode RKE d'ordre 2\footnote{Celle apparaissant dans ImEx222.} 
        et d'une méthode de splitting
        \footnote{Où l'on utilise les méthodes implicites et explicites de la méthode ImEx222 mais dans un contexte de splitting de Strang.}.
        Chaque colonne représente l'analyse d'une méthode différente. 
        La première ligne présente le domaine de stabilité en fonction des indices spectraux $Z_E \in \mathbb{R}$ et $Z_I \in \mathbb{R}^-$. 
        Les points bleus représentent les couples d'indices spectraux intervenant dans la résolution de l'équation de Nagumo 
        pour les paramètres d'équation choisis ($D$ et $k$) et les paramètres de discrétisation retenus ($\Delta t$ et $\Delta x$). 
        La seconde ligne n'est qu'un zoom de la première autour de ces indices spectraux. 
        La dernière colonne (splitting) présente une disposition différente, puisque les opérateurs sont totalement découplés.
        La première ligne correspond à la fonction de stabilité de la méthode explicite (avec un zoom autour des indices spectraux de la réaction).
        la seconde ligne représente la fonction de stabilité de la méthode implicite. 
        Dans les deux cas, l'intervalle tracé en bleu représente la plage de valeurs d'indices spectraux balayés par chaque opérateur.