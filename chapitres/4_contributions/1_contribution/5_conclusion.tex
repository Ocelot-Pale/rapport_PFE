Cette première contribution a comparé un schéma de splitting ERK2+IRK2 à deux schémas ImEx ARK sur des questions de stabilités de convergence. 
L'étude de convergence à été réalisées dans deux contextes différents : sans adaptation spatiale et avec adaptation spatiale par MRA.
Les principaux résultats sont : 
\begin{itemize}
    \item[$\diamond$] \textbf{Stabilité:} Tandis que par nature le splitting 
    découple les problématiques de stabilités, celle des méthodes ARK résulte d'un couplage entre le spectre des deux opérateurs.  
    Cela pourrait être exploité astucieusement puisque dans certains cas (\textit{cf.} \ref{par:analyse_generale_stab_nagumo})
    un opérateur implicité très raide peut stabiliser la méthode explicite. Ce serait particulièrement intéressement pour des problèmes de diffusion-réaction
    où une réaction implicite très raide pourrait étendre la stabilité d'une diffusion explicite.
    \item[$\diamond$] \textbf{Convergence:} 
    \item[$\diamond$] \textbf{Couplage avec l'adaptation spatiale:} Il a été monté empiriquement
    que les méthode ImEx peuvent interagir de manière néfaste avec l'adaptation spatiale par MRA.
    Il est probable que les méthode de splitting interagissent également, mais il semble que cette interaction soit plus limités.
    Comme montré au chapitre suivant (chapitre \ref{par:contrib_2}) ce type d’interaction (méthode en temps - adaptation spatiale)
    sont complexes et dépendent énormément des schémas temporelles et des modalités de mises en oeuvre de la MRA. 
    Donc cette étude ne suffit pas à tirer des conclusions générales,
    en revanche elle confirme l'existence de tels couplage et montre qu'ils peuvent être important car  (\textit{cf} \ref{par:couplage_temps_adaptation}) une méthode ImEx meilleure que le splitting sur un schéma non-adapté 
    devient moins bonne que le splitting pour un schéma adapté (dès lors que les erreurs de MRA ne sont plus négligeables).
\end{itemize}