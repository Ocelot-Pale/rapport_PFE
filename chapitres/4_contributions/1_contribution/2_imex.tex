La classe de méthodes ImEx étudiée sont les méthodes de Runge et Kutta additives (RK-ImEx ou RK-additive).
Ces méthodes consistent à sommer plusieurs méthodes de Runge et Kutta appliquées chacune à un opérateur différent.
L'objectif est d'employer des RK explicites (RKE) et des RK implicites (RKI), en adéquation avec les besoin de chaque opérateur.
\subsubsection{Un exemple}
    Pour introduire aux méthodes de Runge et Kutta additives, commençons par un exemple simple et usons d'une méthode RK-ImEx
    d'ordre un, résultant de la somme de deux méthodes RK à un étages (RK1). Nous notons cette méthode ImEx111 \cite{ASCHER1997151}. 
    Les méthodes RK1 servant de briques élémentaires à la RK111 sont: un schéma d'Euler explicite et un schéma d'Euler implicite.
    Supposons que l'on cherche à approcher une équation d'évolution faisant intervenir deux opérateurs: $A^E$ se prêtant à des méthodes explicites\footnote{Par exemple, un opérateur peu raide mais non local.}
    et $A^I$ se prêtant aux méthodes implicites\footnote{Par exemple un opérateur raide mais local.}. L'équation cible serait de la forme: 
    \begin{align}
        \dt{u} = A^E u + A^I u.
    \end{align}
    \paragraph{Résolution par approche monolithique}
        Rappelons d'abord comme le problème serait résolu en n'utilisant qu'une seule RK1 pour tout le problème (approche monolithique).
        \subparagraph{Euler explicite}
            En résolvant avec Euler explicite, le schéma s'écrit: 
            \begin{align}
                u^{n+1} = u^n + \Delta t (A^E + A^I) u^n.
            \end{align}
            Mais si l'opérateur $A^I$ est très raide, la stabilité risque d'imposer un pas de temps très restrictif risquant de rendre la méthode non viable.
        \subparagraph{Euler implicite}
            En résolvant avec Euler implicite, le schéma s'écrit:
            \begin{align}
                u^{n+1} = \bigl(Id - \Delta t (A^E + A^I)\bigr)^{-1} u^n.
            \end{align}
            Mais si l'opérateur $A^E$ est rend l'inversion coûteuse;
            par exemple s'il est non-local (impliquant la résolution d'un gros système au lieu de plusieurs petits systèmes), 
            ou s'il est non linéaire (nécessite d'être réinverser à chaque pas de temps);
            alors cette méthode ne sera pas viable non plus.
    \paragraph{Résolution par une méthode ImEx: une Runge et Kutta Additive}
            Mettons en oeuvre la méthode ImEx111. 
            L’approximation au pas de temps $n+1$ s'écrit en sommant une contribution issue de la méthode Euler explicite (RKE1)
            et une contribution issue de la méthode Euler implicite (RKI1):
            \begin{align}
                u^{n+1} = u^n + \Delta t (\underbrace{k_1}_{\text{RKE1}} + \underbrace{k_1'}_{\text{RKI1}})
            \end{align}
            La contribution RKE1 s'écrit:
            \begin{align}
                k_1 = A^E u^n.
            \end{align}
            La contribution RKI1 s'écrit:
            \begin{align}
                k_1' = A^I u^{n+1}
            \end{align}
            Ainsi: 
            \begin{align}
                &u^{n+1} = u^{n+1} = u^n + \Delta t ( A^E u^n +  A^I u^{n+1}),\\\notag
                \text{donc: }&u^{n+1} - \Delta t  A^I u^{n+1} = u^n + \Delta t  A^E u^n,\\\notag
                \text{et donc: }&u^{n+1} = (Id - \Delta t A^I)^{-1} \circ (Id + \Delta t A^E) u^n.
            \end{align}
            Ainsi dans cette méthode seul l’opérateur $Id- \Delta t A^I$ doit être inversé. Ce qui était bien l’objectif. Les opérateurs ont été 
            découplés lors de la résolution.
\subsubsection{Cadre mathématique général}
    Pour construire des méthodes plus complexes et d'ordres supérieurs introduisons le formalisme de \cite{ASCHER1997151} pour traiter les méthodes RK-additives. 
    Ici, nous travaillons uniquement sur méthodes ImEx pour deux opérateurs mais théoriquement, il est possible de construire des méthodes ImEx pour traiter 
    autant d'opérateurs que l'on le souhaite \cite{KENNEDY2003139}.
    \paragraph{Notation}%AFAIRE -> ajouter une référence interne vers la notion de méthode DIRK et SDIRK
        Une méthode ImEx additive est construite à partir d'une méthode implicite à $s$ étages (une méthode DIRK et si possible SDIRK) et d'une méthode explicite à $s+1$ étages
        \footnote{Au besoin, la méthode explicite peut être à $s$ étages, qui est un cas particulier d'une méthode à $s+1$ étages.}.
        Pour uniformiser, le tableau de Butcher de la méthode implicite est complété par une ligne et une colonne de zéros afin que les deux méthodes
        s'écrivent comme si elles avaient le même nombre d'étages.
        Les tableaux de Butcher des deux méthodes s'écrivent alors :
        
        \subparagraph{Méthode RKE, $s+1$ étages:}
        \begin{align}
        \text{RKE} : \quad
        \begin{array}{c|c}
        \tilde{c} & \tilde{A} \\
        \hline
        & \tilde{b}^T
        \end{array}
        =
        \begin{array}{c|ccccc}
        0 & 0 & 0 & 0 & \cdots & 0 \\
        \tilde{c}_1 & \tilde{a}_{10} & 0 & 0 & \cdots & 0 \\
        \tilde{c}_2 & \tilde{a}_{20} & \tilde{a}_{21} & 0 & \cdots & 0 \\
        \vdots & \vdots & \vdots & \ddots & \ddots & \vdots \\
        \tilde{c}_s & \tilde{a}_{s0} & \tilde{a}_{s1} & \tilde{a}_{s2} & \cdots & 0 \\
        \hline
        & \tilde{b}_0 & \tilde{b}_1 & \tilde{b}_2 & \cdots & \tilde{b}_s
        \end{array}
        \end{align}
        
        \subparagraph{Méthode RKI (DIRK) $s$ étages:}
        \begin{align}
        \text{RKI} : \quad
        \begin{array}{c|c}
        c & A \\
        \hline
        & b^T
        \end{array}
        =
        \begin{array}{c|ccccc}
        0 & 0 & 0 & 0 & \cdots & 0 \\
        c_1 & 0 & a_{11} & 0 & \cdots & 0 \\
        c_2 & 0 & a_{21} & a_{22} & \cdots & 0 \\
        \vdots & \vdots & \vdots & \ddots & \ddots & \vdots \\
        c_s & 0 & a_{s1} & a_{s2} & \cdots & a_{ss} \\
        \hline
        & 0 & b_1 & b_2 & \cdots & b_s
        \end{array}
        \end{align}
        
        où les coefficients $\tilde{a}_{ij}$, $\tilde{b}_i$, $\tilde{c}_i$ définissent la méthode explicite et 
        les coefficients $a_{ij}$, $b_i$, $c_i$ définissent la méthode implicite DIRK. 
        
    \paragraph{Schéma général d'une méthode RK-additive}
        Une étape de la méthode RK-additive appliquée au système 
        $\frac{du}{dt} = A^E u + A^I u$ s'écrit :
        
        \subparagraph{Calcul des étages :}
        En initialisant $u_0 = u^n$,
        \begin{align}
        u_i &= u^n + \Delta t \sum_{j=0}^{i-1} \tilde{a}_{ij} A^E u_j + \Delta t \sum_{j=1}^{i} a_{ij} A^I u_j, \quad &i = 0, 1, \ldots, s
        \end{align}
        Soit en mettant en lumière le caractère implicite de la méthode sur $A^I$:
        \begin{align}
        (Id - \Delta t a_{ii} A^I) u_i &= u^n + \Delta t \sum_{j=0}^{i-1} \tilde{a}_{ij} A^E u_j + a_{ij} A^I u_j, \quad &i= 0, 1, \ldots, s
        \end{align}
        
        \subparagraph{Solution à l'étape suivante :}
        \begin{align}
        u^{n+1} &= u^n + \Delta t \sum_{i=0}^{s} \tilde{b}_i A^E u_i + \Delta t \sum_{i=1}^{s} b_i A^I u_i
        \end{align}
        
        Cette formulation générale permet de construire des méthodes d'ordre élevé. 
    \paragraph{Ordre de convergence}
        L'ordre d'une méthode RK-additive est bien sur borné par l'ordre le plus faible des méthodes RK individuelles qu'elle convoque.
        Naturellement, cette borne n'est pas nécessairement atteintes pour toute méthodes RK additionnée. Des conditions d'ordre liant
        les coefficients des méthodes individuelles entre eux doivent être respectées. Le nombre de ses conditions augmente (très) rapidement avec 
        l'ordre de la méthode et le nombre d'opérateurs à résoudre \cite{KENNEDY2003139}, le lecteur motivé se référera par exemple à \cite{Hairer1981}.