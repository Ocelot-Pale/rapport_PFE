La classe de méthodes ImEx étudiée sont les méthodes de Runge et Kutta additives (RK-ImEx).
Ces méthodes consistent à sommer plusieurs méthodes de Runge et Kutta appliquées chacune à un opérateur différent.
L'objectif est d'employer des RK explicites (RKE) et des RK implicites (RKI), en adéquation avec les besoin de chaque opérateur.
\subsubsection{Un exemple}
    Pour introduire aux méthodes de Runge et Kutta additives, commençons par un exemple simple et usons d'une méthode RK-ImEx
    d'ordre un, résultant de la somme de deux méthodes RK à un étages (RK1). Nous notons cette méthode ImEx111 \cite{ASCHER1997151}. 
    Les méthodes RK1 servant de briques élémentaires à la RK111 sont: un schéma d'Euler explicite et un schéma d'Euler implicite.
    Supposons que l'on cherche à approcher une équation d'évolution faisant intervenir deux opérateurs: $A^E$ se prêtant à des méthodes explicites\footnote{Par exemple, un opérateur peu raide mais non local.}
    et $A^I$ se prêtant aux méthodes implicites\footnote{Par exemple un opérateur raide mais local.}. L'équation cible serait de la forme: 
    \begin{align}
        \dt{u} = A^E u + A^I u.
    \end{align}
    \paragraph{Résolution par approche monolithique}
        Rappelons d'abord comme le problème serait résolu en n'utilisant qu'une seule RK1 pour tout le problème (approche monolithique).
        \subparagraph{Euler explicite}
            En résolvant avec Euler explicite, le schéma s'écrit: 
            \begin{align}
                u^{n+1} = u^n + \Delta t (A^E + A^I) u^n.
            \end{align}
            Mais si l'opérateur $A^I$ est très raide, la stabilité risque d'imposer un pas de temps très restrictif risquant de rendre la méthode non viable.
        \subparagraph{Euler implicite}
            En résolvant avec Euler implicite, le schéma s'écrit:
            \begin{align}
                u^{n+1} = \bigl(Id - \Delta t (A^E + A^I)\bigr)^{-1} u^n.
            \end{align}
            Mais si l'opérateur $A^E$ est rend l'inversion coûteuse;
            par exemple s'il est non-local (impliquant la résolution d'un gros système au lieu de plusieurs petits systèmes), 
            ou s'il est non linéaire (nécessite d'être réinverser à chaque pas de temps);
            alors cette méthode ne sera pas viable non plus.
    \paragraph{Résolution par une méthode ImEx: une Runge et Kutta Additive}
            Mettons en oeuvre la méthode ImEx111. 
            L’approximation au pas de temps $n+1$ s'écrit en sommant une contribution issue de la méthode Euler explicite (RKE1)
            et une contribution issue de la méthode Euler implicite (RKI1):
            \begin{align}
                u^{n+1} = u^n + \Delta t (\underbrace{k_1}_{\text{RKE1}} + \underbrace{k_1'}_{\text{RKI1}})
            \end{align}
            La contribution RKE1 s'écrit:
            \begin{align}
                k_1 = A^E u^n.
            \end{align}
            La contribution RKI1 s'écrit:
            \begin{align}
                k_1' = A^I u^{n+1}
            \end{align}
            Ainsi: 
            \begin{align}
                &u^{n+1} = u^{n+1} = u^n + \Delta t ( A^E u^n +  A^I u^{n+1}),\\\notag
                \text{donc: }&u^{n+1} - \Delta t  A^I u^{n+1} = u^n + \Delta t  A^E u^n,\\\notag
                \text{et donc: }&u^{n+1} = (Id - \Delta t A^I)^{-1} \circ (Id + \Delta t A^E) u^n.
            \end{align}
            Ainsi dans cette méthode seul l’opérateur $Id- \Delta t A^I$ doit être inversé. Ce qui était bien l’objectif. Les opérateurs ont été 
            découplés lors de la résolution. 