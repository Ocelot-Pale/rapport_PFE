\label{par:ImEx_presentation}
Les méthodes ImEx étudiées sont les méthodes de Runge et Kutta additives (RK-ImEx ou RK-additive).
Ces méthodes consistent à sommer plusieurs méthodes de Runge et Kutta appliquées chacune à un opérateur différent.
\textit{L'objectif est d'intégrer chaque opérateurs avec des méthodes RK différentes (implicites ou explicites), en accord
les spécificité de chaque opérateur et cela, indépendamment des autres opérateurs.}
\subsubsection{Un exemple}
    Pour introduire la méthodes de Runge et Kutta additives, on commence par un exemple simple usant d'une méthode RK-ImEx
    d'ordre un. Cette ImEx naît de la conjugaison de deux méthodes Runge et Kutta à un étages (RK1). 
    Cette méthode est notée ImEx111 \cite{ASCHER1997151}.
    Les méthodes RK1 servant de briques élémentaires à la RK111 sont: un schéma d'Euler explicite et un schéma d'Euler implicite.
    Soit une équation d'évolution faisant intervenir deux opérateurs:
    \begin{itemize}
        \item[$\diamond$] L'opérateur $A^E$ se prêtant aux méthodes explicites (par exemple, un opérateur peu raide mais non local).
        \item[$\diamond$] L'opérateur $A^I$ se prêtant aux méthodes implicites (par exemple un opérateur raide mais local).
    \end{itemize}
    L'équation cible serait alors de la forme: 
    \begin{align}
        \dt{u} = A^E u + A^I u.
    \end{align}
    \paragraph{Résolution par approche monolithique}
        En n'utilisant qu'une seule RK1 pour tout le problème (approche monolithique), la dynamique serait approchée d'une des façon suivante.
            En simulant avec un schéma d'Euler explicite monolithique, la méthode s'écrit alors:
            \begin{align}
                u^{n+1} = u^n + \Delta t (A^E + A^I) u^n.
            \end{align}
            Mais si l'opérateur $A^I$ est très raide, la stabilité risque d'imposer un pas de temps très restrictif risquant de rendre la méthode \underline{non viable}.
            En résolvant avec un schéma d'Euler implicite monolithique, la méthode s'écrit:
            \begin{align}
                u^{n+1} = \bigl(Id - \Delta t (A^E + A^I)\bigr)^{-1} u^n.
            \end{align}
            Mais si l'opérateur $A^E$ rend l'inversion coûteuse;
            par exemple s'il est non-local (impliquant la résolution d'un gros système au lieu de plusieurs petits systèmes), 
            et/ou s'il est non linéaire (nécessite d'être réinverser à chaque pas de temps);
            alors cette méthode ne sera \underline{pas viable} non plus.
    \paragraph{Résolution par une méthode ImEx: une Runge et Kutta Additive}
            Lorsque la méthode ImEx111 est choisie,
            l’approximation au pas de temps $n+1$ s'écrit en sommant une contribution issue de la méthode Euler explicite (RKE1)
            et une contribution issue de la méthode Euler implicite (RKI1):
            \begin{align}
                u^{n+1} = u^n + \Delta t (\underbrace{k_1}_{\text{RKE1}} + \underbrace{k_1'}_{\text{RKI1}})
            \end{align}
            La contribution issue de la RKE1 appliquée à $A^E$ s'écrit (Euler explicit):
            \begin{align}
                k_1 = A^E u^n.
            \end{align}
            La contribution issue de la RKI1 appliquée à $A^I$ s'écrit (Euler implicit):
            \begin{align}
                k_1' = A^I u^{n+1}
            \end{align}
            Ainsi: 
            \begin{align}
                &u^{n+1} = u^n + \Delta t ( A^E u^n +  A^I u^{n+1}),\\\notag
                \text{donc: }&u^{n+1} - \Delta t  A^I u^{n+1} = u^n + \Delta t  A^E u^n,\\\notag
                \text{et donc: }&u^{n+1} = (Id - \Delta t A^I)^{-1} \circ (Id + \Delta t A^E) u^n.
            \end{align}
            Ainsi dans cette méthode seul l’opérateur $Id- \Delta t A^I$ est inversé et les problèmes de raideurs sont résolus;
            ce qui était l’objectif. Les traitements sur les opérateurs sont bien découplés lors de la résolution.
\subsubsection{Cadre mathématique général}
    Pour construire des méthodes plus complexes et d'ordres supérieurs introduisons le formalisme de \cite{ASCHER1997151} pour traiter les méthodes RK-additives. 
    Ici, nous travaillons uniquement sur méthodes ImEx pour deux opérateurs mais théoriquement, il est possible de construire des méthodes ImEx pour traiter 
    autant d'opérateurs que l'on le souhaite \cite{KENNEDY2003139}.
    \paragraph{Notations}
        Une méthode ImEx additive est construite à partir d'une méthode implicite à $s$ étages (une méthode DIRK et si possible SDIRK \cite{nasa_DIRK}) et d'une méthode explicite à $s+1$ étages
        \footnote{Au besoin, la méthode explicite peut être à $s$ étages, qui est un cas particulier d'une méthode à $s+1$ étages.}.
        Pour uniformiser, le tableau de Butcher de la méthode implicite est complété par une ligne et une colonne de zéros afin que les deux méthodes
        s'écrivent comme si elles avaient le même nombre d'étages.
        Les tableaux de Butcher des deux méthodes s'écrivent alors :
        
        \subparagraph{Méthode RKE, $s+1$ étages:}
        \begin{align}
        \text{RKE} : \quad
        \begin{array}{c|c}
        \tilde{c} & \tilde{A} \\
        \hline
        & \tilde{b}^T
        \end{array}
        =
        \begin{array}{c|ccccc}
        0 & 0 & 0 & 0 & \cdots & 0 \\
        \tilde{c}_1 & \tilde{a}_{10} & 0 & 0 & \cdots & 0 \\
        \tilde{c}_2 & \tilde{a}_{20} & \tilde{a}_{21} & 0 & \cdots & 0 \\
        \vdots & \vdots & \vdots & \ddots & \ddots & \vdots \\
        \tilde{c}_s & \tilde{a}_{s0} & \tilde{a}_{s1} & \tilde{a}_{s2} & \cdots & 0 \\
        \hline
        & \tilde{b}_0 & \tilde{b}_1 & \tilde{b}_2 & \cdots & \tilde{b}_s
        \end{array}
        \end{align}
        
        \subparagraph{Méthode RKI (DIRK) $s$ étages:}
        \begin{align}
        \text{RKI} : \quad
        \begin{array}{c|c}
        c & A \\
        \hline
        & b^T
        \end{array}
        =
        \begin{array}{c|ccccc}
        0 & 0 & 0 & 0 & \cdots & 0 \\
        c_1 & 0 & a_{11} & 0 & \cdots & 0 \\
        c_2 & 0 & a_{21} & a_{22} & \cdots & 0 \\
        \vdots & \vdots & \vdots & \ddots & \ddots & \vdots \\
        c_s & 0 & a_{s1} & a_{s2} & \cdots & a_{ss} \\
        \hline
        & 0 & b_1 & b_2 & \cdots & b_s
        \end{array}
        \end{align}
        
        où les coefficients $\tilde{a}_{ij}$, $\tilde{b}_i$, $\tilde{c}_i$ définissent la méthode explicite et 
        les coefficients $a_{ij}$, $b_i$, $c_i$ définissent la méthode implicite DIRK. 
        
    \paragraph{Schéma général d'une méthode RK-additive}
        Une étape de la méthode RK-additive appliquée entre les pas de temps $n$ et $n+1$ au système 
        $\frac{du}{dt} = A^E u + A^I u$ s'écrit :
        
        \subparagraph{Calcul des approximations intermédiaires:}
        Les calcul des approximations aux pas de temps intermédiaires $\left( u_i\right)_{i \in \left\{ 0,\ldots,s \right\}}$ 
        se fait grâce à la relation:
        \begin{align}
        u_i &= u^n + \Delta t \sum_{j=0}^{i-1} \tilde{a}_{ij} A^E u_j + \Delta t \sum_{j=0}^{i} a_{ij} A^I u_j, \quad &i = 0, 1, \ldots, s,\\\notag
        \text{En initialisant $u_0 = u^n$.}
        \end{align}
        Soit en mettant en lumière le caractère implicite de la méthode sur $A^I$:
        \begin{align}
        (Id - \Delta t a_{ii} A^I) u_i &= u^n + \Delta t \sum_{j=0}^{i-1} \left(\tilde{a}_{ij} A^E u_j + a_{ij} A^I u_j\right), \quad &i= 0, 1, \ldots, s
        \end{align}
        
        \subparagraph{Calcul de l'approximation déinitive:}
        \begin{align}
        u^{n+1} &= u^n + \Delta t \sum_{i=0}^{s} \tilde{b}_i A^E u_i + \Delta t \sum_{i=0}^{s} b_i A^I u_i
        \end{align}
        
        Cette formulation générale permet de construire des méthodes d'ordre élevé. 
    \paragraph{Ordre de convergence}
        L'ordre d'une méthode RK-additive est évidemment borné par l'ordre des méthodes RK individuelles convoquées.
        Naturellement, cette borne n'est pas nécessairement atteintes; des conditions d'ordre liant
        les coefficients des méthodes individuelles entre eux doivent être respectées. Le nombre de ses conditions augmente (très) rapidement avec 
        l'ordre de la méthode et le nombre d'opérateurs à résoudre \cite{KENNEDY2003139}, le lecteur motivé se référera par exemple à \cite{Hairer1981}.