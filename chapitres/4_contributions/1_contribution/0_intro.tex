\label{par:contrib_imex}
L'objectif est de comparer la pertinence des méthodes Runge et Kutta implicites-explicites par rapport au \textit{splitting} d'opérateurs traditionnel ainsi
que leurs éventuelles interactions avec l'adaptation en espace par multirésolution adaptative.
Pour introduire cette première étude l'équation de Nagumo est d'abord présentée comme un cas
test idéal pour éprouver les méthodes de résolution des équations de diffusion-réaction (\ref{par:analyser_operateurs_nagumo}).
Dans un second temps les méthodes ImEx utilisées sont détaillées (\ref{par:ImEx_presentation}).
Par la suite (\ref{par:contrib1:stab}) leur stabilité est évaluée dans un contexte général; puis en se focalisant sur l'équation de Nagumo, où elles sont comparée 
à une méthode de séparation d'opérateur.
Ceci permet de valider la pertinence \textit{a priori} de ces méthodes sur les équations de réaction-diffusion, 
et mène naturellement à une étude de convergence expérimentale avec et sans adaptation de maillage (\ref{par:contrib1:convergence}).