L'objectif est de comparer la pertinence des méthodes RK implicites-explicites au \textit{splitting} d'opérateurs traditionnel.
Pour introduire cette première étude l'équation de Nagumo est d'abord présentée comme un excellent cas
test pour éprouver les méthodes de résolution des équations d'advection-diffusion-réaction. 
Dans un second temps les méthodes ImEx utilisées sont détaillées. Par la suite leur stabilité
est évaluée dans un contexte général; puis en se focalisant sur l'équation de Nagumo, où elles sont comparée 
à une méthode de séparation d'opérateur classique (splitting de Strang).
Ceci permet de valider la pertinence \textit{a priori} de ces méthodes sur les équations de réaction-diffusion, 
et mène naturellement à une étude de convergence expérimentale.